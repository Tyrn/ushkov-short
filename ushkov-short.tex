\documentclass[11pt,a4paper]{memoir}

\pagestyle{plain}
\usepackage{longtable}

\usepackage[usenames,dvipsnames,svgnames,table]{xcolor}
\usepackage{fontspec}
\usepackage{xltxtra}
% code borrowed from Polyglossia documentation — Thanks!
\definecolor{myblue}{rgb}{0.02,0.04,0.48}
\definecolor{lightblue}{rgb}{0.61,.8,.8}
\definecolor{myred}{rgb}{0.65,0.04,0.07}

\usepackage{polyglossia}
\setmainlanguage{russian}
\setotherlanguages{churchslavonic,english}
\usepackage{churchslavonic}
\usepackage{lettrine}

\setmainfont[Mapping=tex-text]{Liberation Serif}
\setsansfont[Mapping=tex-text]{Liberation Sans}
\setmonofont[Mapping=tex-text]{Liberation Mono}

\newfontfamily\churchslavonicfont[Script=Cyrillic,Ligatures=TeX,Scale=1.33333333,HyphenChar="005F]{PonomarUnicode.otf} 
\newfontfamily{\slv}[Scale=MatchLowercase]{Ponomar Unicode TT}
\newfontfamily{\ust}[Scale=MatchLowercase]{Menaion Unicode TT}
\newfontfamily{\ind}{Indiction Unicode TT}

\usepackage{indentfirst}
\frenchspacing
\clubpenalty=10000
\widowpenalty=10000

\sloppy

%% Ensure sequential numbering of subsubsections.
\setsecnumdepth{subsubsection}
\counterwithout{subsubsection}{subsection}
\renewcommand{\thechapter}{\arabic{chapter}}
\renewcommand{\thesection}{\arabic{section}}
\renewcommand{\thesubsection}{\Roman{subsection}}
\renewcommand{\thesubsubsection}{\S\arabic{subsubsection}}

\setcounter{tocdepth}{3} % Must not precede the above
\usepackage[hidelinks]{hyperref}

%% Suppress \hline on demand
\newcommand{\hln}{}

\begin{document}
    
    \tableofcontents
    
    \chapter*{}
    \markboth{}{}
        \section*{Благолепие церковнославянского языка}
        \label{sec:blago}
        \addcontentsline{toc}{section}{\nameref{sec:blago}}

    Прошло уже более тысячи лет со времени Крещения Руси. Воспринятое ею Православие, благодаря трудам святых Мефодия и Кирилла, совершает свои благоговейные богослужения на церковнославянском языке. Он по своей структуре наиболее близок к греческому, и это позволило сохранить существующий в Греческой церкви строй богослужения, что явилось для многочисленных славянских народов великим благом и живительным источником благочестия и совершенствования нравственного в духе истины правой веры, а также руководством в земных делах: образования, иконографии, храмо- и градостроительства. Наша Русская Православная Церковь восприняла в неповрежденности и чистоте от Греческой Матери Церкви все догматическое, литургическое, духовное, святоотеческое наследие и предание, связав все это, с помощью славянского языка, с духом народа христианского на Руси, с жизнью и бытом и просвещением, ибо на славянский язык были переведены все богослужебные книги и писания греческих отцов, прием сохранялась точность, ясность и истинность, хотя перевод и являлся творческим делом.
    
    Славянский язык помогал памятовать о Христе, хранить Его живой образ и Его учение в сердце и душе и приносить плоды живого творческого восприятия слова Божия. В Церкви шло становление и образование самого славянского языка, его внутренняя христианизация и воцерковление, преображение самой стихии славянской мысли и слова, славянского голоса, души народа, На долю русского народа выпало редкое счастье принять христианство тогда, когда прогремели громом своим Вселенские Соборы и утверждено в чистоте апостольское Православие, и в Церкви Греческой утвердились и литургические, и догматические, и нравственные истины и самый язык славянский, ничем не запятнанный и чистый, воспринял для своего народа сосуд благодати, и в благодати Христовой народ верный обновляется, и просвещается, и спасается, и становится причастным вечной Истине, и славянский язык богатство веры и культуры передал многим полудиким племенам и народам, и в Церкви они нашли единение и братство Христово как дети Божии. На Руси потрудились многие ревнители духовного просвещения: и князь Владимир~---~просветитель, и св. Андрей Боголюбский, и свв. Александр Невский, Даниил Московский, святитель Петр, митрополит Алексий, преподобный Сергий Радонежский и многие другие подвижники благочестия. Созидались монастыри, храмы, школы. Создавались библиотеки, делались переводы на славянский язык, составлялись азбуки для малых и полудиких народностей, писались летописи.
    
    Славянский язык хранил церковное единство и православные традиции. 2 августа 1581 г. впервые была напечатана полная Библия на славянском языке в г. Остроге, что способствовало глубокому изучению Священного Писания. Русская Церковь в XVI в., когда протестанты превращали христианство в систему философских построений, как плод похи Возрождения язычества, начала вести длительную и тяжелую борьбу за сохранение чистоты веры, церковнославянского языка, церковно-национальной культуры.
    
    Славянская Библия занимает самое первое место по своей полноте и точности воспроизведения библейских древних писаний, о чем свидетельствуют и списки Кумранских рукописей, написанных за 200---300 лет до Р.Х., когда не было нужды искажать тексты Священного Писания. Древние рукописи своим текстом совпадают с острожской славянской Библией. В славянской Библии точно сохранено летоисчисление~---~5508 лет от сотворения мира до Рождества Христова, и все мессианские места. Известно, что текст 70 толковников, взятый в основу славянской Библии, долгое время считался в александрийских синагогах у евреев священным переводом, и его читали на греческом языке 200---300 лет. Масоретский же текст Библии идет от Акибы (около II века после Р.Х.).
    
    Кирилло-Мефодианский перевод Евангелия на славянский язык отличается точностью, верностью подлиннику и ясностью, чистотою выражения христианских понятий.
    
    Славянский текст Священного Писания и православного богослужения~---~свидетель первого тысячелетия, то есть неразделенного христианства, и в то же время он является драгоценным наследием равноапостольских трудов свв. Кирилла и Мефодия, на котором воспитываются чада Русской Православной Церкви.
    
    Церковнославянский богослужебный текст является важным и святым. Христиане разных конфессий стремятся осмыслить свои традиции в сравнении с традициями других исповеданий, ведь священные книги Нового Завета переведены на 1050 языков, и среди этих переводов самое важное место занимает славянский. Поскольку перевод Писания был сделан с греческого языка, а греческая текстология устанавливала единый печатный греческий текст XVII в., потом в XX веке его вытеснили из употребления, то в вопросе изысканий и оценки конечных результатов многих реконструкций текстов важную роль играет славянский текст как свидетель текстовой традиции первого тысячелетия, то есть времени неразделенного христианства.
    
    Острожская Библия 1581 года на славянском языке ни по форме, ни по содержанию не изменилась и является эталоном, по которому можно проследить реконструкцию греческого текста по времени.
    
    Анализ и сопоставление текстов показывает, что славянский текст стоит на твердой, нерушимой основе древней богослужебной практики Православия, а западное христианство отошло от древних традиций.
    
    Святые равноапостольные братья Кирилл и Мефодий использовали для перевода церковный византийский текст, который не искажался, был под контролем и оберегался всегда и всюду в Церкви, как нечто постоянное, неизменное и непреложное, как сама Истина, как церковно-текстуальное предание, уходящее в апостольскую древность. Это предание Церкви святое, соборное, апостольское, как поток, исходит из свежего родника живой воды евангельской, течет по чистому руслу под ясным небом и при тихой погоде, и его прозрачные воды менее доступны для загрязнения человеческими мудрованиями во имя кичащегося ума и в ущерб созидающей любви.
    
    В текстах встречаются несколько десятков слов труднопонятных, как тектон~---~дровосек, усмен~---~кожан, измена~---~выкуп, выну~---~всегда и др., но здравый человек за неделю их запомнит, ведь даже иностранные языки учат, да не один, а два-три языка, заучивая 40-45 слов в день, а язык, когда-то бывший родным, разговорным языком наших предков, язык, освященный веками, надо любить и хранить.
    
    Славянский язык в Церкви на Руси освящен непрерывным богослужение, и внешняя гармония, чистота, сердечность, молитвенность, мир Христов и утешение благодатное водворяются в душе православного христианина, как плод веры и любви.
    
    Большое значение славянскому языку для православного богослужения придает и Грузинская Церковь. Архимандрит Рафаил (Карелин) из Тбилиси выразил то так: <<В настоящее время в литургической жизни Грузинской Автокефальной Церкви славянский язык является по своему значению вторым, после древнегрузинского, богослужебным языком. Церковная славистика имеет для нас практическую важность и теоретический интерес>>.
    
    В связи с этим необходимо остановиться на нескольких вопросах, касающихся значения славянского языка для современного православного богослужения, рассматривая при этом славянский язык обобщенно и условно, как единую структурную и знаковую систему.
    
    В данном докладе славянский язык рассматривается обобщенно и условно, как единая структурная и знаковая система.
    
    Славянский язык насчитывает более чем тысячелетнюю литературную историю. Он не стал мертвым языком, не сделался достоянием одних археологов и лингвистов; славянский язык звучит на всех континентах, на нем совершается богослужение в православных церквах, на нем миллионы людей воссылают молитвы к Богу.
    
    В последнее время неоднократно поднимался вопрос: не является ли славянский язык литургическим архаизмом, преградой к пониманию текстов, и не следует ли перевести все богослужебные книги на современные языки?
    
    Такие попытки предпринимались в России и других странах.
    
    В России языковые литургические реформы не встретили поддержки основной массы верующих, и на сегодняшний день богослужебным языком остается церковнославянский. Можно ли объяснить это явление только одной приверженностью к традиции, или же оно имеет еще другие основания?
    
    Полагаем, что эти основания следующие.
    
    I. Славянскому языку, как и другим древним языкам (имеется в виду древнегреческий и древнегрузинский), присуща особая динамическая структура; он лучше передает пульс религиозной жизни, глубже выражает молитвенные чувства, чем современные языки.
    
    Древние языки более подходят к цельному синтетическому восприятию, новые~---~к аналитическому, дробному; древние~---~к созерцанию, новые~---~к логизированию; древние языки полны энергии и эмоций, новые, в сравнении с ними, носят рационалистический, описательный характер. Древние языки дают большую возможность соприкоснуться с глубиной явлений, с духовными субстанциями, сделать человека участником событий, современные~---~изложить явления в определенной системе и дать их анализ.
    
    Сравнивая лексическую и грамматическую ткань древних и новых языков, мы видим, что древние языки располагают меньшим по объему словарным фондом, но их грамматический строй отличается большим многообразием, пластичностью и совершенством.
    
    Мобильность и экспрессивность глагольных форм, богатство аффиксов и флексий, не имеющих аналогии в современных языках, лаконизм и динамика синтаксических структур, этимологическая глубина лексики создают неповторимую выразительность и, по словам патриарха Пимена, <<особую красоту>> славянского языка. Эти возможности славянского языка позволяют выразить в богослужебных текстах многоплановость событий, синергизм и параллелизм действий, объединить различные хронологические периоды в единые смысловые циклы (как бы преодолев линейную протяженность времени), проявить и усилить подтекст каждого предложения.
    
    Сложность и гибкость глагольных форм славянского языка делают одну и ту же по словарному составу фразу то ажурно-легкой и полетной, то~---~до физической ощутимости тяжелой и твердой, как бы изваянной из мрамора. Лаконизм, внутренняя чеканность и в то же время как бы внешняя незавершенность и обрывистость предложений, часто без последовательных переходов от одного предмета к другому, наделяют смысловые паузы эмоциональным содержанием, подчеркивая, что богослужебный текст~---~это не монолог, а диалог, таинственная беседа души с Божеством, что в священных событиях присутствует, как их постоянный субъект, личность каждого молящегося.
    
    Говоря образно, динамика древних языков созвучна динамике света. Богослужение~---~это симфония из лучей Божественных энергий. Священная история изображается в этой Божественной симфонии на светящемся фоне вечности, земные реалии~---~в их логоистическом преображении.
    
    Следует отметить, что в богослужебных текстах мастерски использован точно и строго очертанный круг изобразительных средств. Это делалось древними гимнографами для того, чтобы не подменить духовных переживаний эстетическими, т.е. душевными, не обременить ума молящегося художественной информацией и, тем самым, не оземлить молитву.
    
    Древние гимнографы мыслили Церковь как идею единства в Ее реальном и мистическом воплощении. Унификация изобразительных средств~---~одно из условий общности молитвы, духовного познания, эмоций и сопереживаний верующих~---~тела Церкви.
    
    Динамичные структуры, внутренние смысловые емкости и скульптурная пластичность славянского языка помогают воспринимать традиционные символы и метафористические образы в разных контекстах как вечно новые и неповторимые.
    
    Одним словом, древние языки более приспособлены для выражения явлений и динамики духовной жизни. Это первая и главная причина их сохранения в православном богослужении.
    
    II. Второе основание~---~трудность самого перевода.
    
    Богослужебные тексты~---~это шедевры священной поэзии особого типа и порядка. Переводчик должен хорошо знать святоотеческое богословие, именно святоотеческое, а не схоластическое, чтобы понять глубокие по догматическому содержанию литургические тексты. Далее, переводчик должен обладать одновременно глубокой филологической подготовкой и незаурядным поэтическим дарованием. Без этого перевод богослужебных книг будет казаться обесцвеченным и прозаическим пересказом великих поэм.
    
    Православные церковные службы называют опоэтизированным, иконографическим, поющим богословием.
    
    И, наконец, самое главное, переводчик должен иметь живую веру, церковность и религиозную интуицию, которая дала бы возможность почувствовать, пережить, оценить каждую фразу и слово переводимого текста.
    
    Интересно, что Петербургский митрополит Гавриил и Московский митрополит Филарет, руководившие работами по переводу святоотеческих произведений с греческого языка на славянский и русский, считали нужным давать рукописи переводов для проверки простым старцам-монахам, которые не знали греческого языка, но, обладая общностью внутреннего духовного опыта с древними писателями-подвижниками и находясь с ними, так сказать, на одной духовной волне, интуитивно чувствовали всякую фальшь и ошибку перевода.
    
    III. Третье основание~---~традиция. Традиция~---~это кристаллизация истории. Это актуальное бытие прошлого в настоящем. Любовь и уважение к прошлому сохранили до наших дней старинные соборы и храмы в их первозданном величии, иконы и фрески изумительной духовной глубины. Живая традиция сохранила нам дивное, неповторимое православное богослужение.
    
    Само слово <<Православие>> значит правое, правильное богопочитание, основным признаком которого является верность апостольскому преданию.
    
    Церковная традиция возникла из общности духовного опыта христиан, опыта emph{богообщения}. emph{Одинаковое религиозное чувство создало} одинаковые или очень схожие формы. Церковное богослужение и храмовое устройство~---~это синтез жизни Церкви в эпоху ее духовного расцвета. Содержание создает формы, но и формы хранят и оберегают содержание. В данном случае древние языки имеют вспомогательное, но важное значение для сохранения в чистоте и внутренней целостности одного из видов церковного предания~---~богослужебного канона.
    
    Чем устойчивее и тверже канон, тем глубже и чище он выражает общечеловеческую и духовную потребность: каноническое есть церковное, церковное~---~соборное, соборное же~---~всечеловеческое.
    
    Славянский язык, наряду с другими древними языками, стал сакральным, священным языком Церкви.
    
    Вопрос о ясности богослужебных славянских текстов для верующих приобретает все большее значение.
    
    Следует обратить особое внимание на научную сторону новых изданий богослужебных книг, снабдить их справочным материалом, в том числе словарем, где наряду с буквальным переводом давалась бы этимология славянской лексики. Не только Священное Писание, но и следующий за ним духовный пласт~---~богослужебный канон~---~также нуждается в толкованиях и комментариях. Перед нами стоит задача составления литургической экзегезы, отвечающей интеллектуальному уровню и духовным запросам современных верующих.
    
    Славянское православное богослужение~---~это бесценное сокровище мистического гносиса, огромный потенциал духовных сил и энергии, который мы должны сохранить не только для себя, но и для будущих поколений.
    
    Велико значение церковнославянского языка для православного богослужения и православного пастыря. Но церковнославянский язык имеет очень важное значение и для православного хистианина. На этом языке совершается богослужение в нашей Православной Русской Церкви и написаны наши священные и богослужебные книги. По своему возвышенному характеру, по своей силе и звучности церковнославянский язык является наиболее совершенным средством для выражения религиозных настроений православного русского человека. Высшие стремления души и глубокие чувствования, отрешенные от земного и направленные к небесному, чистому и вечному, получают наиболее соответствующее выражение в этом языке, далеком от всего обычного, житейского. Но для этого необходимо, чтобы все читаемое в храме Божием достигало своей цели, т.е. доходило бы до сознания души христианина, научая ее истинному благочестию и указывая ей путь спасения. А потому нужно, чтобы те, которые употребляют язык Церкви~---~чтецы, певцы церковно- и священнослужители~---~в совершенстве понимали то, что произносят их уста. Кроме того, без полного знания языка Матери Церкви нельзя дать ответа вопрошающим о недоуменных словах и выражениях церковного языка. А таких слов и выражений в церковнославянском языке достаточно.
    
    Итак, все, что читается и поется в храмах Божиих, должно быть истолковано людьми, стоящими на высоте своего призвания, а такими людьми, прежде всего, должны быть православные пастыри. Только тогда у православного русского человека укрепится сознательное и вместе с тем благоговейное отношение к богослужению нашей Православной Церкви, и само его мировоззрение обогатится достаточным запасом религиозно-церковных понятий
    
    Славянский язык назван именно славянским потому, что на нем говорили наши предки славяне.
    
    Славяне, обитавшие первоначально около Карпат, долгое время были непросвещенным языческим народом. Очень часто им случалось соприкасаться с культурными народами Византийской и Римской империй. Мало-помалу славяне стали подчиняться влиянию этих уже просвещенных христианством народов. От греков и римлян к славянам проникло и быстро у них распространилось христианство. Но, слушая богослужения на чуждых им греческом или латинском языках, славяне не могли себе вполне усвоить истины нового для них учения, и потому многие из них были христианами лишь по имени, а по существу сохранили прежнюю грубость нравов и держались старых языческих суеверий. Так было до IX века.
    
    В половине IX века святые братья Кирилл (в миру Константин) и Мефодий, родом греки, задумали облегчить славянам понимание христианского богослужения и перевели для них богослужебные книги с греческого языка на славянский. Так как в то время у славян не было еще письменности, то святые братья сами составили славянскую азбуку, взяв за образец греческую азбуку и дополнив ее недостающими буквами. Таким образом, богослужебные книги были переведены ими на славянский язык, а сами они стали апостолами~---~просветителями славян.
    
    Святые братья Кирилл (827---869 гг.) и Мефодий (\dag 885 г.) были детьми знатного вельможи и родились в Солуни (Фессалониках)~---~главном городе Македонии, страны, населенной по преимуществу славянами. Старший брат, Мефодий, после домашнего воспитания занимал сначала военные, а потом административные должности, но впоследствии принял монашество, поселившись на горе Олимп. Младший, Кирилл, отличался блестящими способностями к учению, обучался словесным, философским и математическим наукам. Его ожидали почести в свете, но он не прельстился этим, а принял сан священника и место библиотекаря в библиотеке при храме св. Софии. Впоследствии св. Кирилл удалился к брату на Олимп. Предполагают, что в это время жизни на Олимпе святые братья и начали переводить богослужебные книги на славянский язык.
    
    Около 862 года, через посредничество Византийского императора Михаила III, святые братья были приглашены Моравским князем Ростиславом в Моравию для проповеди христианства. Здесь начинается главная деятельность свв. Кирилла и Мефодия. В княжествах Моравском и Панионском (где ныне Венгрия и Чехия) христианство было уже проповедано среди тамошних славян немецким, латинским духовенством. Но латинское духовенство совершало богослужение на чуждом славянам латинском языке. А так как латинский язык был непонятен славянам, то, конечно, проповедь немецкого духовенства оказалась безуспешной.
    
    Святые братья, начав учить славян Христовой вере, дали им священные книги на славянском языке. На славянский язык были переведены и богослужебные книги. На славянском языке стали они совершеть богослужение, а для обучения славян письменности стали заводить и школы. Христианство между славянами начало быстро распространяться.
    
    Немецкое духовенство, видя успехи святых братьев в деле распространения ими христианства среди славян, из зависти подало на них жалобу в Рим. Римский папа Николай I вызвал к себе святых братьев на суд (в то время не было еще разделения Церкви, все были православные, и только что начинались несогласия между Римом и Византией). Святые братья отправились в Рим, но уже не застали папу Николая I в живых, а преемник его Андриан очень благосклонно отнесся к свв. Кириллу и Мефодию. Кирилл представил папе Евангелие на славянском языке, и папа положил славянский перевод в соборе св. Петра, где святые братья даже отслужили торжественную литургию на славянском языке. Св. Кирилл в Риме заболел и скончался. Умирая, он убеждал св. Мефодия не покидать дела проповеди и распространения веры среди славян. Папа Андриан рукоположил Мефодия в епископа Моравии, отпустил его с честью и дал грамоту, в которой признал богослужение на славянском языке целесообразным.
    
    По возвращении св. Мефодия к Моравийским славянам враги не оставили его в покое. Они восстановили против него немецкого императора Людовика и добились того, что Мефодий был заточен в темницу, где и пробыл в заточении два с половиной года. Когда ему возвращена была свобода, то его вновь оклеветали перед папой. Пришлось Мефодию опять отправляться в Рим и защищать себя. Оправданный папой, св. Мефодий продолжал свое великое дело.
    
    Когда св. Мефодий скончался, то на учеников его было воздвигнуто гонение. Они, изгнанные из Моравии, ушли в другие славянские земли, главным образом, в Болгарию. Болгария заботливо хранила труды святых просветителей и потом, когда в конце X века, при св. князе Владимире, Русь приняла христианство, списки этих книг передала нашему отечеству. В те древние времена славянские наречия очень мало отличались друг от друга, и книги одного славянского племени были совершенно пригодны для богослужения и чтения другого.
    
    О высоком значении церковнославянского языка убедительно говорит Соколов Г.А. в своем <<Слове в защиту церковнославянского языка>> (Астрахань, 1910).
    
    <<Славянское слово имеет определенную силу как фактор религиозно-нравственного развития, вот уже 1000 лет просвещающего Русь Православную. Его воспитательное значение сохраняется и до сих пор, ибо те же чувства, то же настроение господствует среди чад Русской Православной Церкви и непосредственная сила религиозного наставления, выраженная в формах славянской речи, особенно чувствуется ими>>.
    
    Содержание церковных книг неотделимо от формы. Церковнославянский язык является в данном случае идеальной формой. Отделить форму от идеи~---~значило бы ослабить ее, лишить ее всей полноты ее силы и значения. Сбросить эту форму без ущерба для содержания невозможно.
    
    Слово церковное слилось с жизнью Православной России.
    
    Церковнославянский язык~---~это драгоценный сосуд, созданный для высокого содержания, вполне ему по основным своим свойствам соответствующий. Ведь не напрасно историки пришли к выводу, что труд переводов Мефодия и Кирилла, <<их великое дело было пережито нами в собственной истории: в русской мысли, в русском чувстве, в русском подвиге>>.
    
    Насколько вера искреннее и глубже, настолько острее и потребность в форме именно славянской речи, в простом и величавом ее течении, в ее дивной методичности и музыкальности.
    
    Сила мысли и сила чувства находятся в связи с силой церковнославянской речи.
    
    Быть в церкви, возноситься душою к Богу, поучаться стало не только потребностью, но и влечением, свойством души, природой.
    
    Славянское слово исполнено веры и высшей духовной правды.
    
    Церковнославянская речь не только своим духом, исполненным непосредственной веры и умиления, но даже видом своих письмен влечет в горний мир возвышенных созерцаний и настроений, христианского подвига и нравственного совершенства.
    
    Церковнославянский язык как бы впитал в себя те свойства, какие принадлежат самому тексту, и самим своим видом и особенностями способствует восприятию текста.
    
    История и жизнь убеждают нас, что славянский язык~---~это сокровище, которое нужно хранить, которым нужно дорожить.
    
    Благодать Божия поставила славянский язык в один ряд с тремя превосходными языками: греческим, латинским и еврейским.
    
    Некоторая оторванность языка литургического от разговорной речи, от обыденной жизни,~---~составляет достоинство молитвенной речи. Иметь для религиозных целей язык понятный и в то же время торжественный, далекий от того, в котором встречаются вульгарности (а в наше время вульгарность~---~уже принятый тон, потому что речь изобилует грубостью и грязью), и даже от слов и оборотов обыденного языка~---~это общечеловеческая потребность.
    
    Как первые детские впечатления кладут самые глубокие основы нравственной и душевной жизни человека и остаются на всю жизнь милы и дороги, так и первоначальные священные и богослужебные тексты навсегда остаются дороги и священны для народа, т.к. от их впечатлений и влияний возникли первые начала самой живой, именно религиозной стороны духовной жизни народа.
    
    Славянский язык необходим для обнаружения религиозных стремлений.
    
    Славянский язык~---~язык Церкви и религиозной письменности.
    
    Славянский язык~---~лучшее средство, чтобы привести к крепкой вере и жизни по вере.
    
    До принятия христианства на Руси не было устойчивого языческого мировоззрения, твердо выработанной языческой космогонии.
    
    Расшатать и разорвать связь между религиозными и нравственными стремлениями и славянским языком~---~не значило бы поколебать те основы духовных устремлений, которые были заложены еще 1000 лет тому назад.
    
    Учение веры и нравственности отлилось в формы славянской речи.
    
    Славянский язык принес неоценимые религиозно-просветительные блага русскому народу. Славянский язык дал те привычки, которые у народа русского стали второй природой. В чем особенности этой привычки? Своей искренностью, непосредственностью, глубоко проникнутая верой в Бога и Его промыслительные действия, речь на этом языке самим складом своим, самим выбором слов и выражений, тем духом, который, ее проникает, производит теперь это уже привычное влияние на душу верующего. Далее, язык церковнославянский, уже много веков имея единственной задачей выражать Божественное учение и повествования о домостроительстве Божием в направлении жизни людей, так сроднился с тем, для выражения чего он служит, что воспитанный на нем народ особенно восприимчив именно к тому, что он выражает, и в самом его звуковом составе, в движении речи ощущает этот высший смысл, слышит голос Бога, сам к Нему на этом языке обращается и веками усвоил то убеждение, что славянский язык есть орудие, средство общения на нем с Богом.
    
    Там, где у нас на Руси для целей религиозных вместо славянского языка употребляют русский, там уже нет Православия. Есть у нас секты, где на собраниях, имеющих характер богослужения, читают и поют по-русски. Но кто же их назовет православными? И вот те именно причины, которые отделили их от Православия и сделали их по мировоззрению и религиозным понятиям иными~---~они-то привели их к предпочтению русского языка языку славянскому. Когда под влиянием новых учений и критического отношения к вере отцов изменилось настроение, прекратилась восприимчивость к живым и непосредственным воздействиям славянской речи, тогда православное богослужение и обряды потеряли для них свое значение. Дальнейшим результатом всего этого явилось стремление проникнутый духом православной веры язык славянский заменить языком живой, разговорной речи, более удовлетворяющим требованиям сознания. Но, несомненно, он уже не то говорит воображению и чувству. Когда явилось стремление к замене для религиозных нужд славянской речи речью русской, там, очевидно, уже произошли духовные перемены, ведущие к измене Православию. Склонность к тому или иному языку есть только симптом и результат этих внутренних и глубоких перемен в вере и мировоззрении. Лишь перемены в верованиях ведут за собой измену языку прежних верований.
    
    Православные, пока они православные, не могут желать этой перемены уже по складу своей духовной природы; и чем тверже, искреннее и глубже вера их, тем дальше они от этих желаний. Они непосредственно чувствуют, что перемена языка была бы связана с переменой в самом духе православной религии. В славянской речи они слышат молитвенный голос своих отцов и дедов и сливаются с ними в единстве многовековой истории, в единстве молитвы всей России, всех славян.
    
    Те, кто желает перевода богослужебных текстов на русский язык, желают этого не в силу своих собственных религиозных потребностей. Они почти все равнодушны и к богослужению, и к языку Церкви. Они желают только в силу теоретических соображений о большей доступности текстов богослужебного языка, о более сознательном отношении к читаемому и т.д. Но те, кто живут религиозной жизнью, кто нуждается в Церкви и службах ее, те именно и не желают перевода, да и не могут желать, не изменившись в своих отношениях к религии и к тем чувствам, которые вызываются славянским языком.
    
    Заменить один язык другим~---~значит изменить свойства и особенности содержания, придавши ему новый характер, несоответствующий тому, какой вложили духоносные переводчики в славянские тексты; это значило бы принизить его содержание, приблизить к мирской жизни, а нередко придать и характер вульгарности.
    
    Всякое слово православного на славянском языке знакомо, близко, сродни всякому человеку, воспитанному в Церкви.
    
    Церковнославянский язык далек от утилитарности, он возбуждает чувство великого, возвышенного и вековечного. Он не колеблется и не изменяется~---~сообразно с тем, как неизменна и устойчива вера православная.
    
    Перевести на частный язык, каким является русский, Божественные истины~---~значило бы поставить их в общий поток жизни, внося в перевод перемены в языке, вызванные нуждами жизни, т.е. обусловленные социальными движениями, изменениями в искусствах, науке, торговле и вообще во всех сферах деятельности, обусловливающих развитие языка. Но как вера устойчива и неизменна в своем существе, так должен быть устойчив и язык религии, не меняясь с каждым поколением, согласно движению общественной жизни. Язык общеупотребительный в житейских делах с каждым поколением уже не тот, и одно столетие жизни вносит в него значительные перемены. Неужели и язык Церкви с каждым поколением должен подновляться и изменяться, внося новые и новые перемены вопреки неизменяемости Кафолической Церкви?
    
    Когда на Русь была перенесена церковнославянская письменность, то на этом языке совершалось богослужение, читались священные книги, писались жития русских святых, летописи, писались проповеди и другие религиозного и нравственно-назидательного содержания творения. И только акты государственные, договоры и другие официальные бумаги писались по-русски.
    
    В то время между языками теми существовало значительное сходство. Языки влияли один на другой вплоть до времени Петра I, когда возмогли возникать сомнения, считать ли главный отдел литературы (летописи, сказания, хождения и т.д.) высшим стилем русской речи или низшим стилем речи церковнославянской.
    
    Церковнославянский язык дает созерцание высоких религиозных идей и образов, влияет на сердце, возбуждает нравственные чувства, влечет к улучшению жизни на началах любви, самопожертвования и служения своим близким. Он имеет свои особенные задачи. Не принижаясь, не ослабевая, религиозная вера не может заковываться в формы, свойственные знанию.
    
    Церковнославянский язык достоин сохранения и как лучшая форма выражения великой идеи христианства в том виде, в каком оно сложилось в Православной Церкови и осуществилось в душе русского народа.
    
    Мысль о богослужении на русском языке родилась у людей, стоящих вдали от Православия, равнодушных к Церкви, не переживавших на себе влияния славянской речи.
    
    Славянский язык мы защищаем не по праву его давности, а лишь потому, что он имеет благотворное значение и свои преимущества перед русской речью.
    
    Различие во взглядах на славянский язык в конце концов связывается с различием мировоззрения и отношения к религии. Со стороны отрицателей славянского языка сказывается и отрицательное отношение к религии. Не является ли отрицание славянского языка в богослужении посягательством на самую религию, а не только на формы ее выражения?
    
    Нерасположенные к славянскому языку относятся не сочувственно и к религии.
    
    \section*{Краткий учебник церковнославянского языка\\
        \emph{предисловие}}
    \label{sec:brief}
    \addcontentsline{toc}{section}{\nameref{sec:brief}}
    
    Составитель сего учебника доброй памяти emph{профессор Анатолий Васильевич Ушков}.
    
    Анатолий Васильевич Ушков родился 7 августа 1894 года в г. Самаре в семье сотрудника Самарской Духовной Консистории. По окончании четырех классов Самарской Духовной семинарии, он в 1912 году успешно сдал вступительные экзамены в Казанский университет и был зачислен на физико-математический факультет. После университета, в 1916 году, А. Ушков был мобилизован на военную службу; проходил обучение в Киевском артиллерийском училище. С 1918 по 1945 г. он преподавал математику и физику в средних и специальных учебных заведениях в городах Самаре, Красноярске и Москве. Заведовал учебной частью и был руководителем методических совещаний по вопросам преподавания физики и математики. С 1939 по 1943 г. заочно учился в Московском государственном педагогическом институте на факультете русского языка и литературы, закончив который, некоторое время, наряду с преподаванием математики, вел курс литературы в средних учебных заведениях г. Москвы.
    
    В 1945 г. он осуществил свою давнишнюю мечту~---~послужить Святой Церкви и поступил в Московскую Духовную Академию. По окончании академии, в 1949 г., со званием кандидата богословия, полученным за сочинение <<Душа и ее бессмертие по христианскому учению>>, он преподавал катехизис, а затем церковнославянский язык в духовной семинарии. В 1964 г., по прочтении двух пробных лекций <<Древний мир перед пришествием Христа Спасителя>> и <<Основные положения формальной логики>>, А.В. Ушкову было присвоено звание доцента и поручено чтение лекций по логике в академии. Им составлен ряд учебных пособий для духовных школ: <<Краткая пасхалия>>, <<Логика в курсе академического образования>>, <<Краткая пасхалия в общедоступном изложении>>, <<Астрономический справочник>>, <<Краткое описание солнечной системы>>, <<Календарный счет времени>>, <<Счет и мера с древнейших времен до наших дней>>. Анатолием Васильевичем был подготовлен учебный курс церковнославянского языка, за который Совет академии присвоил ему 15 декабря 1969 г. степень магистра богословия и утвердил в должности профессора.
    
    Анатолий Васильевич отличался большой скромностью, чутким, внимательным отношением к людям; поражала его необычайная работоспособность, величайшее трудолюбие. Жажда проповеди слова Божия была его духовной потребностью. Он произнес 65 проповедей в праздничные и воскресные дни в академическом храме, в них~---~яркое выражение его глубоких христианских чувств и безграничной, сердечной любви к Богу и людям. В начале 1971---1972 учебного года Анатолий Васильевич по болезни вынужден был оставить работу в академии и перейти на пенсию.
    
    Скончался А.В. Ушков 14 января 1972 г., на 78-м году жизни, приобщившись Святых Христовых Тайн. Похоронен на кладбище в Сергиевом Посаде.
    
    Подготовил к печати доцент-иерей В. Москвич.
    
    \chapter*{1-й класс}
    \label{ch:firstgrade}
    \addcontentsline{toc}{chapter}{\nameref{ch:firstgrade}}
        \section*{Введение}
        \label{sec:intro}
        \addcontentsline{toc}{section}{\nameref{sec:intro}}
                \subsubsection[Значение церковнославянского языка]{Значение церковнославянского языка для православного богослужения и православного пастыря}
                
    Церковнославянский язык имеет очень важное значение для православного христианина. На этом языке совершается богослужение в нашей Православной Русской Церкви и написаны наши священные и богослужебные книги. По своему возвышенному характеру, по своей силе и звучности церковнославянский язык является наиболее совершенным средством для выражения религиозных настроений православного русского человека. Высшие стремления духа и глубокие чувствования, отрешенные от земного и направленные к небесному, чистому и вечному, получают наиболее соответствующее выражение в этом языке, далеком от обычного, житейского.
    
    Но для этого необходимо, чтобы все читаемое в храме Божием достигало своей цели, т.е. доходило бы до сознания души христианина, научая ее истинному благочестию и указывая ей путь спасения. А поэтому нужно, чтобы те, которые употребляют язык Церкви~---~чтецы, певцы, церковно- и священнослужители~---~в совершенстве понимали то, что произносят их уста. Да кроме того, без полного изучения языка матери-Церкви нельзя дать ответа вопрошающим о недоуменных словах и выражениях церковного языка. А таких слов и выражений в церковнославянском языке достаточно. Взять хотя бы такие слова: \emph{абие, амо, аще, выну, вкупе, ей, ктому, нань, леть, паче, паки, рамо, сице, тацы, убо, уне, яко}; или такие выражения: \emph{Змий сей, его же создал еси ругахуся ему; Чермнует бо ся небо; И бяху выну в церкви; Обыде нас последняя бездна; Любовию же, Дево, песни ткати спротяженно сложенныя неудобно есть; Противодышущу росоносному Духу, со огнем сущу пояху} и многие другие.
    
    Итак, все что читается и поется в храмах Божиих, должно быть истолковано людьми, стоящими на высоте своего призвания, а такими людьми, прежде всего, должны быть православные пастыри. Только тогда у православного русского человека укрепится сознательное и вместе с тем благоговейное отношение к богослужению нашей Православной Церкви и самое его мировоззрение обогатится достаточным запасом религиозно-церковных понятий.
    
                \subsubsection{Возникновение письменности у славян}
                
    Славянский язык назван именно \emph{славянским} потому, что на нем говорили наши предки славяне.
    
    Славяне, обитавшие первоначально около Карпат, долгое время были непросвещенным языческим народом. Очень часто им случалось соприкасаться с культурными народами Византийской и Римской империй. Мало-помалу славяне стали подчиняться влиянию этих уже просвещенных христианских народов. От греков и римлян к славянам проникло и быстро у них распространилось христианство. Но, слушая богослужение на чуждом им греческом или латинском языке, славяне не могли себе вполне усвоить истины нового для них учения, и потому многие из них были христианами лишь по имени, а по существу сохраняли прежнюю грубость нравов и держались старых языческих суеверий. Так было до половины IX века.
    
    В Половине IX века святые братья Кирилл (в миру Константин) и Мефодий, родом греки, задумали облегчить славянам понимание христианского богослужения~---~и перевели для них богослужебные книги с греческого языка на славянский. Так как у славян в то время не было еще письменности, то святые братья сами составили славянскую азбуку, взяв за образец греческую азбуку и дополнив ее недостающими буквами. Таким образом, богослужебные книги были переведены ими на славянский язык, а сами они стали апостолами-просветителями славян.
    
                \subsubsection{Деятельность святых братьев Кирилла и Мефодия}

    Святые братья Кирилл (827---869 гг.) и Мефодий (\dag 885 г.) были детьми знатного вельможи и родились в Солуни (Фессалониках)~---~главном городе Македонии, страны, населенной по преимуществу славянами. Старший брат, Мефодий, после домашнего воспитания занимал сначала военные, а потом административные должности, но впоследствии принял монашество, поселившись на горе Олимп. Младший, Кирилл, отличался блестящими способностями к учению, обучался словесным, философским и математическим наукам. Его ожидали почести в свете, но он не прельстился этим, а принял сан священника и место библиотекаря в библиотеке при храме св. Софии. Впоследствии св. Кирилл удалился к брату на Олимп. Предполагают, что в это время жизни на Олимпе святые братья и начали переводить богослужебные книги на славянский язык.

    Около 862 года, через посредничество Византийского императора Михаила III, святые братья были приглашены Моравским князем Ростиславом в Моравию для проповеди христианства. Здесь начинается главная деятельность свв. Кирилла и Мефодия. В княжествах Моравском и Панионском (где ныне Венгрия и Чехия) христианство было уже проповедано среди тамошних славян немецким, латинским духовенством. Но латинское духовенство совершало богослужение на чуждом славянам латинском языке. А так как латинский язык был непонятен славянам, то, конечно, проповедь немецкого духовенства оказалась безуспешной.

    Святые братья, начав учить славян Христовой вере, дали им священные книги на славянском языке. На славянский язык были переведены и богослужебные книги. На славянском языке стали они совершать богослужение, а для обучения славян письменности стали заводить и школы. Христианство между славянами начало быстро распространяться.

    Немецкое духовенство, видя успехи святых братьев в деле распространения ими христианства среди славян, из зависти подало на них жалобу в Рим. Римский папа Николай I вызвал к себе святых братьев на суд (в то время не было еще разделения Церкви, все были православные, и только что начинались несогласия между Римом и Византией). Святые братья отправились в Рим, но уже не застали папу Николая I в живых, а преемник его Андриан очень благосклонно отнесся к свв. Кириллу и Мефодию. Кирилл представил папе Евангелие на славянском языке, и папа положил славянский перевод в соборе св. Петра, где святые братья даже отслужили торжественную литургию на славянском языке. Св. Кирилл в Риме заболел и скончался. Умирая, он убеждал св. Мефодия не покидать дела проповеди и распространения веры среди славян. Папа Андриан рукоположил Мефодия в епископа Моравии, отпустил его с честью и дал грамоту, в которой признал богослужение на славянском языке целесообразным.

    По возвращении св. Мефодия к Моравийским славянам враги не оставили его в покое. Они восстановили против него немецкого императора Людовика и добились того, что Мефодий был заточен в темницу, где и пробыл в заточении два с половиной года. Когда ему возвращена была свобода, то его вновь оклеветали перед папой. Пришлось Мефодию опять отправляться в Рим и защищать себя. Оправданный папой, св. Мефодий продолжал свое великое дело.

    Когда св. Мефодий скончался, то на учеников его было воздвигнуто гонение. Они, изгнанные из Моравии, ушли в другие славянские земли, главным образом, в Болгарию. Болгария заботливо хранила труды святых просветителей и потом, когда в конце X века, при св. князе Владимире, Русь приняла христианство, списки этих книг передала нашему отечеству. В те древние времена славянские наречия очень мало отличались друг от друга, и книги одного славянского племени были совершенно пригодны для богослужения и чтения другого.


        \section{Церковнославянское буквоначертание}
                \subsubsection{Церковнославянская азбука}

    Устная наша речь состоит из членораздельных \textbf{звуков}.

    Изображения звуков речи условными знаками в письме или в печати называются \textbf{буквами}.

    Собрание всех букв языка, расположенных в общепринятом порядке, называется \textbf{азбукой} (от названия двух первых славянских букв <<аз>> и <<буки>>). Церковнославянская азбука состоит из 39 букв.


    \begin{center}
        \renewcommand*{\arraystretch}{1.4}    \begin{longtable}{|c|c|c|c|}
            \caption*{Буквы церковнославянской азбуки}\\
            \hline
                Прописная буква & Малая буква & Название & Произношение\\
            
            \hline
            \endfirsthead
            \multicolumn{4}{l}%
            {
%                \tablename\ \thetable\ -- 
                \footnotesize\textit{Начало на предыдущей странице}
            } \\
            \hline
                Прописная буква & Малая буква & Название & Произношение\\
            \hline
            \endhead
            \hline \multicolumn{4}{r}{\footnotesize\textit{Продолжение на следующей странице}} \\
            \endfoot
            \hline
            \endlastfoot
            {\slv{А}} & {\slv{а}} & аз & а\\\hln
            {\slv{Б}} & {\slv{б}} & буки & б\\\hln
            {\slv{В}} & {\slv{в}} & веди & в\\\hln
            {\slv{Г}} & {\slv{г}} & глаголь & г\\\hln
            {\slv{Д}} & {\slv{д}} & добро & д\\\hln
            {\slv{Е}} & {\slv{є, е}} & есть & е\\\hln
            {\slv{Ж}} & {\slv{ж}} & живете & ж\\\hln
            {\slv{Ѕ}} & {\slv{ѕ}} & зело & з\\\hln
            {\slv{З}} & {\slv{з}} & земля & з\\\hln
            {\slv{И}} & {\slv{и}} & иже & и\\\hln
            {\slv{І}} & {\slv{ї}} & и & и\\\hln
            {\slv{К}} & {\slv{к}} & како & к\\\hln
            {\slv{Л}} & {\slv{л}} & люди & л\\\hln
            {\slv{М}} & {\slv{м}} & мыслете & м\\\hln
            {\slv{Н}} & {\slv{н}} & наш & н\\\hln
            {\slv{Ѻ}} & {\slv{ѻ, о}} & он & о\\\hln
            {\slv{П}} & {\slv{п}} & покой & п\\\hln
            {\slv{Р}} & {\slv{р}} & рцы & р\\\hln
            {\slv{С}} & {\slv{с}} & слово & с\\\hln
            {\slv{Т}} & {\slv{т}} & твердо & т\\\hln
            {\slv{Оу}} & {\slv{ᲂу, ꙋ, у}} & ук & у\\\hln
            {\slv{Ф}} & {\slv{ф}} & ферт & ф\\\hln
            {\slv{Х}} & {\slv{х}} & хер & х\\\hln
            {\slv{Ц}} & {\slv{ц}} & цы & ц\\\hln
            {\slv{Ч}} & {\slv{ч}} & червь & ч\\\hln
            {\slv{Ш}} & {\slv{ш}} & ша & ш\\\hln
            {\slv{Щ}} & {\slv{щ}} & ща & щ\\\hln
            {\slv{Ъ}} & {\slv{ъ}} & ер & --\\\hln
            {\slv{Ы}} & {\slv{ы}} & ы & ы\\\hln
            {\slv{Ь}} & {\slv{ь}} & ерь & --\\\hln
            {\slv{Ѣ}} & {\slv{ѣ}} & ять & е\\\hln
            {\slv{Ю}} & {\slv{ю}} & ю & ю\\\hln
            {\slv{Ꙗ, Ѧ}} & {\slv{ꙗ, ѧ}} & я & я\\\hln
            {\slv{Ѡ}} & {\slv{ѡ}} & омега & о\\\hln
            {\slv{Ѿ}} & {\slv{ѿ}} & от & от\\\hln
            {\slv{Ѯ}} & {\slv{ѯ}} & кси & кс\\\hln
            {\slv{Ѱ}} & {\slv{ѱ}} & пси & пс\\\hln
            {\slv{Ѳ}} & {\slv{ѳ}} & фита & ф\\\hln
            {\slv{Ѵ}} & {\slv{ѵ}} & ижица & и, в\\\hln
%            {\slv{}} & {\slv{}} & & \\\hln
        \end{longtable}
    \end{center}

    \begin{flushright}
        \emph{Упражнение 1}
    \end{flushright}
    \medskip
    
    \textbf{Славянские или церковные буквы, сходные с русскими}
    
    {\slv{а б в г д е ж з и к л м н о п р с т ф х ц ч ш щ ъ ы ь ю}}
    \medskip
    
    \begin{center}
        \begin{slv}
            Бо́гъ.~\textemdash~Спа́съ.~\textemdash~Сы́нъ.~\textemdash~Ма́-ти.~\textemdash~Не́-бо.~\textemdash~Ча̑-да.~\textemdash~Лю́-ди.~\textemdash~По-сты̀.~\textemdash~Ѻ҆-те́цъ.~\textemdash~Це́р-ковь.~\textemdash~Гос-по́дь.~\textemdash~Мо-ли̑т-вы.~\textemdash~Пра́зд-ни-цы.~\textemdash~Спа-си́-тель.~\textemdash~Бла-го-сло-ве́нъ Бо́гъ на́шъ.~\textemdash~Бо́гъ є҆-динъ є҆́сть.~\textemdash~Го́с-по-ди, бла-го-сло-вѝ.~\textemdash~Въ це́рк-вахъ бла-го-сло-ви́-те Бо́-га.~\textemdash~Гла́съ бы́сть съ не-бе-сѐ: Ты̀ є҆-сѝ Сы́нъ Мо́й воз-лю́б-лен-ный.~\textemdash~Бо́гъ на́шъ на не-бе-сѝ и на зе-млѝ.~\textemdash~Воз-лю́-би-ши Го́с-по-да Бо́-га тво-е-го̀.~\textemdash~Ще́дръ и ми́-лос-тивъ Гос-по́дь.~\textemdash~Бра́т-ство воз-лю-би́-те.~\textemdash~Чтѝ ѻ҆т-ца̀ тво-е-го̀ и ма́-терь тво-ю̀.~\textemdash~Сы́нъ не по-кор-ли́-вый в по-ги́-бель.
        \end{slv}
    \end{center}


                \subsubsection{Основные правила церковнославянского чтения и письма}
                
    При церковнославянском чтении нужно соблюдать следующие основные правила:
    
    1. Читать нужно внятно и произносить слова так, как они напечатаны в книге. Например, слово {\slv{є҆го̀}} нужно читать <<его>> (а не <<ево>>), {\slv{моли́тва}} (а не <<малитва>>), {\slv{єди́наго}} (а не <<единова>>).
    
    2. Звука <<ё>> в церковнославянском языке совсем нет, а потому букву {\slv{є}} или {\slv{е}} нужно всегда произносить как <<е>>, а не как <<ё>>. Например, {\slv{твоѐ}} (а не <<твоё>>), {\slv{моѐ}} (а не <<моё>>), {\slv{тве́рдый}} (а не <<твёрдый>>).
    
    3. В церковнославянских словах на одном из слогов ставится \textbf{ударение}, чтобы показать, что этот слог нужно усилить голосом или, как говорят, сделать на нем ударение. Ударение обозначается знаком~~{\slv{́}}, или~~{\slv{̀}}, или~~{\slv{̑}}, например: {\slv{влады́ко, хвала̀, сїѧ̑}}. Если же слово начинается с гласного звука, то над буквой этого звука ставится \textbf{придыхание}, которое изображается знаком~~{\slv{҆}}, например: {\slv{о҆де́жда. є҆сѝ}}.
    
    4. В церковнославянском языке начальное слово предложения (после точки) пишется с прописной (большой) буквы, например: {\slv{Въ нача́лѣ сотворѝ бг҃ъ не́бо и҆ зе́млю. Землѧ̀ же бѣ неви́дима и҆ неꙋстро́ена}} (Быт. 1, 1--2). Прописная буква пишется иногда в начале каждого стиха, даже если стих начинается после запятой или другого какого-нибудь знака помимо точки.
    
    5. Священные имена лиц, собственные имена, т.е. имена, названия городов, стран, гор, рек, морей и пр., если они стоят в середине или в конце предложение (т.е. не после точки), пишутся с малой буквы, например: {\slv{Рече́ же гдⷭ҇ь къ мѡѷсе́ю и҆ а҆арѡ́нꙋ въ землѝ є҆гѵ́петстѣй}} (Исх. 12, 1).


                \subsubsection{Особенности правописания и произношения некоторых букв в церковнославянском языке}

    Буква {\slv{Г}} (<<глаголь>>)

%                \subsubsection{Правописание и произношение букв, перешедших в церковнославянский язык из греческого языка}
%                \subsubsection{Правописание букв\chs{ꙗ}и\chs{ѧ}}
%                \subsubsection{Надстрочные знаки}
%                \subsubsection{Буквы как числовые знаки (цифры)}
%        \section{Общие понятия о грамматических формах}
%                \subsubsection{Разделение звуков}
%                \subsubsection{Чередование звуков}
%                \subsubsection{Состав слова}
%                \subsubsection{Слова простые и сложные}
%                \subsubsection{Славянское неполногласие}
%                \subsubsection{Части речи и их грамматические формы}
%                \subsubsection{Понятие о предложении}
%                \subsubsection{Пунктуация}
%        \section{Глагол}
%                \subsubsection{Понятие о глаголе}
%                \subsubsection{Глаголы архаические}
%                \subsubsection{Неопределенная форма глаголов}
%                \subsubsection{Спряжение в настоящем времени вспомогательного глагола\chs{бы҆́ти}}
%                \subsubsection{Спряжение в настоящем времени прочих глаголов}
%                \subsubsection{Глагольные основы}
%                \subsubsection{Глаголы тематические и разделение их на два спряжения}
%                \subsubsection{Преходящее время вспомогательного глагола\chs{бы҆́ти}}
%                \subsubsection{Аорист вспомогательного глагола\chs{бы҆́ти}}
%                \subsubsection{Преходящее время тематических глаголов}
%                \subsubsection{Аорист тематических глаголов}
%                \subsubsection{Глаголы вида совершенного и несовершенного}
%                \subsubsection{Будущее время глаголов}
%                \subsubsection{Понятие о наклонениях глагола}
%                \subsubsection{Желательное наклонение глаголов}
%                \subsubsection{Повелительное наклонение глаголов}
%                \subsubsection{Спряжение архаических глаголов}
%                \subsubsection{Будущее время описательное}
%        \section{Склоняемые части речи}
%            \subsection{Имя существительное}
%                \subsubsection{Родовые окончания имен существительных}
%                \subsubsection{Первое склонение имен существительных}
%                \subsubsection{Общие замечания к первому склонению имен существительных}
%                \subsubsection{Имена существительные первого склонения с основой на шипящие и на\chs{ц}}
%                \subsubsection{Второе склонение имен существительных}
%                \subsubsection{Общие замечания ко второму склонению имен существительных}
%                \subsubsection{Имена существительные второго склонения с основой на шипящие и на\chs{ц}}
%                \subsubsection{Третье склонение имен существительных}
%            \subsection{Имя прилагательное}
%                \subsubsection{Разделение имен прилагательных}
%                \subsubsection{Краткие имена прилагательные}
%                \subsubsection{Склонение кратких имен прилагательных с твердым окончанием}
%                \subsubsection{Склонение кратких имен прилагательных с мягким окончанием}
%                \subsubsection{Склонение кратких имен прилагательных с основой на шипящие}
%            \subsection{Местоимение}
%                \subsubsection{Понятие о местоимении}
%                \subsubsection{Склонение личных местоимений\chs{а҆́зъ}и\chs{ты̀}и возвратного\chs{себѐ}}
%                \subsubsection{Особенности глагольного сказуемого в предложении}
%                \subsubsection{Согласование слов в предложении}
%                \subsubsection{Управление слов в предложении}
%                \subsubsection{Обращение}
%    \chapter*{2-й класс}
%    \label{ch:secondgrade}
%    \addcontentsline{toc}{chapter}{\nameref{ch:secondgrade}}
%    \setcounter{section}{0}
%    \setcounter{subsubsection}{0}
%        \section{Склоняемые части речи}
%            \subsection{Имя существительное}
%                \subsubsection{Имена существительные неравносложные}
%                \subsubsection{Имена существительные разносклоняемые}
%            \subsection{Местоимение}
%                \subsubsection{Склонение личного местоимения 3-го лица \chs{ѻ҆́нъ}}
%                \subsubsection{Склонение притяжательных местоимений}
%                \subsubsection{Склонение определительного местоимения\chs{ве́сь}}
%                \subsubsection{Склонение указательного местоимения\chs{се́й}}
%                \subsubsection{Склонение вспомогательных местоимений\chs{кто̀}и\chs{что̀}}
%                \subsubsection{Склонение указательного местоимения\chs{то́й}(тот)}
%                \subsubsection{Склонение указательного местоимения\chs{ѻ҆́нъ, ѻ҆́ный}}
%            \subsection{Имя прилагательное}
%                \subsubsection{Склонение полных имен прилагательных}
%                \subsubsection{Общие замечания к склонению полных имен прилагательных}
%                \subsubsection{Понятие о степенях сравнения имен прилагательных}
%                \subsubsection{Сравнительная степень имен прилагательных}
%                \subsubsection{Превосходная степень имен прилагательных}
%                \subsubsection{Неправильные степени сравнения}
%            \subsection{Имя числительное}
%                \subsubsection{Понятие об имени числительном}
%                \subsubsection{Имена числительные количественные}
%                \subsubsection{Имена числительные порядковые}
%                \subsubsection{Склонение количественных числительных}
%        \section{Глагол}
%                \subsubsection{Понятие о причастии}
%                \subsubsection{Несклоняемое (спрягаемое) причастие}
%                \subsubsection{Прошедшее совершенное время глаголов}
%                \subsubsection{Давнопрошедшее время глаголов}
%                \subsubsection{Условное наклонение глаголов}
%                \subsubsection{Безличные глаголы}
%                \subsubsection{Возвратная форма глагола}
%                \subsubsection{Страдательная форма глагола}
%                \subsubsection{Склоняемые причастия}
%                \subsubsection{Краткие действительные причастия настоящего времени}
%                \subsubsection{Полные действительные причастия настоящего времени}
%                \subsubsection{Краткие действительные причастия прошедшего времени}
%                \subsubsection{Полные действительные причастия прошедшего времени}
%                \subsubsection{Краткие страдательные причастия настоящего времени}
%                \subsubsection{Полные страдательные причастия настоящего времени}
%                \subsubsection{Краткие страдательные причастия прошедшего времени}
%                \subsubsection{Полные страдательные причастия прошедшего времени}
%                \subsubsection{Сложная страдательная форма глаголов}
%        \section{Неизменяемые части речи}
%                \subsubsection{Наречие}
%                \subsubsection{Предлог}
%                \subsubsection{Союз}
%                \subsubsection{Частицы}
%                \subsubsection{Междометия}
%        \section{Некоторые особенности церковнославянского синтаксиса}
%                \subsubsection{Двойной винительный падеж}
%                \subsubsection{Падеж родительный-разделительный}
%                \subsubsection{Замена притяжательных местоимений личными и возвратными в дательном падеже}
%                \subsubsection{Употребление местоимения и имени прилагательного в смысле имени существительного}
%                \subsubsection{Отсутствие отрицания при глаголе при наличии местоимений\chs{никто̀, ничто̀}}
%                \subsubsection{Славянский член и его употребление}
%                \subsubsection{Особенности славянского придаточного предложения}
%                \subsubsection{Особенности славянских придаточных предложений с союзными словами \chs{є҆́же, во є҆́же, ѡ є҆́же}}
%                \subsubsection{Особенности славянского придаточного предложения}
%                \subsubsection{Оборот <<Дательный самостоятельный>>}
%                \subsubsection{Оборот <<Дательный с неопределенным>>}
%                \subsubsection{Оборот <<Винительный с неопределенным>>}

\end{document}