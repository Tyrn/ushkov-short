\documentclass[11pt,a4paper,oneside]{memoir}

\pagestyle{plain}

%\usepackage{adjustbox}
%\usepackage{showframe}
%\usepackage{changepage}
\usepackage{datetime2}
\usepackage{longtable}
\usepackage{makecell}
\usepackage{multirow}
\usepackage[usenames,dvipsnames,svgnames,table]{xcolor}
\usepackage{fontspec}
\usepackage{xltxtra}
\usepackage{pifont}
% code borrowed from Polyglossia documentation — Thanks!
\definecolor{myblue}{rgb}{0.02,0.04,0.48}
\definecolor{lightblue}{rgb}{0.61,.8,.8}
\definecolor{myred}{rgb}{0.65,0.04,0.07}

\usepackage{polyglossia}
\setmainlanguage{russian}
\setotherlanguages{churchslavonic,english}
\setotherlanguage[variant=polytonic]{greek}
\usepackage{churchslavonic}
\usepackage{lettrine}

\setmainfont[Mapping=tex-text]{Liberation Serif}
\setsansfont[Mapping=tex-text]{Liberation Sans}
\setmonofont[Mapping=tex-text]{Liberation Mono}

\newfontfamily\churchslavonicfont[Script=Cyrillic,Ligatures=TeX,Scale=1.33333333,HyphenChar="005F]{PonomarUnicode.otf} 
\newfontfamily{\slv}[Scale=MatchLowercase]{Ponomar Unicode TT}
\newfontfamily{\ust}[Scale=MatchLowercase]{Menaion Unicode TT}
\newfontfamily{\ind}{Indiction Unicode TT}

\usepackage{indentfirst}
\frenchspacing
\clubpenalty=10000
\widowpenalty=10000

\sloppy

%%\/ Borrowed from titlepages by Peter Wilson \/%%

%\usepackage{pst-text}

%%%% Additional font macros
\makeatletter
%%%% light series
%% e.g., s:12
\DeclareRobustCommand\ltseries
{\not@math@alphabet\ltseries\relax
    \fontseries\ltdefault\selectfont}
%% e.g., t:32
\newcommand{\ltdefault}{l}
%% e.g., v:19
\DeclareTextFontCommand{\textlt}{\ltseries}

% heavy(bold) series
\DeclareRobustCommand\hbseries
{\not@math@alphabet\hbseries\relax
    \fontseries\hbdefault\selectfont}
%% e.g., t:32
\newcommand{\hbdefault}{hb}
%% e.g., v:19
\DeclareTextFontCommand{\texthb}{\hbseries}
\makeatother

\newcommand*{\isbn}{{\small\textsc{ISBN}}}

%%% for the Web-O-Mints fonts
\newcommand*{\wb}[2]{\fontsize{#1}{#2}\usefont{U}{webo}{xl}{n}}
%\renewcommand*{\wb}[2]{}%    probably kills Web-O-Mints (and some layouts?)
%%% for the Fontsite 500 fonts
\newcommand*{\FSfont}[1]{%
    \fontencoding{T1}\fontfamily{#1}\selectfont}
%\renewcommand*{\FSfont}[1]{}%    kills special font selections

\newcommand*{\labelit}[1]{\phantomsection\label{#1}}
\newcommand*{\refit}[1]{(graphic on page~\pageref{#1})}

\chapterstyle{dash}\renewcommand*{\chaptitlefont}{\normalfont\itshape\LARGE}
\setlength{\beforechapskip}{2\onelineskip}
\setsecheadstyle{\normalfont\Large\raggedright}
\makeindex
\renewcommand*{\indexname}{Index of Designers}
\makeatletter
\newcommand*{\boxminipage}{%
    \@ifnextchar [%]
    \@ibxminipage
    {\@iiibxminipage c\relax[s]}}
\def\@ibxminipage[#1]{%
    \@ifnextchar [%]
    {\@iibxminipage{#1}}%
    {\@iiibxminipage{#1}\relax[s]}}
\def\@iibxminipage#1[#2]{%
    \@ifnextchar [%]
    {\@iiibxminipage{#1}{#2}}%
    {\@iiibxminipage{#1}{#2}[#1]}}
\let\@bxminto\@empty
\def\@iiibxminipage#1#2[#3]#4{%
    \ifx\relax#2\else
    \setlength\@tempdimb{#2}%
    \def\@bxminto{to\@tempdimb}%
    \fi
    \leavevmode
    \@pboxswfalse
    \if #1b\vbox
    \else
    \if #1t\vtop
    \else
    \ifmmode \vcenter
    \else \@pboxswtrue $\vcenter
    \fi
    \fi
    \fi
    %  \@bxminto
    \bgroup%          outermost vbox
    \hsize #4
    \hrule\@height\fboxrule
    \hbox\bgroup%   inner hbox
    \vrule\@width\fboxrule \hskip\fboxsep 
    \vbox \@bxminto
    \bgroup% innermost vbox
    \vskip\fboxsep
    \advance\hsize -2\fboxrule \advance\hsize -2\fboxsep
    \textwidth\hsize \columnwidth\hsize
    \@parboxrestore
    \def\@mpfn{mpfootnote}\def\thempfn{\thempfootnote}\c@mpfootnote\z@
    \let\@footnotetext\@mpfootnotetext
    \let\@listdepth\@mplistdepth \@mplistdepth\z@
    \@minipagerestore\@minipagetrue
    \everypar{\global\@minipagefalse\everypar{}}}

\def\endboxminipage{%
    \par\vskip-\lastskip
    \ifvoid\@mpfootins\else
    \vskip\skip\@mpfootins\footnoterule\unvbox\@mpfootins\fi
    \vskip\fboxsep
    \egroup%    end innermost vbox
    \hskip\fboxsep \vrule\@width\fboxrule
    \egroup%    end hbox
    \hrule\@height\fboxrule
    \egroup%    end outermost vbox
    \if@pboxsw $\fi}
\makeatother

\DeclareRobustCommand{\cs}[1]{\texttt{\char`\\#1}}
\newlength{\tpheight}\setlength{\tpheight}{0.9\textheight}
\newlength{\txtheight}\setlength{\txtheight}{0.9\tpheight}
\newlength{\tpwidth}\setlength{\tpwidth}{0.9\textwidth}
\newlength{\txtwidth}\setlength{\txtwidth}{0.9\tpwidth}
\newlength{\drop}

\newenvironment{showtitle}{%
    \begin{boxminipage}[c][\tpheight]{\tpwidth}
        \centering\begin{vplace}\begin{minipage}[c][\txtheight]{\txtwidth}}%
            {\end{minipage}\end{vplace}\end{boxminipage}}

\newcommand*{\titleCC}{\begingroup% City of Cambridge
    \drop=0.1\txtheight
    \vspace*{\drop}
    \centering 
    {\Large\itshape КРАТКИЙ УЧЕБНИК}\\[0.5\drop]
    {\textcolor{Red}{\HUGE\bfseries {\slv{✠}}}}\par
    \vspace{\drop}
    {\LARGE\itshape ЦЕРКОВНОСЛАВЯНСКОГО ЯЗЫКА}\par
    \vfill
    {\Large УШКОВ А.В.}\par
    \vfill
%    {\plogo}\\[0.5\baselineskip]
    {\itshape Веркола}\par
    {\scshape 2017}\par
    %\vfill
    \vspace*{\drop}
    \endgroup}

%%/\ Borrowed from titlepages by Peter Wilson /\%%

%%\/ placedtabular courtesy of Skillmon \/%%

\usepackage{pgfkeys}
\makeatletter
\newif\if@placedleft
\newif\if@placedright
\newif\if@placedmiddle
\newif\if@placedcapused
\newif\if@placedscapused
\newif\if@placedfontused
\newif\if@placedcfontused
\newdimen\@placedpadding
\pgfkeys{/placedt/.is family,/placedt,
    default/.style={%
        place=left,
        stretch=1.4,
        padding=1cm,
    },
    stretch/.estore in=\arraystretch,
    place/.is choice,
        place/left/.code=\@placedlefttrue\@placedrightfalse\@placedmiddlefalse,
        place/right/.code=\@placedrighttrue\@placedleftfalse\@placedmiddlefalse,
        place/middle/.code=\@placedmiddletrue\@placedleftfalse\@placedrightfalse,
    padding/.estore in=\@placedpadding,
    caption/.code=\@placedcapusedtrue\def\@placedcap{#1},
    scaption/.code=\@placedscapusedtrue\def\@placedscap{#1},
    font/.store in=\@placedfont,
    captionfont/.store in=\@placedcapfont,
    font/.code=\@placedfontusedtrue\def\@placedfont{#1},
    captionfont/.code=\@placedcfontusedtrue\def\@placedcapfont{#1},
}

\newenvironment{placedtabular}[2][]{%
    \par\medskip\noindent\begin{minipage}{\textwidth}%
        \pgfkeys{/placedt, default, #1}%
        \if@placedleft%
            \newdimen\@placedrest%
            \@placedrest=\textwidth%
            \advance\@placedrest by -\@placedpadding%
            \hskip\@placedpadding%
            \begin{minipage}{\@placedrest}%
                \raggedright%
        \else\if@placedmiddle%
            \centering%
        \else\if@placedright%
            \newdimen\@placedrest%
            \@placedrest=\textwidth%
            \advance\@placedrest by -\@placedpadding%
            \begin{minipage}{\@placedrest}%
                \raggedleft%
        \fi\fi\fi%
        \if@placedcapused%
            {\if@placedcfontused\normalsize\normalfont\@placedcapfont\fi\@placedcap}%
            \if@placedscapused%
                \addcontentsline{lot}{table}{\@placedscap}%
            \else%
                \addcontentsline{lot}{table}{\@placedcap}%
            \fi%
            \\[0.2\baselineskip]%
        \fi%
        \if@placedfontused\normalsize\normalfont\@placedfont\fi%
        \begin{tabular}{#2}%
    }{
        \end{tabular}%
        \if@placedmiddle\else%
            \end{minipage}%
        \fi%
    \end{minipage}%
    }
\makeatother

%%/\ placedtabular courtesy of Skillmon /\%%

%%\/ A tiny toolbox \/%%

\newfontfamily{\mock}[Scale=0.97]{DejaVu Serif Condensed Italic}
%% Style for mock items
\newcommand{\mockitem}[1]{{\mock{#1}}}

%% Suppress \paragraph{title} on demand
\newcommand{\exercise}{}

%% Quote style for Slavonic exercise text
\newcommand{\exquote}{quote}

%% Exercise answer example logo
\newcommand{\exanswer}{\ding{242}}

%% Suppress \hline on demand
\newcommand{\hln}{}

%% Parenthesize it
\newcommand{\pstyle}{\textenglish}
\newcommand{\pxp}[1]{\pstyle{(}#1\pstyle{)}}

%% Dash it
\newcommand{\sdash}{\textenglish{\textemdash}}

%% Table caption size
\newcommand{\tabcaptsize}{\footnotesize}

%% Shrunken table size
\newcommand{\shrunkensize}{\footnotesize}

%% Some fixed horizontal spaces
\newcommand{\hstba}{1cm}
\newcommand{\hstbb}{0.75cm}
\newcommand{\hspca}{\hspace{2.4em}}
\newcommand{\hspcb}{\hspace{3.5em}}

%% \makecell resized
\newcommand{\mkcella}{\scriptsize\makecell}
\newcommand{\mkcellb}{\footnotesize\makecell}

%% Slavonic resized
\newcommand{\slva}[1]{\scriptsize\slv{#1}}

%% Rotate cell
\newcommand{\spheading}[2][10em]{% \spheading[<width>]{<stuff>}
    \rotatebox{90}{\parbox{#1}{\raggedright #2}}}

%%/\ A tiny toolbox /\%%

%%\/ Build document structure, ToC, links et al. \/%%

%% Ensure sequential numbering of subsubsections and paragraphs.
\setsecnumdepth{paragraph}
\counterwithout{paragraph}{subsubsection}
\counterwithout{subsubsection}{subsection}
\setcounter{tocdepth}{3} % Must not precede the above

%%%\/ Set section/paragraph title appearance \/%%%

\setparaindent{0pt}
\setparaheadstyle{\bfseries\itshape\raggedleft}
\setafterparaskip{1ex}

\renewcommand{\thechapter}{\arabic{chapter}}
\renewcommand{\thesection}{\arabic{section}}
\renewcommand{\thesubsection}{\Roman{subsection}}
\renewcommand{\thesubsubsection}{\S\arabic{subsubsection}}
\renewcommand{\theparagraph}{{Упражнение }\arabic{paragraph}}

%\usepackage[explicit]{titlesec}

%\titleformat{<command>}[<shape>]{<format>}{<label>}{<sep>}{<before>}[<after>]
%\titleformat{\paragraph}[hang]{\raggedleft\normalfont\small\itshape\bfseries}{}{0pt}{#1 \arabic{paragraph}}

%%%/\ Set section/paragraph title appearance /\%%%

\iffalse
    Courtesy of Heiko Oberdiek:
    
    Unique anchor names

    The trouble with the links is caused by anchor names that are not unique. This is caused by the counter reset. Both counters subsection and subsubsection are reusing its numbers. But hyperref needs unique numbers to generate the anchor names to generate the link destinations.

    Uniqueness can be provided by a proper definition of the companion macros \theH<counter> that are used by hyperref if they are available. The following example adds a macro \ChapterAnchorPrefix to be inserted into the definitions of \theH<counter> for section and subsection. If a counter is reset, then \ChapterAnchorPrefix must be assigned a new value.
\fi

\usepackage[hidelinks]{hyperref}
\usepackage{bookmark}

\newcommand*{\ChapterAnchorPrefix}{chapter}
\def\theHsection{\ChapterAnchorPrefix.\the\value{section}}
\def\theHsubsubsection{\ChapterAnchorPrefix.\the\value{subsubsection}}

%%/\ Build document structure, ToC, links et al. /\%%

\begin{document}
    \begin{titlingpage}
        %% Borrowed from titlepages by Peter Wilson
        \begin{showtitle}
            \titleCC
        \end{showtitle}
        \labelit{CC}
        {
            \par\vspace{0.2\baselineskip}
            %        \footnotesize \verb?\titleCC?
        }
        %% From titlepages by Peter Wilson, end
    \end{titlingpage}
    
    \tableofcontents*
    
    \chapter*{}
        \section*{Краткий учебник церковнославянского языка}
        \label{sec:brief}
        \addcontentsline{toc}{section}{\nameref*{sec:brief}}
    
    Составитель сего учебника доброй памяти \emph{профессор Анатолий Васильевич Ушков}.
    
    Анатолий Васильевич Ушков родился 7 августа 1894 года в г. Самаре в семье сотрудника Самарской Духовной Консистории. По окончании четырех классов Самарской Духовной семинарии, он в 1912 году успешно сдал вступительные экзамены в Казанский университет и был зачислен на физико-математический факультет. После университета, в 1916 году, А. Ушков был мобилизован на военную службу; проходил обучение в Киевском артиллерийском училище. С 1918 по 1945 г. он преподавал математику и физику в средних и специальных учебных заведениях в городах Самаре, Красноярске и Москве. Заведовал учебной частью и был руководителем методических совещаний по вопросам преподавания физики и математики. С 1939 по 1943 г. заочно учился в Московском государственном педагогическом институте на факультете русского языка и литературы, закончив который, некоторое время, наряду с преподаванием математики, вел курс литературы в средних учебных заведениях г. Москвы.
    
    В 1945 г. он осуществил свою давнишнюю мечту~---~послужить Святой Церкви и поступил в Московскую Духовную Академию. По окончании академии, в 1949 г., со званием кандидата богословия, полученным за сочинение <<Душа и ее бессмертие по христианскому учению>>, он преподавал катехизис, а затем церковнославянский язык в духовной семинарии. В 1964 г., по прочтении двух пробных лекций <<Древний мир перед пришествием Христа Спасителя>> и <<Основные положения формальной логики>>, А.В. Ушкову было присвоено звание доцента и поручено чтение лекций по логике в академии. Им составлен ряд учебных пособий для духовных школ: <<Краткая пасхалия>>, <<Логика в курсе академического образования>>, <<Краткая пасхалия в общедоступном изложении>>, <<Астрономический справочник>>, <<Краткое описание солнечной системы>>, <<Календарный счет времени>>, <<Счет и мера с древнейших времен до наших дней>>. Анатолием Васильевичем был подготовлен учебный курс церковнославянского языка, за который Совет академии присвоил ему 15 декабря 1969 г. степень магистра богословия и утвердил в должности профессора.
    
    Анатолий Васильевич отличался большой скромностью, чутким, внимательным отношением к людям; поражала его необычайная работоспособность, величайшее трудолюбие. Жажда проповеди слова Божия была его духовной потребностью. Он произнес 65 проповедей в праздничные и воскресные дни в академическом храме, в них~---~яркое выражение его глубоких христианских чувств и безграничной, сердечной любви к Богу и людям. В начале 1971--1972 учебного года Анатолий Васильевич по болезни вынужден был оставить работу в академии и перейти на пенсию.
    
    Скончался А.В. Ушков 14 января 1972 г., на 78-м году жизни, приобщившись Святых Христовых Тайн. Похоронен на кладбище в Сергиевом Посаде.
    
    \medskip
    Подготовил к печати доцент-иерей В. Москвич.
    
    \chapter*{1-й класс}
    \label{ch:firstgrade}
    \addcontentsline{toc}{chapter}{\nameref*{ch:firstgrade}}
        \section*{Введение}
        \label{sec:intro}
        \addcontentsline{toc}{section}{\nameref*{sec:intro}}
                \subsubsection[Значение церковнославянского языка]{Значение церковнославянского языка для православного богослужения и православного пастыря}
                
    Церковнославянский язык имеет очень важное значение для православного христианина. На этом языке совершается богослужение в нашей Православной Русской Церкви и написаны наши священные и богослужебные книги. По своему возвышенному характеру, по своей силе и звучности церковнославянский язык является наиболее совершенным средством для выражения религиозных настроений православного русского человека. Высшие стремления духа и глубокие чувствования, отрешенные от земного и направленные к небесному, чистому и вечному, получают наиболее соответствующее выражение в этом языке, далеком от обычного, житейского.
    
    Но для этого необходимо, чтобы все читаемое в храме Божием достигало своей цели, т.е. доходило бы до сознания души христианина, научая ее истинному благочестию и указывая ей путь спасения. А поэтому нужно, чтобы те, которые употребляют язык Церкви~---~чтецы, певцы, церковно- и священнослужители~---~в совершенстве понимали то, что произносят их уста. Да кроме того, без полного изучения языка матери-Церкви нельзя дать ответа вопрошающим о недоуменных словах и выражениях церковного языка. А таких слов и выражений в церковнославянском языке достаточно. Взять хотя бы такие слова: \emph{абие, амо, аще, выну, вкупе, ей, ктому, нань, леть, паче, паки, рамо, сице, тацы, убо, уне, яко}; или такие выражения: \emph{Змий сей, его же создал еси ругахуся ему; Чермнует бо ся небо; И бяху выну в церкви; Обыде нас последняя бездна; Любовию же, Дево, песни ткати спротяженно сложенныя неудобно есть; Противодышущу росоносному Духу, со огнем сущу пояху} и многие другие.
    
    Итак, все что читается и поется в храмах Божиих, должно быть истолковано людьми, стоящими на высоте своего призвания, а такими людьми, прежде всего, должны быть православные пастыри. Только тогда у православного русского человека укрепится сознательное и вместе с тем благоговейное отношение к богослужению нашей Православной Церкви и самое его мировоззрение обогатится достаточным запасом религиозно-церковных понятий.
    
                \subsubsection{Возникновение письменности у славян}
                
    Славянский язык назван именно \emph{славянским} потому, что на нем говорили наши предки славяне.
    
    Славяне, обитавшие первоначально около Карпат, долгое время были непросвещенным языческим народом. Очень часто им случалось соприкасаться с культурными народами Византийской и Римской империй. Мало-помалу славяне стали подчиняться влиянию этих уже просвещенных христианских народов. От греков и римлян к славянам проникло и быстро у них распространилось христианство. Но, слушая богослужение на чуждом им греческом или латинском языке, славяне не могли себе вполне усвоить истины нового для них учения, и потому многие из них были христианами лишь по имени, а по существу сохраняли прежнюю грубость нравов и держались старых языческих суеверий. Так было до половины IX века.
    
    В Половине IX века святые братья Кирилл (в миру Константин) и Мефодий, родом греки, задумали облегчить славянам понимание христианского богослужения~---~и перевели для них богослужебные книги с греческого языка на славянский. Так как у славян в то время не было еще письменности, то святые братья сами составили славянскую азбуку, взяв за образец греческую азбуку и дополнив ее недостающими буквами. Таким образом, богослужебные книги были переведены ими на славянский язык, а сами они стали апостолами-просветителями славян.
    
                \subsubsection{Деятельность святых братьев Кирилла и Мефодия}

    Святые братья Кирилл (827--\,869 гг.) и Мефодий (\dag 885 г.) были детьми знатного вельможи и родились в Солуни (Фессалониках)~---~главном городе Македонии, страны, населенной по преимуществу славянами. Старший брат, Мефодий, после домашнего воспитания занимал сначала военные, а потом административные должности, но впоследствии принял монашество, поселившись на горе Олимп. Младший, Кирилл, отличался блестящими способностями к учению, обучался словесным, философским и математическим наукам. Его ожидали почести в свете, но он не прельстился этим, а принял сан священника и место библиотекаря в библиотеке при храме св. Софии. Впоследствии св. Кирилл удалился к брату на Олимп. Предполагают, что в это время жизни на Олимпе святые братья и начали переводить богослужебные книги на славянский язык.

    Около 862 года, через посредничество Византийского императора Михаила III, святые братья были приглашены Моравским князем Ростиславом в Моравию для проповеди христианства. Здесь начинается главная деятельность свв. Кирилла и Мефодия. В княжествах Моравском и Панионском (где ныне Венгрия и Чехия) христианство было уже проповедано среди тамошних славян немецким, латинским духовенством. Но латинское духовенство совершало богослужение на чуждом славянам латинском языке. А так как латинский язык был непонятен славянам, то, конечно, проповедь немецкого духовенства оказалась безуспешной.

    Святые братья, начав учить славян Христовой вере, дали им священные книги на славянском языке. На славянский язык были переведены и богослужебные книги. На славянском языке стали они совершать богослужение, а для обучения славян письменности стали заводить и школы. Христианство между славянами начало быстро распространяться.

    Немецкое духовенство, видя успехи святых братьев в деле распространения ими христианства среди славян, из зависти подало на них жалобу в Рим. Римский папа Николай I вызвал к себе святых братьев на суд (в то время не было еще разделения Церкви, все были православные, и только что начинались несогласия между Римом и Византией). Святые братья отправились в Рим, но уже не застали папу Николая I в живых, а преемник его Андриан очень благосклонно отнесся к свв. Кириллу и Мефодию. Кирилл представил папе Евангелие на славянском языке, и папа положил славянский перевод в соборе св. Петра, где святые братья даже отслужили торжественную литургию на славянском языке. Св. Кирилл в Риме заболел и скончался. Умирая, он убеждал св. Мефодия не покидать дела проповеди и распространения веры среди славян. Папа Андриан рукоположил Мефодия в епископа Моравии, отпустил его с честью и дал грамоту, в которой признал богослужение на славянском языке целесообразным.

    По возвращении св. Мефодия к Моравийским славянам враги не оставили его в покое. Они восстановили против него немецкого императора Людовика и добились того, что Мефодий был заточен в темницу, где и пробыл в заточении два с половиной года. Когда ему возвращена была свобода, то его вновь оклеветали перед папой. Пришлось Мефодию опять отправляться в Рим и защищать себя. Оправданный папой, св. Мефодий продолжал свое великое дело.

    Когда св. Мефодий скончался, то на учеников его было воздвигнуто гонение. Они, изгнанные из Моравии, ушли в другие славянские земли, главным образом, в Болгарию. Болгария заботливо хранила труды святых просветителей и потом, когда в конце X века, при св. князе Владимире, Русь приняла христианство, списки этих книг передала нашему отечеству. В те древние времена славянские наречия очень мало отличались друг от друга, и книги одного славянского племени были совершенно пригодны для богослужения и чтения другого.


        \section{Церковнославянское буквоначертание}
                \subsubsection{Церковнославянская азбука}

    Устная наша речь состоит из членораздельных \textbf{звуков}.

    Изображения звуков речи условными знаками в письме или в печати называются \textbf{буквами}.

    Собрание всех букв языка, расположенных в общепринятом порядке, называется \textbf{азбукой} (от названия двух первых славянских букв <<аз>> и <<буки>>). Церковнославянская азбука состоит из 39 букв.

    \begin{center}
        \renewcommand*{\arraystretch}{1.4}    
        \begin{longtable}{|c|c|c|c|}
%            \caption*{Буквы церковнославянской азбуки}\\
            \Xcline{1-4}{\arrayrulewidth}
            
            Прописная буква
            & Малая буква
            & Название
            & Произношение
            \\
            \Xcline{1-4}{\arrayrulewidth}
            \endfirsthead
            
            \multicolumn{4}{l}{\footnotesize\textit{Начало на предыдущей странице}}
            \\
            \Xcline{1-4}{\arrayrulewidth}
            
            Прописная буква
            & Малая буква
            & Название
            & Произношение
            \\
            \Xcline{1-4}{\arrayrulewidth}
            \endhead
            \Xcline{1-4}{\arrayrulewidth}
            
            \multicolumn{4}{r}{\footnotesize\textit{Продолжение на следующей странице}}
            \\
            
            \endfoot
            \Xcline{1-4}{\arrayrulewidth}
            \endlastfoot
            
            {\slv{А}}     & {\slv{а}}       & аз      & а\\\hln
            {\slv{Б}}     & {\slv{б}}       & буки    & б\\\hln
            {\slv{В}}     & {\slv{в}}       & веди    & в\\\hln
            {\slv{Г}}     & {\slv{г}}       & глаголь & г\\\hln
            {\slv{Д}}     & {\slv{д}}       & добро   & д\\\hln
            {\slv{Е}}     & {\slv{є, е}}    & есть    & е\\\hln
            {\slv{Ж}}     & {\slv{ж}}       & живете  & ж\\\hln
            {\slv{Ѕ}}     & {\slv{ѕ}}       & зело    & з\\\hln
            {\slv{З}}     & {\slv{з}}       & земля   & з\\\hln
            {\slv{И}}     & {\slv{и}}       & иже     & и\\\hln
            {\slv{І}}     & {\slv{ї}}       & и       & и\\\hln
            {\slv{К}}     & {\slv{к}}       & како    & к\\\hln
            {\slv{Л}}     & {\slv{л}}       & люди    & л\\\hln
            {\slv{М}}     & {\slv{м}}       & мыслете & м\\\hln
            {\slv{Н}}     & {\slv{н}}       & наш     & н\\\hln
            {\slv{Ѻ, Ѽ}} & {\slv{ѻ, о}}    & он      & о\\\hln
            {\slv{П}}     & {\slv{п}}       & покой   & п\\\hln
            {\slv{Р}}     & {\slv{р}}       & рцы     & р\\\hln
            {\slv{С}}     & {\slv{с}}       & слово   & с\\\hln
            {\slv{Т}}     & {\slv{т}}       & твердо  & т\\\hln
            {\slv{Оу}}    & {\slv{ᲂу, ꙋ, у}} & ук      & у\\\hln
            {\slv{Ф}}     & {\slv{ф}}       & ферт    & ф\\\hln
            {\slv{Х}}     & {\slv{х}}       & хер     & х\\\hln
            {\slv{Ц}}     & {\slv{ц}}       & цы      & ц\\\hln
            {\slv{Ч}}     & {\slv{ч}}       & червь   & ч\\\hln
            {\slv{Ш}}     & {\slv{ш}}       & ша      & ш\\\hln
            {\slv{Щ}}     & {\slv{щ}}       & ща      & щ\\\hln
            {\slv{Ъ}}     & {\slv{ъ}}       & ер      & --\\\hln
            {\slv{Ы}}     & {\slv{ы}}       & ы       & ы\\\hln
            {\slv{Ь}}     & {\slv{ь}}       & ерь     & --\\\hln
            {\slv{Ѣ}}     & {\slv{ѣ}}       & ять     & е\\\hln
            {\slv{Ю}}     & {\slv{ю}}       & ю       & ю\\\hln
            {\slv{Ꙗ, Ѧ}} & {\slv{ꙗ, ѧ}}    & я       & я\\\hln
            {\slv{Ѡ}}     & {\slv{ѡ}}       & омега   & о\\\hln
            {\slv{Ѿ}}     & {\slv{ѿ}}       & от      & от\\\hln
            {\slv{Ѯ}}     & {\slv{ѯ}}       & кси     & кс\\\hln
            {\slv{Ѱ}}     & {\slv{ѱ}}       & пси     & пс\\\hln
            {\slv{Ѳ}}     & {\slv{ѳ}}       & фита    & ф\\\hln
            {\slv{Ѵ}}     & {\slv{ѵ}}       & ижица   & и, в\\\hln

        \end{longtable}
    \end{center}

                    \paragraph{\exercise}
    
    \textbf{Славянские или церковные буквы, сходные с русскими}
    
    {\slv{а б в г д е ж з и к л м н о п р с т ф х ц ч ш щ ъ ы ь ю}}
    \medskip
    
    \begin{\exquote}\begin{slv}
        Бо́гъ.~\sdash~Спа́съ.~\sdash~Сы́нъ.~\sdash~Ма́-ти.~\sdash~Не́-бо.~\sdash~Ча̑-да.~\sdash~Лю́-ди.~\sdash~По-сты̀.~\sdash~Ѻ҆-те́цъ.~\sdash~Це́р-ковь.~\sdash~Гос-по́дь.~\sdash~Мо-ли̑т-вы.~\sdash~Пра́зд-ни-цы.~\sdash~Спа-си́-тель.~\sdash~Бла-го-сло-ве́нъ Бо́гъ на́шъ.~\sdash~Бо́гъ є҆-динъ є҆́сть.~\sdash~Го́с-по-ди, бла-го-сло-вѝ.~\sdash~Въ це́рк-вахъ бла-го-сло-ви́-те Бо́-га.~\sdash~Гла́съ бы́сть съ не-бе-сѐ: Ты̀ є҆-сѝ Сы́нъ Мо́й воз-лю́б-лен-ный.~\sdash~Бо́гъ на́шъ на не-бе-сѝ и на зе-млѝ.~\sdash~Воз-лю́-би-ши Го́с-по-да Бо́-га тво-е-го̀.~\sdash~Ще́дръ и ми́-лос-тивъ Гос-по́дь.~\sdash~Бра́т-ство воз-лю-би́-те.~\sdash~Чтѝ ѻ҆т-ца̀ тво-е-го̀ и ма́-терь тво-ю̀.~\sdash~Сы́нъ не по-кор-ли́-вый в по-ги́-бель.
    \end{slv}\end{\exquote}


                \subsubsection{Основные правила церковнославянского чтения и письма}
                
    При церковнославянском чтении нужно соблюдать следующие основные правила:
    
    1. Читать нужно внятно и произносить слова так, как они напечатаны в книге. Например, слово {\slv{є҆го̀}} нужно читать <<его>> (а не <<ево>>), {\slv{моли́тва}} (а не <<малитва>>), {\slv{єди́наго}} (а не <<единова>>).
    
    2. Звука <<ё>> в церковнославянском языке совсем нет, а потому букву {\slv{є}} или {\slv{е}} нужно всегда произносить как <<е>>, а не как <<ё>>. Например, {\slv{твоѐ}} (а не <<твоё>>), {\slv{моѐ}} (а не <<моё>>), {\slv{тве́рдый}} (а не <<твёрдый>>).
    
    3. В церковнославянских словах на одном из слогов ставится \textbf{ударение}, чтобы показать, что этот слог нужно усилить голосом или, как говорят, сделать на нем ударение. Ударение обозначается знаком~~{\slv{́}}, или~~{\slv{̀}}, или~~{\slv{̑}}, например: {\slv{влады́ко, хвала̀, сїѧ̑}}. Если же слово начинается с гласного звука, то над буквой этого звука ставится \textbf{придыхание}, которое изображается знаком~~{\slv{҆}}, например: {\slv{о҆де́жда. є҆сѝ}}.
    
    4. В церковнославянском языке начальное слово предложения (после точки) пишется с прописной (большой) буквы, например: {\slv{Въ нача́лѣ сотворѝ бг҃ъ не́бо и҆ зе́млю. Землѧ̀ же бѣ неви́дима и҆ неꙋстро́ена}} (Быт. 1, 1--2). Прописная буква пишется иногда в начале каждого стиха, даже если стих начинается после запятой или другого какого-нибудь знака помимо точки.
    
    5. Священные имена лиц, собственные имена, т.е. имена, названия городов, стран, гор, рек, морей и пр., если они стоят в середине или в конце предложение (т.е. не после точки), пишутся с малой буквы, например: {\slv{Рече́ же гдⷭ҇ь къ мѡѷсе́ю и҆ а҆арѡ́нꙋ въ землѝ є҆гѵ́петстѣй}} (Исх. 12, 1).


                \subsubsection{Особенности правописания и произношения некоторых букв в церковнославянском языке}

    Буква {\slv{Г}} (<<глаголь>>) перед {\slv{г, к, х}} произносится как звук <<и>>, например: {\slv{сѷгклі́тъ}} (<<синклит>>), {\slv{а҆сѷгкрі́тъ}} (<<асинкрит>>). Исключаются из этого правила слова: {\slv{а҆гге́й}} (<<Аггей>>) и {\slv{а҆́гге́лъ}} (<<аггел>>, в значении злого духа).
    
    Буквы {\slv{є, е}}. Буква {\slv{є}} (так называемое <<удлиненное есть>>) пишется всегда в начале слова; в середине же и на конце слова обычно пишется буква {\slv{е}} (<<простое есть>>), например: {\slv{є҆́зеро, є҆ле́нь, се́рдце}}. Но иногда буква {\slv{є}} пишется и в середине слова и даже в конце слова. Эти случаи будут рассмотрены впоследствии.
    
    Буква {\slv{ѕ}} (<<зело>>) произносится как русское <<з>> и пишется только в семи коренных словах: {\slv{ѕвѣзда̀, ѕвѣ́рь, ѕе́лїе, ѕла́къ, ѕмі́й, ѕло̀, ѕѣлѡ̀}}, а также в производных от этих слов: {\slv{ѕлы́й, ѡ҆ѕло́бити, ѕла́чный, ѕвѣ́рскїй}} и др.
    
    Буква {\slv{ї}} (<<и>>) пишется пред гласными, например: {\slv{воскресе́нїе}}. Но в словах иностранных (еврейских и греческих) эта буква пишется и перед согласными, например: {\slv{і҆́долъ, вїно̀}}. Слово <<мир>> в значении <<вселенная>> в славянском языке пишется через {\slv{ї}}, например: {\slv{ненави́дитъ ва́съ мі́ръ}} (вселенная) (Ин. 15, 19), а это же самое слово, но в значении <<покой>>, <<тишина>> пишется через {\slv{и}} (<<иже>>), например: {\slv{ми́ръ мо́й даю̀ ва́мъ}} (т.е. даю благодатное умиротворение, покой) (Ин. 14, 27).
    
    Буквы {\slv{ѻ, о, ѽ}}. Буква {\slv{ѻ}} (<<он польск\'ое>>) пишется в начале слова, например: {\slv{ѻ҆́гнь, ѻ҆́ко}}. Исключение составляет слово {\slv{і҆ѻрда́нъ}}. Буква {\slv{о}} (<<он простое>>) пишется в середине и в конце слова, например: {\slv{со́нмъ, не́бо}}. Буква {\slv{ѽ}} употребляется только в качестве междометия.
    
    Буквы {\slv{ᲂу, ꙋ, у}} (<<ук>>) изображают звук <<у>>, причем буква {\slv{ᲂу}} пишется в начале слова, а {\slv{ꙋ}}~---~в середине и на конце слова, например: {\slv{ᲂу҆че́нїе, ᲂу҆́бѡ, дꙋша̀, рꙋкꙋ̀}}. Буква {\slv{у}} (без буквы {\slv{о}}) в словах совсем не пишется, а употребляется в значении цифры, о чем будет сказано далее.
    
    Буква {\slv{ѣ}} (<<ять>>) первоначально изображала собою звук как бы средний между нашими звуками <<е>> и <<я>> (почему и названа <<ять>>). Но с утратой этого звука в произношении буква {\slv{ѣ}} в последствии в церковнославянском языке стала означать звук <<е>>.
    
    Буквы {\slv{ъ}} (<<ер>>) и {\slv{ь}} (<<ерь>>). Буква {\slv{ъ}} первоначально обозначала краткий звук <<о>>, а буква {\slv{ь}}~---~краткий звук <<е>>. Эти звуки были потому краткие, что произносились более кратко и слабо, чем другие согласные звуки. В некоторых случаях они были особенно слабы и стали совсем исчезать из произношения. В письме сохранилась постановка буквы {\slv{ъ}} в конце слов, которая теперь показывает твердое произношение конечного согласного звука, например: {\slv{хра́мъ, пра́ведникъ}}. Иногда в этом случае букву {\slv{ъ}} заменяет особый знак <<ерок>>, или <<ерик>>, который обозначается~~{\slv{̾}} и ставится над последней буквой слова. Как правило, ерок ставится в односложных предлогах, оканчивающихся на согласный звук, как-то: {\slv{ѡ҆б̾, и҆з̾, без̾, над̾, под̾}} и др. Над этими предлогами (кроме предлога {\slv{бли́з̾}}) также не ставят ударений. Когда эти предлоги соединяются в качестве приставок с другим словом, начинающимся с гласного звука, в одно слово, то ерок, смотря по произношению, удерживается (в качестве разделительного знака), например: {\slv{ѡ҆б̾ѧтїѧ, и҆з̾ѧсни́ти}}, или опускается, например: {\slv{и҆зы́де}}. Предлоги, состоящие только из согласных звуков: {\slv{въ, къ, съ}}, при раздельном их написании всегда пишутся с буквой {\slv{ъ}}, а не с ероком. Буква {\slv{ь}}, исчезнувшая как гласный звук из произношения, осталась лишь знаком мягкости предыдущего согласного звука, а потому сохранила свою постановку как в середине, так и на конце слов.

                \subsubsection{Правописание и произношение букв, перешедших в церковнославянский язык из греческого языка}

    Буквы, перешедшие с греческого языка в церковнославянский язык, следующие: {\slv{ѡ, ѿ, ѯ, ѱ, ѳ, ѵ}}.

    Буква {\slv{ѡ}} (<<омега>>) пишется в предлогах {\slv{ѡ҆}} и {\slv{ѡ҆б̾}} и в словах, начинающихся этими предлогами (в качестве приставок), например: {\slv{ѡ҆чище́нїе, ѡ҆бличи́ти}}. Если слово, начинающееся буквой {\slv{ѡ}}, соединяется с приставкой в одно слово, то в полученном составном слове буква {\slv{ѡ}} почти всегда удерживается, например: {\slv{преѡбраже́нїе}}. Буква {\slv{ѡ}} пишется также в словах, заимствованных из других языков (греческого и еврейского), например: {\slv{ѡ҆са́нна, і҆кѡ́на, канѡ́нъ, і҆ѡа́ннъ}} и др. Наконец, буква {\slv{ѡ}} пишется в некоторых наречиях, оканчивающихся звуком <<о>>, союзах, а также и в других случаях, о чем будет сказано впоследствии.

    Буква {\slv{ѿ}} (<<от>>), как уже показывает и ее название, представляет соединение двух букв {\slv{ѡ}} и {\slv{т}}, что соответствует и ее начертанию. Эта составная буква употребляется в качестве предлога {\slv{ѿ}}, а также в словах, начинающихся этим предлогом (в качестве приставки), например: {\slv{ѿра́да, ѿмще́нїе}}.

    Буквы {\slv{ѯ}} (<<кси>>), {\slv{ѱ}} (<<пси>>), {\slv{ѳ}} (<<фита>>) употребляются только в словах, заимствованных с греческого языка, например: {\slv{а҆леѯа́ндръ, ѱалти́рь, ѳома̀}}.

    Буква {\slv{ѵ}} (<<ижица>>) пишется в словах, взятых с греческого языка, причем произносится или как звук <<и>>, или как звук <<в>>. Если буква {\slv{ѵ}} стоит в начале слова или после согласной буквы, то она произносится как наш гласный звук <<и>>. В этом случае над нею ставятся две наклонные черточки, например: {\slv{сѷно́дъ}}. Эти черточки заменяются или придыханием, когда с буквы {\slv{ѵ}} начинается слово, или ударением, когда оно падает на слог, содержащий {\slv{ѵ}}, например: {\slv{ѵ҆ссѡ́пъ, тѵ́хѡнъ}}. После гласного звука буква {\slv{ѵ}} произносится как согласный звук <<в>>, например: {\slv{є҆́ѵа, лаѵре́нтїй, леѵі́тъ}}. Исключение составляют некоторые собственные имена, например: {\slv{мѡѷсе́й}}, а также составные слова, например: {\slv{триѷпоста́сный}}. Слово {\slv{мѵ́ро}} пишется через {\slv{ѵ}} для различения его в косвенных падежах от слов {\slv{мі́ръ}} и {\slv{ми́ръ}}.


                \subsubsection{Правописание букв {\large\slv{ꙗ}} и {\large\slv{ѧ}}}

    Буква {\slv{ꙗ}} представляет собою соединение двух звуков: {\slv{ї}} и {\slv{а}}. Прежде буква {\slv{ї}} обозначала тот звук, который изображается в русском языке иногда буквой <<й>> (<<и краткое>>), например в словах <<мой>>, <<лейка>>. Этот звук (а не буква) называется <<йотом>>, а гласные, перед которыми он слышится называются <<йотированными>>. При соединении звука <<йота>> со звуком {\slv{а}} получалось созвучие {\slv{їа}} (<<йа>>), которое и стали изображать буквой {\slv{ꙗ}}, а произносить: <<я>>. Обычно йотированная буква {\slv{ꙗ}} пишется только в начале слова, например: {\slv{ꙗ҆́рость, ꙗ҆́кѡ}}.
    
    Буква {\slv{ѧ}} первоначально изображала носовой звук, произносимый как <<ем>> или <<ен>>. Впоследствии этот носовой звук, называемый в древности <<малым юсом>>, перешел в звук <<я>>, но только не йотированный, т.е. без соединения с ним звука <<йота>>. Но нейотированный звук <<я>> может произноситься только после согласного звука, а потому буква {\slv{ѧ}} пишется в середине и конце слова, например: {\slv{мѧте́жъ, стезѧ̀}}.
    
    Но в слове <<язык>>, которое в славянском языке имеет много значений, как-то: член тела, орган речи, наречие или говор, народ, племя, то условились, что если слово <<язык>> обозначает народ или племя, писать букву {\slv{ꙗ}}, например: {\slv{ꙗ҆зы́къ самарі́йскїй}} (т.е. народ самарийский). Если же слово <<язык>> обозначает член тела, орган речи, наречие (говор), то условились писать букву{\slv{ѧ}}, например: {\slv{Прильпнѝ ѧ҆зы́къ мо́й горта́ни моемꙋ̀}} (Пс. 136, 6); {\slv{церко́вно-славѧ́нскїй ѧ҆зы́къ}}.
    
                    \paragraph{\exercise}

    \textbf{Славянские или церковные буквы, не сходные с русскими}
%    \medskip
    
    \begin{\exquote}\begin{slv}
        Оу҆́хо.~\sdash~Оу҆́тро.~\sdash~Оу҆ста̀.~\sdash~Оу҆́тренѧ.~\sdash~Оу҆спе́ние.~\sdash~Оу҆пова́нїе.~\sdash~Дꙋ́хъ.~\sdash~Дꙋша̀.~\sdash~Слꙋ́жба.~\sdash~І҆исꙋ́съ.~\sdash~Ꙗ҆́сли.~\sdash~Ꙗ҆зы́къ.~\sdash~И҆́мѧ.~\sdash~Сѣ́ьѧ.~\sdash~Марі́ѧ.~\sdash~Клѧ́тва.~\sdash~Вече́рнѧ.~\sdash~Лїтꙋргі́ѧ.~\sdash~І҆ѡ́на.~\sdash~І҆ѡа́ннъ.~\sdash~І҆ѡ́сифъ.~\sdash~І҆ѻрда́нъ.~\sdash~І҆ѡанні́кїй.
        
        Го́споди, поми́лꙋй.~\sdash~Приклонѝ ᲂу҆́хо Твоѐ мнѣ̀ и҆ ᲂуслы́ши глаго́лы моѧ̀.~\sdash~Ѕаꙋ́тра ᲂу҆слы́ши гла́съ мо́й, Царю̀ мо́й и҆ Бо́же мо́й.~\sdash~Оу҆слы́ши, Го́споди, пра́вдꙋ мою̀, вонмѝ моле́нїю моемꙋ̀.~\sdash~Да возра́дꙋетсѧ дꙋша̀ моѧ̀ ѡ҆ Го́сподѣ.~\sdash~Положѝ, Го́споди, хране́нїе ᲂу҆стѡ́мъ мои̑мъ.~\sdash~Сла́ва Тебѣ̀ Бо́гꙋ Благода́телю на́шемꙋ во вѣ́ки вѣко́въ.~\sdash~Сла́ва Ѻ҆тцꙋ̀ и҆ Сы́нꙋ и҆ Свѧто́мꙋ Дꙋ́хꙋ.~\sdash~Бо́же, ми́лостивъ бꙋ́ди мнѣ̀ грѣ́шникꙋ.
        
        Ꙗ҆вѝ нам, Го́споди, ми́лость Твою̀ и҆ спасе́нїе Твоѐ да́ждь на́мъ.~\sdash~И҆сповѣ́дайтесѧ Го́споди, ꙗ҆́кѡ благ: ꙗ҆́кѡ въ вѣ́къ ми́лость є҆гѡ́.~\sdash~Да бꙋ́детъ во́лѧ твоѧ̀, ꙗ҆́кѡ на небесѝ и҆ на землѝ.~\sdash~А҆́нгели, ᲂу҆спѣ́нїе Пречи́стыѧ ви́дѣвше, ᲂу҆диви́шасѧ.
        
        Благословѝ, дꙋшѐ моѧ̀, Го́спода.~\sdash~Го́споди, воззва́хъ къ Тебѣ̀, ᲂу҆слы́ши мѧ̀.~\sdash~Поми́лꙋй мѧ̀, Бо́же, поми́лꙋй мѧ̀.~\sdash~Пресвѧта́ѧ Тро́ице, поми́лꙋй на́съ.~\sdash~Свѧ́тъ, Свѧ́тъ, Свѧ́тъ є҆сѝ Бо́же, Богоро́дицею поми́лꙋй на́съ.
        
        Го́споди, ѡ҆чи́сти грѣхѝ на́шѧ.~\sdash~Моли́ Бо́га ѡ҆ на́съ, свѧти́телю ѻ҆́тче Нїко́лае, ꙗ҆́кѡ мы̀ ᲂу҆се́рднѡ къ Тебѣ̀ прибега́емъ.~\sdash~Ѽ, Всепѣ́таѧ Ма́ти, ро́ждшаѧ всѣ́хъ свѧты́хъ Свѧтѣ́йшее Сло́во!~\sdash~Ѽ, пресла́дкїй и҆ всеще́дрый І҆исꙋ́се, прїимѝ ны́нѣ ма́лое моле́нїе сїѐ на́ше, ꙗ҆́коже прїѧ́лъ є҆сѝ вдови́цы два̀ ле́пта.
    \end{slv}\end{\exquote}

                \subsubsection{Надстрочные знаки}

    В церковнославянском языке употребляются следующие надстрочные знаки: \textbf{ударения}, \textbf{придыхание} и \textbf{титла}.

    \bigskip
    \mockitem{1. Ударения}
    \medskip

    Ударения ставятся над гласными буквами для усиления слога. Первый слог слова, если он начинается с гласного звука, и последний слог слова, если он оканчивается гласным звуком, называются \emph{открытыми} слогами; все же остальные слоги слова называются \emph{закрытыми} слогами. Ударение ставится также над односложными словами.
    
    Ударения в церковнославянском языке бывают: \textbf{острое} ({\slv{~́}}), \textbf{тупое} ({\slv{~̀}}) и \textbf{облегченное} ({\slv{~̑}}).
    
    Острое ударение может стоять на любом слоге слова, кроме последнего открытого слога, например: {\slv{си́ла, мо́ре, любо́вь, горта́нь, є҆диномꙋ́дрие}}.
    
    Тупое ударение ставится всегда только на последнем открытом слоге, например: {\slv{рꙋка̀, млеко̀, сотворѝ, рцы̀}}.
    
    Облегченное ударение может стоять на всех слогах слова, кроме первого открытого слога. Это ударение вообще употребляется для различения созвучных грамматических форм, о чем будет сказано в надлежащем месте.
    
    Если за словом с ударением на последнем открытом слоге следует одно из односложных слов, оканчивающихся на гласный звук, т.е. состоящих из одного открытого слога, а именно: {\slv{бо̀, мѧ̀, мѝ, тѧ̀, тѝ, сѧ̀, сѝ, жѐ}} (но не предлоги {\slv{во, на, по, при}} и проч.), то ударение над этими односложными словами опускается, а предшествующее им слово меняет тупое ударение на острое. Например: {\slv{Сꙋди́ ми, бж҃е}} (Пс. 42, 1); {\slv{Бѣ́ же во дни̑ во це́ркви ᲂу҆чѧ̀}} (Лк. 21, 37).
    
    Односложные предлоги и частицы {\slv{не, ни, но, на, по, да, же, при, пре, бо, во, ко, со, за}} и др. не имеют над собою никакого ударения. Исключение из этих правил составляет слово {\slv{сѐ}} (всю), которое сохраняет над собою ударение, и иногда частица {\slv{ни}}, когда она употребляется в значении русского слова <<нет>>, например: {\slv{Бꙋ́ди же сло́во ва́ше: є҆́й, є҆́й: нѝ, нѝ}} (Мф. 5, 37).
    
    \bigskip
    \mockitem{2. Придыхание}
    \medskip
    
    Когда слово начинается с открытого слога, то над гласной буквой этого слога ставится придыхание ({\slv{~҆}}), например: {\slv{є҆ле́й, и҆мѣ́ти, ᲂу҆чени́къ, ꙗ҆зы́къ}}.
    
    В словах, где ударение падает на начальный открытый слог, то над гласной буквой ставится придыхание вместе с острым ударением, например: {\slv{ѻ҆́ко, а҆́ще, и҆́мѧ, є҆́зеро}}. Такое соединение придыхания с острым ударением называется словом {\slv{и҆́со}}.
    
    Союзы {\slv{и҆}} и {\slv{а҆}}, а также предлоги {\slv{ᲂу҆}} и {\slv{ѡ҆}} пишутся только с одним придыханием (без ударения).
    
    Иногда бывает соединение придыхания с тупым ударением. Такое соединение называется словом {\slv{а҆́построфь}}.
    
    Апострофь ставится на начальном открытом ударном слоге для различения смысла некоторых созвучных слов. Например: {\slv{и҆}} (союз <<и>>) и {\slv{и҆̀}} (он, его), {\slv{ᲂу҆}} (предлог <<у>>) и {\slv{ᲂу҆̀}} (еще), {\slv{и҆́же}} (который) и {\slv{и҆̀же}} (которые) и т.п.

    Эти апострофорные слова сохраняют свое ударение и после слов, оканчивающихся ударным открытым слогом. Например: {\slv{И̑ а҆́бїе воззва̀ ѧ҆̀}} (Мк. 1, 20).
    
    \bigskip
    \mockitem{3. Титла}
    \medskip
    
    В церковнославянском языке некоторые слова пишутся сокращенно: с пропуском одной или нескольких букв, а иногда и целых слогов в середине слова. Сокращение это обозначается особыми знаками \textbf{титлами}.
    
    Различают простое титло и буквенные титла.
    
    \emph{Простое} титло изображается знаком {\slv{~҃}}, который ставится вверху слова над той буквой, перед которой или после которой имеется пропуск одной или нескольких букв. Например: {\slv{бг҃ъ}} (Бог), {\slv{ѻ҆ц҃ъ}} (Отец), {\slv{бл҃года́ть}} (благодать).
    
    \emph{Буквенное} титло представляет собою дугообразный титловой знак {\slv{~҇}} с какой-либо подписной буквой, именно с той, какая буква пропущена или находится среди пропущенных (если их несколько). Буквенные титла следующие:

    \bigskip\autorows{l}{1}{l}{
        \hspca{{\slv{~ⷢ҇}}~---~глаголь-титло: {\slv{є҆ѵⷢ҇лїе}} (Евангелие),},
        \hspca{{\slv{~ⷪ҇}}~---~он-титло: {\slv{прⷪ҇ро́къ}} (пророк),},
        \hspca{{\slv{~ⷬ҇}}~---~рцы-титло: {\slv{и҆мⷬ҇къ}} (имярек),},
        \hspca{{\slv{~ⷭ҇}}~---~слово-титло: {\slv{гдⷭ҇ь}} (Господь).}
    }

    К буквенным титлам еще относится добро-титло, которое изображается без титлового знака {\slv{~҇}}, а просто ставится буква <<добро>> вверху слова. Например: {\slv{бцⷣа}} (Богородица), {\slv{влⷣко}} (Владыко).
    
    Общее правило постановки титл таково. Под титло подводится слово тогда, когда оно относится к лицам и предметам священным, почитаемым; наоборот, те же самые слова, но относящиеся к обыкновенным именам, предметам и понятиям, в особенности же к языческим, пишутся без титла. Например:

    \bigskip\autorows[-1pt]{l}{2}{l}{
        \hspca{{\slv{бг҃ъ}} (истинный Бог)}, {{\slv{бо́гъ}} (бог языческий)},
        \hspca{{\slv{і҆и҃съ}} (имя Господа)}, {{\slv{і҆исꙋ́съ}} (имя человека)},
        \hspca{{\slv{мр҃і́а}} (Пресвятая Дева Мария)}, {{\slv{марі́а}} (имя женщины)},
        \hspca{{\slv{мт҃и}} (Матерь Господа)}, {{\slv{ма́ти}} (земная мать)},
        \hspca{{\slv{ѻ҆ц҃ъ}} (Отец Небесный)}, {{\slv{ѻ҆те́цъ}} (земной отец)},
        \hspca{{\slv{дв҃а}} (Дева Пресвятая)}, {{\slv{дѣ́ва}} (девица)},
        \hspca{{\slv{дх҃ъ}} (Дух Святый)}, {{\slv{дꙋ́хъ}} (дыхание)},
        \hspca{{\slv{а҆́гг҃лъ}} (Ангел)}, {{\slv{а҆́ггелъ}} (аггел, или злой дух)}
    }

    Изредка под титлами пишутся и обыкновенные слова, имеющие в отдельных случаях священное значение или священную память:
    
    {\slv{мцⷭ҇ъ}} (месяц), {\slv{чл҃вѣ́къ}} (человек), {\slv{нн҃ѣ}} (ныне), {\slv{срⷣце}} (сердце), {\slv{сл҃нце}} (солнце), {\slv{дн҃ь}} (день).
    
    Есть сокращения на особые случаи:
    
    1. В венце образа Христа Спасителя встречаются надписи: {\slv{о҆ ѡ҆́н}}. Это греческое слово (\textgreek{ὁ ὠν}), написанное славянскими буквами и означающее {\slv{сы́й}}, т.е. сущий.
    
    2. В венце образа распятого на кресте Христа Спасителя имеются такие буквы: {\slv{І҆.Н҃.Ц҃.І҆.}}, или {\slv{І҆Н҃Ц҃І҆}}. Это означает: <<Иисус Назарянин Царь Иудейский>>.
    
    3. На образе Божией Матери встречается надпись {\slv{мр҃ ѳꙋ҃}}, т.е. по-гречески (\textgreek{Μήτηρ τοῦ Θεοῦ})~---~Матерь Бога.
    \pagebreak
    
                \subsubsection{Буквы как числовые знаки (цифры)}

    Буквы церковнославянской азбуки служат одновременно и цифрами. Чтобы показать, что данная буква является цифрой, над буквой ставится простое титло~{\slv{~҃}}.

    \begin{placedtabular}[%
%        caption={\tabcaptsize Числа от 1 до 1\,000\,000}
        ]{|c|c|c|c|}
        \hline
        
        \emph{Единицы}
        & \emph{Десятки}
        & \emph{Сотни}
        & \emph{Тысячи}
        \\
        
        \mkcella{изображаются\\следующими\\буквами}
        & \mkcella{изображаются\\следующими\\буквами}
        & \mkcella{изображаются\\следующими\\буквами}
        & \mkcella{изображаются теми же буквами,\\какими единицы,\\десятки и сотни,\\но с прибавлением впереди\\буквы значка}
        \\
    
        &
        &
        & \huge{\slv{҂}}
        \\\hline
        
        {\slv{а҃}} (1) & {\slv{і҃}} (10) & {\slv{р҃}} (100) & {\slv{҂а҃}} (1000) \\\hln
        {\slv{в҃}} (2) & {\slv{к҃}} (20) & {\slv{с҃}} (200) & {\slv{҂в҃}} (2000) \\\hln
        {\slv{г҃}} (3) & {\slv{л҃}} (30) & {\slv{т҃}} (300) & {\slv{҂г҃}} (3000) \\\hln
        {\slv{д҃}} (4) & {\slv{м҃}} (40) & {\slv{у҃}} (400) & {\slv{҂і҃}} (10\,000) \\\hln
        {\slv{є҃}} (5) & {\slv{н҃}} (50) & {\slv{ф҃}} (500) & {\slv{҂к҃}} (20\,000) \\\hln
        {\slv{ѕ҃}} (6) & {\slv{ѯ҃}} (60) & {\slv{х҃}} (600) & {\slv{҂л҃}} (30\,000) \\\hln
        {\slv{з҃}} (7) & {\slv{ѻ҃}} (70) & {\slv{ѱ҃}} (700) & {\slv{҂р҃}} (100\,000) \\\hln
        {\slv{и҃}} (8) & {\slv{п҃}} (80) & {\slv{ѿ}} (800) & {\slv{҂с҃}} (200\,000) \\\hln
        {\slv{ѳ҃}} (9) & {\slv{ч҃}} (90) & {\slv{ц҃}} (900) & {\slv{҂҂а҃}} (1\,000\,000) \\\hline
        
    \end{placedtabular}

    \bigskip
    Десять тысяч ({\slv{҂і҃}}) произносится по-славянски {\slv{тьма̀}}, двадцать тысяч ({\slv{҂к҃}})~---~{\slv{двѣ̀ тьмы̀}}, тридцать тысяч ({\slv{҂л҃}})~---~{\slv{трѝ тьмы̀}} и т.д. Сто тысяч ({\slv{҂р҃}}) называется по-славянски {\slv{легеѡ́нъ}}, двести тысяч ({\slv{҂с҃}})~---~{\slv{два̀ легеѡ́на}} и т.д.
    
    Числа от 10 до 20 составляются следующим образом: сначала ставят единицы и справа приписывают десятки, причем титло ставится над единицами. Вот нумерация чисел второго десятка: {\slv{а҃і}} (11), {\slv{в҃і}} (12), {\slv{г҃і}} (13), {\slv{д҃і}} (14), {\slv{є҃і}} (15), {\slv{ѕ҃і}} (16), {\slv{з҃і}} (17), {\slv{и҃і}} (18), {\slv{ѳ҃і}} (19).
    
    Числа от 20 до 100 составляются порядком, т.е. сперва пишут десятки, а справа приписывают единицы; титло ставится над десятками: {\slv{к҃а}} (21), {\slv{л҃в}} (32), {\slv{м҃д}} (44), {\slv{н҃ѕ}} (56), {\slv{ѯ҃г}} (63), {\slv{ѻ҃ѳ}} (79).
    
    Таким же обычным порядком составляются числа и от 100 до 1000 и далее; титло во всех этих числах ставится всегда на предпоследней цифре (второй цифре справа): {\slv{сл҃є}} (235), {\slv{тн҃з}} (357), {\slv{фп҃ѕ}} (586), {\slv{хч҃д}} (694), {\slv{ѱм҃ф}} (749), {\slv{ѿо҃ⷢⷢг}} (873), {\slv{цк҃в}} (922), {\slv{҂ацн҃з}} (1957), {\slv{҂аѿп҃ѕ}} (1886), {\slv{҂аѿч҃д}} (1894), {\slv{҂зтч҃а}} (7391), {\slv{҂кцг҃і}} (20913), {\slv{҂ѻєу҃ѕ}} (75406).
    \bigskip
    \pagebreak

    \textbf{Упражнения в чтении употребительнейших слов под титлами}

                    \paragraph{\exercise}
    
    \begin{\exquote}\begin{slv}
        Бг҃ъ \pxp{Бо́гъ}, Бг҃а \pxp{Бо́га}, Бг҃ꙋ \pxp{Бо́гꙋ}, Бг҃ови \pxp{Бо́гови}, Бг҃омъ \pxp{Бо́гомъ}, Бж҃е \pxp{Бо́же}, ѡ҆ Бз҃ѣ \pxp{Бо́зѣ}.
        
        Бг҃ъ любы̀ є҆́сть.~\sdash~Бг҃ъ застꙋ́прикъ мо́й.~\sdash~По́йте Бг҃ꙋ на́шемꙋ, по́йте!~\sdash~А҆́ще кто̀ рече́тъ, ꙗ҆́кѡ люблю̀ Бг҃а, а҆ бра́та своего̀ ненави́дитъ, ло́жъ є҆́сть.~\sdash~Бж҃е, ѡ҆чи́сти мѧ̀ грѣ́шнаго.~\sdash~Ѡ҆ Бз҃ѣ спасе́нїе моѐ и҆ сла́ва моѧ̀.~\sdash~Бж҃їѧ никто́же вѣ́сть.
        
        Гдⷭ҇ь \pxp{Госпо́дь}, Гдⷭ҇а \pxp{Го́спода}, Гдⷭ҇ꙋ \pxp{Го́сподꙋ}, Гдⷭ҇ви \pxp{Го́сподеви}, Гдⷭ҇и \pxp{Го́споди}, ѡ҆ Гдⷭ҇ѣ \pxp{Го́споде}.
        
        Гдⷭ҇ь мнѣ̀ помо́щникъ.~\sdash~Не во́змеши и҆́мене Гдⷭ҇а Бг҃а твоегѡ̀ всꙋ́е.~\sdash~Гдⷭ҇ꙋ помо́лимсѧ! Гдⷭ҇и, поми́лꙋй.~\sdash~Гдⷭ҇и, не введѝ менѐ въ напа́сть.~\sdash~Да возра́дꙋетсѧ дꙋша̀ моѧ̀ ѡ҆ Гдⷭ҇ѣ.
        
        І҆и҃съ \pxp{І҆исꙋ̀съ} Хрⷭ҇то́съ \pxp{Хрїсто́съ}, І҆и҃са Хрⷭ҇та̀, І҆и҃сомъ Хрⷭ҇то́мъ, ѡ҆ І҆и҃сѣ Хрⷭ҇тѣ̀, Сп҃съ \pxp{Спа́съ}, Сп҃си́тель \pxp{Спаси́тель}, Хрⷭ҇то́въ \pxp{Христо́въ}.
        
        Воскр҃съ І҆и҃съ ѿ гро́ба, ꙗ҆́кѡже проречѐ.~\sdash~Вѣ́рꙋю во Є҆ди́наго Гдⷭ҇а І҆и҃са Хрⷭ҇та̀.~\sdash~І҆и҃се, Сы́не Бж҃їй, поми́лꙋй мѧ̀.~\sdash~І҆и҃се, Сп҃се мо́й, спасѝ мѧ҃!~\sdash~Возра́довасѧ дх҃ъ мо́й ѡ҆ Бз҃ѣ Сп҃сѣ мое́мъ.~\sdash~Дрꙋ́къ дрꙋ́га тѧгѡты̀ носи́те, и҆ та́кѡ и҆спо́лните зако́нъ Хрⷭ҇то́въ.
    \end{slv}\end{\exquote}

                    \paragraph{\exercise}
    
    \begin{\exquote}\begin{slv}
        Ст҃ъ \pxp{Свѧ́тъ}, Ст҃ы́й \pxp{Свѧты́й}, Ст҃а́гѡ, Ст҃о́мꙋ, Ст҃и, Ст҃а́ѧ, Ст҃ы̑мъ.
        
        Ст҃ъ Гдⷭ҇ь Бг҃ъ на́шъ.~\sdash~Ст҃ъ, Ст҃ъ, Ст҃ъ Гдⷭ҇ь Вседержи́тель.~\sdash~Поклони́мсѧ Ст҃о́мꙋ Гдⷭ҇ꙋ І҆и҃сꙋ, Є҆ди́номꙋ безгрѣ́шномꙋ.~\sdash~Ст҃ы́й Бж҃е, Ст҃ы́й Крѣ́пкїй, Ст҃ы́й Безсме́ртный поми́лꙋй на́съ!~\sdash~Свѧ́ти бꙋ́дите, ꙗ҆́кѡ А҆́зъ ст҃ъ \pxp{є҆́смь} Гдⷭ҇ь Бг҃ъ ва́шъ.~\sdash~Ст҃а̑ѧ ст҃ы̑мъ.
        
        Ѻ҆цъ \pxp{Ѻ҆те́цъ}, Сн҃ъ \pxp{Сы́нъ}, Дх҃ъ \pxp{Дꙋ́хъ}, Дш҃а̀ \pxp{Дꙋша̀}, Трⷪ҇ца \pxp{Тро́ица}, Нн҃ѣ \pxp{Ны́нѣ}, Прⷭ҇нѡ \pxp{При́снѡ}.
        
        Ѻ҆ц҃ъ, Сн҃ъ и҆ Ст҃ы́й Дх҃ъ Трⷪ҇ца Ст҃а́ѧ.~\sdash~Оу҆пова́нїе моѐ Ѻ҆ц҃ъ, прибѣ́жище моѐ Сн҃ъ, покро́въ мо́й Дх҃ъ Ст҃ы́й, Трⷪ҇ца Ст҃а́ѧ, сла́ва Тебѣ̀.~\sdash~Во и҆́мѧ Ѻ҆ц҃а̀ и҆ Сн҃а и҆ Ст҃а́гѡ Дх҃а. А҆ми́нь.~\sdash~Сла́ва Ѻ҆ц҃ꙋ̀ и҆ Сн҃ꙋ и҆ Ст҃о́мꙋ Дх҃ꙋ, и҆ нн҃ѣ и҆ прⷭ҇нѡ, и҆ во вѣ́ки вѣкѡ́въ. А҆ми́нь.~\sdash~Вѣ́рꙋю и҆ въ Дх҃а Ст҃а́го, Гдⷭ҇а животворѧ́щаго, и҆́же со Ѻ҆ц҃емъ и҆ Сн҃омъ спокланѧ́ема и҆ ссла́вима.~\sdash~Прест҃а́ѧ Трⷪ҇це, Бж҃е на́шъ, сла́ва Тебѣ̀.~\sdash~Ѿ сна̀ воста́въ, бл҃годарю̀ Тѧ, Ст҃а́ѧ Трⷪ҇це.~\sdash~Да возра́дꙋетсѧ дш҃а̀ моѧ̀ ѡ҆ Гдⷭ҇ѣ, ѡ҆блече́ бо мѧ̀ въ ри́зꙋ спасе́нїѧ.
        
        Бцⷣа \pxp{Богоро́дица}, Дв҃а \pxp{Дѣ́ва}, Мр҃і́ѧ \pxp{Марі́ѧ}, Мт҃и \pxp{Ма́ти}, Прⷭ҇нодв҃а \pxp{Приснодѣ́ва}.
        
        Мл҃твами Бцⷣы, млⷭ҇тиве, ѡ҆чи́сти мно́жество согрѣше́нїй на́шихъ.~\sdash~Пребл҃гослове́нна є҆сѝ Бцⷣе Дв҃о.~\sdash~Ра́дꙋйсѧ, бл҃года́тнаѧ Мр҃і́е.~\sdash~Спасѝ ѿ бѣ́дъ рабы̑ твоѧ̑, Бцⷣе.~\sdash~Пресла́внаѧ Прⷭ҇нодв҃о, Мт҃и Хрⷭ҇та̀ Бг҃а, принесѝ на́шꙋ мл҃твꙋ Сн҃ꙋ Твоемꙋ̀ и҆ Бг҃ꙋ на́шемꙋ, да спасе́тъ Тобо́ю дш҃ы на́ша.
    \end{slv}\end{\exquote}
    \medskip

                    \paragraph{\exercise}

    \begin{\exquote}\begin{slv}
        А҆́гг҃лъ \pxp{А҆́нгелъ}, Влⷣко \pxp{Влады́ко}, Влⷣчца \pxp{Влады́чица}, Бл҃гъ \pxp{Бла́гъ}, Цр҃ь \pxp{Ца́рь}, Цр҃ица \pxp{Цари́ца}.
        
        Ѡ҆полчи́тсѧ А҆́гг҃лъ Гдⷭ҇ень ѡ҆́крестъ боѧ́щихсѧ є҆гѡ̀, и҆ и҆зба́витъ и҆̀хъ.~\sdash~Ст҃і́и а҆рха́г҃гли, моли́те Бг҃а ѡ҆ на́съ.~\sdash~А҆́гг҃лѡмъ свои̑мъ заповѣ́сть ѡ҆ тебѣ̀, сохрани́ти тѧ̀ во всѣ́хъ пꙋте́хъ твои́хъ.~\sdash~По́йте Бг҃ꙋ на́шемꙋ, по́йте Цр҃е́ви на́шемꙋ, по́йте: ꙗ҆́кѡ цр҃ь всеѧ̀ землѝ Бг҃ъ.~\sdash~А҆́гг҃ле, храни́телю мо́й ст҃ы́й.~\sdash~Бл҃гословѝ, Влⷣка, ст҃ы́й вхо́ди.~\sdash~Прест҃а́ѧ Влⷣчце Бцⷣе, молѝ ѡ҆ на́съ грѣ́шныхъ.~\sdash~Гдⷭ҇и и҆ Влⷣко живота́ моегѡ̀, дꙋ́хъ пра́здности, ᲂу҆ны́нїѧ, любонача́лїѧ, и҆ праздносло́вїѧ не да́ждь мѝ.~\sdash~Бл҃гъ мнѣ̀ зако́нъ ᲂу҆́стъ Твои́хъ, па́че ты́сѧщъ зла́та и сребра̀.~\sdash~Црⷭ҇тво Твоѐ црⷭ҇тво всѣ́хъ вѣкѡ́въ.
        
        Є҆ѵⷢ҇лі́стъ \pxp{Є҆ѵангелі́стъ}, Є҆пⷭ҇кпъ \pxp{Є҆пі́скопъ}, І҆и҃ль \pxp{І҆зра́иль}, Крⷭ҇тъ \pxp{Кре́стъ}, І҆ерⷭ҇ли́мъ \pxp{І҆ерꙋсали́мъ}.
        
        Велича́емъ тѧ̀, а҆пⷭ҇ле Хрⷭ҇то́въ и҆ є҆ѵⷢ҇лі́сте І҆ѡа́нне Бг҃осло́ве.~\sdash~Крⷭ҇тꙋ̀ Твоемꙋ̀ поклонѧ́емсѧ, Влⷣко!~\sdash~Сла́ва, Гдⷭ҇и, крⷭ҇тꙋ̀ Твоемꙋ̀ чтⷭ҇но́мꙋ.~\sdash~И̑сповѣ́дꙋю є҆ди́но кр҃ще́нїе во ѡ҆ставле́нїе грѣхѡ́въ.~\sdash~Ѡ̑гради́ мѧ, Гдⷭ҇и, си́лою чтⷭ҇на́гѡ и҆ животворѧ́щагѡ Твоегѡ̀ крⷭ҇та̀, и҆ сохрани́ мѧ ѿ всѧ́кагѡ ѕла̀.
    \end{slv}\end{\exquote}
    \medskip

                    \paragraph{\exercise}

    \begin{\exquote}\begin{slv}
        Мрⷣость \pxp{Мꙋ̀дрость}, Премⷣръ \pxp{Премꙋ́дръ}, Млⷭ҇ть \pxp{Ми́лость}, Мл҃тва \pxp{Моли́тва}, Прⷭ҇то́лъ \pxp{Престо́лъ}.
        
        И҆́стинꙋ возлюби́лъ є҆сѝ, безвѣ́стнаѧ и҆ та́йнаѧ премⷣрости Твоеѧ̀ ꙗ҆ви́лъ мѝ є҆сѝ.~\sdash~Оу҆тверди́лъ є҆́сть Гдⷭ҇ь млⷭ҇ть Свою̀ на боѧ́щихсѧ Є҆гѡ́.~\sdash~Прⷭ҇то́лъ Тво́й, Бж҃е, въ вѣ́къ вѣ́ка.
        
        Всѧ̑ нбⷭ҇ныѧ си̑лы ст҃ы́хъ а҆́гг҃лъ и҆ а҆рха̑гг҃лъ, моли́те ѡ҆ на́съ грѣ́шныхъ.~\sdash~Пра́ведницы просвѣтѧ́тсѧ ꙗ҆́кѡ со́лнце, въ Ца́рствїи Ѻ҆ц҃а̀ и҆́хъ.~\sdash~Сщ҃е́нницы Твоѝ ѡ҆блекꙋ̀тсѧ пра́вдою, и҆ пре́пбнїи Твоѝ возра́дꙋютсѧ.~\sdash~Ржⷭ҇тво̀ Твоѐ, Хрⷭ҇те Бж҃е нашъ, возсїѧ̀ мі́рови свѣ́тъ ра́зꙋма.~\sdash~Нб҃о прⷭ҇то́лъ Мо́й, землѧ̀ же подно́жїе но́гъ Мои́хъ.~\sdash~Па́мѧть прⷣвныхъ съ похвала́ми: и҆́мѧ же нечести́выхъ ᲂу҆гаса́етъ.~\sdash~И̑з̾ ᲂу҆́стъ младє́нецъ и҆ ссꙋ́щихъ соверши́лъ є҆сѝ хвалꙋ̀.~\sdash~Совѣ́тъ Гдⷭ҇ень во вѣ́къ пребыва́етъ, помышлє҆нїѧ срⷣца Є҆гѡ́ въ ро́дъ и҆ ро́дъ.~\sdash~Въ нача́лѣ сотворѝ Бг҃ъ не́бо и҆ зе́млю.~\sdash~Ѻ҆́чи Гдⷭ҇ни на прⷣвныѧ, и҆ ᲂу҆́ши Є҆гѡ́ въ моли́твꙋ и҆̀хъ.~\sdash~Млⷭ҇рдїѧ двє́ри ѿве́рзи на́мъ, Бл҃гослове́ннаѧ Бцⷣе!~\sdash~Не клени́тесѧ ни нб҃ом, ни земле́ю, ни и҆но́ю ко́ею клѧ́твою.~\sdash~Во Црⷭ҇твїи Твое́мъ помѧнѝ на́съ, Гдⷭ҇и, є҆гда̀ прїи́деши во Црⷭ҇твїи Твое́мъ.
    \end{slv}\end{\exquote}
    \pagebreak

        \section[Общие понятия о грамматических формах]{Общие понятия о грамматических формах церковнославянского языка}
                \subsubsection{Разделение звуков}

    \medskip
    Звуки церковнославянского языка, как и в русском языке, разделяются на \textbf{гласные} и \textbf{согласные}.
    \bigskip
    
    Гласные звуки бывают:
    
    \bigskip\autorows{l}{1}{l}{
        \hspca{\emph{твердые}: {\slv{а, о, ꙋ, ы}};},
        \hspca{\emph{мягкие}: {\slv{ꙗ}} ({\slv{ѧ}}){\slv{, е, ю, и}} ({\slv{ї, ѵ}}){\slv{, ѣ}}.}
    }

    Согласные звуки делятся на:
    
    \bigskip\autorows{l}{1}{l}{
        \hspca{\emph{губные}: {\slv{б, в, м, п, ф}} ({\slv{ѳ}}){\slv{, ѵ}};},
        \hspca{\emph{гортанные}: {\slv{г, к, х}};},
        \hspca{\emph{зубные}: {\slv{д, з}} ({\slv{ѕ}}){\slv{, с, т, ц}}~---~эти звуки, кроме {\slv{д}} и {\slv{т}},}
    }
    
    называются еще свистящими;
    
    \bigskip\autorows{l}{1}{l}{
        \hspca{\emph{шипящие}: {\slv{ж, ч, ш, щ}};},
        \hspca{\emph{плавные}: {\slv{л, м, н, р}}; плавные звуки {\slv{м}} и {\slv{н}}}
    }

    называются еще носовыми.
    \medskip

                \subsubsection{Чередование звуков}

    При изменении слов гласные и согласные звуки могут чередоваться ({\slv{берꙋ̀~\sdash~соб{\Large и}ра́ю, сꙋ́хъ~\sdash~сꙋ{\Large ш}и́ти}}). Особенно часто чередуются твердые согласные звуки с более мягкими. Это чередование происходит или посредством замены твердых согласных мягкими, или посредством вставок мягкого согласного перед твердым или после него. Такое чередование согласных носит название \emph{смягчение согласных звуков}.
    
    Вот основные случаи смягчения:

    \bigskip
    \mockitem{1. Смягчение гортанных звуков}
    \medskip
    
    Гортанные звуки {\slv{г, к, х}} перед гласными звуками {\slv{е}} и {\slv{и}} смягчаются посредством замены их шипящими, а именно:

    \begin{adjustwidth}{\hstbb}{0cm}
        \renewcommand*{\arraystretch}{1.2}
        \begin{tabular}[l]{lcl}

            {\slv{г~\sdash~ж:}}
            & ~~~~~
            & {\slv{дрꙋ́гъ~\sdash~дрꙋ́же~\sdash~дрꙋжи́ти}}
            \\
            
            {\slv{к~\sdash~ч:}}
            & ~~~~~
            & {\slv{ре́къ~\sdash~речѐ~\sdash~рѣ́чи}}
            \\
            
            {\slv{х~\sdash~ш:}}
            & ~~~~~
            & {\slv{дꙋ́хъ~\sdash~дꙋ́ше}}
            \\
            
        \end{tabular}
    \end{adjustwidth}

    \medskip
    Но эти же самые гортанные звуки, находясь перед падежными окончаниями {\slv{ѣ}} и {\slv{и}} ({\slv{ы}}), а также перед {\slv{и}} ({\slv{ы}}) в некоторых глагольных формах,~---~смягчаются несколько иначе, посредством замены их свистящими:

    \begin{adjustwidth}{\hstbb}{0cm}
        \renewcommand*{\arraystretch}{1.2}
        \begin{tabular}[l]{lcl}
            
            {\slv{г~\sdash~з:}}
            & ~~~~~
            & {\slv{дрꙋ́гъ~\sdash~дрꙋ́зи~\sdash~ѡ҆ дрꙋзѣ́хъ}}
            \\
            
            {\slv{к~\sdash~ц:}}
            & ~~~~~
            & {\slv{ре́къ~\sdash~рцы̀}}
            \\
            
            {\slv{х~\sdash~с:}}
            & ~~~~~
            & {\slv{лꙋ́хъ~\sdash~лꙋ́си~\sdash~ѡ҆ лꙋ́сѣх}}
            \\
            
        \end{tabular}
    \end{adjustwidth}

    \medskip
    Итак, гортанные звуки могут смягчаться или шипящими (1-й закон смягчения), или свистящими (2-й закон смягчения.)
    
    \bigskip
    \mockitem{2. Смягчение зубных звуков}
    \medskip
    
    Зубные звуки смягчаются шипящими, а именно:

    \begin{adjustwidth}{\hstbb}{0cm}
        \renewcommand*{\arraystretch}{1.2}
        \begin{tabular}[l]{lcl}
            
            {\slv{з~\sdash~ж:}}
            & ~~~~~
            & {\slv{вѧза́ти~\sdash~вѧжꙋ̀}}
            \\
            
            {\slv{с~\sdash~ш:}}
            & ~~~~~
            & {\slv{писа́ти~\sdash~пишꙋ̀}}
            \\
            
            {\slv{т~\sdash~щ:}}
            & ~~~~~
            & {\slv{свѣ́тъ~\sdash~свѣщꙋ̀~\sdash~свѣща̀}}
            \\
            
            {\slv{ц~\sdash~ч:}}
            & ~~~~~
            & {\slv{ѻ҆те́цъ~\sdash~ѻ҆те́чествїе}}
            \\
            
        \end{tabular}
    \end{adjustwidth}

    \medskip
    Зубной звук {\slv{д}} смягчается вставкой перед ним звука {\slv{ж}}:
    
    \begin{adjustwidth}{\hstbb}{0cm}
        \renewcommand*{\arraystretch}{1.2}
        \begin{tabular}[l]{l}
            
            {\slv{ходи́ти~\sdash~хождꙋ̀~\sdash~хожде́нїе}}
            \\
            
        \end{tabular}
    \end{adjustwidth}
    
    \bigskip
    \mockitem{3. Смягчение губных звуков}
    \medskip

    Губные звуки {\slv{б, в, м, п}} перед некоторыми гласными смягчаются вставкой после губных плавного звука {\slv{л}}:
    
    \begin{adjustwidth}{\hstbb}{0cm}
        \renewcommand*{\arraystretch}{1.2}
        \begin{tabular}[l]{l}
            
            {\slv{люби́ти~\sdash~люблю̀}}\\
            {\slv{ꙗзви́ти~\sdash~ꙗ҆звлю̀}}\\
            {\slv{ломи́ти~\sdash~ломлю̀}}\\
            {\slv{спа́ти~\sdash~сплю̀}}\\
            
        \end{tabular}
    \end{adjustwidth}

                \subsubsection{Состав слова}

    Слова образуются через сочетания отдельных звуков.
    
    Соединение гласного звука с одним или несколькими согласными составляет \textbf{слог}. Очевидно, что в слове столько слогов, сколько в нем гласных звуков, а поэтому слова могут быть \emph{односложные, двухсложные} и вообще \emph{многосложные}.
    
    Одни из звуков, из которых образуются слова, являются неизменными, другие же могут изменяться. Так, например, в словах: {\slv{да́ти, дарова́нїе, воздаѧ́нїе}} и т.д. Звуки {\slv{да}} во всех этих словах не изменяются, остальные же звуки: {\slv{-ти, -рованїе, воз-, -ѧнїе}}~---~представляют собою различные сочетания.
    
    Неизменяемые звуки в слове составляют корень слова. Существенным признаком корня слова служит его \emph{односложность}.
    
    Часть слова, стоящая перед корнем, называется \textbf{приставкой}, например: в слове {\slv{воздаѧ́нїе}} слог {\slv{воз-}}, стоящий перед корнем {\slv{-да-}}, есть приставка.
    
    Часть слова, стоящая после корня и указывающая на определенный падеж в именах или на лицо в глаголах, называется \textbf{окончанием}, например в словах {\slv{сы́нъ, сы́на, сы́нꙋ}} или {\slv{нестѝ, несꙋ́тъ}} звуки {\slv{-ъ, -а, -ꙋ, -ти, -ꙋтъ}} являются окончаниями.
    
    Иногда между корнем слова и окончанием вставляется один или несколько звуков, которые образуют вставочные слоги в слове. Такие вставочные слоги в слове называются \textbf{суффиксами}; так, например, в слове {\slv{дарова́нїе}} слоги {\slv{-ро-, -ва-, -нї-}} являются суффиксами.
    
    Окончание слова меняется в зависимости от падежей и чисел в именах или в зависимости от времени, лица, числа в глаголах. Остальные же части слова (корень, приставка, суффиксы) в этих случаях не меняются и в своей совокупности называются \textbf{основой} слова. Например, в слове {\slv{воздаѧ́нїе}} основой слова будет {\slv{воздаѧ́нї}}, а окончанием~---~{\slv{е}}.

                \subsubsection{Слова простые и сложные}

    Слова бывают простые и сложные.
    
    Если в состав слова входит только один корень, то такое слово называется \textbf{простым}. Например: {\slv{вѣ́ра, ѡ҆бходи́ти}}.
    
    Если же слово произошло от соединения двух или более простых слов, то такое слово называется \textbf{сложным}. Возьмем, например, два таких простых слова: {\slv{смире́нїе}} и {\slv{мꙋ̀дрость}}. Из этих двух слов мы можем составить одно сложное слово: {\slv{смиренномꙋ́дрїе}}. Ясно, что в сложном слове столько корней, сколько в состав этого слова входит простых слов. В приведенном сложном слове {\slv{смиренномꙋ́дрїе}} два корня: {\slv{-мир-}} и {\slv{-мꙋдр-}}.
    
    Простые слова соединяются в сложные посредством так называемой \textbf{соединительной гласной}. Такими соединительными гласными служат звуки {\slv{е}} или {\slv{о}}, например: {\slv{пꙋт{\Large е}ше́ствие, благ{\Large о}лѣ́пїе}}.

                \subsubsection{Славянское неполногласие}

    \textbf{Славянским неполногласием} называется наличие в словах некоторых звуковых сочетаний между согласными. Эти звуковые сочетания следующие: {\slv{ра, ла, ре, ле}}.
    
    Неполногласием эти звуковые сочетания названы в противоположность соответствующим звуковым сочетаниям: {\slv{оро, оло, ере, еле}}, которые называются \textbf{русским полногласием}.
    
    Вот несколько примеров славянского неполногласия:

    \bigskip\autorows[-1pt]{l}{3}{l}{
        \hspca{{\slv{гра́дъ}}~---~город},{{\slv{сребро̀}}~---~серебро},{{\slv{брада̀}}~---~борода},
        \hspca{{\slv{зла́то}}~---~золото},{{\slv{млеко̀}}~---~молоко},{{\slv{глава̀}}~---~голова},
        \hspca{{\slv{дре́во}}~---~дерево},{{\slv{кла́съ}}~---~колос},{{\slv{врата̀}}~---~ворота}
    }

                \subsubsection{Части речи и их грамматические формы}

    В церковнославянском языке, как и в русском, десять частей речи. Все части речи разделяются на изменяемые и неизменяемые.

    К \textbf{изменяемым частям речи} относятся имена существительные, прилагательные, числительные, местоимения и глаголы. Это слова самостоятельные, обозначающие предметы, признаки, количество, действия и т.д.

    {\mock{Имя существительное}} бывает \emph{трех родов}:

    \bigskip\autorows[-1pt]{l}{4}{l}{
        \hspca{мужского:}, {\slv{человѣ́къ}}, {\slv{ѻ҆лта́рь}}, {\slv{хра́мъ}},
        \hspca{женского:}, {\slv{жена̀}}, {\slv{це́рковь}}, {\slv{си́ла}},
        \hspca{среднего:}, {\slv{вре́мѧ}}, {\slv{село̀}}, {\slv{благоволе́нїе}}
    }

    {\mock{Имя прилагательное}} и некоторые имена числительные, а также некоторые местоимения в свою очередь сами могут изменяться по родам. Например:

    \bigskip\autorows[-1pt]{l}{4}{l}{
        \hspca{мужской род:}, {\slv{благі́й}}, {\slv{пѧ́тый}}, {\slv{ѻ҆́нъ}},
        \hspca{женский род:}, {\slv{блага́ѧ}}, {\slv{пѧ́таѧ}}, {\slv{ѻна̀}},
        \hspca{средний род:}, {\slv{благо́е}}, {\slv{пѧ́тое}}, {\slv{ѻ҆но̀}}
    }

    Эти четыре части речи изменяются также по числам и падежам.

    \emph{Чисел} в церковнославянском языке три: единственное, двойственное, множественное.
    
    \emph{Падежей} в церковнославянском языке семь: именительный, родительный, дательный, винительный, звательный, творительный, предложный.
    
    Звательный падеж служит обращением к лицу или предмету и во множественном числе почти всегда сходен с именительным (имена числительные и местоимения звательного падежа не имеют).
    
    В двойственном числе во всех именах (кроме некоторых числительных и местоимений) сходны между собою следующие падежи:
    
    \bigskip\begin{adjustwidth}{\hstbb}{0cm}
        \renewcommand*{\arraystretch}{1.0}
        \begin{tabular}[l]{ll}
            
            1) & именительный, винительный, звательный\\
            2) & родительный, предложный\\
            3) & дательный, творительный\\
            
        \end{tabular}
    \end{adjustwidth}

    \medskip
    Изменение слов по падежам называется \emph{склонением}.
    
    Все четыре перечисленные части речи называются \emph{склоняемыми}.
    
    {\mock{Глагол}} изменяется по временам, лицам и числам.
    
    Действия или состояния предметов, выражаемые глаголами, совершаются \emph{во времени}. Они могут происходить в настоящем времени, прошедшем времени и будущем времени.
    
    Глагол имеет \emph{три лица}: 1-е, 2-е и 3-е.
    
    \emph{Чисел} церковнославянский глагол имеет три: единственное, двойственное, множественное.
    
    Двойственное число глагола имеет особенность, заключающуюся в том, что это число изменяется по родам: у женского и среднего рода окончания сходны, но различаются от окончаний мужского рода; кроме того, 2-е и 3-е лица в каждом роде одинаковы по окончаниям.
    
    Изменения глагола по временам, лицам и числам, в двойственном числе и по родам называется \emph{спряжением}.
    
    Глагол имеет еще особые формы, о которых будет сказано впоследствии.
    
    К \textbf{неизменным частям речи} относятся наречия, предлоги, союзы, частицы и междометия. Из них только наречия являются самостоятельными словами, а предлоги, союзы и частицы представляют собою так называемые \emph{служебные слова}, которые придают самостоятельным словам надлежащие взаимные соотношения или различные оттенки смысла. Что же касается междометий, то они занимают совершенно отдельное место, не являются сами по себе ни самостоятельными, ни служебными словами, а выражают только различные чувства и переживания.

                \subsubsection{Понятие о предложении}

    Мысль, выраженная словами, называется \textbf{предложением}.
    
    Слова, входящие в состав предложения, бывают связаны между собою известными сочетаниями. Возьмем, например, такое предложение:
    
    \bigskip\autorows{l}{1}{l}{
        \hspca{{\slv{Небеса̀ повѣ́даютъ сла́вꙋ бж҃їю}} (Пс. 18, 2)}
    }
    
    Здесь слово {\slv{небеса̀}} связано (или, как говорят, согласовано) со словом {\slv{повѣ́даютъ}}, а слово {\slv{сла́вꙋ}}~---~со словом {\slv{бж҃їю}}.
    
    Каждое слово в предложении (если это слово не является служебным словом или междометием) отвечает на какой-то вопрос; так, в приведенном предложении
    
    \medskip\begin{adjustwidth}{\hstbb}{0cm}
        \renewcommand*{\arraystretch}{1.2}
        \begin{tabular}[l]{llcl}
            
            слово
            & {\slv{небеса̀}}
            & отвечает на вопрос
            & \emph{что?} (имен. падеж)
            \\
            
            & {\slv{повѣ́даютъ}}
            & --
            & \emph{что делают?}
            \\
            
            & {\slv{сла́вꙋ}}
            & --
            & \emph{что?} (винит. падеж)
            \\
            
            & {\slv{бж҃їю}}
            & --
            & \emph{чью?}
            \\
            
        \end{tabular}
    \end{adjustwidth}

    \medskip
    Самостоятельные слова, входящие в состав предложения и отвечающие на какой-нибудь вопрос, называются \textbf{членами предложения}.
    
    В каждом предложении говорится о ком- или о чем-либо.
    
    То, о чем говорится в предложении, называется \emph{подлежащим}. Подлежащее всегда отвечает на вопрос именительного падежа: {\large кто}? или {\large что}? В данном предложении слово {\slv{небеса́}} является подлежащим.
    
    То, что говорится о подлежащем, называется \emph{сказуемым}. Сказуемое отвечает на вопрос: {\large что делает подлежащее}? или {\large что с ним делается}? или {\large что оно такое}? В приведенном предложении сказуемым является слово {\slv{повѣ́даютъ}}.
    
    Подлежащее и сказуемое могут иметь при себе объяснительные слова. В предложении слова {\slv{сла́вꙋ бж҃їю}} являются объяснительными словами.
    
    Подлежащее и сказуемое называются \emph{главными членами}, а объяснительные слова~---~\emph{второстепенными членами}.
    
    Из второстепенных членов различают дополнения, определения и обстоятельства.
    
    \emph{Дополнением} в предложении называется слово, относящееся к сказуемому и отвечающее на вопросы косвенных падежей, т.е. всех падежей, кроме именительного и звательного. В том же предложении дополнением служит слово {\slv{сла́вꙋ}} ({\large что}?~---~винит. падеж).
    
    \emph{Определение} может отвечать на вопросы: {\large какой}? {\large чей}? {\large который}? {\large сколько}? В данном предложении определением служит слово {\slv{бж҃їю}} ({\large чью}? {\large какую}?).
    
    \emph{Обстоятельства} выражают различные условия действия или состояния подлежащего. Они могут обозначать место действия, время, образ, цель и причину действия, отвечая на соответствующие вопросы.

                \subsubsection{Пунктуация}

    \emph{Пунктуацией} называется правило расстановки знаков препинания между словами в предложении. В церковнославянской письменности знаки препинания следующие: точка, запятая, двоеточие, знак вопроса, из которых точка и запятая ставятся в тех же случаях, как и в русском языке.
    
    Двоеточие ({\slv{:}}) употребляется перед собственной речью, или когда последующее предложение объясняет предыдущее. Но в церковнославянском тексте двоеточие употребляется и в том случае, когда по-русски должна стоять точка с запятой, например:
    
    \bigskip\autorows{l}{1}{l}{
        \hspca{\slv{Воскли́кнемъ бг҃ꙋ сп҃си́телю на́шемꙋ: предвари́мъ лицѐ є҆гѡ̀}},
        \hspca{{\slv{во и҆сповѣ́данїи}} (Пс. 94, 1--2)}
    }

    Кроме того, двоеточие употребляется вместо русского многоточия,  когда указывается только начало церковного чтения или песнопения, например:
    
    \begin{quote}
        {\slv{Сла́ва, и҆ ны́нѣ: Вѣ́рꙋю во є҆ди́наго бг҃а ѻ҆ц҃а̀: ({\footnotesize листъ} г҃). трис҃тое. Прес҃та́ѧ трⷪ҇це: По ѻ҆́ч҃е на́шъ: {\footnotesize свѧще́нникъ:} Ꙗ҆́кѡ твоѐ є҆́сть црⷭ҇тво:}}
    \end{quote}
    \pagebreak
    
    Для изображения знака вопрос в церковнославянском тексте употребляется знак <<точка с запятой>> ({\slv{;}}). Например:

    \bigskip\autorows{l}{1}{l}{
        \hspca{\slv{А҆́зъ тре́бꙋю тобо́ю крⷭ҇ти́тисѧ, и҆ ты́ ли}},
        \hspca{{\slv{грѧде́ши ко мнѣ̀;}} (Мф. 3, 14)}
    }

    Иногда знак вопроса в церковнославянском тексте обозначает знак восклицания ({\slv{!}}). Например:

    \bigskip\autorows{l}{1}{l}{
        \hspca{{\slv{Гдⷭ҇и, что́ сѧ ᲂу҆мно́жиша стꙋжа́ющїи мѝ;}} (Пс. 3, 2)},
        \hspca{(Господи, как умножились враждующие против меня!)}
    }

        \section{Глагол}
                \subsubsection{Понятие о глаголе}

    Глагол есть самая основная часть речи во всех языках. В предложении глагол фигурирует почти всегда в качестве главного члена предложения~---~сказуемого. Поэтому-то, прежде изучения других самостоятельных частей речи церковнославянского языка, рекомендуется предварительно изучить простейшие церковнославянские глагольные формы.
    
    \textbf{Глаголом} называется часть речи, выражающая действие или состояние предмета или явления. Например:

    \bigskip\autorows{l}{1}{l}{
        \hspca{{\slv{И̑щи́те и҆ ѡбрѧ́щете}} (Мф. 7, 7)},
        \hspca{{\slv{А҆́зъ ᲂуснꙋ́хъ и҆ спа́хъ, воста́хъ}} (Пс. 3, 6)}
    }


                \subsubsection{Глаголы архаические}

    Глаголы подчиняются известным законам образования и спряжения. Но в церковнославянском языке существует несколько глаголов, которые не подчиняются общим законам образования и спряжения. У этих глаголов сохранились еще древние первоначальные формы славянского языка, а потому такие глаголы получили название \emph{архаических}, т.е. древних, старинных. К таким глаголам относятся следующие пять: {\slv{бы́ти, вѣ́дѣти, ꙗ҆́сти, да́ти, и҆мѣ́ти}}.
    
    Из этих архаических глаголов глагол {\slv{бы́ти}} имеет особо важное значение. В силу такого исключительного значения архаического глагола {\slv{бы́ти}} его принято называть \emph{вспомогательным глаголом}.
    \pagebreak
    
                \subsubsection{Неопределенная форма глаголов}

    У всех глаголов есть такая форма, которая не спрягается и служит лишь указанием действия или состояния. Например: {\slv{писа́ти, хвали́ти, нестѝ}}.
    
    Такая глагольная форма называется \emph{неопределенной формой} глагола и считается \emph{началом глагола}.
    
    Неопределенная форма глагола всегда имеет окончание {\slv{-ти}}. Если иногда неопределенная форма глагола и имеет окончание {\slv{-щи}} ({\slv{мощѝ, рещѝ}}), то такое окончание не представляет собою никакого исключения: это то же окончание {\slv{-ти}}, но только с предшествующим гортанным {\slv{-г-}} или {\slv{-к-}} ({\slv{мог-тѝ, рек-тѝ}}), а эти гортанные ({\slv{-г-}} и {\slv{-к-}}), соединяясь с {\slv{-т-}}, переходят, по закону смягчения, вместе с ним в {\slv{-щ-}}.
    
    Таким образом, вместо {\slv{могтѝ}} получилось {\slv{мощѝ}} ({\slv{г}} + {\slv{т}} = {\slv{щ}}), а вместо {\slv{ректѝ}}~---~{\slv{рещѝ}} ({\slv{к}} + {\slv{т}} = {\slv{щ}}).

                \subsubsection{Спряжение в настоящем времени вспомогательного глагола {\slv{бы́ти}}}

    Глагол в настоящем времени служит для выражения действия или состояния в данный момент. Например:

    \bigskip\autorows{l}{1}{l}{
        \hspca{{\slv{Гдⷭ҇ь пасе́тъ мѧ̀}} (Пс. 22, 1)}
    }

    Важное значение глагола {\slv{бы́ти}} среди других глаголов, как глагола вспомогательного, заставляет изучать его спряжение прежде спряжения других глаголов.

    \medskip\begin{placedtabular}[%
%        caption={\tabcaptsize Спряжение глагола {\slv{бы́ти}} в настоящем времени}
        ]{|c|c|c|c|c|}
        \hline

        \multirow{2}{*}{\spheading[2em]{\scriptsize{Лицо}}}
        & \multirow{2}{*}{\mkcella{Единственное\\число}}
        & \multicolumn{2}{c|}{\mkcella{Двойственное число}}
        & \multirow{2}{*}{\mkcella{Множественное\\число}}
        \\
        
        \cline{3-4}
        
        &
        & \mkcella{Мужской род}
        & \mkcella{Жен. и сред. род}
        &
        \\
        
        \hline
        
        1
        & \makecell{{\footnotesize\slv{а҆́зъ}} {\slv{є҆́смь}}}
        & \makecell{{\footnotesize\slv{мы̀}} {\slv{є҆сва́}}\\({\scriptsize или} {\slv{є҆сма̀}})}
        & \makecell{{\footnotesize\slv{мы̀}} {\slv{є҆свѣ̀}}}
        & \makecell{{\footnotesize\slv{мы̀}} {\slv{є҆смы̀}}}
        \\\hline
        
        2
        & \makecell{{\footnotesize\slv{ты̀}} {\slv{є҆сѝ}}}
        & \makecell{{\footnotesize\slv{вы̀}} {\slv{є҆ста̀}}}
        & \makecell{{\footnotesize\slv{вы̀}} {\slv{є҆стѣ̀}}}
        & \makecell{{\footnotesize\slv{вы̀}} {\slv{є҆стѐ}}}
        \\\hline
        
        3
        & \makecell{{\footnotesize\slv{ѻ҆́нъ, ѻ҆на̀, ѻ҆но̀}}\\{\slv{є҆́сть}}}
        & \makecell{{\footnotesize\slv{ѻ҆́на}} {\slv{є҆ста̀}}}
        & \makecell{{\footnotesize\slv{ѻ҆́нѣ}} {\slv{є҆стѣ̀}}}
        & \makecell{{\footnotesize\slv{ѻ҆нѝ, ѻ҆нѣ̀, ѻ҆нѝ}}\\{\slv{сꙋ́ть}}}
        \\\hline
    
    \end{placedtabular}

    \bigskip
    Если при глаголе {\slv{бы́ти}} находится отрицание {\slv{не}}, то в настоящем времени {\slv{не}} и {\slv{є҆́смь}} сливаются. Происходит так называемое \emph{стяжание} двух звуков {\slv{е}}. При стяжании двойной звук {\slv{ее}} переходит в {\slv{ѣ}}, и образуется слово {\slv{нѣсмь}}. В таком положении глагол {\slv{бы́ти}} с отрицанием {\slv{не}} и спрягается в настоящем времени во всех лицах и числах: {\slv{нѣ́смь, нѣ́си, нѣ́сть}}\ldots~{\slv{нѣ́смы, нѣ́сте}}\ldots~Исключение из этого правила составляет только 3-е лицо множественного числа, где отрицание {\slv{не}} пишется раздельно: {\slv{не сꙋ́ть}}.
    \pagebreak
    
                    \paragraph{\exercise}

    Поставьте приведенные в скобках слова в соответствующую форму.
    
    \medskip\begin{adjustwidth}{\hstbb}{0cm}
        \renewcommand*{\arraystretch}{1.2}
        \begin{tabular}[l]{rl}
            
            \emph{Образец}:
            & \makecell[l]{В начале ты, Господи, основал землю,\\и небеса _____ ({\slv{бы́ти}}) дело рук Твоих (Пс. 101, 26).}
            \\
            
            &
            \\
            
            \exanswer
            & \makecell[l]{В начале ты, Господи, основал землю,\\и небеса {\slv{сꙋ́ть}} дело рук Твоих.}
            \\
            
        \end{tabular}
    \end{adjustwidth}

    \medskip
    Правильность ответа проверьте в церковнославянском тексте Нового Завета.
    
    1. Петр же поднял его, говоря: встань; я тоже _____ ({\slv{бы́ти}}) человек (Деян. 10, 26).
    
    2. Грех не должен над вами господствовать, ибо вы _____ ({\slv{бы́ти}}) не под законом, но под благодатью (Рим. 6, 14).
    
    3. Но злой дух сказал в ответ: Иисуса знаю, и павел мне известен, а вы кто _____ ({\slv{бы́ти}})? (Деян. 19, 15).
    
    4. Итак, смотрите, поступайте осторожно, не как неразумные, потому что дни лукавы _____ ({\slv{бы́ти}}) (Еф. 5, 15).
    
    5. Ибо теперь мы живы _____ ({\slv{бы́ти}}), когда вы стоите в Господе (1 Фес. 3, 8).
    
    6. Итак, неизвинителен ты _____ ({\slv{бы́ти}}), всякий человек (Рим. 2, 1).
    
    7. Потому что немудрое Божие премудрее человеков _____ ({\slv{бы́ти}}), и немощное Божие сильнее человеков _____ ({\slv{бы́ти}}) (1 Кор. 1, 25).
    
    8. Что же? станем ли грешить, потому что мы _____ ({\slv{бы́ти}}) не под законом, а под благодатью? (Рим. 6, 15).
    
    9. Ибо нет _____ ({\slv{бы́ти}}) лицеприятия у Бога (Рим. 2, 11).
    
    10. Ибо все вы _____ ({\slv{бы́ти}}) сыны света и сыны дня: мы не _____ ({\slv{бы́ти}}) сыны ночи, ни тьмы (1 Фес. 5, 5).
    \pagebreak
    
                \subsubsection{Спряжение в настоящем времени прочих глаголов}

    Для образца спряжения прочих глаголов в настоящем времени возьмем глаголы {\slv{нестѝ}} и {\slv{хвали́ти}}.
    
    \medskip\begin{placedtabular}[%
%        caption={\tabcaptsize Спряжение в настоящем времени глаголов\protect\\{\slv{нестѝ}} и {\slv{хвали́ти}}}
        ]{|c|c|c|}
        \hline
        
        \footnotesize\makecell{Лицо}
        & \footnotesize\makecell{Единственное число}
        & \footnotesize\makecell{Множественное число}
        \\\hline
        
        \makecell{1}
        & \makecell{{\slv{несꙋ̀, хвалю̀}}}
        & \makecell{{\slv{несе́мъ, хва́лимъ}}}
        \\\hln
        
        \makecell{2}
        & \makecell{{\slv{несе́ши, хва́лиши}}}
        & \makecell{{\slv{несе́те, хва́лите}}}
        \\\hln
        
                    
        \makecell{3}
        & \makecell{{\slv{несе́тъ, хва́литъ}}}
        & \makecell{{\slv{несꙋ́тъ, хва́лѧтъ}}}
        \\\hline

        
        \multicolumn{3}{|c|}{\footnotesize\makecell{Двойственное число}}
        \\\hline

        & \footnotesize\makecell{Мужской род}
        & \footnotesize\makecell{Женский и средний род}
        \\\hline
        
        \makecell{1}
        & \makecell{{\slv{несе́ва, хва́лива}}}
        & \makecell{{\slv{несе́вѣ, хва́ливѣ}}}
        \\\hline
        
        \makecell{2}
        & \multirow{2}{*}{{\slv{несе́та, хва́лита}}}
        & \multirow{2}{*}{{\slv{несе́тѣ, хва́литѣ}}}
        \\\cline{1-1}
        
        \makecell{3}
        &
        &
        \\\hline

    \end{placedtabular}

    \bigskip
    Необходимо заметить:
    
    1. Во 2-м и 3-м лице двойственного числа женского и среднего рода глаголы оканчиваются на {\slv{ѣ}}, а во 2-м лице множественного числа~---~на {\slv{е}}; нужно помнить, что первоначально в славянском языке звуки {\slv{ѣ}} и {\slv{е}} звучали не одинаково, когда же эти звуки впоследствии стали однозвучны, то буквы (но не звуки) {\slv{ѣ}} (в двойственном числе) и {\slv{е}} (во множественном числе) стали служить лишь для различия числовых форм друг от друга; такое правописание для этих чисел распространяется на спряжение глаголов и в остальных временах.
    
    Второе и третье лицо двойственного числа в настоящем времени (а также и в других временах) имеют одинаковую форму.
    
    2. Если в 1-м лице настоящего времени основа глагола оканчивается гортанным звуком {\slv{г}} или {\slv{к}}, то эти звуки перед другими личными окончаниями (кроме 3-го лица множественного числа) соответственно смягчаются в {\slv{ж}} или {\slv{ч}}. Например:
    
    \medskip\autorows{l}{1}{l}{
        \hspca{\slv{могꙋ̀~\sdash~мо́жеши, рекꙋ̀~\sdash~рече́ши}}
    }
    
    3. Если основа неопределенной формы оканчивается на юсовское {\slv{ѧ}}, то это {\slv{ѧ}} перед окончаниями настоящего времени разлагается на свои древние звуки. Например:
    
    \medskip\autorows{l}{1}{l}{
        \hspca{{\slv{клѧ́-ти~\sdash~клен-ꙋ̀, клен-е́-ши}} и т.д.},
        \hspca{{\slv{ꙗ҆́-ти~\sdash~є҆́мл-ю, є҆́мл-е-ши}} и т.д.}
    }

    В этом последнем случае губной звук {\slv{м}} смягчается плавным звуком {\slv{л}}.

                    \paragraph{\exercise}
    
    Восстановите в нижеследующих примерах правильные окончания церковнославянских глаголов. В скобках приведена неопределенная форма соответствующих глаголов. Правильность выполнения проверьте по церковнославянскому тексту, ссылки на который приводятся в конце каждого предложения.
    
    \medskip\begin{adjustwidth}{\hstbb}{0cm}
        \renewcommand*{\arraystretch}{1.2}
        \begin{tabular}[l]{rl}
            
            \emph{Образец}:
            & \makecell[l]{Когда Иисус шел оттуда, за ним следовали двое слепых\ldots\\И говорит им Иисус: {\slv{Верꙋ}}\ldots~({\slv{ве́ровати}}) ли, яко могу\\сие сотворите? (Мф. 9, 27--28).}
            \\
            
            &
            \\
            
            \exanswer
            & \makecell[l]{Когда Иисус шел оттуда, за ним следовали \textbf{двое} слепых\ldots\\И говорит им Иисус: {\slv{Ве́рꙋета}} ли, яко могу\\сие сотворите? (Мф. 9, 27--28).}
            \\
            
        \end{tabular}
    \end{adjustwidth}

    \medskip
    (Выделенное в примере слово \textbf{двое} указывает на то, что глагол стоит в двойственном числе. Подробнее о двойственном числе см. в книге: В.И. Супрун. <<Учебник старославянского языка>>, с. 63--66).
    
    1. Тогда подошли к Нему сыновья Зеведеевы \textbf{Иаков} и \textbf{Иоанн} и сказали: Учитель! {\slv{Хоще}}\ldots~({\slv{хотѣ́ти}}), чтобы Ты сделал нам, о чем {\slv{проси}}\ldots~({\slv{проси́ти}}). Он сказал им: что {\slv{хоще}}\ldots~({\slv{хотѣ́ти}}), чтобы Я сделал вам? Они сказали Ему: дай нам {\slv{да сѧде}}\ldots~({\slv{сѣдѣ́ти}}) у Тебя, одному по правую сторону, а другому по левую в славе Твоей. Но Иисус сказал им: не знаете, чего {\slv{проси}}\ldots~({\slv{про́сити}}). {\slv{Мож}}\ldots~({\slv{мощѝ}}) ли пить чашу, которую Я пью, и креститься крещением, которым Я крещусь? Они отвечали: {\slv{мож}}\ldots~({\slv{мощѝ}}). Иисус же сказал им: чашу, которую Я пью, {\slv{испї}}\ldots~({\slv{испи́ти}}) и крещением, которым Я крещусь, {\slv{крест}}\ldots~({\slv{крести́тисѧ}}) (Мк. 10, 35--39).
    
    Истинно также говорю вам, что если \textbf{двое} из вас {\slv{совѣща}}\ldots~({\slv{совѣща́ти}}) на земле просить о всяком деле, то, чего бы ни {\slv{прос}}\ldots~({\slv{проси́ти}}), будет им от Отца Моего Небесного (Мф. 18, 19).
    
    3. Господь~---~престол Его на небесах, \textbf{очи} Его {\slv{призира}}\ldots~({\slv{призира́ти}}) на нищего; \textbf{вежды} Его {\slv{испыта}}\ldots~({\slv{испыта́ти}}) сынов человеческих (Пс. 10, 4).
    
    4. \textbf{Фавор} и \textbf{Ермон} о имени Твоем {\slv{возрадꙋ}}\ldots~({\slv{возра́доватисѧ}}) (Пс. 88, 13).

                \subsubsection{Глагольные основы}

    В глаголах нужно различать две основы: основу неопределенной формы и основу настоящего времени. Эти две основы имеют очень важное значение, потому что от них образуются все глагольные формы.
    \pagebreak
    
    Основа неопределенной формы глагола~---~это та его часть, которая останется, если отбросить окончание {\slv{-ти}}. Так, например, в глаголах:
    
    \medskip\begin{adjustwidth}{\hstbb}{0cm}
        \begin{tabular}[l]{l|c|l}
            
            {\slv{бра́-ти}} & основа & {\slv{бра-}} \\
            {\slv{зва́-ти}} &        & {\slv{зва-}} \\
            
        \end{tabular}
    \end{adjustwidth}

    \medskip
    Основа настоящего времени глагола~---~это та его часть, которая останется, если отбросить личное окончание 1-го лица единственного числа. Например, в этих же глаголах:
    
    \medskip\begin{adjustwidth}{\hstbb}{0cm}
        \begin{tabular}[l]{l|c|l}
            
            {\slv{бер-ꙋ̀}} & основа & {\slv{бер-}} \\
            {\slv{зов-ꙋ̀}} &        & {\slv{зов-}} \\
            
        \end{tabular}
    \end{adjustwidth}

    \medskip
    Но не всегда основа неопределенной формы глагола отличается от основы настоящего времени; нередко обе основы совпадают, а иногда они совпадают даже с корнем глагола. Например, в глаголах:
    
    \bigskip\autorows[-1pt]{l}{2}{l}{
        \hspca{\slv{дѣ́ла-ти}}, {обе основы},
        \hspca{\slv{дѣ́ла-ю}}, {\slv{дѣла-}},
        \hspca{\slv{нес-ти́}}, {обе основы совпадают},
        \hspca{\slv{нес-ꙋ̀}}, {с корнем {\slv{нес-}}}
    }

                \subsubsection{Глаголы тематические и разделение их на два спряжения}

    По основе настоящего времени различают глаголы: тематические и нетематические (архаические).
    
    \textbf{Тематические} глаголы~---~это те, в которых между основой настоящего времении личным окончанием (кроме 1-го лица единственного числа и 3-го лица множественного числа) находится гласный звук (\emph{тема}) {\slv{-е-}} (например, {\slv{зов-е́-ши}}) или {\slv{-и-}} (например, {\slv{хва́л-и-ши}}).
    
    \textbf{Нетематические} же (архаические) глаголы присоединяют личные окончания непосредственно к корню.
    
    Тематические глаголы разделяются на два \textbf{спряжения}.
    
    К \textbf{первому спряжению} относятся те глаголы, которые в настоящем времени принимают личные окончания с помощью тематической гласной {\slv{е}}.
    
    Ко \textbf{второму спряжению} относятся те глаголы, которые в настоящем времени принимают личные окончания с помощью тематической гласной {\slv{и}}.
    
    Первое спряжение тематических глаголов характеризуется (за очень малыми исключениями) еще тем, что в 3-м лице множественного числа настоящего времени они имеют окончание {\slv{-ꙋтъ}} или {\slv{-ютъ}} ({\slv{нес-ꙋ́тъ, повелѣва́-ютъ}}), а глаголы второго спряжения в том же числе того же времени имеют окончание {\slv{-атъ}} или {\slv{-ѧтъ}} ({\slv{де́рж-атъ, хва́л-ѧтъ}}).

                \subsubsection{Преходящее время вспомогательного глагола {\slv{бы́ти}}}

    \textbf{Преходящим временем} называется такое прошедшее время глагола, которое выражает действие, долго продолжавшееся или несколько раз повторявшееся. Например:
    
    \bigskip\autorows{l}{1}{l}{
        \hspca{{\slv{Во́ды мно́ги бѧ́хꙋ тꙋ̀}} (Ин. 3, 23)}
    }

    Рассмотрим преходящее время вспомогательного глагола {\slv{бы́ти}}.


    \begin{center}
%        {\tabcaptsize Преходящее время вспомогательного глагола {\slv{бы́ти}}}
        \renewcommand*{\arraystretch}{1.2}
        \begin{tabular}[c]{|c|c|c|c|c|}
            \hline

            \multirow{2}{*}{\spheading[2em]{\scriptsize{Лицо}}}
            & \multirow{2}{*}{\mkcella{Единственное\\число}}
            & \multicolumn{2}{c|}{\mkcella{Двойственное число}}
            & \multirow{2}{*}{\mkcella{Множественное\\число}}
            \\
            
            \cline{3-4}
            
            &
            & \mkcella{Мужской род}
            & \mkcella{Жен. и сред. род}
            &
            \\
            
            \hline
            
            1
            & \makecell{\slv{бѧ́хъ, бѣ́хъ}}
            & \makecell{\slv{бѧ́хова, бѣ́хова}}
            & \makecell{\slv{бѧ́ховѣ, бѣ́ховѣ}}
            & \makecell{\slv{бѧ́хомъ, бѣ́хомъ}}
            \\\hline
            
            2
            & \makecell{\slv{бѣ̀}}
            & \multirow{2}{*}{{\slv{бѧ́ста, бѣ́ста}}}
            & \multirow{2}{*}{{\slv{бѧ́стѣ, бѣ́стѣ}}}
            & \makecell{\slv{бѧ́сте, бѣ́сте}}
            \\\cline{1-2}\cline{5-5}
            
            3
            & \makecell{\slv{бѧ́ше, бѣ̀}}
            &
            &
            & \makecell{\slv{бѧ́хꙋ, бѣ́хꙋ, бѣ́ша}}
            \\\hline
            
        \end{tabular}
    \end{center}

    \textbf{Примечание}. Преходящее время имеет в учебной литературе по церковнославянскому языку несколько различных наименований. Его называют также прошедшим продолжительным, прошедшим несовершенным, прошедшим многократным временем или \emph{имперфектом}.

    Оно используется преимущественно для всякого рода описаний. В этом времени написаны, например, описание земли в начале сотворения мира (Быт. 1, 2) и описание Овчей купели в Евангелии от Иоанна (Ин. 5, 2--4). Более понятной его функция станет ниже, когда мы познакомимся с \emph{аористом}.
    
                    \paragraph{\exercise}

    Поставьте приведенные в скобках церковнославянские слова в соответствующую форму преходящего времени. Правильность решения можно проверить, обратившись к церковнославянскому тексту. Ответ считается правильным, если приведена одна из форм, перечисленных в таблице выше. Например, если в тексте написано {\slv{бѧ́хомъ}}, а обучающийся написал {\slv{бѣ́хомъ}}, ответ считается правильным, так как в таблице это две равноправные формы 1-го лица множественного числа.
    
    \medskip\begin{adjustwidth}{\hstbb}{0cm}
        \renewcommand*{\arraystretch}{1.2}
        \begin{tabular}[l]{rl}
            
            \emph{Образец}:
            & \makecell[l]{В Антиохии, в тамошней церкви _____ ({\slv{бы́ти}}) некоторые\\пророки и учители (Деян. 13, 1).}
            \\
            
            &
            \\
            
            \exanswer
            & \makecell[l]{В Антиохии, в тамошней церкви {\slv{бѧ́хꙋ}} некоторые пророки\\и учители.}
            \\
            
        \end{tabular}
    \end{adjustwidth}

    \medskip
    1. Ибо вы _____ ({\slv{бы́ти}}), как овцы блуждающие (не имея пастыря), но возвратились ныне к Пастырю и Блюстителю душ ваших (1 Пет. 2, 25).
    
    2. И вот, в тот самый час три человека стали перед домом, в котором я _____ ({\slv{бы́ти}}), посланные из Кесарии ко мне (Деян. 11, 11).
    
    3. Скиния свидетельства _____ ({\slv{бы́ти}}) у отцов наших в пустыне, как повелел Говоривший Моисею сделать ее по образцу, им виденному (Деян. 4, 44).
    
    4. А до пришествия веры мы заключены _____ ({\slv{бы́ти}}) под стражею закона, до того времени, как надлежало открыться вере (Гал. 3, 23).
    
    5. \ldots и видит отверстое небо и сходящий к нему некоторый сосуд, как бы большое полотно, привязанное за четыре угла и опускаемое на землю; в нем _____ ({\slv{бы́ти}}) всякие четвероногие земные, звери, пресмыкающиеся и птицы небесные (Деян. 10, 11--12).
    
    6. Ибо, когда вы были рабами греха, тогда _____ ({\slv{бы́ти}}) свободны от праведности (Рим. 6, 20).
    
    7. А за что убил его? За то, что дела его _____ ({\slv{бы́ти}}) злы, а дела брата его праведны (1 Ин. 3, 12).
    
    8. Церквам Христовым в Иудее лично я не _____ ({\slv{бы́ти}}) известен (Гал. 1, 22).
    
    9. У множества же уверовавших _____ ({\slv{бы́ти}}) одно сердце и одна душа; и никто ничего из имения своего не называл своим, но все у них было общее (Деян. 4, 32).

                \subsubsection{Аорист вспомогательного глагола {\slv{бы́ти}}}

    \textbf{Аористом} называется такое прошедшее время глагола, которое характеризует действие вполне законченное. Например:
    
    
    \bigskip\autorows{l}{1}{l}{
        \hspca{{\slv{То́й речѐ и҆}} {\slv{\large бы́ша}} (Пс. 32, 9)}
    }

    \textbf{Примечание}: \emph{Аорист} не имеет прямого аналога в русском языке. Его называют также прошедшим совершенным, прошедшим однократным временем. Однако эти названия отражают его сущность лишь отчасти. Наиболее точно будет сказать, что аорист используется для \textbf{ведения рассказа}, а \emph{имперфект} (преходящее) употребляется для обозначения сопутствующих и производных от основных фактов (см. Иеромонах Алипий. Грамматика церковно-славянского языка М., 1991, с. 200--204):
    
    Царь Ирод, \textbf{услышав} об Иисусе,~---~ибо имя Его стало гласно,~---~\emph{говорил}: это Иоанн Креститель воскрес из мертвых, и потому чудеса делаются им. Другие \emph{говорили}: это Илия, а иные \emph{говорили}: это пророк, или как один из пророков. Ирод же, услышав, \textbf{сказал}: это Иоанн, которого я обезглавил; он воскрес из мертвых. Ибо сей Ирод, послав, \textbf{взял} Иоанна и \textbf{заключил} его в темницу за Иродиаду, жену Филиппа, брата своего, потому что \textbf{женился} на ней. Ибо Иоанн \emph{говорил} Ироду: не должно тебе иметь жену брата твоего. Иродиада же \emph{злобясь} на него, \emph{желала} убить его; но не \emph{могла}. Ибо Ирод \emph{боялся} Иоанна, зная, что он муж праведный и святой, и \emph{берег} его; многое \emph{делал}, слушаясь его, и с удовольствием \emph{слушал} его (Мк. 6, 14--20).
    \pagebreak
    
    Выделенные в тексте жирным шрифтом формы переданы в церковнославянском тексте аористом и образуют как бы \textbf{костяк} действия:

    \bigskip\autorows{l}{1}{l}{
        \hspca{услышал~---~сказал~---~взял~---~заключил~---~женился.}
    }

    Последний факт является как бы напоминанием о том, что было раньше; формы, выделенные курсивом, образуют некий \textbf{фон} указанного действия.
    
    Рассмотрим аорист вспомогательного глагола {\slv{бы́ти}}.

    \medskip\begin{placedtabular}[%
%        caption={\tabcaptsize Аорист вспомогательного глагола {\slv{бы́ти}}}
        ]{|c|c|c|c|c|}
        \hline
        
        \multirow{2}{*}{\spheading[2em]{\scriptsize{Лицо}}}
        & \multirow{2}{*}{\mkcella{Единственное\\число}}
        & \multicolumn{2}{c|}{\mkcella{Двойственное число}}
        & \multirow{2}{*}{\mkcella{Множественное\\число}}
        \\
        
        \cline{3-4}
        
        &
        & \mkcella{Мужской род}
        & \mkcella{Жен. и сред. род}
        &
        \\
        
        \hline
        
        1
        & \makecell{\slv{бы́хъ}}
        & \makecell{\slv{бы́хова, бы́сва}}
        & \makecell{\slv{бы́ховѣ, бы́свѣ}}
        & \makecell{\slv{бы́хомъ}}
        \\\hline
        
        2
        & \makecell{\slv{бы̀}}
        & \multirow{2}{*}{{\slv{бы́ста}}}
        & \multirow{2}{*}{{\slv{бы́стѣ}}}
        & \makecell{\slv{бы́сте}}
        \\\cline{1-2}\cline{5-5}
        
        3
        & \makecell{\slv{бы́сть, бы̀}}
        &
        &
        & \makecell{\slv{бы́ша}}
        \\\hline
        
    \end{placedtabular}

    \bigskip
    При переводе некоторых славянских предложений на русский язык, где фигурирует глагол {\slv{бы́ти}} в аористе 3-го лица единственного числа в своей форме {\slv{бы́сть}}, эта форма иногда переводится русским безличным глаголом \emph{стало} или \emph{случилось}. Например:

    \bigskip\autorows{l}{1}{l}{
        \hspca{{\slv{Бы́сть же во дни̑ ты̑ѧ}} (Лк. 2, 1) (Случилось же в те дни)}
    }

                    \paragraph{\exercise}

    Поставьте приведенные в скобках церковнославянские слова в соответствующую форму аориста.
    
    \medskip\begin{adjustwidth}{\hstbb}{0cm}
        \renewcommand*{\arraystretch}{1.2}
        \begin{tabular}[l]{rl}
            
            \emph{Образец}:
            & \makecell[l]{И угодно _____ ({\slv{бы́ти}}) это предложение всему собранию\\(Деян. 6, 5).}
            \\
            
            &
            \\
            
            \exanswer
            & \makecell[l]{И угодно {\slv{бы́сть}} это предложение всему собранию.}
            \\
            
        \end{tabular}
    \end{adjustwidth}

    \medskip
    1. И _____ ({\slv{бы́ти}}) я у вас в немощи и в страхе и в великом трепете \mbox{(1 Кор. 2, 3)}.

    2. Иосиф открылся братьям своим, и известен _____ ({\slv{бы́ти}}) фараону род Иосифов (Деян. 7, 13).

    3. Как и вы некогда были непослушны Богу, а ныне помилованы _____ (\mbox{\slv{бы́ти}}) (Рим. 11, 30).

    4. Иаков перешел в Египет, и скончался сам и отцы наши; и перенесены \mbox{_____} (\mbox{\slv{бы́ти}}) в Сихем и положены во гробе (Деян. 7, 15--16).

    5. Злословят нас, мы благословляем; гонят нас, мы терпим; хулят нас, мы молим; мы _____ ({\slv{бы́ти}}) как сор для мира, как прах, всеми попираемый доныне (1 Кор. 4, 12--13).

    6. Для иудеев я был как иудей, чтобы приобрести иудеев; для подзаконных _____ ({\slv{бы́ти}}) как подзаконный, чтобы приобрести подзаконных (1Кор. 9, 20).

    7. Ибо если, избегнув скверн мира чрез познание Господа и Спасителя нашего Иисуса Христа, опять запутываются в них и побеждаются ими, то последнее _____ ({\slv{бы́ти}}) для таковых хуже первого (2 Пет. 2, 20).

    8. Они убили предвозвестивших пришествие Праведника, Которого предателями и убийцами _____ ({\slv{бы́ти}}) ныне вы (Деян. 7, 52).

    9. Ныне радуюсь в страданиях моих за вас и восполняю недостаток в плоти моей скорбей Христовых за Тело Его, которое есть Церковь, которой _____ ({\slv{бы́ти}}) я служителем (Кол. 1, 24--25).

    \emph{Пояснение к упражнению 10}. Два предыдущих упражнения наглядно показывают различие между преходящим временем и аористом. Если в упражнении 9 значения глагола {\slv{бы́ти}} практически совпадают с русскими, то в упражнении 10 каждый пример приобретает дополнительное значение, непонятное на основе только русского текста. Приведем списком дополнительные значения:
    
    \medskip\begin{adjustwidth}{\hstbb}{0cm}
        \renewcommand*{\arraystretch}{1.2}
        \begin{tabular}[l]{rl}
            
            \emph{Образец}:
            & \makecell[l]{({\slv{бы́ти}}) угодно, т.е. \textbf{понравиться};}
            \\
            
            & 1: я у вас \textbf{оказался} в состоянии немощи\ldots
            \\
            
            & 2: \textbf{стал} известен;
            \\

            
            & 3: вам было \textbf{предоставлено} помилование;
            \\
            
                        
            & 4: наших отцов \textbf{перенесли} в Сихем\ldots
            \\
            
                        
            & 5: мы \textbf{превратились} в сор для мира;
            \\
            
                        
            & 6: я \textbf{становился} иудеем и подзаконным;
            \\
            
                        
            & 7: \textbf{стало} хуже;
            \\
            
                        
            & 8: \textbf{стали} вы;
            \\
            
                        
            & 9: я \textbf{стал} служителем.
            \\

        \end{tabular}
    \end{adjustwidth}

    \medskip
    Суть явления здесь состоит в том, что глагол {\slv{бы́ти}}, сам по себе обозначающий \textbf{состояние}, через форму аориста приобратает значение \textbf{события}, а конкретный русский перевод (стать, превратиться и т.д.) зависит уже от контекста. После этого необходимого пояснения выполним упражнение 11, которое поможет уяснить рассмотренное различие.

                    \paragraph{\exercise}

    Поставьте приведенные в скобках церковнославянские слова в соответствующую форму преходящего времени или аориста.
    
    1. При наступлении дня Пятидесятницы все они были единодушно вместе. И внезапно _____ ({\slv{бы́ти}}) шум с неба, как бы от несущегося сильного ветра, и наполнил весь дом (Деян. 2, 1--2).
    
    2. Вместе с ним лицемерили и прочие Иудеи, так что даже Варнава _____ ({\slv{бы́ти}}) увлечен их лицемением (Гал. 2, 13).
    
    3. Когда я _____ ({\slv{бы́ти}}) младенцем, то по-младенчески говорил, по-младенчески мыслил, по-младенчески рассуждал; а как _____ ({\slv{бы́ти}}) мужем, то оставил младенческое (1 Кор. 13, 11).
    
    4. \ldots вы _____ ({\slv{бы́ти}}) в то время без Христа, отчуждены от общества Израильского, чужды заветов обетования, не имели надежды и были безбожники в мире. А теперь во Христе Иисусе вы, бывшие некогда далеко, _____ ({\slv{бы́ти}}) близки Кровию Христовою (Еф. 2, 12--13).
    
    5. В Иоппии находилась одна ученица, именем Тавифа; она _____ ({\slv{бы́ти}}) исполнена добрых дел и творила много милостынь (Деян. 9, 36).
    
    6. Они же, пригрозив, отпустили их, не находя возможности наказать их, по причине народа; потому что все прославляли Бога за происшедшее. Ибо лет более сорока _____ ({\slv{бы́ти}}) тому человеку, над которым сделалось сие чудо исцеления (Деян. 4, 21--22).
    
    7. Руками же апостолов совершались в народе многие знамения и чудеса; и все единодушно _____ ({\slv{бы́ти}}) в притворе Соломоновом (Деян. 5, 12).

                \subsubsection{Преходящее время тематических глаголов}

    Рассмотрим спряжение в преходящем времени тематических глаголов {\slv{нестѝ}} и {\slv{хвали́ти}}.

    \medskip\begin{placedtabular}[%
%        caption={\tabcaptsize Спряжение в настоящем времени тематических глаголов\protect\\{\slv{нестѝ}} и {\slv{хвали́ти}}}
        ]{|c|c|c|c|c|}
        \hline
        
        \multirow{2}{*}{\spheading[2em]{\scriptsize{Лицо}}}
        & \multirow{2}{*}{\mkcella{Единственное\\число}}
        & \multicolumn{2}{c|}{\mkcella{Двойственное число}}
        & \multirow{2}{*}{\mkcella{Множественное\\число}}
        \\
        
        \cline{3-4}
        
        &
        & \mkcella{Мужской род}
        & \mkcella{Жен. и сред. род}
        &
        \\
        
        \hline
        
        1
        & \makecell{{\slv{несѧ́хъ}}\\{\slv{хвалѧ́хъ}}}
        & \makecell{{\slv{несѧ́хова}}\\{\slv{хвалѧ́хова}}}
        & \makecell{{\slv{несѧ́ховѣ}}\\{\slv{хвалѧ́ховѣ}}}
        & \makecell{{\slv{несѧ́хомъ}}\\{\slv{хвалѧ́хомъ}}}
        \\\hline
        
        2
        & \multirow{2}{*}{\makecell{{\slv{несѧ́ше}}\\{\slv{хвалѧ́ше}}}}
        & \multirow{2}{*}{\makecell{{\slv{несѧ́ста}}\\{\slv{хвалѧ́ста}}}}
        & \multirow{2}{*}{\makecell{{\slv{несѧ́стѣ}}\\{\slv{хвалѧ́стѣ}}}}
        & \makecell{{\slv{несѧ́сте}}\\{\slv{хвалѧ́сте}}}
        \\\cline{1-1}\cline{5-5}
        
        3
        &
        &
        &
        & \makecell{{\slv{несѧ́хꙋ}}\\{\slv{хвалѧ́хꙋ}}}
        \\\hline
        
    \end{placedtabular}

    \bigskip
    Преходящее время тематических глаголов образуется от неопределенной формы глаголов следующим образом:
    
    а) если в неопределенной форме глагола перед окончанием {\slv{-ти}} находится гласный звук {\slv{а}}, то {\slv{-ти}} меняется на окончание {\slv{-хъ}}, например:

    \bigskip\autorows{l}{1}{l}{
        \hspca{\slv{велича́-ти~\sdash~велича́-хъ}}
    }

    б) во всех же остальных тематических глаголах окончание неопределенной формы {\slv{-ти}} меняется (вместе с простым суффиксом, если он имеется) на окончание {\slv{-ѧхъ}}, например:

    \bigskip\autorows{l}{1}{l}{
        \hspca{\slv{нес-тѝ~\sdash~нес-ѧ́хъ}},
        \hspca{\slv{хвал-и́-ти~\sdash~хвал-ѧ́хъ}}
    }

    Гортанные {\slv{г}} и {\slv{к}}, смягчаясь перед {\slv{-ѧхъ}} соответственно в {\slv{ж}} и {\slv{ч}}, принимают после себя {\slv{а}}, а не {\slv{ѧ}}, например:

    \bigskip\autorows{l}{1}{l}{
        \hspca{\slv{мощѝ}}({\slv{мо{\Large г}тѝ}}){\slv{~\sdash~мо{\Large ж}а́хъ}},
        \hspca{\slv{тещѝ}}({\slv{те{\Large к}тѝ}}){\slv{~\sdash~те{\Large ч}а́хъ}}
    }

    Губные согласные смягчаются вставкой после них плавного звука {\slv{л}}, например:

    \bigskip\autorows{l}{1}{l}{
        \hspca{\slv{люби́ти~\sdash~люблѧ́хъ}},
        \hspca{\slv{лови́ти~\sdash~ловлѧ́хъ}}
    }

    2-е и 3-е лицо единственного числа тематических глаголов в преходящем времени имеют одинаковую форму
    
    От глаголов {\slv{ꙗ҆́ти}} и {\slv{клѧ́ти}} преходящее время будет: {\slv{є҆́млѧхъ, кленѧ́хъ}}.

                    \paragraph{\exercise}
                    
    Поставьте приведенные в скобках церковнославянские слова в соответствующую форму преходящего времени. Русский глагол пропускается. (Русский текст приводится по переводу П. Юнгерова).
    
    \medskip\begin{adjustwidth}{\hstbb}{0cm}
        \renewcommand*{\arraystretch}{1.2}
        \begin{tabular}[l]{rl}
            
            \emph{Образец}:
            & \makecell[l]{Когда сокрушались кости мои, \textbf{поносили} ({\slv{поноша́ти}})\\меня враги мои (Пс. 41, 11).}
            \\
            
            &
            \\
            
            \exanswer
            & \makecell[l]{Когда сокрушались кости мои, {\slv{поноша́хꙋ}} меня враги мои.}
            \\
            
        \end{tabular}
    \end{adjustwidth}

    \medskip
    1. Поношения и страдания \textbf{ожидала} ({\slv{ча́ѧти}}) душа моя (Пс. 68, 21).
    
    2. Это я вспоминал и \textbf{изливал} ({\slv{излиѧ́ти}}) душу мою в себя (Пс. 41, 5).
    
    3. Против меня \textbf{шептали} ({\slv{шепта́ти}}) враги мои, на меня \textbf{замышляли} ({\slv{помышлѧ́ти}}) зло (Пс. 40, 8).
    
    4. Ты\ldots~с которым мы \textbf{ходили} ({\slv{ходи́ти}}) в дом Божий единомысленно (Пс. 54, 15).
    
    5. Освободил Он от тяжестей хребет его; руки его \textbf{работали} ({\slv{порабо́тати}}) корзиною (Пс. 80, 7).
    
    6. Когда же я \textbf{говорил} ({\slv{глаго́лати}}) с ними, они без вины враждовали (борити) со мною (Пс. 119, 7).
    
    7. Тайно клевещущего на ближнего своего~---~сего я \textbf{изгонял} ({\slv{и҆згонѧ́ти}}); со смотрящим гордо и ненасытным~---~с ним я не \textbf{ел} ({\slv{ꙗ҆́сти}})\ldots~Не \textbf{жил} ({\slv{жи́ти}}, в форме {\slv{живѧ}})\ldots~внутри моего дома поступающий гордо, говорящий неправду \textbf{не был прав} ({\slv{и҆справлѧ́ти}}) пред глазами моими. Поутру \textbf{избивал} ({\slv{и҆збива́ти}}) всех грешников земли\ldots~(Пс. 100, 5--8).
    
    8. Устами своими \textbf{благословляли} ({\slv{благословлѧ́ти}}), а сердцем своим \textbf{проклинали} ({\slv{клѧ́ти}}) (Пс. 61, 5).

                \subsubsection{Аорист тематических глаголов}

    Для образца спряжения в аористе тематических глаголов рассмотрим уже знакомые нам глаголы {\slv{нестѝ}} и {\slv{хвали́ти}}.

    \medskip\begin{placedtabular}[%
%        caption={\tabcaptsize Аорист тематических глаголов {\slv{нестѝ}} и {\slv{хвали́ти}}}
        ]{|c|c|c|c|c|}
        \hline
        
        \multirow{2}{*}{\spheading[2em]{\scriptsize{Лицо}}}
        & \multirow{2}{*}{\mkcella{Единственное\\число}}
        & \multicolumn{2}{c|}{\mkcella{Двойственное число}}
        & \multirow{2}{*}{\mkcella{Множественное\\число}}
        \\
        
        \cline{3-4}
        
        &
        & \mkcella{Мужской род}
        & \mkcella{Жен. и сред. род}
        &
        \\
        
        \hline
        
        1
        & \makecell{{\slv{несо́хъ}}\\{\slv{хвали́хъ}}}
        & \makecell{{\slv{несо́хова}}\\{\slv{хвали́хова}}}
        & \makecell{{\slv{несо́ховѣ}}\\{\slv{хвали́ховѣ}}}
        & \makecell{{\slv{несо́хомъ}}\\{\slv{хвали́хомъ}}}
        \\\hline
        
        2
        & \multirow{2}{*}{\makecell{{\slv{несѐ}}\\{\slv{хвалѝ}}}}
        & \multirow{2}{*}{\makecell{{\slv{несо́ста}}\\{\slv{хвали́ста}}}}
        & \multirow{2}{*}{\makecell{{\slv{несо́стѣ}}\\{\slv{хвали́стѣ}}}}
        & \makecell{{\slv{несо́сте}}\\{\slv{хвали́сте}}}
        \\\cline{1-1}\cline{5-5}
        
        3
        &
        &
        &
        & \makecell{{\slv{несо́ша}}\\{\slv{хвали́ша}}}
        \\\hline
        
    \end{placedtabular}

    \bigskip
    Аорист тематических глаголов образуется от неопределенной формы следующим образом:
    
    а) если основа неопределенной формы оканчивается на гласный звук, то окончание {\slv{-ти}} непосредственно меняется на окончание {\slv{-хъ}}:

    \bigskip\autorows{l}{1}{l}{
        \hspca{\slv{хвали́-ти~\sdash~хвали́-хъ}}
    }

    б) если же основа неопределенной формы оканчивается на согласный звук, то окончание аориста {\slv{-хъ}} присоединяется к этой основе посредством соединительной гласной {\slv{-о-}}:

    \bigskip\autorows{l}{1}{l}{
        \hspca{\slv{нес-тѝ~\sdash~нес-о́-хъ}}
    }

    При окончании аориста (вместе с соединительной гласной {\slv{-о-}}) на {\slv{-охъ}} 2-е и 3-е лицо единственного числа оканчивается на {\slv{-е}}, причем при основе на гортанные {\slv{г}} и {\slv{к}} последние соответственно смягчаются в {\slv{ж}} и {\slv{ч}}. Например:

    \bigskip\autorows{l}{1}{l}{
        \hspca{\slv{нес-о́хъ~\sdash~нес-ѐ}},
        \hspca{\slv{поверг-о́хъ~\sdash~пове́рж-е}},
        \hspca{\slv{тек-о́хъ~\sdash~теч-ѐ}},
        \hspca{\slv{мог-о́хъ~\sdash~мо́ж-е}}
    }
    
    В остальных же глаголах 2-е и 3-е лицо единственного числа оканчивается на конечный гласный звук основы. Например:

    \bigskip\autorows{l}{1}{l}{
        \hspca{\slv{глаго́ла-хъ~\sdash~глаго́ла}},
        \hspca{\slv{хвали́-хъ~\sdash~хвалѝ}}
    }

    Глагол {\slv{ꙗ҆́ти}} в 1-м лице единственного числа аориста имеет форму {\slv{ꙗхъ}}, а во 2-м и 3-м лице к корню {\slv{ꙗ-}} присоединяется окончание {\slv{-тъ}}: {\slv{ꙗ҆́тъ}}.
    \pagebreak
    
    Если этот глагол употребляется с приставками {\slv{при-}}, {\slv{под-}} и т.п., то 2-е лицо единственного числа оканчивается на конечный гласный звук основы (т.е. на корень {\slv{-ꙗ-}}), а в 3-м лице присоединяется к основе окончание {\slv{-тъ}}. Например:

    \bigskip\autorows{l}{4}{l}{
        \hspca{1-е лицо:}, {\slv{прїѧ́-хъ}}, {\slv{под̾ѧ́-хъ}}, {\slv{взѧ́-хъ}},
        \hspca{2-е лицо:}, {\slv{прїѧ̀}}, {\slv{под̾ѧ}}, {\slv{взѧ̀}},
        \hspca{3-е лицо:}, {\slv{прїѧ́-тъ}}, {\slv{под̾ѧ́-тъ}}, {\slv{взѧ́-тъ}}
    }
    
    Глагол {\slv{рещѝ}} в аористе во всех числах имеет две формы:

    \medskip\begin{placedtabular}[%
%        caption={\tabcaptsize Аорист тематического глагола {\slv{рещѝ}}}
        ]{|c|c|c|c|c|}
        \hline
        
        \multirow{2}{*}{\spheading[2em]{\scriptsize{Лицо}}}
        & \multirow{2}{*}{\mkcella{Единственное\\число}}
        & \multicolumn{2}{c|}{\mkcella{Двойственное число}}
        & \multirow{2}{*}{\mkcella{Множественное\\число}}
        \\
        
        \cline{3-4}
        
        &
        & \mkcella{Мужской род}
        & \mkcella{Жен. и сред. род}
        &
        \\
        
        \hline
        
        1
        & \makecell{{\slv{реко́хъ}}\\{\slv{рѣ́хъ}}}
        & \makecell{{\slv{реко́хова}}\\{\slv{рѣ́хова}}}
        & \makecell{{\slv{реко́ховѣ}}\\{\slv{рѣ́ховѣ}}}
        & \makecell{{\slv{реко́хомъ}}\\{\slv{рѣ́хомъ}}}
        \\\hline
        
        2
        & \multirow{2}{*}{\makecell{{\slv{ре́че}}\\~}}
        & \multirow{2}{*}{\makecell{{\slv{реко́ста}}\\{\slv{рѣ́ста}}}}
        & \multirow{2}{*}{\makecell{{\slv{реко́стѣ}}\\{\slv{рѣ́стѣ}}}}
        & \makecell{{\slv{реко́сте}}\\{\slv{рѣ́сте}}}
        \\\cline{1-1}\cline{5-5}
        
        3
        &
        &
        &
        & \makecell{{\slv{реко́ша}}\\{\slv{рѣ́ша}}}
        \\\hline
        
    \end{placedtabular}

                    \bigskip\paragraph{\exercise}

Поставьте приведенные в скобках церковнославянские слова в соответствующую форму аориста. Русский глагол пропускается.
    
    \medskip\begin{adjustwidth}{\hstbb}{0cm}
        \renewcommand*{\arraystretch}{1.2}
        \begin{tabular}[l]{rl}
            
            \emph{Образец}:
            & \makecell[l]{И ныне, как только Он \textbf{возвысил} ({\slv{вознестѝ}}) главу мою\\над врагами, я \textbf{обошел} ({\slv{ѡ҆бытѝ}}) и принес в скинии\\\textbf{жертву} ({\slv{пожре́ти}}) хвалы и восклицания (Пс. 26, 6).}
            \\
            
            &
            \\
            
            \exanswer
            & \makecell[l]{И ныне, как только Он {\slv{вознесѐ}} главу мою\\над врагами, я {\slv{ѡ҆быдо́хъ}} и {\slv{пожро́хъ}} в скинии\\жертву хвалы и восклицания.}
            \\
            
        \end{tabular}
    \end{adjustwidth}

    \medskip
    1. \textbf{Услышал} ({\slv{слы́шати}}) Господь и \textbf{помиловал} ({\slv{поми́ловати}}) меня, Господь \textbf{стал} ({\slv{бы́ти}}) помощником мне (Пс. 29, 11).
    
    2. Беззаконие мое я \textbf{создал} ({\slv{позна́ти}}) и греха моего не \textbf{скрыл} ({\slv{покры́ти}}), \textbf{сказал} ({\slv{рещѝ}}): <<Исповедуюсь Господу в беззаконии моем\ldots>> (Пс. 31, 5).
    
    3. Но я на Тебя, Господи, \textbf{уповал} ({\slv{ᲂу҆пова́ти}}), сказал: Ты~---~Бог мой! (Пс. 30, 15).
    
    4. Ибо они без вины \textbf{скрыли} ({\slv{скры́ти}}) для меня пагубную сеть своего, напрасно \textbf{поносили} ({\slv{поноси́ти}}) душу мою (Пс. 34, 7).
    
    5. Ибо \textbf{окружили} ({\slv{ѡ҆бытѝ}}) меня беды, коим нет числа, \textbf{постигли} ({\slv{пости́гнꙋти}}) меня беззакония мои, так что я не \textbf{мог} ({\slv{возмощѝ}}) смотреть, \textbf{умножились} ({\slv{ᲂу҆мно́житисѧ}}) более волос на голове моей, и сердце мое \textbf{оставило} ({\slv{ѡ҆ста́вити}}) меня (Пс. 39, 13).
    
    6. Все сие \textbf{постигло} (букв.: <<постигли>>~---~глагол {\slv{прїитѝ}}) нас, но мы не забыли ({\slv{забы́ти}}) Тебя и не \textbf{нарушили} ({\slv{непра́вдовати}}) завета Твоего (Пс. 43, 18).
    
    7. \textbf{Восшел} ({\slv{взы́ти}}) Бог при восклицании, Господь при звуке трубном (Пс. 46, 6).
    
    8. И \textbf{убоялся} ({\slv{ᲂу҆боѧ́тисѧ}}) всякий человек и \textbf{возвестили} ({\slv{возвѣсти́ти}}) о делах Бога и \textbf{уразумели} ({\slv{разꙋмѣ́ти}}) действия Его (Пс. 63, 10).
    
    9. Войду в дом Твой со всесожжением, исполню пред Тобою обеты мои, которые \textbf{изрекли} ({\slv{и҆зрещѝ}}) уста моя и \textbf{произнес} (букв.: <<произнесли>>~---~{\slv{глаго́лати}}) язык мой в скорби моей (Пс. 65, 13--14).
    
    10. К нему устами моими я \textbf{воззвал} ({\slv{воззва́ти}}) и \textbf{превознес} ({\slv{вознестѝ}}) Его языком моим (Пс. 65, 17).
    
    11. И \textbf{отверг} ({\slv{ѿри́нꙋти}}) селение Иосифа и колена Ефремова не \textbf{избрал} ({\slv{и҆збра́ти}}) (Пс. 77, 67).
    
                    \paragraph{\exercise}
                    
    Поставьте приведенные в скобках церковнославянские слова в соответствующую форму преходящего времени или аориста.
    
    1. Гордые до крайности \textbf{преступали закон} ({\slv{законопрестꙋпова́ти}}), а я не \textbf{уклонялся} ({\slv{ᲂу҆клони́тисѧ}}) от закона Твоего (Пс. 118, 51).
    
    2. Воздающим мне злом за добро \textbf{клеветали} ({\slv{ѡ҆болга́ти}}) на меня, так как я \textbf{следовал} ({\slv{гонѧ́ти}}) добру (Пс. 37, 21).
    
    3. Хотя они честь мою \textbf{замыслили} ({\slv{совѣща́ти}}) низринути, жадно \textbf{спешили} ({\slv{тещѝ}}), устами своими \textbf{благословляли} ({\slv{благословлѧ́ти}}), а сердцем своим \textbf{проклинали} ({\slv{клѧ́ти}}) (Пс. 61, 5).
    
    4. Но они, вскричав громким голосом, \textbf{затыкали} ({\slv{затыка́ти}}) уши свои и единодушно \textbf{устремились} ({\slv{ᲂу҆стреми́тисѧ}}) на него (Деян. 7, 57).
    
    5. И когда Петр \textbf{пришел} ({\slv{взы́ти}}) в Иерусалим, обрезанные \textbf{упрекали} ({\slv{препира́тисѧ}}) его (Деян. 11, 2).
    
    6. Язычники, слыша это, \textbf{радовались} ({\slv{ра́доватисѧ}}) и \textbf{прославляли} ({\slv{сла́вити}}) (Деян. 13, 48).
    
    7. Дойдя до Мисии, \textbf{предпринимали} ({\slv{покꙋша́тисѧ}}) идти в Вифинию; но Дух \textbf{не допустил} ({\slv{ѡ҆ста́вити}}) их (Деян. 16, 7).
    
    8. И, придя, \textbf{извинились} ({\slv{ᲂу҆моли́ти}}) перед ними и, выведя, \textbf{просили} ({\slv{моли́ти}}) удалиться из города (Деян. 16, 39).
    
    9. Услышав о воскресении мертвых, одни \textbf{насмехались} ({\slv{рꙋга́тисѧ}}) а другие \textbf{говорили} ({\slv{рещѝ}}): об этом послушаем тебя в другое время (Деян. 17, 32).
    
    10. Весь город \textbf{пришел в движение} ({\slv{подви́гнꙋтисѧ}}), и \textbf{сделалось} ({\slv{бы́ти}}) стечение народа; и, схватив Павла, \textbf{повлекли} ({\slv{влещѝ}}) его вон из храма, и тотчас \textbf{заперты были} ({\slv{затвори́тисѧ}}) двери (Деян. 21, 30).
    
    11. В приведенном ниже церковнославянском тексте замените формы выделенных глаголов, стоящих во множественном числе, на соответствующие формы единственного числа (подобные случаи нередко встречаются в богослужебной практике):
    
    И аще что яко человецы плоть носяще, и в мире живуще, от диавола \textbf{прельстишаяся}. Аще же в слове или в деле или в ведении или в неведении, или слово священническое \textbf{попраша}, или под клятвою священническом \textbf{быша}, или под свою анафему \textbf{падоша}, или под клятву ведошася, Сам яко благ и незлобивый Владыко, сия рабы Твоя словом разрешися благоволи (Требник. М., 1991, с. 76).

                    \paragraph{\exercise}
                    
    В приведенном ниже церковнославянском тексте замените формы выделенных глаголов, стоящих во единственном числе, на соответствующие формы множественного числа (подобные случаи нередко встречаются в богослужебной практике):
    
    \ldots и аще что \textbf{согреши} словом, или делом, или помышлением, или в нощи, или во дни; или под клятву священническую или своему проклятию \textbf{подпаде}; или клятвою \textbf{огоричся}, и \textbf{проклят} себе; Тебе просим, и Тебе молимся, ослаби, остави, прости ему, Боже\ldots~и аще что от заповедей Твоих \textbf{преступи} или \textbf{согреши}\ldots~Сам яко Благ и Человеколюбец Бог прости\ldots~(Требник. М., 1991, с. 351--352).

                \subsubsection{Глаголы вида совершенного и несовершенного}

    Глаголы в церковнославянском языке бывают вида совершенного и несовершенного.
    
    \textbf{Совершенный вид} глаголов выражает действие или состояние с определенной продолжительностью, под условием оконченности совершения его. Например: {\slv{взѧ́ти, повелѣ́ти}}.
    
    \textbf{Несовершенный вид} глаголов выражает действие или состояние с оттенком неопределенной продолжительности. Например: {\slv{плы́ти, повелѣва́ти}}.
    
    Совершенный вид глаголов, кроме вспомогательного глагола {\slv{бы́ти}}, не может иметь настоящего времени и преходящего.
    \pagebreak
    
                \subsubsection{Будущее время глаголов}

    Глагол в будущем времени служит для выражения действия или состояния, которое должно совершиться впоследствии. Например:

    \bigskip\autorows{l}{1}{l}{
        \hspca{{\slv{Ѿ со́нмищъ {\large и҆жденꙋ́тъ} вы̀}} (Ин. 16, 2)}
    }

    Будущее время в церковнославянском языке может быть \textbf{простое} и \textbf{составное} (сложное).
    
    \bigskip
    \mockitem{1. Будущее время простое}
    \medskip
    
    Будущее время простое имеет глаголы совершенного вида. Образование и спряжение этого времени вполне сходно с образованием и спряжением настоящего времени глаголов несовершенного вида, например: {\slv{несꙋ̀~\sdash~понесꙋ̀, творю̀~\sdash~сотворю̀}}.
    
    Вспомогательный глагол {\slv{бы́ти}} в этом времени также имеет окончания, сходные с окончаниями тематических глаголов.
    
    Вот образец этого спряжения:

    \begin{placedtabular}[%
%        caption={\tabcaptsize Будущее время вспомогательного глагола {\slv{бы́ти}}}
        ]{|c|c|c|c|c|}
        \hline
        
        \multirow{2}{*}{\spheading[2em]{\scriptsize{Лицо}}}
        & \multirow{2}{*}{\mkcella{Единственное\\число}}
        & \multicolumn{2}{c|}{\mkcella{Двойственное число}}
        & \multirow{2}{*}{\mkcella{Множественное\\число}}
        \\
        
        \cline{3-4}
        
        &
        & \mkcella{Мужской род}
        & \mkcella{Жен. и сред. род}
        &
        \\
        
        \hline
        
        1
        & \makecell{{\slv{бꙋ́дꙋ}}}
        & \makecell{{\slv{бꙋ́дева}}}
        & \makecell{{\slv{бꙋ́девѣ}}}
        & \makecell{{\slv{бꙋ́демъ}}}
        \\\hline
        
        2
        & \makecell{{\slv{бꙋ́деши}}}
        & \multirow{2}{*}{{\slv{бꙋ́дета}}}
        & \multirow{2}{*}{{\slv{бꙋ́детѣ}}}
        & \makecell{{\slv{бꙋ́дете}}}
        \\\cline{1-2}\cline{5-5}
        
        3
        & \makecell{{\slv{бꙋ́детъ}}}
        &
        &
        & \makecell{{\slv{бꙋ́дꙋтъ}}}
        \\\hline
        
    \end{placedtabular}

    \bigskip
    \mockitem{2. Будущее время составное}
    \medskip
    
    Будущее время составное имеют глаголы несовершенного вида. Образуется оно из будущего времени глагола {\slv{бы́ти}} и неопределенной формы глагола. Например:
    
    \bigskip\autorows{l}{1}{l}{
        \hspca{{\slv{бꙋ́дꙋ нестѝ, бꙋ́деши нестѝ}} и т.д.},
        \hspca{{\slv{бꙋ́дꙋ слы́шати, бꙋ́деши слы́шати}} и т.д.}
    }

    Вспомогательный глагол {\slv{бы́ти}}, как глагол совершенного вида, составного будущего времени не имеет.

                \subsubsection{Понятие о наклонениях глагола}

    \textbf{Наклонениями глагола} называются такие его формы, которые служат для выражения различных способов действий или состояний.
    
    До сего времени рассматривались формы изъявительного наклонения.
    
    \textbf{Изъявительное наклонение} выражает определенное действие или состояние предмета или явления в настоящем, или прошедшем, или будущем времени. Например: {\slv{спаса́ю, спасо́хъ, спа́сꙋ}}.

                \subsubsection{Желательное наклонение глаголов}

    Желательное наклонение глаголов выражает действие, которое можно пожелать в будущем. Поэтому оно имеет только одно будущее время, которое образуется из частицы {\slv{да}} и настоящего или будущего времени спрягаемого глагола. Например: {\slv{да просвѣща́ю, да просвещꙋ̀}}.
    
    Вот образцы спряжения в этом наклонении глаголов {\slv{бы́ти}} и {\slv{хвали́ти}}:

\begin{center}
%    {\tabcaptsize Спряжение в желательном наклонении глаголов {\slv{бы́ти}} и {\slv{хвали́ти}}}
    \renewcommand*{\arraystretch}{1.2}
    \begin{tabular}[c]{|c|c|c|c|c|c|}
        \hline
        
        \multirow{2}{*}{\spheading[2.5em]{\scriptsize{Глагол}}}
        &\multirow{2}{*}{\spheading[2em]{\scriptsize{Лицо}}}
        & \multirow{2}{*}{\mkcella{Единственное\\число}}
        & \multicolumn{2}{c|}{\mkcella{Двойственное число}}
        & \multirow{2}{*}{\mkcella{Множественное\\число}}
        \\
        
        \cline{4-5}
        
        &
        &
        & \mkcella{Мужской род}
        & \mkcella{Жен. и сред. род}
        &
        \\
        
        \hline
        
        \multirow{3}{*}{\spheading[3em]{\slv{бы́ти}}}
        &1
        & \makecell{{\slv{да бꙋ́дꙋ}}}
        & \makecell{{\slv{да бꙋ́дева}}}
        & \makecell{{\slv{да бꙋ́девѣ}}}
        & \makecell{{\slv{да бꙋ́демъ}}}
        \\\cline{2-6}
        
        &2
        & \makecell{{\slv{да бꙋ́деши}}}
        & \multirow{2}{*}{{\slv{да бꙋ́дета}}}
        & \multirow{2}{*}{{\slv{да бꙋ́детѣ}}}
        & \makecell{{\slv{да бꙋ́дете}}}
        \\\cline{2-3}\cline{6-6}
        
        &3
        & \makecell{{\slv{да бꙋ́детъ}}}
        &
        &
        & \makecell{{\slv{да бꙋ́дꙋтъ}}}
        \\\hline

        \multirow{3}{*}{\spheading[5.5em]{\slv{хвали́ти}}}
        &1
        & \makecell{{\slv{да хвалю̀}}\\{\slv{да похвалю̀}}}
        & \makecell{{\slv{да хва́лива}}\\{\slv{да похва́лива}}}
        & \makecell{{\slv{да хва́ливѣ}}\\{\slv{да похва́ливѣ}}}
        & \makecell{{\slv{да хва́лимъ}}\\{\slv{да похва́лимъ}}}
        \\\cline{2-6}
        
        &2
        & \makecell{{\slv{да хва́лиши}}\\{\slv{да похва́лиши}}}
        & \multirow{2}{*}{\makecell{{\slv{да хва́лита}}\\{\slv{да похва́лита}}}}
        & \multirow{2}{*}{\makecell{{\slv{да хва́литѣ}}\\{\slv{да похва́литѣ}}}}
        & \makecell{{\slv{да хва́лите}}\\{\slv{да похва́лите}}}
        \\\cline{2-3}\cline{6-6}
        
        &3
        & \makecell{{\slv{да хва́литъ}}\\{\slv{да похва́литъ}}}
        &
        &
        & \makecell{{\slv{да хва́лѧтъ}}\\{\slv{да похва́лѧтъ}}}
        \\\hline
        
    \end{tabular}
\end{center}

                \subsubsection{Повелительное наклонение глаголов}

    Повелительное наклонение глаголов выражает просьбу или требование совершить действие. Например: {\slv{спасѝ! повели́те!}}
    
    Особенность этого наклонения та, что оно не имеет времен; кроме того, в единственном числе имеет только второе лицо, а в двойственном или множественном~---~только 1-е и 2-е.
    
    Рассмотрим спряжение в этом наклонении тех же глаголов {\slv{бы́ти, хвали́ти}}.

    \medskip\begin{placedtabular}[%
%        caption={\tabcaptsize Спряжение в повелительном наклонении глаголов {\slv{бы́ти}} и {\slv{хвали́ти}}}
        ]{|c|c|c|c|c|c|}
        \hline
        
        \multirow{2}{*}{\spheading[3.0em]{\scriptsize{Глагол}}}
        &\multirow{2}{*}{\spheading[2.5em]{\scriptsize{Лицо}}}
        & \multirow{2}{*}{\mkcella{Единственное\\число}}
        & \multicolumn{2}{c|}{\mkcella{Двойственное число}}
        & \multirow{2}{*}{\mkcella{Множественное\\число}}
        \\
        
        \cline{4-5}
        
        &
        &
        & \mkcella{Мужской род}
        & \mkcella{Жен. и сред. род}
        &
        \\
        
        \hline
        
        \multirow{2}{*}{\spheading[2.5em]{\scriptsize\slv{бы́ти}}}
        & 1
        & \makecell{--}
        & \makecell{{\slv{бꙋ́дива}}}
        & \makecell{{\slv{бꙋ́дивѣ}}}
        & \makecell{{\slv{бꙋ́димъ}}}
        \\\cline{2-6}
        
        & 2
        & \makecell{{\slv{бꙋ́дꙋ}}}
        & \makecell{{\slv{бꙋ́дита}}}
        & \makecell{{\slv{бꙋ́дитѣ}}}
        & \makecell{{\slv{бꙋ́дите}}}
        \\\hline
        
        \multirow{2}{*}{\spheading[3.3em]{\scriptsize\slv{хвали́ти}}}
        & 1
        & \makecell{--}
        & \makecell{{\slv{хвали́ва}}}
        & \makecell{{\slv{хвали́вѣ}}}
        & \makecell{{\slv{хвали́мъ}}}
        \\\cline{2-6}
        
        & 2
        & \makecell{{\slv{хвалѝ}}}
        & \makecell{{\slv{хвали́та}}}
        & \makecell{{\slv{хвали́тѣ}}}
        & \makecell{{\slv{хвали́те}}}
        \\\hline
        
    \end{placedtabular}

    \bigskip
    Повелительное наклонение тематических глаголов образуется от неопределенной формы глаголов следующим образом: окончание {\slv{-ти}} (и с простым суффиксом, если он имеется) отбрасывается и к основе присоединяется окончание {\slv{-и}}. Например: {\slv{нес-тѝ~\sdash~нес-ѝ, повел-ѣ́-ти~\sdash~повел-ѝ, хвал-и́-ти~\sdash~хвал-ѝ, ꙗ҆́-ти~\sdash~є҆́мл-и}}.
    
    Если же основа глагола оканчивается на гласный {\slv{а}}, то вместо окончания неопределенной формы {\slv{-ти}} непосредственно присоединяется окончание {\slv{й}}. Например: {\slv{повелѣва́-ти~\sdash~повелѣва́-й, пита́-ти~\sdash~пита́-й}}.
    
    Окончание 2-го лица единственного числа {\slv{-и}} иногда меняется на {\slv{-ь}}. Например: {\slv{да́ждь}} (вместо {\slv{да́жди}}), {\slv{ви́ждь}} (вместо {\slv{ви́жди}}).
    
    При наличии в основе гортанных {\slv{г}} и {\slv{к}} последние перед гласным {\slv{и}} смягчаются соответственно в {\slv{з}} и {\slv{ц}} по второму закону смягчения, причем после {\slv{ц}} пишется не {\slv{-и}}, но {\slv{-ы}}. Например: {\slv{помощѝ}} ({\slv{помог-тѝ}})~---~{\slv{помозѝ}}, {\slv{тещѝ}} ({\slv{тек-тѝ}})~---~{\slv{тецы̀}}, {\slv{рещѝ}} ({\slv{рек-тѝ}})~---~{\slv{рцы̀}}.
    
    Третье лицо желательного наклонения употребляется иногда в значении третьего лица повелительного наклонения. Например:
    
    \bigskip\autorows{l}{1}{l}{
        \hspca{{\slv{Да бꙋ́детъ свѣ́тъ}} (Быт. 1, 3)}
    }
    \pagebreak
    
                \subsubsection{Спряжение архаических глаголов}

    Рассмотрим особенности спряжения архаических глаголов: {\slv{вѣ́дѣти, ꙗ҆́сти, да́ти}} и {\slv{и҆мѣ́ти}}.
    
    Эти особенности касаются настоящего времени изъявительного наклонения глаголов {\slv{вѣ́дети, ꙗ҆́сти}} и {\slv{и҆мѣ́ти}}; в отношении глагола {\slv{да́ти}} особенности касаются аориста и будущего времени. В отношении же повелительного наклонения все четыре глагола имеют свои особенности.

    \medskip\begin{placedtabular}[%
%        caption={\tabcaptsize Спряжение архаических глаголов {\slv{вѣ́дѣти}}, {\slv{ꙗ҆́сти}} и {\slv{и҆мѣ́ти}}}
        ]{|c|c|c|c|c|c|}
        \hline
        
        \multirow{2}{*}{\spheading[2.5em]{\scriptsize{~}}}
        &\multirow{2}{*}{\spheading[2.5em]{\scriptsize{Лицо}}}
        & \multirow{2}{*}{\mkcella{Единственное\\число}}
        & \multicolumn{2}{c|}{\mkcella{Двойственное число}}
        & \multirow{2}{*}{\mkcella{Множественное\\число}}
        \\\cline{4-5}
        
        &
        &
        & \mkcella{Мужской род}
        & \mkcella{Жен. и сред. род}
        &
        \\\hline
        
        \multicolumn{6}{|c|}{{\slv{вѣ́дети}}}
        \\\hline
        
        \multirow{2}{*}{\spheading[4.7em]{\scriptsize Настоящее время}}
        &{\small 1}
        & \makecell{{\slv{вѣ́мъ}}}
        & \makecell{{\slv{вѣ́ва}}}
        & \makecell{{\slv{вѣ́вѣ}}}
        & \makecell{{\slv{вѣ́мы}}}
        \\\cline{2-6}
        
        &{\small 2}
        & \makecell{{\slv{вѣ́си}}}
        & \multirow{2}{*}{{\slv{вѣ́ста}}}
        & \multirow{2}{*}{{\slv{вѣ́стѣ}}}
        & \makecell{{\slv{вѣ́сте}}}
        \\\cline{2-3}\cline{6-6}
        
        &{\small 3}
        & \makecell{{\slv{вѣ́сть}}}
        &
        &
        & \makecell{{\slv{вѣ́дѧтъ}}}
        \\\hline
        
        \multirow{2}{*}{\spheading[3.3em]{\scriptsize Повелит. наклон.}}
        &{\small 1}
        & \makecell{--}
        & \makecell{{\slv{вѣ́дива}}}
        & \makecell{{\slv{вѣ́дивѣ}}}
        & \makecell{{\slv{вѣ́димъ}}}
        \\\cline{2-6}
        
        &{\small 2}
        & \makecell{{\slv{вѣ́ждь}}}
        & \makecell{{\slv{вѣ́дита}}}
        & \makecell{{\slv{вѣ́дитѣ}}}
        & \makecell{{\slv{вѣ́дите}}}
        \\\hline

        \multicolumn{6}{|c|}{{\slv{ꙗ҆́сти}}}
        \\\hline
        
        \multirow{2}{*}{\spheading[4.7em]{\scriptsize Настоящее время}}
        &{\small 1}
        & \makecell{{\slv{ꙗ҆́мъ}}}
        & \makecell{{\slv{ꙗ҆́ва}}}
        & \makecell{{\slv{ꙗ҆́вѣ}}}
        & \makecell{{\slv{ꙗ҆́мы}}}
        \\\cline{2-6}
        
        &{\small 2}
        & \makecell{{\slv{ꙗ҆́си}}}
        & \multirow{2}{*}{{\slv{ꙗ҆́ста}}}
        & \multirow{2}{*}{{\slv{ꙗ҆́стѣ}}}
        & \makecell{{\slv{ꙗ҆́сте}}}
        \\\cline{2-3}\cline{6-6}
        
        &{\small 3}
        & \makecell{{\slv{ꙗ҆́стъ}}}
        &
        &
        & \makecell{{\slv{ꙗ҆́дѧтъ}}}
        \\\hline
        
        \multirow{2}{*}{\spheading[3.3em]{\scriptsize Повелит. наклон.}}
        &{\small 1}
        & \makecell{--}
        & \makecell{{\slv{ꙗ҆ди́ва}}}
        & \makecell{{\slv{ꙗ҆ди́вѣ}}}
        & \makecell{{\slv{ꙗ҆ди́мъ}}}
        \\\cline{2-6}
        
        &{\small 2}
        & \makecell{{\slv{ꙗ҆́ждь}}}
        & \makecell{{\slv{ꙗ҆ди́та}}}
        & \makecell{{\slv{ꙗ҆ди́тѣ}}}
        & \makecell{{\slv{ꙗ҆ди́те}}}
        \\\hline

        \multicolumn{6}{|c|}{{\slv{и҆мѣ́ти}}}
        \\\hline
        
        \multirow{2}{*}{\spheading[5em]{\scriptsize Настоящее время}}
        &{\small 1}
        & \makecell{{\slv{и҆́мамъ}}}
        & \makecell{{\slv{и҆́мава}}}
        & \makecell{{\slv{и҆́мавѣ}}}
        & \makecell{{\slv{и҆́мамы}}}
        \\\cline{2-6}
        
        &{\small 2}
        & \makecell{{\slv{и҆́маши}}}
        & \multirow{2}{*}{{\slv{и҆́мата}}}
        & \multirow{2}{*}{{\slv{и҆́матѣ}}}
        & \makecell{{\slv{и҆́мате}}}
        \\\cline{2-3}\cline{6-6}
        
        &{\small 3}
        & \makecell{{\slv{и҆́мать}}}
        &
        &
        & \makecell{{\slv{и҆́мꙋтъ}}\\{\slv{и҆мѣ́ютъ}}}
        \\\hline
        
        \multirow{2}{*}{\spheading[3.7em]{\scriptsize Повелит. наклон.}}
        &{\small 1}
        & \makecell{--}
        & \makecell{{\slv{и҆мѣ́йва}}}
        & \makecell{{\slv{и҆мѣ́йвѣ}}}
        & \makecell{{\slv{и҆мѣ́имъ}}}
        \\\cline{2-6}
        
        &{\small 2}
        & \makecell{{\slv{и҆мѣ́й}}\\{\slv{и҆мѝ}}}
        & \makecell{{\slv{и҆мѣ́йта}}\\{\slv{и҆ми́та}}}
        & \makecell{{\slv{и҆мѣ́йтѣ}}\\{\slv{и҆ми́тѣ}}}
        & \makecell{{\slv{и҆мѣ́йте}}\\{\slv{и҆ми́те}}}
        \\\hline
        
    \end{placedtabular}

    \bigskip
    При рассмотрении спряжения архаических глаголов {\slv{вѣ́дѣти, ꙗ҆́сти}}, и \mbox{\slv{и҆мѣ́ти}} необходимо заметить:
    
    1. Глагол {\slv{вѣ́дѣти}}, как глагол несовершенного вида, имеет только будущее составное время ({\slv{бꙋ́дꙋ вѣ́дѣти}}). В будущем простом он употребляется с какой-либо приставкой (например, {\slv{ᲂу҆вѣ́дети}}), т.е. приобретает совершенный вид и спрягается по образу настоящего времени.
    
    2. Глагол {\slv{ꙗ҆́сти}}, как глагол также несовершенного вида, употребляется в будущем простом времени с приставкой {\slv{с}}; причем звук {\slv{ꙗ}} после {\slv{с}} переходит в {\slv{ѣ}} и между ними для благозвучия вставляется согласный звук {\slv{н}}; получается глагол совершенного вида {\slv{снѣ́сти}}.
    
    3. Глагол {\slv{имѣ́ти}}, как глагол несовершенного вида, будущего простого времени не имеет.
    
    Глагол {\slv{да́ти}}, как глагол совершенного вида, в изъявительном наклонении не имеет настоящего времени, преходящего и будущего составного. Для этих глагольных форм он заменяется производным глаголом несовершенного вида {\slv{даѧ́ти}}, который представляет собою уже тематический глагол 1-го спряжения.
    
    Во 2-м и 3-м лице единственного числа аориста глагол {\slv{да́ти}} имеет форму {\slv{дадѐ}} (вместо {\slv{да}}), а в 3-м лице множественного числа того же времени имеет две формы: {\slv{да́ша}} и {\slv{дадо́ша}}.

    \begin{placedtabular}[%
%        caption={\tabcaptsize Спряжение глагола совершенного вида {\slv{да́ти}}}
        ]{|c|c|c|c|c|c|}
        \hline
        
        \multirow{2}{*}{\spheading[2.5em]{\scriptsize{~}}}
        &\multirow{2}{*}{\spheading[2.5em]{\scriptsize{Лицо}}}
        & \multirow{2}{*}{\mkcella{Единственное\\число}}
        & \multicolumn{2}{c|}{\mkcella{Двойственное число}}
        & \multirow{2}{*}{\mkcella{Множественное\\число}}
        \\\cline{4-5}
        
        &
        &
        & \mkcella{Мужской род}
        & \mkcella{Жен. и сред. род}
        &
        \\\hline
        
        \multirow{3}{*}{\spheading[4.3em]{\scriptsize Аорист}}
        & {\small 1}
        & \makecell{{\slv{да́хъ}}}
        & \makecell{{\slv{да́хова}}}
        & \makecell{{\slv{да́ховѣ}}}
        & \makecell{{\slv{да́хомъ}}}
        \\\cline{2-6}
        
        & {\small 2}
        & \makecell{{\slv{дадѐ}}}
        & \multirow{2}{*}{{\slv{да́ста}}}
        & \multirow{2}{*}{{\slv{да́стѣ}}}
        & \makecell{{\slv{да́сте}}}
        \\\cline{2-3}\cline{6-6}
        
        & {\small 3}
        & \makecell{{\slv{дадѐ}}}
        &
        &
        & \makecell{{\slv{да́ша}}\\{\slv{дадо́ша}}}
        \\\hline
        
        \multirow{3}{*}{\spheading[4.4em]{\scriptsize Будущее время}}
        & {\small 1}
        & \makecell{{\slv{да́мъ}}}
        & \makecell{{\slv{да́дива}}}
        & \makecell{{\slv{да́дивѣ}}}
        & \makecell{{\slv{да́мы}}}
        \\\cline{2-6}
        
        & {\small 2}
        & \makecell{{\slv{да́си}}}
        & \multirow{2}{*}{{\slv{дади́та}}}
        & \multirow{2}{*}{{\slv{дади́тѣ}}}
        & \makecell{{\slv{дади́те}}}
        \\\cline{2-3}\cline{6-6}
        
        & {\small 3}
        & \makecell{{\slv{да́стъ}}}
        &
        &
        & \makecell{{\slv{дадꙋ́тъ}}\\{\slv{дадѧ́ть}}}
        \\\hline
        
        \multirow{2}{*}{\spheading[3.3em]{\scriptsize Повелит. наклон.}}
        & {\small 1}
        & \makecell{--}
        & \makecell{{\slv{дади́ва}}}
        & \makecell{{\slv{дади́вѣ}}}
        & \makecell{{\slv{дади́мъ}}}
        \\\cline{2-6}
        
        & {\small 2}
        & \makecell{{\slv{да́ждь}}}
        & \makecell{{\slv{дади́та}}}
        & \makecell{{\slv{дади́тѣ}}}
        & \makecell{{\slv{да́жд̾те}}}
        \\\hline
        
    \end{placedtabular}

    \addtocounter{paragraph}{1} % Following the original typo.
    
                    \bigskip\paragraph{\exercise}
    
    В приведенных предложениях замените выделенный глагол на соответствующую форму церковнославянского глагола {\slv{вѣ́дѣти}}.
    
    \medskip\begin{adjustwidth}{\hstbb}{0cm}
        \renewcommand*{\arraystretch}{1.2}
        \begin{tabular}[l]{rl}
            
            \emph{Образец}:
            & \makecell[l]{Вы сами \textbf{знаете}, что я всеми силами служил\\отцу вашему (Быт. 31, 6).}
            \\
            
            &
            \\
            
            \exanswer
            & \makecell[l]{Вы сами {\slv{вѣ́сте}}, что я всеми силами служил\\отцу вашему.}
            \\
            
        \end{tabular}
    \end{adjustwidth}

    \medskip
    1. Пойти пойду с тобою; только \textbf{знай}, что не тебе уже будет слава на сем пути (Суд. 4, 9).
    
    2. И если \textbf{знаешь}, что между ними есть способные люди, поставь их смотрителями над моим скотом (Быт. 47, 6).
    
    3. Путь же беззаконных~---~как тьма; они не \textbf{знают}, обо что споткнутся (Притч. 4, 19).
    
    4. Пришельца не обижай и не притесняй его: вы \textbf{знаете} душу пришельца (Исх. 23, 9).
    
    5. Но Я \textbf{знаю}, что фараон, царь Египетский, не позволит вам идти (Исх. 3, 19).
    
    6. Но доколе не придем туда, мы не \textbf{знаем}, что принести в жертву Господу Богу нашему (Исх. 10, 26).
    
    7. Так, и когда вы увидите то сбывающимся, \textbf{знайте}, что близко Царствие Божие (Лк. 21, 31).
    
    8. Слышала ли ты, что Адония, сын Аггифин, сделался царем, а господин наш Давид не \textbf{знает} о том? (3 Цар. 1, 11).
    
                \paragraph{\exercise}

    В приведенных предложениях замените выделенный глагол на соответствующую форму церковнославянского глагола {\slv{ꙗ҆́сти}}.
    
    \medskip\begin{adjustwidth}{\hstbb}{0cm}
        \renewcommand*{\arraystretch}{1.2}
        \begin{tabular}[l]{rl}
            
            \emph{Образец}:
            & \makecell[l]{Итак берегись, не пей вина и сикера, и не \textbf{ешь} ничего\\нечистого (Суд. 13, 4).}
            \\
            
            &
            \\
            
            \exanswer
            & \makecell[l]{Итак берегись, не пей вина и сикера, и не {\slv{ꙗ҆́ждь}} ничего\\нечистого.}
            \\
            
        \end{tabular}
    \end{adjustwidth}

    \medskip
    1. Горе тебе, земля, когда царь твой отрок, и когда князья твои \textbf{едят} рано (Еккл. 10, 16).
    
    2. И когда вы \textbf{едите} и когда пьете, не для себя ли вы \textbf{едите}, не для себя ли вы пьете? (Зах. 7, 6).
    
    3. И если какой человек \textbf{ест} и пьет, и видит доброе во всяком труде своем, то это~---~дар Божий (Еккл. 3, 13).
    
    4. \textbf{Ем} ли Я мясо волов и пью ли кровь козлов? (Пс. 49, 13).
    
    5. Вот это оставлено, положи пред собою и \textbf{ешь} (1 Цар. 9, 24).
    
    6. Пища не приближает нас к Богу: ибо, \textbf{едим} ли мы, ничего не приобретаем; не \textbf{едим}, ничего не теряем (1 Кор. 8, 8).
    
    7. И приноси жертвы мирные, и \textbf{ешь} и насыщайся там, и веселись пред Господом Богом твоим (Втор. 27, 7).
    
    8. И сказал Моисей: \textbf{ешьте} его сегодня, ибо сегодня суббота Господня; сегодня не найдете его на поле (Исх. 16, 25).
    
                        \paragraph{\exercise}

    В приведенных предложениях замените выделенный глагол на соответствующую форму церковнославянского глагола {\slv{и҆мѣ́ти}}.
    
    \medskip\begin{adjustwidth}{\hstbb}{0cm}
        \renewcommand*{\arraystretch}{1.2}
        \begin{tabular}[l]{rl}
            
            \emph{Образец}:
            & \makecell[l]{\ldots каждый оставит войну в своей собственной стране,\\которую они \textbf{имеют} между собою (3 Езд. 13, 33).}
            \\
            
            &
            \\
            
            \exanswer
            & \makecell[l]{\ldots каждый оставит войну в своей собственной стране,\\которую они {\slv{и҆́мꙋтъ}} между собою.}
            \\
            
        \end{tabular}
    \end{adjustwidth}

    \medskip
    1. Иудеи отвечали ему: мы \textbf{имеем} закон (Ин. 19, 7).
    
    2. Хотя бы он и оскудел разумом, \textbf{имей} снисхождение (Сир. 3, 13).
    
    3. Пилат сказал им: \textbf{имеете} стражу; пойдите, охраняйте, как знаете (Мф. 27, 65).
    
    4. \textbf{Имейте} в себе соль, и мир \textbf{имейте} между собою (Мк. 9, 50).
    
    5. Ибо я не \textbf{имею} никого равно усердного, кто бы столь искренно заботился о вас (Флп. 2, 20).
    
    6. Пойди, все, что \textbf{имеешь}, продай и раздай нищим, и будешь \textbf{иметь} сокровище на небесах (Мк. 10, 21).
    
    7. И не могут они, заметив это, оставить их, потому что не \textbf{имеют} смысла (Посл. Иер. 1, 41).
    
    8. Велик он и не \textbf{имеет} конца, высок и неизмерим (Вар. 3, 25).
    
                    \medskip\paragraph{\exercise}
                    
    В приведенных предложениях замените выделенный глагол на соответствующую форму церковнославянского глагола {\slv{да́ти}}.
    
    \medskip\begin{adjustwidth}{\hstbb}{0cm}
        \renewcommand*{\arraystretch}{1.2}
        \begin{tabular}[l]{rl}
            
            \emph{Образец}:
            & \makecell[l]{Лаван расседлал верблюдов и \textbf{дал} соломы и корму\\верблюдам (Быт. 24, 32).}
            \\
            
            &
            \\
            
            \exanswer
            & \makecell[l]{Лаван расседлал верблюдов и {\slv{да́де}} соломы и корму\\верблюдам.}
            \\
            
        \end{tabular}
    \end{adjustwidth}

    \medskip
    1. Если ты праведен, что \textbf{даешь} Ему? (Иов. 35, 7).
    
    2. Я истреблю Израиля с лица земли Моей, которую Я \textbf{дал} им (2 Пар. 7, 20).
    
    3. Она сказала: \textbf{дай} мне благословение (Нав. 15, 19).
    
    4. Господь \textbf{даст} силу народу Своему (Пс. 28, 11).
    
    5. Ибо вы знаете, какие мы \textbf{дали} вам заповеди от Господа Иисуса (1 Фес. 4, 2).
    
    6. Я \textbf{дам} тебе две тысячи коней (4 Цар. 18, 23).
    
    7. И \textbf{дала} кушанье и хлеб, которые она приготовила, в руки Иакову, сыну своему (Быт. 27, 17).
    
    8. И \textbf{дали} на устроение дома Божия пять тысяч талантов (1 Пар. 29, 7).

                \subsubsection{Будущее время описательное}

    Будущее время описательное является разновидностью будущего составного времени и отличается от него только тем, что вместо вспомогательного глагола {\slv{бы́ти}} берется в формах настоящего времени архаический глагол {\slv{имѣ́ти}}, спрягаемый же глагол ставится в неопределенной форме. Глагол {\slv{бы́ти}} также имеет будущее описательное время, так как оно приложимо для глаголов как несовершенного, так и совершенного вида. Например:
    
    \bigskip\autorows{l}{1}{l}{
        \hspca{{\slv{и҆́мамъ бы́ти, и҆́маши бы́ти}} и т.д.},
        \hspca{{\slv{и҆́мамъ хвали́ти, и҆́маши хвали́ти}} и т.д.}
    }

    У глагола {\slv{и҆мѣ́ти}} в будущем описательном времени в качестве вспомогательного глагола присоединяется тематический глагол {\slv{хотѣ́ти}}. Например: {\slv{хощꙋ̀ и҆мѣ́ти, хо́щеши и҆мѣ́ти}} и т.д.
    
    Будущее описательное время большей частью употребляется тогда, когда нужно не только ограничиться передачей мысли, но сделать ее и более наглядной, образной. Например:
    
    {\slv{Глаго́лю же ва́мъ, ꙗ҆́кѡ не и҆́мамъ пи́ти ѿны́нѣ ѿ плода̀ ло́знагѡ}} (Мф. 26, 29).

        \section{Склоняемые части речи}
            \subsection{Имя существительное}
                \subsubsection{Родовые окончания имен существительных}

    \textbf{Именем существительным} называется часть речи, обозначающая предмет, явление или понятие. Например: {\slv{землѧ̀, па́стырь, до́ждь, вѣ́тр, помышле́нїе}}.
    
    Каждому роду имен существительных соответствуют особые окончания именительного падежа единственного числа, которые поэтому называются \emph{родовыми} окончаниями. Так, родовые окончания в именах
    
    \bigskip\autorows[-1pt]{l}{2}{l}{
        \hspca{мужского рода:}, {{\slv{-ъ, -ь, -й}} ({\slv{до́мъ, па́стырь, мра́вїй}});},
        \hspca{женского рода:}, {{\slv{-а, -ѧ, -ь}} ({\slv{вода̀, ладїѧ̀, ма́терь}});},
        \hspca{среднего рода:}, {{\slv{-о, -е, -ѧ}} ({\slv{чело̀, мо́ре, и҆́мѧ}}).}
    }

    Имена существительные в церковнославянском языке имеют \textbf{три склонения}. Так как склоняемые имена могут иметь или твердые, или мягкие окончания, то и некоторые склонения подразделяются на \emph{твердые} и \emph{мягкие}.

                \subsubsection{Первое склонение имен существительных}

    К первому склонению имен существительных относятся имена \textbf{женского} рода с окончанием на {\slv{-а}} и {\slv{-ѧ}}. Но в некоторых случаях эти окончания имеют и имена существительные по значению мужского рода, например: {\slv{воево́да, сꙋдїѧ̀}}, которые также склоняются по первому склонению.
    
    Имена существительные с окончанием на {\slv{-а}} (при основе на согласный) принадлежат к твердому склонению, а с окончанием на {\slv{ѧ}} (а также и на {\slv{-а}} с основой на гласный)~---~к мягкому склонению.
    
    Возьмем для образца первого склонения существительные: {\slv{жена̀, пꙋсты́нѧ}} и {\slv{ски́нїѧ}} ({\slv{ски́нїа}}).
    
    \begin{center}
%        {\tabcaptsize Первое склонение имен существительных {\slv{жена̀, пꙋсты́нѧ}} и {\slv{ски́нїѧ}} ({\slv{ски́нїа}})}
        \renewcommand*{\arraystretch}{1.4}
        \footnotesize\begin{tabular}[c]{|c|c|c|c|c|c|c|}
            \hline
            
            \makecell{Па-\\деж}
            & \multicolumn{3}{c|}{Единственное число}
            & \multicolumn{3}{c|}{Множественное число}
            \\\hline
            
            И.
            & {\slv{жена̀}}
            & {\slv{пꙋсты́нѧ}}
            & \makecell{{\slv{ски́нїѧ}}\\{\slv{ски́нїа}}}
            & {\slv{жєны̀}}
            & {\slv{пꙋсты̑ни}}
            & {\slv{ски̑нїи}}
            \\\hline
            
            Р.
            & {\slv{жены̀}}
            & \multirow{2}{*}{{\slv{пꙋсты́ни}}}
            & \multirow{2}{*}{{\slv{ски́нїи}}}
            & {\slv{же́нъ}}
            & {\slv{пꙋсты́нь}}
            & {\slv{ски́ний}}
            \\\cline{1-2}\cline{5-7}
            
            Д.
            & {\slv{женѣ̀}}
            &
            &
            & {\slv{жена́мъ}}
            & {\slv{пꙋсты́нѧмъ}}
            & {\slv{ски́нїѧмъ}}
            \\\hline
    
            В.
            & {\slv{женꙋ̀}}
            & {\slv{пꙋсты́ню}}
            & {\slv{ски́нїю}}
            & {\slv{жєны}}
            & {\slv{пꙋсты̑ни}}
            & {\slv{ски̑нїи}}
            \\\hline
    
            З.
            & {\slv{же́но}}
            & \makecell{{\slv{пꙋсты́не}}\\{\slv{пꙋсты́нѧ}}}
            & \makecell{{\slv{ски́нїе}}\\{\slv{ски́нїѧ}}}
            & {\slv{же́ны}}
            & {\slv{пꙋсты̑ни}}
            & {\slv{ски̑нїи}}
            \\\hline
            
            Т.
            & {\slv{жено́ю}}
            & {\slv{пꙋсты́нею}}
            & {\slv{ски́нїею}}
            & {\slv{жена́ми}}
            & {\slv{пꙋсты́нѧми}}
            & {\slv{ски́нїѧми}}
            \\\hline
            
            П.
            & {\slv{ѡ҆ женѣ̀}}
            & {\slv{ѡ҆ пꙋсты́ни}}
            & {\slv{ѡ҆ ски́нїи}}
            & \makecell{{\slv{ѡ҆ жена́хъ}}\\{\slv{ѡ҆ женѣ́хъ}}}
            & {\slv{ѡ҆ пꙋсты́нѧхъ}}
            & {\slv{ѡ҆ ски́нїѧхъ}}
            \\\hline
            
            \makecell{~\\~}
            & \multicolumn{3}{c|}{Двойственное число}
            \\\cline{1-4}
    
            \makecell{И.\\В. З.}
            & {\slv{жєны̀}}
            & {\slv{пꙋсты̑ни}}
            & {\slv{ски̑нїи}}
            \\\cline{1-4}
            
            \makecell{Р. П.}
            & \makecell{\slv{жєнꙋ̀}}
            & {\slv{пꙋсты́ню}}
            & {\slv{ски̑нїю}}
            \\\cline{1-4}
    
            \makecell{Д. Т.}
            & \makecell{{\slv{жена́ма}}\\{\slv{жено́ма}}}
            & \makecell{{\slv{пꙋсты́нѧма}}\\{\slv{пꙋсты́ньми}}}
            & {\slv{ски́нїѧма}}
            \\\cline{1-4}
            
        \end{tabular}
    \end{center}

                \subsubsection{Общие замечания к первому склонению имен существительных}

    1. Звательный падеж единственного числа в мягком склонении большей частью оканчивается на {\slv{-е}}: {\slv{землѐ, пꙋсты́не}}, но иногда он бывает сходен и с именительным падежом единственного числа: {\slv{землѧ̀, пꙋсты́нѧ}}.
    
    2. Дательный и творительный падежи двойственного числа иногда, в виде исключения, в твердом склонении оканчиваются на {\slv{-ома}} ({\slv{жено́ма}}), а предложный падеж множественного числа~---~на {\slv{-ѣхъ}} ({\slv{ѡ҆ женѣ́хъ}}).
    
    3. Формы родительного падежа множественного числа в твердом склонении весьма разнообразны: {\slv{мольба̀~\sdash~моле́бъ, татьба̀~\sdash~та́тьбъ, тьма̀~\sdash~те́мъ, та́йна~\sdash~та́инъ, смо́ква~\sdash~смо́квїй}} (и {\slv{смо́квей}}) и т.п.
    
    4. Некоторые существительные первого склонения не имеют или двойственного и множественного чисел, например: {\slv{мгла̀, мзда̀, прѧ̀, лжа̀, тлѧ̀}} и др., или единственного и двойственного чисел: {\slv{вери́ги, ѡ҆ко́вы}} и др.
    
    5. Некоторые падежи двойственного и множественного чисел, созвучные с некоторыми падежами единственного числа, различаются от них или иным начертанием букв ({\slv{е}} или {\slv{о}}~---~в единственном числе и {\slv{є}} или {\slv{ѡ}}~---~в двойственном и множественном числах), или ударением (острое или тупое~---~в единственном числе и облегченное~---~в двойственном и множественном числах). Эта особенность в начертании букв или постановке ударений соблюдается и в прочих склонениях, а также в склонениях других частей речи.
    
    6. Имена существительные с основой на гортанные {\slv{г, к, х}} смягчаются перед {\slv{ѣ}} и {\slv{и}} ({\slv{ы}}) соответственно в {\slv{з, ц, с}}, например:
    
    \bigskip\autorows{l}{1}{l}{
        \hspca\slv{но{\large г}а̀~\sdash~{но́\large з}ѣ~\sdash~но́{\large з}и},
        \hspca\slv{рꙋ{\large к}а̀~\sdash~{рꙋ́\large ц}ѣ~\sdash~рꙋ́{\large ц}ы},
        \hspca\slv{сно{\large х}а̀~\sdash~сно{\large с}ѣ̀~\sdash~сно́{\large с}и}
    }

    Исключение из этого правила составляет слово {\slv{па́сха}} ({\slv{па́сцѣ}}).


                \subsubsection{Имена существительные первого склонения с основой на шипящие и на {\slv{ц}}}

    Имена существительные с окончанием на {\slv{-а}}, но с основой на шипящие {\slv{ж, ч, ш, щ}}, а также на {\slv{ц}} имеют в некоторых падежах окончания твердого склонения, а в некоторых~---~мягкого.
    
    Возьмем для образца склонения таких существительных: {\slv{дꙋша̀}} и {\slv{дѣви́ца}}.
    
    \medskip\begin{placedtabular}[%
%        caption={\tabcaptsize Склонения существительных {\slv{дꙋша̀}} и {\slv{дѣви́ца}}}
        ]{|c|c|c|c|c|}
        \hline
        
        \mkcella{Па-\\деж}
        & \multicolumn{2}{c|}{\mkcellb{Единственное число}}
        & \multicolumn{2}{c|}{\mkcellb{Множественное число}}
        \\\hline
        
        И.
        & {\slv{дꙋша̀}}
        & {\slv{дѣви́ца}}
        & {\slv{дꙋ́ши}}
        & {\slv{дѣви̑цы}}
        \\\hline
        
        Р.
        & {\slv{дꙋшѝ}}
        & {\slv{дѣви́цы}}
        & {\slv{дꙋ́шъ}}
        & {\slv{дѣви́цъ}}
        \\\hline
        
        Д.
        & {\slv{дꙋшѝ}}
        & {\slv{деви́цѣ}}
        & {\slv{дꙋша́мъ}}
        & {\slv{дѣви́цамъ}}
        \\\hline
        
        В.
        & {\slv{дꙋ́шꙋ}}
        & {\slv{дѣви́цꙋ}}
        & {\slv{дꙋ́ши}}
        & {\slv{дѣви́цъ}}
        \\\hline
        
        З.
        & {\slv{дꙋшѐ}}
        & \makecell{{\slv{дѣви́це}}}
        & {\slv{дꙋ́шы}}
        & {\slv{дѣви̑цы}}
        \\\hline
        
        Т.
        & {\slv{дꙋше́ю}}
        & {\slv{дѣви́цею}}
        & {\slv{дꙋша́ми}}
        & {\slv{дѣви́цами}}
        \\\hline
        
        П.
        & {\slv{ѡ҆ дꙋшѝ}}
        & {\slv{ѡ҆ дѣви́цѣ}}
        & \makecell{{\slv{ѡ҆ дꙋша́хъ}}}
        & {\slv{ѡ҆ дѣви́цахъ}}
        \\\hline
        
        \makecell{~}
        & \multicolumn{2}{c|}{\mkcellb{Двойственное число}}
        \\\cline{1-3}
        
        \makecell{И.\\В. З.}
        & {\slv{дꙋши̑}}
        & {\slv{дѣви̑цы}}
        \\\cline{1-3}
        
        \makecell{Р. П.}
        & \makecell{\slv{дꙋ́шꙋ}}
        & {\slv{дѣви̑цꙋ}}
        \\\cline{1-3}
        
        \makecell{Д. Т.}
        & \makecell{{\slv{дꙋша́ма}}}
        & \makecell{{\slv{дѣви́цами}}}
        \\\cline{1-3}
        
    \end{placedtabular}

    \bigskip
    1. Имена существительные с основой на шипящие в дательном и предложном падежах единственного числа имеют окончание {\slv{-и}}, а не {\slv{-ѣ}}, а некоторые из таких существительных имеют также в родительном падеже множественного числа две формы, например: {\slv{кꙋ́ща~\sdash~кꙋ́щей}} и {\slv{кꙋ́щъ}}.
    
    В винительном падеже множественного числа эти существительные имеют окончание {\slv{-ы}} ({\slv{дꙋ́шы}}) для различения этого падежа от именительного того же числа, которые имеют окончание {\slv{-и}} ({\slv{дꙋ́ши}}).
    
    2. Имена существительные с основой на {\slv{ц}} вместо окончания {\slv{-и}} везде имеют окончание {\slv{-ы}} или {\slv{-ѣ}} ({\slv{дѣви́цы, ѡ҆ дѣви́цѣ}}).

                \subsubsection{Второе склонение имен существительных}

    Ко второму склонению имен существительных относятся имена \textbf{мужского} рода с окончанием на {\slv{-ъ}} (твердое склонение) и с окончанием на {\slv{-ь}} и {\slv{-й}} (мягкое склонение), имена \textbf{среднего} рода с окончанием на {\slv{-о}} (твердое склонение) и с окончанием на {\slv{-е}} (мягкое склонение).
    
    \bigskip\mockitem{1. Твердое склонение}
    \medskip
    
    Приведем образцы твердого склонения имен существительных мужского рода: {\slv{сы́нъ, ра́бъ, дꙋ́хъ, проро́къ}}~---~и среднего рода: {\slv{лѣ́то}}.
    
    \begin{placedtabular}[%
%        caption={\tabcaptsize Склонения существительных {\slv{сы́нъ, ра́бъ, дꙋ́хъ, проро́къ, лѣ́то}}}
        font=\shrunkensize
        ]{|c|c|c|c|c|c|c|}
        \hline
        
        \multirow{7}{*}{\spheading[11.4em]{Единственное число}}
        & И.
        & {\slv{сы́нъ}}
        & {\slv{ра́бъ}}
        & {\slv{дꙋ́хъ}}
        & {\slv{проро́къ}}
        & {\slv{лѣ́то}}
        \\\cline{2-7}
        
        & Р.
        & {\slv{сы́на}}
        & {\slv{раба̀}}
        & {\slv{дꙋ́ха}}
        & {\slv{проро́ка}}
        & {\slv{лѣ́та}}
        \\\cline{2-7}
        
        & Д.
        & \makecell{{\slv{сы́нꙋ}}\\{\slv{сы́нови}}}
        & {\slv{рабꙋ̀}}
        & \makecell{{\slv{дꙋ́хꙋ}}\\{\slv{дꙋ́хови}}}
        & {\slv{проро́кꙋ}}
        & {\slv{лѣ́тꙋ}}
        \\\cline{2-7}
        
        & В.
        & {\slv{сы́на}}
        & {\slv{раба̀}}
        & \makecell{{\slv{дꙋ́хъ}}\\{\slv{дꙋ́ха}}}
        & {\slv{проро́ка}}
        & \multirow{2}{*}{{\slv{лѣ́то}}}
        \\\cline{2-6}
        
        & З.
        & {\slv{сы́не}}
        & \makecell{{\slv{ра́бе}}}
        & {\slv{дꙋ́ше}}
        & {\slv{проро́че}}
        &
        \\\cline{2-7}
        
        & Т.
        & {\slv{сы́номъ}}
        & {\slv{рабо́мъ}}
        & {\slv{дꙋ́хомъ}}
        & {\slv{проро́комъ}}
        & {\slv{лѣ́томъ}}
        \\\cline{2-7}
        
        & П.
        & {\slv{ѡ҆ сы́нѣ}}
        & {\slv{ѡ҆ рабѣ̀}}
        & \makecell{{\slv{ѡ҆ дꙋ́се}}}
        & {\slv{ѡ҆ проро́цѣ}}
        & {\slv{ѡ҆ лѣ́тѣ}}
        \\\hline
        
        \multirow{3}{*}{\spheading[4.7em]{Дв. число}}
        & \makecell{И.\\В. З.}
        & {\slv{сы̑на}}
        & {\slv{раба̑}}
        & {\slv{дꙋ̑ха}}
        & {\slv{прорѡ́ка}}
        & {\slv{лѣ̑та}}
        \\\cline{2-7}
        
        & Р. П.
        & {\slv{сы̑нꙋ}}
        & {\slv{рабꙋ̑}}
        & {\slv{дꙋ̑хꙋ}}
        & {\slv{прорѡ́кꙋ}}
        & {\slv{лѣ̑тꙋ}}
        \\\cline{2-7}
        
        & Д. Т.
        & {\slv{сыно́ма}}
        & {\slv{рабо́ма}}
        & {\slv{дꙋ́хома}}
        & {\slv{проро́кома}}
        & {\slv{лѣ́тома}}
        \\\hline
        
        \multirow{7}{*}{\spheading[12.7em]{Множественное число}}
        & И.
        & \makecell{{\slv{сы́ны}}\\{\slv{сы́нове}}}
        & {\slv{рабѝ}}
        & {\slv{дꙋ́си}}
        & {\slv{проро́цы}}
        & {\slv{лѣ̑та}}
        \\\cline{2-7}
        
        & Р.
        & {\slv{сынѡ́въ}}
        & \makecell{{\slv{ра̑бъ}}\\{\slv{рабѡ́въ}}}
        & {\slv{дꙋхѡ́въ}}
        & \makecell{{\slv{проро́кѡвъ}}\\{\slv{прорѡ́къ}}}
        & {\slv{лѣ́тъ}}
        \\\cline{2-7}
        
        & Д.
        & \makecell{{\slv{сынѡ́мъ}}\\{\slv{сыновѡ́мъ}}}
        & {\slv{рабѡ́мъ}}
        & \makecell{{\slv{дꙋхѡ́мъ}}\\{\slv{дꙋховѡ́мъ}}}
        & {\slv{проро́кѡмъ}}
        & {\slv{лѣ́тѡмъ}}
        \\\cline{2-7}
        
        & В.
        & \makecell{{\slv{сыны̀}}\\{\slv{сынѡ́въ}}}
        & \makecell{{\slv{рабы̀}}\\{\slv{рабѡ́въ}}}
        & {\slv{дꙋ́хи}}
        & {\slv{проро́ки}}
        & \multirow{2}{*}{{\slv{лѣ̑та}}}
        \\\cline{2-6}
        
        & З.
        & {\slv{сы́ны}}
        & \makecell{{\slv{рабѝ}}}
        & {\slv{дꙋ́си}}
        & {\slv{проро́цы}}
        &
        \\\cline{2-7}
        
        & Т.
        & \makecell{{\slv{сы̑ны}}\\{\slv{сы́нми}}}
        & \makecell{{\slv{рабы̑}}\\{\slv{раба́ми}}}
        & {\slv{дꙋ̑хи}}
        & \makecell{{\slv{прорѡ́ки}}\\{\slv{проро́ками}}}
        & {\slv{лѣ́ты}}
        \\\cline{2-7}
        
        & П.
        & {\slv{ѡ҆ сынѣ́хъ}}
        & {\slv{ѡ҆ рабѣ́хъ}}
        & \makecell{{\slv{ѡ҆ дꙋ́сѣхъ}}}
        & {\slv{ѡ҆ проро́цѣхъ}}
        & {\slv{ѡ҆ лѣ́техъ}}
        \\\hline
        
    \end{placedtabular}

    \bigskip
    Существительные с основой на {\slv{ц}} в Р., Д., П. падежах ед. ч. имеют окончание твердого варианта склонения ({\slv{о҆трокови́ца~\sdash~о҆трокови́цѣ}}).
    
    \pagebreak\bigskip\mockitem{2. Мягкое склонение}
    \medskip
    
    Приведем образцы мягкого склонения имен существительных мужского рода: {\slv{па́стырь, жре́бїй}}; среднего рода: {\slv{мо́ре, писа́нїе}}.
    
    \begin{placedtabular}[%
%        caption={\tabcaptsize Склонения существительных {\slv{па́стырь, жре́бий, мо́ре, писа́нїе}}}
        ]{|c|c|c|c|c|c|}
        \hline
        
        \multirow{7}{*}{\spheading[11.4em]{\small Единственное число}}
        & И.
        & {\slv{па́стырь}}
        & {\slv{жре́бїй}}
        & {\slv{мо́ре}}
        & {\slv{писа́нїе}}
        \\\cline{2-6}
        
        & Р.
        & {\slv{па́стырѧ}}
        & {\slv{жре́бїѧ}}
        & {\slv{мо́рѧ}}
        & {\slv{писа́нїѧ}}
        \\\cline{2-6}
        
        & Д.
        & \makecell{{\slv{па́стырю}}\\{\slv{па́стыреви}}}
        & {\slv{жре́бїю}}
        & {\slv{мо́рю}}
        & {\slv{писа́нїю}}
        \\\cline{2-6}
        
        & В.
        & {\slv{па́стырѧ}}
        & {\slv{жре́бїй}}
        & \multirow{2}{*}{{\slv{мо́ре}}}
        & {\slv{писа́нїе}}
        \\\cline{2-4}\cline{6-6}
        
        & З.
        & {\slv{па́стырю}}
        & \makecell{{\slv{жре́бїе}}}
        &
        & {\slv{писа́ние}}
        \\\cline{2-6}
        
        & Т.
        & {\slv{па́стыремъ}}
        & {\slv{жре́бїемъ}}
        & {\slv{мо́ремъ}}
        & {\slv{писа́нїем}}
        \\\cline{2-6}
        
        & П.
        & \makecell{{\slv{ѡ҆ па́стырѣ}}\\{\slv{ѡ҆ па́стыри}}}
        & {\slv{ѡ҆ жре́бїи}}
        & {\slv{ѡ҆ мо́ри}}
        & {\slv{ѡ҆ писа́нїи}}
        \\\hline
        
        \multirow{3}{*}{\spheading[5.1em]{\small Дв. число}}
        & \makecell{И.\\В. З.}
        & {\slv{па̑стырѧ}}
        & {\slv{жрє́бїѧ}}
        & {\slv{мѡ́рѧ}}
        & {\slv{писа̑нїѧ}}
        \\\cline{2-6}
        
        & Р. П.
        & {\slv{па̑стырю}}
        & {\slv{жрє́бию}}
        & {\slv{мѡ́рю}}
        & {\slv{писа̑нию}}
        \\\cline{2-6}
        
        & Д. Т.
        & \makecell{{\slv{па́стырема}}\\{\slv{па́стырьма}}}
        & {\slv{жре́бїима}}
        & {\slv{мо́рема}}
        & {\slv{писа́нїима}}
        \\\hline
        
        \multirow{7}{*}{\spheading[12.7em]{\small Множественное число}}
        & И.
        & {\slv{па́стырїе}}
        & {\slv{жрє́бїи}}
        & {\slv{морѧ̀}}
        & {\slv{писа̑нїѧ}}
        \\\cline{2-6}
        
        & Р.
        & {\slv{па́стырей}}
        & \makecell{{\slv{жрє́бїй}}\\{\slv{жре́бїевъ}}}
        & {\slv{море́й}}
        & {\slv{писа́ний}}
        \\\cline{2-6}
        
        & Д.
        & {\slv{па́стырємъ}}
        & {\slv{жре́бїємъ}}
        & {\slv{мѡ́рем}}
        & {\slv{писа́нїємъ}}
        \\\cline{2-6}
        
        & В.
        & \makecell{{\slv{па́стыри}}\\{\slv{па́стырей}}}
        & {\slv{жрє́бїи}}
        & \multirow{2}{*}{{\slv{морѧ̀}}}
        & \multirow{2}{*}{{\slv{писа̑нїѧ}}}
        \\\cline{2-4}
        
        & З.
        & {\slv{па́стырїе}}
        & \makecell{{\slv{жрє́бїи}}}
        &
        &
        \\\cline{2-6}
        
        & Т.
        & \makecell{{\slv{па̑стыри}}\\{\slv{па́стырьми}}}
        & {\slv{жре́бїими}}
        & \makecell{{\slv{мѡ́ри}}\\{\slv{морѧ́ми}}}
        & \makecell{{\slv{писа̑нїи}}\\{\slv{писа́нми}}}
        \\\cline{2-6}
        
        & П.
        & {\slv{ѡ҆ па́стырехъ}}
        & \makecell{{\slv{ѡ҆ жре́бїихъ}}\\{\slv{ѡ҆ жре́бїѧхъ}}}
        & \makecell{{\slv{ѡ҆ мо́рехъ}}\\{\slv{ѡ҆ морѧ́хъ}}}
        & {\slv{ѡ҆ писа́нїихъ}}
        \\\hline
        
    \end{placedtabular}

                \subsubsection{Общие замечания ко второму склонению имен существительных}

    Во втором склонении имен существительных необходимо обратить внимание на следующие особенности:
    
    1. Имена существительные {\slv{ве́рхъ, до́мъ}} в родительном падеже единственного числа оканчиваются на {\slv{-ꙋ}} вместо {\slv{-а}}. Например:
    
    \bigskip\autorows{l}{1}{l}{
        \hspca{{\slv{И̑ ведо́ша є҆го̀}} ({\slv{і҆и҃са}}) {\slv{до верхꙋ̀ горы̀}} (Лк. 4, 29)},
        \hspca{{\slv{Во дво́рѣхъ до́мꙋ гдⷭ҇нѧ}} (Пс. 115, 10)}
    }

    2. Имена существительные собственные, а также обозначающие звание или профессию, в дательном падеже единственного числа принимают наращение {\slv{-ови}} в твердом склонении и {\slv{-еви}} в мягком склонении. Например:
    
    \bigskip\autorows{l}{1}{l}{
        \hspca{\slv{И̑ дꙋ́хови твоемꙋ̀}},
        \hspca{{\slv{Бж҃е, сꙋ́дъ тво́й царе́ви да́ждь}} (Пс. 71, 1)}
    }

    Наращенное окончание {\slv{-ови}} в дательном падеже употребляется также в словах иностранных, вошедших в церковнославянский язык, хотя бы это слово и было мягкого склонения. Например:
    
    \bigskip\autorows{l}{1}{l}{
        \hspca{{\slv{Почерпни́те ны́нѣ и҆ принеси́те а҆рхїтрїклі́нови}} (Ин. 2, 9)},
        \hspca{{\slv{А҆́ще бо бы́сте вѣ́ровали мѡѷсе́ови, вѣ́ровали}}},
        \hspca{{\slv{бы́сте ᲂу҆́бо и҆ мнѣ}} (Ин. 5, 46)},
        \hspca{{\slv{Во́ини же є҆́мше і҆и҃са, ведо́ша къ каїа́фѣ а҆рхїере́ови}} (Мф. 26, 57)}
    }

    3. Винительный падеж единственного числа при именах одушевленных сходен с родительным, а при именах неодушевленных~---~с именительным. Но иногда и при именах одушевленных предметов винительный падеж сходен с именительным. Например:
    
    \bigskip\autorows{l}{1}{l}{
        \hspca{{\slv{Приве́дше теле́цъ ᲂу҆пита́нный заколи́те}} (Лк. 15, 23)}
    }

    4. В именах существительных мужского рода при окончании именительного падежа единственного числа на {\slv{-ъ}} и {\slv{й}} звательный падеж того же числа имеет окончание {\slv{-е}}: {\slv{о҆́тче, фарїсе́е, андре́е}}, а при окончании именительного падежа на {\slv{-ь}}~---~окончание {\slv{-ю}}: {\slv{царю̀, па́стырю, ᲂу҆чи́телю}}; но если основа оканчивается на свистящий {\slv{-з}}, то окончание в звательном падеже бывает {\slv{-е}}: {\slv{кнѧ́зь~\sdash~кнѧ́же}}.
    
    5. Предложный падеж единственного числа может оканчиваться на {\slv{-ѣ}}, {\slv{-и}}, и {\slv{-ꙋ}}. Например: {\slv{ѡ҆ свѣ́тѣ, въ раѝ, въ домꙋ̀, ѡ҆ челѣ̀, ѡ҆ мо́ри}}.
    
    6. Именительный, винительный и звательный падежи двойственного числа имеют окончание {\slv{-а}} в твердом склонении и окончание {\slv{-ѧ}} в мягком склонении. Например:
    
    \bigskip\autorows{l}{1}{l}{
        \hspca{{\slv{Человѣ́къ нѣ́кий и҆мѣ два̀ сы̑на}} (Лк. 15, 11)},
        \hspca{{\slv{И̑ сѐ, два̀ жрє́бїѧ мета́ша}}}
    }

    7. Родительный и предложный падежи этого же числа оканчиваются на {\slv{-ꙋ}} в твердом склонении и на {\slv{-ю}} в мягком склонении. Например:
    
    \bigskip\autorows{l}{1}{l}{
        \hspca{{\slv{Сн҃ъ є҆диноро́дный... во двою̀ є҆стествꙋ̑ несли́тнѡ}}},
        \hspca{{\slv{познава́емый}} (Догм. гл. 6)}
    }

    8. Дательный и творительный падежи двойственного числа оканчиваются на {\slv{-ома}} в твердом склонении и на {\slv{-ема}} или {\slv{-има}} в мягком склонении. Например: {\slv{сыно́ма, царе́ма, писа́нїима}}.
    
    9. Именительный и звательный падежи множественного числа иногда оканчиваются на {\slv{-ове}}, или на {\slv{-їе}}, или на {\slv{-є}}. Например:
    
    \bigskip\autorows{l}{1}{l}{
        \hspca{{\slv{Сы́нове же ца́рствїѧ и҆згна́нии бꙋ́дꙋтъ}} (Мф. 8, 12)},
        \hspca{\slv{Па́стырїе и҆ ᲂу҆чи́телїе вселе́́нныѧ, моли́те}},
        \hspca{{\slv{ѡ҆ насъ грѣ́шныхъ}} (Вел. повеч.)},
        \hspca{{\slv{Фарїсе́є ше́дше, совѣ́тъ сотвори́ша на него̀}} (Мф. 8, 12)}
    }

    10. В родительном падеже множественного числа почти все имена существительные имеют:
    
    а) или окончание {\slv{-ѡвъ, -євъ, -ей, -ъ, -й}}, например: {\slv{рогѡ́въ, крає́въ, царе́й, лѣ́тъ, писа́нїй}};
    
    б) или окончание, созвучное с именительным падежом единственного числа, например:
    
    \bigskip\autorows{l}{1}{l}{
        \hspca{{\slv{Ка́мени запеча́танꙋ ѿ і҆ꙋдє́й}} (вместо: {\slv{ѿ і҆ꙋде́євъ}})},
        \hspca{{\slv{Нѣ́цыи ѿ кни̑жникъ}} (вместо: {\slv{ѿ кни́жникѡвъ}})}
    }

    11. Дательный падеж множественного числа оканчивается на {\slv{-ѡмъ}} (или на {\slv{-овѡмъ}} при наращении), на {\slv{-ємъ}}. Например:
    
    \bigskip\autorows{l}{1}{l}{
        \hspca{{\slv{Речѐ гдⷭ҇ь свои̑мъ ᲂу҆ченикѡмъ}}},
        \hspca{{\slv{Тогда̀ і҆и҃съ нача́тъ поноша́ти градовѡмъ}} (Лк. 11, 20)},
        \hspca{{\slv{Заповѣ́да о҆тцє́мъ сказа́ти сыновѡмъ свои̑мъ}} (Пс. 77, 5)}
    }

    12. Творительный падеж множественного числа часто принимает сокращенную форму, а именно: вместо окончания {\slv{-ами}} или {\slv{ѧми}} принимает окончание {\slv{-ы}} или {\slv{-и}}. Например:
    
    \bigskip\autorows{l}{1}{l}{
        \hspca{{\slv{Досто́инъ є҆сѝ во всѧ̑ времена̀ пѣ́ть бы́ти гла̑сы}}},
        \hspca{{\slv{прпⷣбными}} (Послед. вечерни)},
        \hspca{{\slv{Ꙗ҆́вльшїѧсѧ же ны́нѣ, писа̑нїи прⷪ҇ро́ческими, по повелѣ́нїю}}},
        \hspca{{\slv{вѣ́чнагѡ бг҃а}} (Рим. 14, 25)}
    }

    Конечно, в этом случае эта форма творительного падежа различается от созвучной иногда ему формы винительного падежа или начертанием букв, или облегченным ударением.
    
    13. В предложном падеже множественного числа с основой на согласный (но не шипящий) звук может быть окончание {\slv{-ѣхъ}} и {\slv{-ехъ}}. Окончание {\slv{-ѣхъ}} пишется тогда, когда в русском языке в соответствующем слове пишется окончание \textbf{-ах}, а {\slv{-ехъ}} пишется тогда, когда по-русски пишется \textbf{-ях}. Например: {\slv{ѡ҆ сынѣ́хъ}} (о сынах), {\slv{ѡ҆ па́стырехъ}} (о пастырях).
    
    14. Имена существительные с основой на гортанный звук смягчают его перед гласными {\slv{е}} и {\slv{и}} по известным законам смягчения. Но в винительном и творительном падежах множественного числа, в отличие их от именительного, гортанные не смягчаются. Например:
    
    \medskip\autorows{l}{1}{l}{
        \hspca{{\slv{Вни́де са́мъ}} ({\slv{і҆и҃съ}}) {\slv{и҆ ᲂу҆чн҃цы}} (им. п. мн. ч.) {\slv{є҆гѡ̀}} (Ин. 18, 1)},
        \hspca{{\slv{Пое́мъ}} ({\slv{і҆и҃съ}}) {\slv{ѻ҆бана́десѧте ᲂу҆чн҃ки}} (вин. п. мн. ч.) {\slv{своѧ̑}} (Лк. 18, 31)},
        \hspca{{\slv{Сѣдѧ́ше}} ({\slv{і҆и҃съ}}) {\slv{со ᲂу҆чн҃кѝ}} (тв. п. мн. ч.) {\slv{свои́ми}} (Ин. 6, 3)},
        \hspca{{\slv{Цр҃ь нбный за чл҃вѣколю́иїе на землѝ ꙗви́сѧ и҆ съ человѣ̑ки}}},
        \hspca{(тв. п. мн. ч.) {\slv{поживѐ}} (Догм. гл. 8)}
    }

    15. Имена существительные среднего рода, имеющие перед гласным окончанием две и более согласных, принимают в некоторых косвенных падежах беглый гласный. Например: {\slv{дно̀~\sdash~дѡ́нъ, ѕло̀~\sdash~ѕѡ́лъ}}.
    
    Употребляемые только во множественном числе существительные среднего рода: {\slv{врата̀, чре́сла}} и др.~---~склоняются по образцу существительного {\slv{лѣ́то}} во множественном числе.
    
    16. Имена существительные, взятые с еврейского языка, как например: {\slv{а҆донаі̀, равві̀}} и подобные им, не склоняются. Иногда не склоняется и имя {\slv{і҆и҃съ}}, когда это имя стоит не одно, а в сочетании с именем {\slv{хрⷭ҇то́съ}}. Например:
    
    \bigskip\autorows{l}{1}{l}{
        \hspca{{\slv{Блⷣгть же и҆ и҆́стина і҆и҃съ хрⷭ҇то́мъ бы́сть}} (Ин. 1, 17)}
    }
    \pagebreak

    
                \subsubsection{Имена существительные второго склонения с основой на шипящие и на {\slv{ц}}}

    Приведем образцы склонений таких существительных: {\slv{мꙋ́жъ, лицѐ}}.
    
    \medskip\begin{placedtabular}[%
%        caption={\tabcaptsize Склонения существительных {\slv{мꙋ́жъ}} и {\slv{лицѐ}}}
        ]{|c|c|c|c|c|}
        \hline
        
        \mkcella{Па-\\деж}
        & \multicolumn{2}{c|}{\mkcellb{Единственное число}}
        & \multicolumn{2}{c|}{\mkcellb{Множественное число}}
        \\\hline
        
        И.
        & {\slv{мꙋ́жъ}}
        & {\slv{лицѐ}}
        & {\slv{мꙋ́жїе}}
        & {\slv{ли́ца}}
        \\\hline
        
        Р.
        & {\slv{мꙋ́жа}}
        & {\slv{лица̀}}
        & \makecell{{\slv{мꙋ̑жъ}}\\{\slv{мꙋже́й}}}
        & {\slv{ли́цъ}}
        \\\hline
        
        Д.
        & \makecell{{\slv{мꙋ́жꙋ}}\\{\slv{мꙋ́жеви}}}
        & {\slv{лицꙋ̀}}
        & {\slv{мꙋжє́мъ}}
        & {\slv{ли́цам}}
        \\\hline
        
        В.
        & {\slv{мꙋ́жа}}
        & {\slv{лицѐ}}
        & {\slv{мꙋ́жы}}
        & \makecell{{\slv{ли́ца}}\\{\slv{лицы̀}}}
        \\\hline
        
        З.
        & {\slv{мꙋ́жꙋ}}
        & \makecell{{\slv{лицѐ}}}
        & {\slv{мꙋ́жїе}}
        & \makecell{{\slv{ли́ца}}}
        \\\hline
        
        Т.
        & {\slv{мꙋ́жемъ}}
        & {\slv{лице́мъ}}
        & \makecell{{\slv{мꙋ̑жи}}\\{\slv{мꙋжа́ми}}}
        & \makecell{{\slv{ли̑цы}}\\{\slv{ли́цами}}}
        \\\hline
        
        П.
        & {\slv{ѡ҆ мꙋ́жи}}
        & {\slv{ѡ҆ лицѣ̀}}
        & \makecell{{\slv{ѡ҆ мꙋ́жахъ}}\\{\slv{ѡ҆ мꙋже́хъ}}}
        & {\slv{ѡ҆ ли́цахъ}}
        \\\hline
        
        \mkcella{~\\~}
        & \multicolumn{2}{c|}{\mkcellb{Двойственное число}}
        \\\cline{1-3}
        
        \makecell{И.\\В. З.}
        & {\slv{мꙋ̑жа}}
        & {\slv{лица̑}}
        \\\cline{1-3}
        
        Р. П.
        & {\slv{мꙋ̑жꙋ}}
        & {\slv{лицꙋ̑}}
        \\\cline{1-3}
        
        Д. Т.
        & {\slv{мꙋже́ма}}
        & {\slv{лице́ма}}
        \\\cline{1-3}
        
    \end{placedtabular}

    \bigskip
    Склонение имен существительных с основой на шипящие и на {\slv{ц}} имеет свои особенности.
    
    1. Имена существительные мужского рода с основой на шипящий {\slv{ч}} имеют в именительном падеже единственного числа окончание {\slv{-ь}}. Например: {\slv{ме́чь, вра́чь, клю́чь}}.
    
    2. Звательный падеж единственного числа с основой на шипящие {\slv{ж}} и {\slv{ч}} оканчивается на {\slv{-ꙋ}}: {\slv{мꙋ́жꙋ, врачꙋ̀}}.
    
    3. Предложный падеж единственного числа с основой на {\slv{ж}} и {\slv{щ}} оканчивается на {\slv{-и}}, а с основой на {\slv{ц}} оканчивается на {\slv{ѣ}}. Например: {\slv{ѡ҆ мꙋ́жи, на ло́жи, въ со́нмищи, ѡ҆ лицѣ̀}}.
    
    Но существительное {\slv{се́рдце}} имеет иногда в предложном падеже окончание {\slv{-ы}}: {\slv{Лю́дїе под̾ тобо́ю падꙋ́тъ въ се́рдцы вра̑гъ цр҃е́выхъ}} (Лк. 44, 6).
    
    4. Дательный и творительный падежи двойственного числа существительных {\slv{мꙋ́жъ}} и {\slv{се́рдце}} имеют формы: {\slv{мꙋже́ма, сердца́ма}}.
    
    5. Именительный и звательный падежи множественного числа с основой на {\slv{ч}} имеют окончание {\slv{-еве}}: {\slv{Вра́чеве воскресѧ́тъ и҆ и҆сповѣ́дѧтсѧ тебѣ̀}} (Пс. 87, 11).
    
    6. Родительный падеж множественного числа существительных среднего рода с основой на {\slv{ж, щ, ц}} оканчивается на {\slv{-ъ}}: {\slv{ло́жъ, сокро́вищъ, ли́цъ}}.
    
    7. Имена существительные мужского рода с основой на {\slv{ж}} имеют в винительном падеже множественного числа окончание {\slv{-ы}} для различения от предложного падежа единственного числа, имеющего окончание {\slv{-и}}, а также от творительного падежа множественного числа, тоже имеющего в одной из своих форм окончание {\slv{-и}}. Например:
    
    \bigskip\autorows{l}{1}{l}{
        \hspca{{\slv{Оу҆смотри́те ᲂу҆̀бо бра́тїе, мꙋ́жы}} (вин. п. мн. ч.) {\slv{ѿтъ ва́съ}} (Деян. 6, 3)},
        \hspca{{\slv{Цари́ца ю҆́жскаѧ воста́нетъ на сꙋ́дъ съ мꙋ̑жи}} (тв. п. мн. ч.)},
        \hspca{{\slv{ро́да сегѡ̀}} (Лк. 11, 31)},
        \hspca{{\slv{Гдⷭ҇и, слы́шахъ ѿ мно́гихъ ѡ҆ мꙋ́жи}} (предл. п. ед. ч.)},
        \hspca{{\slv{се́мъ}} (Деян. 9, 13)}
    }

    8. В предложном падеже множественного числа существитльные с основой на {\slv{ж}} и {\slv{ч}} имеют двоякое окончание: {\slv{-ахъ}} и {\slv{-ехъ}} (но не {\slv{-ѣхъ}}), например {\slv{ѡ҆ мꙋ́жахъ, ѡ҆ мꙋже́хъ}}, а с основой на {\slv{щ}} могут оканчиваться на {\slv{-ахъ, -ихъ, -ехъ}}, например: {\slv{въ со́нмищахъ}}, или {\slv{въ со́нмищихъ}}, или {\slv{въ со́нмищехъ}}. И только в существительных с основой на {\slv{ц}} окончание бывает {\slv{-ахъ}}: {\slv{ѡ҆ ли́цахъ, ѡ҆ сердца́хъ}}.

                \subsubsection{Третье склонение имен существительных}

    К третьему склонению имен существительных относятся имена \textbf{женского} рода с окончанием на {\slv{-ь}}. Так как окончание {\slv{-ь}} является мягким, то это склонение только мягкое.
    
    К третьему склонению относятся лишь немногие имена существительные мужского рода на {\slv{-ь}}: {\slv{горта́нь, пꙋ́ть}}.
    
    {\slv{Госпо́дїе}}~---~во множественном числе удерживает окончания по третьему склонению.
    
    Возьмем для образца третьего склонения существительные {\slv{ча́сть}} и {\slv{це́рковь}}.
    
    \medskip\begin{placedtabular}[%
        caption={\tabcaptsize Склонения существительных {\slv{ча́сть}} и {\slv{це́рковь}}}
        ]{|c|c|c|c|c|}
        \hline
        
        \mkcella{Па-\\деж}
        & \multicolumn{2}{c|}{\mkcellb{Единственное число}}
        & \multicolumn{2}{c|}{\mkcellb{Множественное число}}
        \\\hline
        
        \makecell{И.\\В. З.}
        & {\slv{ча́сть}}
        & {\slv{це́рковь}}
        & {\slv{ча̑сти}}
        & {\slv{цє́ркви}}
        \\\hline
        
        Р.
        & {\slv{ча́сти}}
        & {\slv{це́ркве}}
        & {\slv{часте́й}}
        & {\slv{церкве́й}}
        \\\hline
        
        Д.
        & {\slv{ча́сти}}
        & {\slv{це́ркви}}
        & {\slv{часте́мъ}}
        & {\slv{це́рквамъ}}
        \\\hline
        
        Т.
        & {\slv{ча́стїю}}
        & {\slv{це́рковїю}}
        & {\slv{частмѝ}}
        & {\slv{це́рквами}}
        \\\hline
        
        П.
        & {\slv{ѡ҆ ча́сти}}
        & {\slv{ѡ҆ це́ркви}}
        & \makecell{{\slv{ѡ҆ часте́хъ}}}
        & {\slv{ѡ҆ це́рквахъ}}
        \\\hline
        
        \mkcella{~\\~}
        & \multicolumn{2}{c|}{\mkcellb{Двойственное число}}
        \\\cline{1-3}
        
        \makecell{И.\\В. З.}
        & {\slv{ча̑сти}}
        & {\slv{цє́ркви}}
        \\\cline{1-3}
        
        \makecell{Р. П.}
        & \makecell{\slv{ча̑стїю}}
        & {\slv{цє́рковїю}}
        \\\cline{1-3}
        
        \makecell{Д. Т.}
        & \makecell{{\slv{часте́ма}}}
        & \makecell{{\slv{церква́ма}}}
        \\\cline{1-3}
        
    \end{placedtabular}

    \bigskip
    1. По образцу склонения существительного {\slv{ча́сть}} склоняется существительное {\slv{дла́нь}}, а также некоторые имена существительные, употребляемые только во множественном числе, как-то: {\slv{мо́щи, гꙋ́сли, ꙗ҆́сли, пе́рси}} и др.
    
    2. Некоторые имена существительные третьего склонения, склоняющиеся по образцу существительного {\slv{це́рковь}}, имеют иногда в именительном падеже единственного числа двоякое окончание, например: {\slv{любо́вь}} и {\slv{любы̀}}, {\slv{свекро́вь}} и {\slv{свекры̀}}, {\slv{неплодо́вь}} и {\slv{непло́ды}}.
    
    3. Имя существительное {\slv{мꙋ́дрость}}, относимое к Иисусу Христу, имеет в звательном падеже единственного числа окончание {\slv{-е}}. Например: {\slv{Ѽ мꙋ́дросте, и҆ сло́ве бж҃їй, и҆ си́ло!}} (Тропарь 9-й песни канона Пасхи).

            \subsection{Имя прилагательное}
                \subsubsection{Разделение имен прилагательных}

    \textbf{Именем прилагательным} называется часть речи, обозначающая признак предмета, явления или понятия. Например: {\slv{пра́ведный мꙋ́жъ, ве́лїѧ сла́ва, го́рнее мѣ́сто}}.
    
    Имена прилагательные, как обозначающие признаки предметов, явлений или понятий, могут указывать или на их качество, или на отношение предмета к материалу, месту, времени и т.д. Поэтому имена прилагательные разделяются на \emph{качественные} ({\slv{лю́тый звѣ́рь, блага́ѧ вѣ́сть, до́брое дѣ́ло}}) и \emph{относительные} ({\slv{зла́тъ вѣне́цъ, морска́ѧ волна̀, нощно́е вре́мѧ}}).
    
    Особую разновидность относительных имен прилагательных составляют так называемые имена прилагательные {\slv{притяжательные}}, которые имеют значение принадлежности. Например: {\slv{Но́евъ ковче́гъ, хра́мъ і҆ерⷭ҇ли́мль}}.
    
    По характеру окончаний имена прилагательные разделяются на:
    
    \medskip\autorows{l}{1}{l}{
        \hspca{\emph{краткие} ({\slv{до́бръ, зла́тъ}}) и \emph{полные} ({\slv{до́брый, златы́й}})},
        \hspca{\emph{твердые} ({\slv{мꙋ́дръ, мꙋ́дрый}}) и \emph{мягкие} ({\slv{си́нїй, си́нь}})}
    }

                \subsubsection{Краткие имена прилагательные}

    Краткие имена прилагательные по своим окончаниям сходны с родовыми окончаниями имен существительных, а именно:
    
    \bigskip\autorows{l}{3}{l}{
        \hspca{мужской род}, {оканчивается на}, {{\slv{-ъ, -ь}} ({\slv{мꙋ́дръ, си́нь}})},
        \hspca{женский род}, {оканчивается на}, {{\slv{-а, -ѧ}} ({\slv{мꙋ́дра, си́нѧ}})},
        \hspca{средний род}, {оканчивается на}, {{\slv{-о, -е}} ({\slv{мꙋ́дро, си́не}})}
    }

    \noindent
    а потому и склоняются по соответствующим склонениям имен существительных.
    \pagebreak

                \subsubsection{Склонение кратких имен прилагательных с твердым окончанием}

    Для образца возьмем склонение краткого имени прилагательного (совместно с существительным) с твердым окончанием: {\slv{до́бръ}} ({\slv{пло́дъ}}), {\slv{до́бра}} ({\slv{ри́за}}), {\slv{до́бро}} ({\slv{дѣ́ло}}).
    
    \begin{placedtabular}[%
%        caption={\tabcaptsize Склонения прилагательных {\slv{до́бръ}} ({\slv{пло́дъ}}), {\slv{до́бра}} ({\slv{ри́за}}), {\slv{до́бро}} ({\slv{дѣ́ло}})}
        ]{|c|c|c|c|c|}
        \hline

        ~
        & \mkcella{Па-\\деж}
        & \mkcellb{Мужской род}
        & \mkcellb{Женский род}
        & \mkcellb{Средний род}
        \\\hline
        
        \multirow{7}{*}{\spheading[10em]{\small Единственное число}}
        & И.
        & {\slv{до́бръ}} {\slva{пло́дъ}}
        & {\slv{до́бра}} {\slva{ри́за}}
        & {\slv{до́бро}} {\slva{дѣ́ло}}
        \\\cline{2-5}
        
        & Р.
        & {\slv{добра̀}} {\slva{плода̀}}
        & {\slv{добры̀}} {\slva{ри́зы}}
        & {\slv{до́бра}} {\slva{дѣ́ла}}
        \\\cline{2-5}
        
        & Д.
        & {\slv{до́брꙋ}} {\slva{плодꙋ̀}}
        & {\slv{до́брѣ}} {\slva{ри́зѣ}}
        & {\slv{до́брꙋ}} {\slva{дѣ́лꙋ}}
        \\\cline{2-5}
        
        & В.
        & {\slv{до́бръ}}({\slv{а̀}}) {\slva{пло́дъ}}
        & {\slv{до́брꙋ}} {\slva{ри́зꙋ}}
        & \multirow{2}{*}{{\slv{до́бро}} {\slva{дѣ́ло}}}
        \\\cline{2-4}
        
        & З.
        & {\slv{до́бръ}} {\slva{пло́де}}
        & {\slv{до́бра}} {\slva{ри́зо}}
        &
        \\\cline{2-5}
        
        & Т.
        & {\slv{до́брымъ}} {\slva{пло́домъ}}
        & {\slv{до́брою}} {\slva{ри́зою}}
        & {\slv{до́брым}} {\slva{дѣ́ломъ}}
        \\\cline{2-5}
        
        & П.
        & {\slv{ѡ҆ до́брѣ}} {\slva{плодѣ̀}}
        & {\slv{ѡ҆ до́брѣ}} {\slva{ри́зѣ}}
        & {\slv{ѡ҆ до́брѣ}} {\slva{дѣ́лѣ}}
        \\\hline
        
        \multirow{3}{*}{\spheading[4.5em]{\small Дв. число}}
        & \makecell{И.\\В. З.}
        & {\slv{дѡ́бра}} {\slva{плѡда̀}}
        & {\slv{дѡ́брѣ}} {\slva{ри̑зѣ}}
        & {\slv{дѡ́бра}} {\slva{дѣ̑ла}}
        \\\cline{2-5}
        
        & Р. П.
        & {\slv{дѡ́брꙋ}} {\slva{плѡдꙋ̀}}
        & {\slv{дѡ́брꙋ}} {\slva{ри̑зꙋ}}
        & {\slv{дѡ́брꙋ}} {\slva{дѣ̑лꙋ}}
        \\\cline{2-5}
        
        & Д. Т.
        & {\slv{до́брыма}} {\slva{пло́дома}}
        & {\slv{до́брыма}} {\slva{ри́зами}}
        & {\slv{до́брыма}} {\slva{дѣло́ма}}
        \\\hline
        
        \multirow{7}{*}{\spheading[10em]{\small Множественное число}}
        & И.
        & {\slv{до́бри}} {\slva{пло́ди}}
        & {\slv{дѡбры̀}} {\slva{ри̑зы}}
        & {\slv{дѡ́бра}} {\slva{дѣла̀}}
        \\\cline{2-5}
        
        & Р.
        & {\slv{дѡ́бръ}} {\slva{плѡ́дъ}}
        & {\slv{дѡ́бръ}} {\slva{ри́зъ}}
        & {\slv{дѡ́бръ}} {\slva{дѣ́лъ}}
        \\\cline{2-5}
        
        & Д.
        & {\slv{дѡ́брымъ}} {\slva{пло́дѡмъ}}
        & {\slv{дѡ́брымъ}} {\slva{ри́замъ}}
        & {\slv{дѡ́брымъ}} {\slva{дѣлѡ́мъ}}
        \\\cline{2-5}
        
        & В.
        & {\slv{добры̀}} {\slva{плоды̀}}
        & {\slv{дѡбры̀}} {\slva{ри̑зы}}
        & {\slv{дѡбра̀}} {\slva{дѣла̑}}
        \\\cline{2-5}
        
        & З.
        & {\slv{до́бри}} {\slva{пло́ди}}
        & {\slv{дѡбры̀}} {\slva{ри̑зы}}
        & {\slv{до́бри}} {\slva{дѣла̀}}
        \\\cline{2-5}
        
        & Т.
        & {\slv{дѡ́бры}} {\slva{плѡ́ды}}
        & {\slv{до́брыми}} {\slva{ри́зами}}
        & {\slv{дѡ́бры}} {\slva{дѣ́лы}}
        \\\cline{2-5}
        
        & П.
        & {\slv{ѡ҆ до́брыхъ}} {\slva{плодѣ́хъ}}
        & {\slv{ѡ҆ до́брыхъ}} {\slva{ри́захъ}}
        & {\slv{ѡ҆ до́брыхъ}} {\slva{дѣ́лѣхъ}}
        \\\hline
        
    \end{placedtabular}

    \bigskip
    Имена прилагательные в значении существительных в звательном падеже единственного числа принимают их окончания, например: {\slv{млⷭ҇тиве}}. В остальных случаях звательный падеж сходен с именительным.
    
    По образцу склонения кратких прилагательных с твердым окончанием склоняются имена прилагательные притяжательные, например: {\slv{сі́мѡновъ, царе́въ}} и др.
    
    Краткие прилагательные: {\slv{є҆диноро́дъ, сꙋгꙋ́бъ}} и т.п.~---~не склоняются.
    
    Имена прилагательные с основой на гортанные смягчают их в соответствующих падежах:
    
    \bigskip\autorows{l}{1}{l}{
        \hspca{{\slv{мно́гъ~\sdash~мно́зи, бла́гъ~\sdash~бла́зи, крѣ́покъ~\sdash~крѣ́пцы}}}
    }

    Краткие прилагательные в единственном числе творительном падеже мужского и среднего родов принимают окончания полных прилагательных {\slv{-ымъ, -имъ}}: {\slv{мꙋ́дрымъ, госпо́днимъ}}.

                \subsubsection{Склонение кратких имен прилагательных с мягким окончанием}

    Для образца возьмем склонение краткого имени прилагательного (совместно с существительным) с мягким окончанием: {\slv{си́нь}} ({\slv{пла́тъ}}), {\slv{си́нѧ}} ({\slv{пелена̀}}), {\slv{си́не}} ({\slv{мо́ре}}).

    \begin{placedtabular}[%
%        caption={\tabcaptsize Склонения прилагательных {\slv{си́нь}} ({\slv{пла́тъ}}), {\slv{си́нѧ}} ({\slv{пелена̀}}), {\slv{си́не}} ({\slv{мо́ре}})}
        ]{|c|c|c|c|c|}
        \hline
        
        ~
        & \mkcella{Па-\\деж}
        & \mkcellb{Мужской род}
        & \mkcellb{Женский род}
        & \mkcellb{Средний род}
        \\\hline
        
        \multirow{7}{*}{\spheading[10em]{\small Единственное число}}
        & И.
        & {\slv{си́нь}} {\slva{пла́тъ}}
        & {\slv{си́нѧ}} {\slva{пелена̀}}
        & {\slv{си́не}} {\slva{мо́ре}}
        \\\cline{2-5}
        
        & Р.
        & {\slv{си́нѧ}} {\slva{пла́та}}
        & {\slv{си́ни}} {\slva{пелены̀}}
        & {\slv{си́нѧ}} {\slva{мо́рѧ}}
        \\\cline{2-5}
        
        & Д.
        & {\slv{си́ню}} {\slva{пла́тꙋ}}
        & {\slv{си́ни}} {\slva{пеленѣ̀}}
        & {\slv{си́ню}} {\slva{мо́рю}}
        \\\cline{2-5}
        
        & В.
        & {\slv{си́нь}}({\slv{ѧ}}) {\slva{пла́тъ}}
        & {\slv{си́ню}} {\slva{пеленꙋ̀}}
        & \multirow{2}{*}{{\slv{си́не}} {\slva{мо́ре}}}
        \\\cline{2-4}
        
        & З.
        & {\slv{си́нь}} {\slva{пла́те}}
        & {\slv{си́нѧ}} {\slva{пелено̀}}
        &
        \\\cline{2-5}
        
        & Т.
        & {\slv{си́нимъ}} {\slva{пла́том}}
        & {\slv{си́нею}} {\slva{пелено́ю}}
        & {\slv{си́нимъ}} {\slva{мо́ремъ}}
        \\\cline{2-5}
        
        & П.
        & {\slv{ѡ҆ си́ни}} {\slva{пла́тѣ}}
        & {\slv{ѡ҆ си́ни}} {\slva{пеленѣ̀}}
        & {\slv{ѡ҆ си́ни}} {\slva{мо́ри}}
        \\\hline
        
        \multirow{3}{*}{\spheading[4.5em]{\small Дв. число}}
        & \makecell{И.\\В. З.}
        & {\slv{си̑ни}} {\slva{пла̑та}}
        & {\slv{си̑ни}} {\slva{пелєнѣ̀}}
        & {\slv{си̑ни}} {\slva{мѡ́рѧ}}
        \\\cline{2-5}
        
        & Р. П.
        & {\slv{си̑ню}} {\slva{пла̑тꙋ}}
        & {\slv{си̑ню}} {\slva{пелєнꙋ̀}}
        & {\slv{си̑ню}} {\slva{мѡ́рю}}
        \\\cline{2-5}
        
        & Д. Т.
        & {\slv{си́нима}} {\slva{пла́тома}}
        & {\slv{си́нима}} {\slva{пелена́ма}}
        & {\slv{си́нима}} {\slva{мо́рема}}
        \\\hline
        
        \multirow{6}{*}{\spheading[11em]{\small Множественное число}}
        & И. З.
        & {\slv{си̑ни}} {\slva{пла́ти}}
        & {\slv{си̑ни}} {\slva{пелєны̀}}
        & {\slv{си̑нѧ}} {\slva{мо́рѧ}}
        \\\cline{2-5}
        
        & Р.
        & {\slv{си̑нь}} {\slva{пла̑тъ}}
        & {\slv{си̑нь}} {\slva{пеле́нъ}}
        & {\slv{си̑нь}} {\slva{море́й}}
        \\\cline{2-5}
        
        & Д.
        & {\slv{си̑нимъ}} {\slva{пла́тѡмъ}}
        & {\slv{си̑нимъ}} {\slva{пелена́мъ}}
        & {\slv{си̑нимъ}} {\slva{мо́рємъ}}
        \\\cline{2-5}
        
        & В.
        & {\slv{си̑ни}} {\slva{пла̑ты}}
        & {\slv{си̑ни}} {\slva{пелєны̀}}
        & {\slv{си̑нѧ}} {\slva{морѧ̀}}
        \\\cline{2-5}
        
        & Т.
        & \makecell{{\slv{си̑ни}} {\slva{пла̑ты}}\\{\slv{си́ними}}}
        & {\slv{си́нѧми}} {\slva{пелена́ми}}
        & {\slv{си̑ни}} {\slva{мѡ́ри}} ({\slva{морѧ́ми}})
        \\\cline{2-5}
        
        & П.
        & {\slv{ѡ҆ си́нихъ}} {\slva{пла́тѣхъ}}
        & {\slv{ѡ҆ си́нѧхъ}} {\slva{пелена́хъ}}
        & \makecell{{\slv{ѡ҆ си́нихъ}} {\slva{мо́рехъ}}\\({\slva{морѧ́хъ}})}
        \\\hline
        
    \end{placedtabular}

    \bigskip
    По образцу склонения кратких прилагательных с мягким окончанием склоняются имена прилагательные притяжательные, например: {\slv{саꙋ́ль, і҆а́ковль}} и т.п.
    
    Краткие прилагательные, как например {\slv{свобо́дь}} и т.п., не склоняются.

                \subsubsection{Склонение кратких имен прилагательных с основой на шипящие}

    Приведем образец склонения краткого имени прилагательного (также совместно с существительными) с основой на шипящий звук {\slv{щ}}~---~{\slv{то́щь}} (напрасен, тщетен): {\slv{то́щь}} ({\slv{да́ръ}}), {\slv{то́ща}} ({\slv{бра́нь}}), {\slv{то́ще}} ({\slv{помышле́нїе}}).
    
    \begin{placedtabular}[%
%        caption={\tabcaptsize Склонение краткого имени прилагательного {\slv{то́щь}}}
        font=\shrunkensize
        ]{|c|c|c|c|c|}
        \hline
        
        ~
        & \makecell{Па-\\деж}
        & Мужской род
        & Женский род
        & Средний род
        \\\hline
        
        \multirow{7}{*}{\spheading[10em]{Единственное число}}
        & И.
        & {\slv{то́щь}} {\slva{да́ръ}}
        & {\slv{то́ща}} {\slva{бра́нь}}
        & {\slv{то́ще}} {\slva{помышле́нїе}}
        \\\cline{2-5}
        
        & Р.
        & {\slv{то́ща}} {\slva{да́ра}}
        & {\slv{то́щи}} {\slva{бра́ни}}
        & {\slv{то́ща}} {\slva{помышле́нїѧ}}
        \\\cline{2-5}
        
        & Д.
        & {\slv{то́щꙋ}} {\slva{да́рꙋ}}
        & {\slv{то́щи}} {\slva{бра́ни}}
        & {\slv{то́щꙋ}} {\slva{помышле́нїю}}
        \\\cline{2-5}
        
        & В.
        & {\slv{то́щь}}({\slv{а}}) {\slva{да́ръ}}
        & {\slv{то́щꙋ}} {\slva{бра́нь}}
        & \multirow{2}{*}{{\slv{то́ще}} {\slva{помышле́нїе}}}
        \\\cline{2-4}
        
        & З.
        & {\slv{то́щь}} {\slva{да́ре}}
        & {\slv{то́ща}} {\slva{бра́нь}}
        &
        \\\cline{2-5}
        
        & Т.
        & {\slv{то́щимъ}} {\slva{да́ромъ}}
        & {\slv{то́щею}} {\slva{бра́нїю}}
        & {\slv{то́щимъ}} {\slva{помышле́ниемъ}}
        \\\cline{2-5}
        
        & П.
        & {\slv{ѡ҆ то́щи}} {\slva{да́рѣ}}
        & {\slv{ѡ҆ то́щи}} {\slva{бра́ни}}
        & {\slv{ѡ҆ то́щи}} {\slva{помышле́нїи}}
        \\\hline
        
        \multirow{3}{*}{\spheading[4.5em]{Дв. число}}
        & \makecell{И.\\В. З.}
        & {\slv{тѡ́ща}} {\slva{да̑ра}}
        & {\slv{тѡ́щи}} {\slva{бра̑ни}}
        & {\slv{тѡ́щи}} {\slva{помышлє́нїѧ}}
        \\\cline{2-5}
        
        & Р. П.
        & {\slv{тѡ́щꙋ}} {\slva{да̑рꙋ}}
        & {\slv{тѡ́щꙋ}} {\slva{бра̑нїю}}
        & {\slv{тѡ́щꙋ}} {\slva{помышлє́нїю}}
        \\\cline{2-5}
        
        & Д. Т.
        & {\slv{то́щима}} {\slva{да́рома}}
        & {\slv{то́щима}} {\slva{бра́нема}}
        & {\slv{то́щима}} {\slva{помышле́нїима}}
        \\\hline
        
        \multirow{6}{*}{\spheading[10em]{Множественное число}}
        & И. З.
        & {\slv{тѡ́щи}} {\slva{да́ри}}
        & {\slv{тѡ́щи}} {\slva{бра̑ни}}
        & {\slv{тѡ́ща}} {\slva{помышлє́нїѧ}}
        \\\cline{2-5}
        
        & Р.
        & {\slv{тѡ́щь}} {\slva{да̑ръ}}
        & {\slv{тѡ́щь}} {\slva{бра́ней}}
        & {\slv{тѡ́щь}} {\slva{помышле́нїй}}
        \\\cline{2-5}
        
        & Д.
        & {\slv{то́щымъ}} {\slva{дарѡ́мъ}}
        & {\slv{то́щымъ}} {\slva{бра́немъ}}
        & {\slv{то́щымъ}} {\slva{помышле́нїємъ}}
        \\\cline{2-5}
        
        & В.
        & {\slv{то́щы}} {\slva{да́ры}}
        & {\slv{то́щы}} {\slva{бра̑ни}}
        & {\slv{тѡ́ща}} {\slva{помышлє́нїѧ}}
        \\\cline{2-5}
        
        & Т.
        & {\slv{тѡ́щы}} {\slva{да̑ры}}
        & {\slv{тѡ́щы}} {\slva{бра́нми}}
        & {\slv{тѡ́щы}} {\slva{помышлє́нїи}}
        \\\cline{2-5}
        
        & П.
        & {\slv{ѡ҆ то́щихъ}} {\slva{дарѣ́хъ}}
        & {\slv{ѡ҆ то́щихъ}} {\slva{бране́хъ}}
        & {\slv{ѡ҆ то́щихъ}} {\slva{помышле́нїихъ}}
        \\\hline
        
    \end{placedtabular}

    \bigskip
    Имена прилагательные краткие с основой на шипящие хотя и имеют твердое окончание, но склоняются по образцу кратких прилагательных с мягким окончанием, заменяя после шипящих {\slv{ж, ч, ш, щ}} гласные {\slv{ѧ}} на {\slv{а}}, {\slv{ю}} на {\slv{ꙋ}}, {\slv{ѣ}} на {\slv{и}}. Отступления от этого правила допускаются только для различения созвучных форм в различных числах и падежах. Так, например, буква {\slv{ы}} пишется только для различия падежей множественного числа от созвучных падежей единственного числа.
    
    Имена прилагательные с основой на шипящую, как и имена существительные, имеют смешанное склонение.


            \subsection{Местоимение}
                \subsubsection{Понятие о местоимении}

    \textbf{Местоимением} называется часть речи, употребляемая вместо имени существительного или его заменяющих имен. Например:
    
    \medskip\autorows{l}{1}{l}{
        \hspca{{\slv{Кто̀ ѿ ва́съ ѡ҆блича́етъ мѧ̀ ѡ҆ грѣсѣ̀;}} (Мф. 8, 46)}
    }

    Все местоимения разделяются на неродовые и родовые.
    
    \textbf{Неродовые} местоимения~---~это те, которые не изменяются по родам, а следовательно, могут относиться ко всем родам. К таким местоимениям относятся {\slv{а҆́зъ}} и {\slv{ты̀}}. Эти местоимения называются \emph{личными}, так как они относятся к тому или другому лицу; {\slv{а҆́зъ}}~---~это местоимение 1-го лица, а {\slv{ты̀}}~---~местоимение 2-го лица. Кроме того, к неродовым местоимениям относится \emph{возвратное} местоимение {\slv{себє̀}} и \emph{вопросительные} {\slv{кто̀}} и {\slv{что̀}}.
    
    \textbf{Родовыми} местоимениями называются такие, которые изменяются по родам в зависимости от определенного слова.
    
    Все местоимения звательного падежа не имеют.
    
    Рассмотрим склонение некоторых местоимений.

                \subsubsection{Склонение личных местоимений {\slv{а҆́зъ}} и {\slv{ты̀}} и возвратного {\slv{себє̀}}}
    
    \begin{placedtabular}[%
%        caption={\tabcaptsize Склонение личных местоимений {\slv{а҆́зъ}} и {\slv{ты̀}}\protect\\и возвратного {\slv{себє̀}}}
        ]{|c|c|c|c|c|}
        \hline
        
        \mkcella{Па-\\деж}
        & \multicolumn{2}{c|}{\mkcellb{Единственное число}}
        & \multicolumn{2}{c|}{\mkcellb{Множественное число}}
        \\\hline
        
        \makecell{И.}
        & {\slv{а҆́зъ}}
        & {\slv{ты̀}}
        & {\slv{мы̀}}
        & {\slv{вы̀}}
        \\\hline
        
        Р.
        & {\slv{менє̀}}
        & {\slv{тебє̀}}
        & {\slv{на́съ}}
        & {\slv{ва́съ}}
        \\\hline
        
        Д.
        & {\slv{мнѣ̀, мѝ}}
        & {\slv{тебѣ̀, тѝ}}
        & {\slv{на́мъ}}
        & {\slv{ва́мъ}}
        \\\hline
        
        В.
        & {\slv{менѐ, мѧ̀}}
        & {\slv{тебѐ, тѧ̀}}
        & {\slv{на́съ, ны̀}}
        & {\slv{ва́съ, вы̀}}
        \\\hline

        Т.
        & {\slv{мно́ю}}
        & {\slv{тобо́ю}}
        & {\slv{на́ми}}
        & {\slv{ва́ми}}
        \\\hline
        
        П.
        & {\slv{ѡ҆ мнѣ̀}}
        & {\slv{ѡ҆ тебѣ̀}}
        & \makecell{{\slv{ѡ҆ на́съ}}}
        & {\slv{ѡ҆ ва́съ}}
        \\\hline
        
        \mkcella{~\\~}
        & \multicolumn{2}{c|}{\mkcellb{Двойственное число}}
        \\\cline{1-3}
        
        \makecell{И.}
        & {\slv{мы̀}}
        & {\slv{вы̀}}
        \\\cline{1-3}
        
        \makecell{Р. П.}
        & \makecell{\slv{на́ю}}
        & {\slv{ва́ю}}
        \\\cline{1-3}
        
        \makecell{Д. Т.}
        & \makecell{{\slv{на́ма}}}
        & \makecell{{\slv{ва́ма}}}
        \\\cline{1-3}
        
        \makecell{В.}
        & {\slv{ны̀}}
        & {\slv{вы̀}}
        \\\cline{1-3}

    \end{placedtabular}

    \bigskip
    Возвратное местоимение {\slv{себє̀}} именительного падежа не имеет, а также не имеет двойственного и множественного чисел.
    
    Созвучные падежи (родительный и винительный единственного числа) различаются между собою по начертанию: родительный падеж имеет в окончании {\slv{є}} (удлиненное), а винительный падеж~---~{\slv{е}} (обыкновенное).
    
    Дательный и предложный падежи единственного числа имеют окончание {\slv{-ѣ}}.
    
    Дательный и винительный падежи единственного числа, а также винительный падеж множественного числа имеют две формы: \emph{полную} и \emph{энклитическую} ({\slv{мѝ, тѝ, сѝ; мѧ̀, тѧ̀, сѧ̀; ны̀, вы̀}}). Эти формы в сочетании с предыдущим словом как бы составляют с ним одно целое и даже иногда, когда предыдущее слово имеет конечный ударный открытый слог, утрачивают собственное ударение (см. \S9, п. 1). Например: {\slv{Оу҆слы́ши мѧ̀, гдⷭ҇и! Спасѝ мѧ̀, бж҃е мо́й!}}
    
    Такие формы и названы \emph{энклитическими} от греческого \textgreek{ἐγκλίνω}~---~склоняюсь, т.е. применительно к данному случаю~---~утрачиваю ударение

                \subsubsection{Особенности глагольного сказуемого в предложении}

    Глагол в предложении почти всегда является \textbf{сказуемым}. В этом случае глагол-сказуемое может быть простым и составным.
    
    \emph{Простым} глагольным сказуемым называется такое сказуемое, которое выражено только одним глаголом. Например: {\slv{А҆́зъ къ бг҃ꙋ воззва́хъ}} (Пс. 54, 17).
    
    \emph{Составным} глагольным сказуемым называется такое сказуемое, которое выражено или двумя глаголами, или глаголом совместно с именем существительным, кратким прилагательным или местоимением. Например: {\slv{А҆́зъ снидо́хъ и҆з̾ѧти и҆̀хъ}} (Деян. 7, 34). {\slv{Вѣ́ра без̾ дѣ́лъ мертва̀ є҆́сть}} (Иак. 2, 20).
    
    Во втором примере, где составное сказуемое выражено кратким прилагательным совместно с глаголом {\slv{бы́ти}} в настоящем времени, заслуживает особого внимания. Глагол {\slv{бы́ти}} в составе подобных сказуемых называется \emph{связкой}. Глагол-связка выполняет в составном сказуемом лишь вспомогательную роль, устанавливая связь между подлежащим и сказуемым. Основное же значение сказуемого выражается входящим в его состав именным словом.
    
    Если глагол-связка стоит в настоящем времени (как во втором примере), то при переводе такого предложения на русский язык, эта связка опускается. Так, например, второе предложение переводится на русский язык так: <<Вера без дел мертва>>.

                \subsubsection{Согласование слов в предложении}

    \textbf{Согласованием} называется такое сочетание слов в предложении, когда эти слова поставлены или в одинаковом падеже, или в одном роде, числе и падеже, или, наконец, в одном числе и лице.
    
    Подлежащее, выраженное именем существительным, согласуется с глагольным сказуемым в одинаковом числе (и роде, если число двойственное):
    
    \bigskip\autorows{l}{1}{l}{
        \hspca{{\slv{Два̀ ᲂу҆чн҃ка и҆до́ста въ ве́сь}}}
    }

    При подлежащем, выраженном именем существительным собирательным, согласование его с глагольным сказуемым бывает по смыслу, т.е. при таком подлежащем в единственном числе сказуемое ставится во множественном, так как собирательное имя заключает в себе многие предметы. Например:
    
    \bigskip\autorows{l}{1}{l}{
        \hspca{{\slv{Наро́дъ же стоѧ́й слы́шавъ глаго́лахꙋ}} (Ин. 12, 29)}
    }

    При двух подлежащих глагольное сказуемое ставится в предложении большей частью в двойственном числе, причем в роде преимущественном. Например:
    
    \medskip\autorows{l}{1}{l}{
        \hspca{{\slv{Пра́вда}} (жен. р.) {\slv{и҆ ми́ръ}} (муж. р.) {\slv{ѡ҆блобыза́стасѧ}} (муж. р.) (Пс. 84, 11)}
    }

    При подлежащем во множественном числе или при многих подлежащих глагольное сказуемое ставится во множественном числе. Например:
        
    \bigskip\autorows{l}{1}{l}{
        \hspca{{\slv{Всѝ дні́е на́ши ѡ҆скꙋдѣ́ша}} (Пс. 89, 9)}
    }

    Имена прилагательные употребляются в речи совместно с именами существительными, а потому согласуются с ними в роде, числе и падеже, например: {\slv{но́ваѧ за́повѣдь, мꙋ́дри мꙋ́жїе}}, причем полные имена прилагательные бывают определениями, а краткие~---~сказуемыми, например:
        
    \bigskip\autorows{l}{1}{l}{
        \hspca{{\slv{Ско́рбное се́рдце. Се́й чꙋ́ждъ є҆́сть кро́ве}}}
    }

                \subsubsection{Управление слов в предложении}

    \textbf{Управлением} называется сочетание слов в предложении, когда одно слово зависит от другого и выражает свою зависимость определенным падежом. Например: {\slv{Добро̀ сотвори́ти человѣ́кꙋ. Надѣ́ѧтисѧ на бг҃а}}.
    
    Здесь слова {\slv{человѣ́кꙋ}}, {\slv{бг҃а}} поставлены в определенном падеже, а иногда соединены с предлогом по требованию других слов: {\slv{сотвори́ти}}, {\slv{надѣ́ѧтисѧ}}. Эти слова, зависящие от других слов, называются \emph{управляемыми}.

                \subsubsection{Обращение}

    \textbf{Обращением} называется слово, называющее того, к кому или к чему обращаются с речью.
    
    Пунктуация при обращении следует тем же правилам, что и в русском языке:
    
    \bigskip\autorows{l}{1}{l}{
        \hspca{{\slv{Гдⷭ҇и, кто́ ѡ҆бита́етъ въ жили́щи твое́мъ;}} (Пс. 14, 1)},
        \hspca{{\slv{Оу҆слы́ши, гдⷭ҇и, пра́вдꙋ мою̀}} (Пс. 16, 1)},
        \hspca{{\slv{Ко́ль возлю́блєнна селє́нїѧ твоѧ̑, гдⷭ҇и си́лъ!}}}
    }

    Обращение не является членом предложения, т.к. не отвечает ни на какой вопрос и грамматически не связано с ним. Ставится обращение всегда в звательном падеже.

    \chapter*{2-й класс}
    \label{ch:secondgrade}
    \addcontentsline{toc}{chapter}{\nameref*{ch:secondgrade}}
    
    %% Ensures correct hyperref links. See the Preamble
    \renewcommand*{\ChapterAnchorPrefix}{2G}
    
    \setcounter{section}{0}
    \setcounter{subsubsection}{0}
        \section{Склоняемые части речи}
            \subsection{Имя существительное}
                \subsubsection{Имена существительные неравносложные}

    Некоторые имена существительные принимают в косвенных падежах между основой и окончанием \emph{суффикс} (или, как говорят, \emph{наращение}), отсутствующий в именительном падеже единственного числа. Этот суффикс (наращение) образует в слове лишний слог, а потому такие имена существительные называются \textbf{неравносложными}.
    
    Имена существительные могут принимать такие наращения: {\slv{-ер-}}, {\slv{-ес-}}, {\slv{-ат-}} ({\slv{-ѧт-}}), {\slv{-ен-}}.
    
    \bigskip\mockitem{1. Склонение существительных с наращением {\slv{-ер-}}}
    \medskip
    
    К таким именам существительным относятся только два: {\slv{ма́ти}} и {\slv{дщѝ}}.
    
    \medskip\begin{placedtabular}[%
%        caption={\tabcaptsize Склонение существительных с наращением {\slv{-ер-}}}
        ]{|c|c|c|c|c|}
        \hline
        
        \mkcella{Па-\\деж}
        & \multicolumn{2}{c|}{\mkcellb{Единственное число}}
        & \multicolumn{2}{c|}{\mkcellb{Множественное число}}
        \\\hline
        
        \makecell{И. З.}
        & {\slv{ма́ти}}
        & {\slv{дщѝ}} ({\slv{дще́рь}})
        & {\slv{ма́тєри}}
        & {\slv{дщє́ри}}
        \\\hline
        
        Р.
        & {\slv{ма́тере}}
        & {\slv{дще́ре}}
        & {\slv{ма́терей, ма́терїй}}
        & {\slv{дще́рей}}
        \\\hline
        
        Д.
        & {\slv{ма́тери}}
        & {\slv{дще́ри}}
        & {\slv{ма́теремъ}}
        & {\slv{дще́ремъ}}
        \\\hline
        
        В.
        & {\slv{ма́терь}}
        & {\slv{дще́рь}}
        & {\slv{матере́й}}
        & {\slv{дщє́ри}}
        \\\hline
        
        Т.
        & {\slv{ма́терїю}}
        & {\slv{дще́рїю}}
        & {\slv{ма́терьми}}
        & {\slv{дще́рьми}}
        \\\hline
        
        П.
        & {\slv{ѡ҆ ма́тери}}
        & {\slv{ѡ҆ дще́ри}}
        & \makecell{{\slv{ѡ҆ ма́терехъ}}}
        & {\slv{ѡ҆ дще́рехъ}}
        \\\hline
        
        \mkcella{~\\~}
        & \multicolumn{2}{c|}{\mkcellb{Двойственное число}}
        \\\cline{1-3}
        
        \makecell{И.\\В. З.}
        & {\slv{ма́тєри}}
        & {\slv{дщє́ри}}
        \\\cline{1-3}
        
        \makecell{Р. П.}
        & \makecell{\slv{ма́тєрїю}}
        & {\slv{дщє́рїю}}
        \\\cline{1-3}
        
        \makecell{Д. Т.}
        & \makecell{{\slv{ма́терема}}}
        & \makecell{{\slv{дще́рема}}}
        \\\cline{1-3}
        
    \end{placedtabular}

    \bigskip
    Как видно из рассмотрения этого склонения, оно является мягким и принимает окончания третьего склонения имен существительных, причем винительный падеж единственного и множественного чисел сходен с именительным.
        
    \bigskip\mockitem{2. Склонение существительных с наращением {\slv{-ес-}}}
    \medskip
    
    К такому склонению относятся некоторые имена существительные \textbf{среднего} рода, оканчивающиеся на {\slv{-о}}, например: {\slv{не́бо, сло́во, чꙋ́до, тѣ́ло}} и др.
    
    Для образца этого склонения возьмем имя существительное {\slv{не́бо}}.
    
    \medskip\begin{placedtabular}[%
%        caption={\tabcaptsize Склонение существительных с наращением {\slv{-ес-}}}
        ]{|c|c|c|c|c|c|}
        \hline
        
        \multicolumn{2}{|c|}{\mkcellb{Единственное число}}
        & \multicolumn{2}{c|}{\mkcellb{Двойственное число}}
        & \multicolumn{2}{c|}{\mkcellb{Множественное число}}
        \\\hline
        
        \makecell{И. В. З.}
        & {\slv{не́бо}}
        & \makecell{И. В. З.}
        & {\slv{небєсѝ}}
        & \makecell{И. В. З.}
        & {\slv{небеса̀}}
        \\\hline

        \makecell{Р.}
        & {\slv{небесѐ}}
        & \makecell{Р. П.}
        & {\slv{небесꙋ̀}}
        & \makecell{Р.}
        & {\slv{небе́съ}}
        \\\hline

        \makecell{Д.}
        & {\slv{небесѝ}}
        & \multirow{3}{*}{Д. Т.}
        & \multirow{3}{*}{\slv{небесе́ма}}
        & \makecell{Д.}
        & {\slv{небесє́мъ}}
        \\\cline{1-2}\cline{5-6}

        \makecell{Т.}
        & {\slv{небесе́мъ}}
        &
        &
        & \makecell{Т.}
        & {\slv{небесы̀}}
        \\\cline{1-2}\cline{5-6}

        \makecell{П.}
        & {\slv{ѡ҆ небесѝ}}
        &
        &
        & \makecell{П.}
        & {\slv{ѡ҆ небесѣ́хъ}}
        \\\hline

    \end{placedtabular}

    \bigskip
    Существительное {\slv{сло́во}}, если оно означает Сына Божия, наращения {\slv{-ес-}} не принимает, склоняется только в единственном числе по второму склонению и в звательном падеже имеет окончание {\slv{-е}}: {\slv{сло́ве бж҃їй}}.
    
    Склонение имен существительных с наращением {\slv{-ес-}} является смешанным: частью твердым, а частью мягким, как это видно из прилагаемого образца.
        
    \bigskip\mockitem{3. Склонение существительных с наращением {\slv{-ат-}} ({\slv{-ѧт-}})}
    \medskip
    
    К такому склонению относятся некоторые имена существительные \textbf{среднего} рода с окончанием на {\slv{-а}} и {\slv{-ѧ}}, например: {\slv{ѻ҆троча̀, ѻ҆вча̀, ꙗ҆гнѧ̀, ѻ҆слѧ̀, жребѧ̀}} и др.
    
    Приведем склонение имени существительного {\slv{ѻ҆троча̀}}.
    
    \begin{center}
        %        Склонение существительных с наращением {\slv{-ат-}} ({\slv{-ѧт-}})
        \renewcommand*{\arraystretch}{1.4}
        \footnotesize\begin{tabular}[c]{|c|c|c|c|c|c|}
            \hline
            
            \multicolumn{2}{|c|}{Единственное число}
            & \multicolumn{2}{c|}{Двойственное число}
            & \multicolumn{2}{c|}{Множественное число}
            \\\hline
            
            \makecell{И. В. З.}
            & {\slv{ѻ҆троча̀}}
            & \makecell{И. В. З.}
            & {\slv{ѻ҆трѡча́ти}}
            & \makecell{И. В. З.}
            & {\slv{ѻ҆троча́та}}
            \\\hline
            
            \makecell{Р.}
            & {\slv{ѻ҆троча́те}}
            & \makecell{Р. П.}
            & {\slv{ѻ҆троча́тꙋ}}
            & \makecell{Р.}
            & {\slv{ѻ҆троча́тъ}}
            \\\hline
            
            \makecell{Д.}
            & {\slv{ѻ҆троча́ти}}
            & \multirow{3}{*}{Д. Т.}
            & \multirow{3}{*}{\slv{ѻ҆троча́тема}}
            & \makecell{Д.}
            & {\slv{ѻ҆троча́тємъ}} ({\slv{-ѡмъ}})
            \\\cline{1-2}\cline{5-6}
            
            \makecell{Т.}
            & {\slv{ѻ҆троча́темъ}}
            &
            &
            & \makecell{Т.}
            & {\slv{ѻ҆троча́ты}}
            \\\cline{1-2}\cline{5-6}
            
            \makecell{П.}
            & {\slv{ѡ҆ ѻ҆троча́ти}}
            &
            &
            & \makecell{П.}
            & {\slv{ѡ҆ ѻ҆троча́техъ}}
            \\\hline
            
        \end{tabular}
    \end{center}

    Как видно из прилагаемого образца, это склонение в большей своей части~---~твердое, причем предложный падеж множественного числа имеет окончание {\slv{-ехъ}} (а не {\slv{-ѣхъ}}).
    \pagebreak

    \bigskip\mockitem{4. Склонение существительных с наращением {\slv{-ен-}}}
    \medskip
    
    Имена существительные с наращением {\slv{-ен-}} принадлежат к мягкому склонению с окончанием на {\slv{-ѧ}}, если основа оканчивается на согласный {\slv{м}}, например: {\slv{и҆́мѧ, вре́мѧ, сѣ́мѧ, бре́мѧ}} и др. Все такие имена \textbf{среднего} рода. В предложном падеже множественного числа они имеют окончание {\slv{-ехъ}}. 
    
    Возьмем для образца склонение существительного {\slv{и҆́мѧ}}.

    \medskip\begin{placedtabular}[%
%        caption={\tabcaptsize Склонение существительных с наращением {\slv{-ен-}}}
        font=\shrunkensize
        ]{|c|c|c|c|c|c|}
        \hline
        
        \multicolumn{2}{|c|}{Единственное число}
        & \multicolumn{2}{c|}{Двойственное число}
        & \multicolumn{2}{c|}{Множественное число}
        \\\hline
        
        \makecell{И. В. З.}
        & {\slv{и҆́мѧ}}
        & \makecell{И. В. З.}
        & {\slv{и҆́мєни}}
        & \makecell{И. В. З.}
        & {\slv{и҆мена̀}}
        \\\hline
        
        \makecell{Р.}
        & {\slv{и҆́мене}}
        & \makecell{Р. П.}
        & {\slv{и҆менꙋ̀}}
        & \makecell{Р.}
        & {\slv{и҆ме́нъ}}
        \\\hline
        
        \makecell{Д.}
        & {\slv{и҆́мени}}
        & \multirow{3}{*}{Д. Т.}
        & \multirow{3}{*}{\slv{и҆мене́ма}}
        & \makecell{Д.}
        & {\slv{и҆менє́мъ}} ({\slv{-ѡ́мъ}})
        \\\cline{1-2}\cline{5-6}
        
        \makecell{Т.}
        & {\slv{и҆́менемъ}}
        &
        &
        & \makecell{Т.}
        & {\slv{и҆мены̀}}
        \\\cline{1-2}\cline{5-6}
        
        \makecell{П.}
        & {\slv{ѡ҆ и҆́мени}}
        &
        &
        & \makecell{П.}
        & {\slv{ѡ҆ и҆́менехъ}}
        \\\hline
        
    \end{placedtabular}

                \subsubsection{Имена существительные разносклоняемые}

    В церковнославянском языке есть немного имен существительных, из которых каждое склоняется по окончаниям различных склонений. Такие имена существительные обыкновенно называются \textbf{разносклоняемыми}.
    
    Приведем образцы склонений этих разносклоняемых существительных.

    \bigskip\mockitem{1. Склонение имен существительных {\slv{гдⷭ҇ь}} и {\slv{господи́нъ}}}
    \medskip
    
    Существительное {\slv{гдⷭ҇ь}} склоняется только в единственном числе по второму склонению. Но, несмотря на то, что это слово оканчивается на {\slv{-ь}} и потому как бы принадлежит к мягкому склонению, в действительности имеет окончания твердого склонения, причем звательный падеж оканчивается на {\slv{-и}}, например {\slv{гдⷭ҇и, поми́лꙋй}}.

    Существительное {\slv{господи́нъ}} (в смысле~---~земной господин) в единственном и двойственном числах склоняется правильно по второму склонению, но во множественном числе имеет значительные отступления, а также многообразные падежные формы.

    \medskip\begin{placedtabular}[%
%        caption={\tabcaptsize Склонение существительных {\slv{гдⷭ҇ь}} и {\slv{господи́нъ}}}
        ]{|c|c|}
        \hline
        
        \multicolumn{2}{|c|}{\mkcellb{Множественное число}}
        \\\hline
        
        \makecell{И. З.}
        & {\slv{госпо́дїе}}
        \\\hline
        
        \makecell{Р.}
        & {\slv{господе́й, госпо́дїй, госпѡ́дъ}}
        \\\hline
        
        \makecell{Д.}
        & {\slv{господє́мъ, госпо́дїѧмъ, господа́мъ}}
        \\\hline
        
        \makecell{В.}
        & {\slv{господы̀, госпо́дїй}}
        \\\hline

        \makecell{Т.}
        & {\slv{господмѝ, госпо́дїѧми, господа́ми, госпѡды̀}}
        \\\hline
        
        \makecell{П.}
        & {\slv{ѡ҆ господѣ́хъ, ѡ҆ госпо́дїѧхъ}}
        \\\hline
        
    \end{placedtabular}
    \pagebreak
    
    \bigskip\mockitem{2. Склонение имени существительного {\slv{бра́тъ}}}
    \medskip
    
    Существительное {\slv{бра́тъ}} в единственном и двойственном числах склоняется правильно по второму склонению. Но во множественном числе это существительное принимает два различных смысла: {\slv{бра́тїе}}, т.е. несколько братьев, и {\slv{бра́тїѧ}}, т.е. существительное собирательное (в смысле хотя бы монастырской братии), склоняющееся только во множественном числе с некоторыми особенностями по первому склонению имен существительных.
    
    \medskip\begin{placedtabular}[%
%        caption={\tabcaptsize Склонение имени существительного {\slv{бра́тъ}}}
        ]{|c|c|c|}
        \hline
        
        \multicolumn{3}{|c|}{\mkcellb{Множественное число}}
        \\\hline
        
        \makecell{И. З.}
        & {\slv{бра́тїе}}
        & {\slv{бра́тїѧ}}
        \\\hline
        
        \makecell{Р.}
        & {\slv{бра́тїй}}
        & {\slv{бра́тїи}}
        \\\hline
        
        \makecell{Д.}
        & {\slv{бра́тїємъ, бра́тїѧмъ}}
        & {\slv{бра́тїи}}
        \\\hline
        
        \makecell{В.}
        & {\slv{бра́тїй}}
        & {\slv{бра́тїю}}
        \\\hline
        
        \makecell{Т.}
        & {\slv{бра́тїѧми}}
        & {\slv{бра́тїею, бра́тїю}}
        \\\hline
        
        \makecell{П.}
        & {\slv{ѡ҆ бра́тїѧхъ}}
        & {\slv{ѡ҆ бра́тїи}}
        \\\hline
        
    \end{placedtabular}

    \bigskip\mockitem{3. Склонение имени существительного {\slv{де́нь}}}
    \medskip
    
    Существительное {\slv{де́нь}} склоняется с некоторыми отступлениями.

    \medskip\begin{placedtabular}[%
%        caption={\tabcaptsize Склонение имени существительного {\slv{де́нь}}}
        ]{|c|c|c|c|c|c|}
        \hline
        
        \multicolumn{2}{|c|}{\mkcellb{Единственное число}}
        & \multicolumn{2}{c|}{\mkcellb{Двойственное число}}
        & \multicolumn{2}{c|}{\mkcellb{Множественное число}}
        \\\hline
        
        \makecell{И. З.}
        & {\slv{де́нь}}
        & \multirow{2}{*}{И. В. З.}
        & \multirow{2}{*}{\slv{дни̑}}
        & \makecell{И. З.}
        & {\slv{дні́е}}
        \\\cline{1-2}\cline{5-6}
        
        \makecell{Р.}
        & {\slv{днѐ}}
        &
        &
        & \makecell{Р.}
        & {\slv{дні́й}}
        \\\hline
        
        \makecell{Д.}
        & {\slv{днѝ, дне́ви}}
        & \multirow{2}{*}{Р. П.}
        & \multirow{2}{*}{\slv{дню̀, дні́ю}}
        & \makecell{Д.}
        & {\slv{днє́мъ}}
        \\\cline{1-2}\cline{5-6}
        
        \makecell{В.}
        & {\slv{де́нь}}
        &
        &
        & \makecell{В.}
        & {\slv{дни̑}}
        \\\hline
        
        \makecell{Т.}
        & {\slv{дне́мъ}}
        & \multirow{2}{*}{Д. Т.}
        & \multirow{2}{*}{\slv{де́нма}}
        & \makecell{Т.}
        & {\slv{де́нми}}
        \\\cline{1-2}\cline{5-6}

        \makecell{П.}
        & {\slv{ѡ҆ днѝ}}
        &
        &
        & \makecell{П.}
        & {\slv{ѡ҆ дне́хъ}}
        \\\hline
        
    \end{placedtabular}

    \bigskip
    По образу склонения существительного {\slv{де́нь}} склоняются существительные: {\slv{ка́мень, ко́рень, пла́мень, сте́пень}}.
    
    \bigskip\mockitem{4. Склонение имени существительного {\slv{пꙋ́ть}}}
    \medskip
    
    Существительное {\slv{пꙋ́ть}} хотя и мужского рода, но по падежным окончаниям относится к третьему склонению и при этом имеет в окончаниях некоторые отступления.

    \begin{placedtabular}[%
        caption={\tabcaptsize Склонение имени существительного {\slv{пꙋ́ть}}}
        ]{|c|c|c|c|c|c|}
        \hline
        
        \multicolumn{2}{|c|}{\mkcellb{Единственное число}}
        & \multicolumn{2}{c|}{\mkcellb{Двойственное число}}
        & \multicolumn{2}{c|}{\mkcellb{Множественное число}}
        \\\hline
        
        \makecell{И. В.}
        & {\slv{пꙋ́ть}}
        & \multirow{2}{*}{И. В. З.}
        & \multirow{2}{*}{\slv{пꙋти̑}}
        & \makecell{И. З.}
        & {\slv{пꙋтїѐ}}
        \\\cline{1-2}\cline{5-6}
        
        \makecell{Р.}
        & {\slv{пꙋтѝ}}
        &
        &
        & \makecell{Р.}
        & {\slv{пꙋті́й}}
        \\\hline
        
        \makecell{Д.}
        & {\slv{пꙋтѝ}}
        & \multirow{2}{*}{Р. П.}
        & \multirow{2}{*}{\slv{пꙋтїю̀}}
        & \makecell{Д.}
        & {\slv{пꙋтє́мъ}}
        \\\cline{1-2}\cline{5-6}
        
        \makecell{З.}
        & {\slv{пꙋтѝ}}
        &
        &
        & \makecell{В.}
        & {\slv{пꙋти̑}}
        \\\hline
        
        \makecell{Т.}
        & {\slv{пꙋте́мъ}}
        & \multirow{2}{*}{Д. Т.}
        & \multirow{2}{*}{\slv{пꙋтьма̀}}
        & \makecell{Т.}
        & {\slv{пꙋтьмѝ}}
        \\\cline{1-2}\cline{5-6}
        
        \makecell{П.}
        & {\slv{ѡ҆ пꙋтѝ}}
        &
        &
        & \makecell{П.}
        & {\slv{ѡ҆ пꙋте́хъ}}
        \\\hline
        
    \end{placedtabular}

    \bigskip\mockitem{5. Склонение имен существительных {\slv{ѻ҆́ко}} и {\slv{ᲂу҆́хо}}}
    \medskip
    
    Существительные {\slv{ѻ҆́ко}} и {\slv{ᲂу҆́хо}} в единственном числе склоняются по образцу существительного {\slv{не́бо}} двояким образом:
    
    а) или без наращения, например: {\slv{ѻ҆́ка}}, {\slv{ᲂу҆́ха}}; {\slv{ѻ҆́кꙋ}}, {\slv{ᲂу҆́хꙋ}} и т.д.,
    
    б) или с наращением {\slv{-ес-}} в косвенных падежах, например: {\slv{ѻ҆чесѐ}}, {\slv{ᲂу҆шесѐ}}; {\slv{ѻ҆чесѝ}}, {\slv{ᲂу҆шесѝ}} и т.д., причем в соответствующих падежах в том и другом случае основы {\slv{к}} и {\slv{х}} по законам смягчения принимают {\slv{ц}}, {\slv{ч}} и {\slv{ш}}, например: {\slv{во ѻ҆́цѣ}}.
    
    В двойственном числе эти существительные имеют совершенно особые формы.
    
    \begin{placedtabular}[%
%        caption={\tabcaptsize Склонение имен существительных {\slv{ѻ҆́ко}} и {\slv{ᲂу҆́хо}}}
        ]{|c|c|c|}
        \hline
        
        \multicolumn{3}{|c|}{\mkcellb{Двойственное число}}
        \\\hline
        
        \makecell{И. З.}
        & {\slv{ѻ҆́чи}}
        & {\slv{ᲂу҆́ши}}
        \\\hline
        
        \makecell{Р. П.}
        & {\slv{ѻ҆́чїю}}
        & {\slv{ᲂу҆́шїю}}
        \\\hline
        
        \makecell{Д. Т.}
        & {\slv{ѻ҆чи́ма}}
        & {\slv{ᲂу҆ши́ма}}
        \\\hline
        
        \makecell{В.}
        & {\slv{ѻ҆́цѣ}}
        & {\slv{ᲂу҆́ши}}
        \\\hline
        
    \end{placedtabular}

    \bigskip
    Во множественном числе они склоняются по образу существительного {\slv{не́бо}} с наращением {\slv{-ес-}}.
    
    \bigskip\mockitem{6. Склонение имени существительного {\slv{ᲂу҆ста̀}} ({\slv{ᲂу҆стна̀}})}
    \medskip
    
    Существительное {\slv{ᲂу҆ста̀}} ({\slv{ᲂу҆стна̀}}) единственного числа не имеет. В остальных числах имеет особые окончания, причем во множественном числе иеет во всех падежах двоякую форму, как то видно из прилагаемого образца.

    \begin{placedtabular}[%
        caption={\tabcaptsize Склонение имени существительного {\slv{ᲂу҆ста̀}} ({\slv{ᲂу҆стна̀}})}
        ]{|c|c|c|c|c|}
        \hline
        
        \multicolumn{2}{|c|}{\mkcellb{Двойственное число}}
        & \multicolumn{3}{c|}{\mkcellb{Множественное число}}
        \\\hline
        
        \multirow{2}{*}{И. В. З.}
        & \multirow{2}{*}{\slv{ᲂу҆стнѣ̀}}
        & \makecell{И. В. З.}
        & {\slv{ᲂу҆стна̀}}
        & {\slv{ᲂу҆ста̀}}
        \\\cline{3-5}
        
        &
        & \makecell{Р.}
        & {\slv{ᲂу҆сте́нъ}}
        & {\slv{ᲂу҆́стъ}}
        \\\hline
        
        \multirow{2}{*}{Р. П.}
        & \multirow{2}{*}{\slv{ᲂу҆стнꙋ̀}}
        & \makecell{Д.}
        & {\slv{ᲂу҆стна́мъ}}
        & {\slv{ᲂу҆стѡ́мъ}}
        \\\cline{3-5}
        
        &
        & \makecell{Т.}
        & {\slv{ᲂу҆стна́ми}}
        & {\slv{ᲂу҆сты̀}}
        \\\hline
        
        Д. Т.
        & {\slv{ᲂу҆стна́ма}}
        & П.
        & \makecell{{\slv{ѡ҆ ᲂу҆стнѣ́хъ,}}\\{\slv{ѡ҆ ᲂу҆стна́хъ}}}
        & \makecell{{\slv{ѡ҆ ᲂу҆стѣ́хъ,}}\\{\slv{ѡ҆ ᲂу҆ста́хъ}}}
        \\\hline

    \end{placedtabular}

            \subsection{Местоимение}
                \subsubsection{Склонение личного местоимения 3-го лица {\slv{ѻ҆́нъ}}}

    Как уже известно, местоимения могут быть \emph{неродовые} и \emph{родовые}. Из неродовых местоимений нами уже рассмотрены ранее личные местоимения 1-го лица {\slv{а҆́зъ}} и 2-го лица {\slv{ты̀}}. Но личное местоимение 3-го лица уже родовое. К рассмотрению его склонения мы и переходим.
    
    \medskip\begin{placedtabular}[%
%        caption={\tabcaptsize Склонение местоимения {\slv{ѻ҆́нъ}}}
        ]{|c|c|c|c|c|c|c|c|}
        \hline
        
        \multicolumn{4}{|c|}{\mkcellb{Единственное число}}
        & \multicolumn{4}{c|}{\mkcellb{Множественное число}}
        \\\hline
        
        \mkcellb{Па-\\деж}
        & \mkcellb{м. р.}
        & \mkcellb{ж. р.}
        & \mkcellb{ср. р.}
        & \mkcellb{Па-\\деж}
        & \mkcellb{м. р.}
        & \mkcellb{ж. р.}
        & \mkcellb{ср. р.}
        \\\hline
        
        И.
        & {\slv{ѻ҆́нъ}}
        & {\slv{ѻ҆на̀}}
        & {\slv{ѻ҆но̀}}
        & И.        
        & {\slv{ѻ҆нѝ}}
        & {\slv{ѻ҆нѣ̀}}
        & {\slv{ѻ҆нѝ}}
        \\\hline

        Р.
        & {\slv{є҆гѡ̀}}
        & {\slv{є҆ѧ̀}}
        & {\slv{є҆гѡ̀}}
        & Р.        
        & \multicolumn{3}{c|}{{\slv{и҆́хъ}}}
        \\\hline
        
        Д.
        & {\slv{є҆мꙋ̀}}
        & {\slv{є҆́й}}
        & {\slv{є҆мꙋ̀}}
        & Д.        
        & \multicolumn{3}{c|}{{\slv{и҆̀мъ}}}
        \\\hline
        
        В.
        & {\slv{є҆го̀, и҆̀}}
        & {\slv{ю҆̀}}
        & {\slv{є҆̀}}
        & В.        
        & {\slv{и҆̀хъ, ѧ҆̀}}
        & \multicolumn{2}{c|}{{\slv{ѧ҆̀}}}
        \\\hline
        
        Т.
        & {\slv{и҆́мъ}}
        & {\slv{є҆́ю}}
        & {\slv{и҆́мъ}}
        & Т.        
        & \multicolumn{3}{c|}{{\slv{и҆́ми}}}
        \\\hline
        
        П.
        & {\slv{ѡ҆ не́мъ}}
        & {\slv{ѡ҆ не́й}}
        & {\slv{ѡ҆ не́мъ}}
        & П.        
        & \multicolumn{3}{c|}{{\slv{ѡ҆ ни́хъ}}}
        \\\hline

        \multicolumn{4}{|c|}{\mkcellb{Двойственное число}}
        \\\cline{1-4}
        
        \makecell{И.}
        & {\slv{ѻ҆́на}}
        & {\slv{ѻ҆́нѣ}}
        & {\slv{ѻ҆́на}}
        \\\cline{1-4}
        
        \makecell{Р. П.}
        & \multicolumn{3}{c|}{{\slv{є҆ю̀}}}
        \\\cline{1-4}

        \makecell{Д. Т.}
        & \multicolumn{3}{c|}{{\slv{и҆́ма}}}
        \\\cline{1-4}

        \makecell{В.}
        & \multicolumn{3}{c|}{{\slv{ѧ҆̀}}}
        \\\cline{1-4}

    \end{placedtabular}

    \bigskip
    Личное местоимение 3-го лица первоначально произносилось: {\slv{и҆̀}}~---~в мужском роде, {\slv{ꙗ҆̀}}~---~в женском роде и {\slv{є҆̀}}~---~в среднем роде. Эта первоначальная форма именительного падежа осталась в церковнославянском языке в мужском и среднем роде только в винительном падеже единственного числа, в женском же роде в том же падеже и числе осталась первоначальная форма {\slv{ю҆̀}}.
    
    После предлогов {\slv{ѡ҆, въ, за, на}} и др. местоимение {\slv{ѻ҆́нъ}} в видах благозвучия принимает согласный звук {\slv{н}}, например: {\slv{на ню̀, на нѐ, на нѧ̀, ѡ҆ не́мъ, въ не́мъ}} и т.п., а в винительном падеже единственного числа мужского рода первоначальная форма {\slv{и҆̀}} в этом случае переходит в {\slv{ь}}, например: {\slv{на́нь}}, где предлог {\slv{на}} пишется слитно с местоимением.
    
    По образцу склонения личного местоимения 3-го лица {\slv{ѻ҆́нъ}} склоняется относительное местоимение {\slv{и҆́же}} (который). Это местоимение состоит из первоначальной формы {\slv{и҆̀}} (он) и частицы {\slv{же}}. При склонении частица {\slv{же}} присоединяется к падежным окончаниям без изменения. Например:
    
    \bigskip
    Единственное число:
    
    \bigskip\autorows[-1pt]{l}{2}{l}{
        \hspca{И.}, {\slv{и҆́же, ꙗ҆́же, є҆́же}},
        \hspca{Р.}, {\slv{є҆гѡ́же, є҆ѧ́же, є҆гѡ́же}},
        \hspca{Т.}, {\slv{и҆́мже, є҆́юже, и҆́мже}}
    }

    Двойственное число:

    \bigskip\autorows[-1pt]{l}{2}{l}{
        \hspca{И. В.}, {\slv{ꙗ҆̀же}},
        \hspca{Р. П.}, {\slv{є҆ю́же}} во всех родах,
        \hspca{Д. П.}, {\slv{и҆́маже}}
    }
    
    Множественное число:
    
    \bigskip\autorows[-1pt]{l}{2}{l}{
        \hspca{Д.}, {\slv{и҆̀мже}} во всех родах и т.д.
    }

                    \paragraph{\exercise}

    В приведенных ниже предложениях замените формы местоимений 3-го лица мужского рода на соответствующие формы женского рода:
    
    \medskip\begin{adjustwidth}{\hstbb}{0cm}
        \renewcommand*{\arraystretch}{1.2}
        \begin{tabular}[l]{rl}
            
            \emph{Образец}:
            & \makecell[l]{Прости \textbf{ему} согрешения \textbf{его}.}
            \\
            
            &
            \\
            
            \exanswer
            & \makecell[l]{Прости {\slv{є҆й}} согрешения {\slv{є҆ѧ̀}}.}
            \\
            
        \end{tabular}
    \end{adjustwidth}

    \medskip
    1. Исцели и раба Твоего (и рабу Твою) от обдержащия \textbf{его} телесныя и душевныя немощи, и оживотвори \textbf{его} благодатию Христа Твоего (Требник. М., 1991, с. 318).
    
    2. Аще что согрешил есть словом, делом или помышлением, прости, очисти \textbf{его} и чиста сотвори от всякого греха и, присно спребывая \textbf{ему}, сохрани прочее лето живота \textbf{его} ходяща во оправданиях Твоих, во еже не ктому быть \textbf{ему} посмеяние диаволу; яко и в \textbf{нем} прославится пресвятое имя Твое (Там же, с. 324).
    
    3. Воздвигни \textbf{его} от одра болезненнаго и от ложа озлобления цела и всесовершенна, даруй \textbf{его} Церкви Твоей благоухающа и творяща волю Твою (Там же, с. 329).
    
    4. Посети \textbf{его} милостию и щедротами Твоими, отжени от \textbf{него} всякую болезнь и немощь (Там же, с. 333).
    
    5. Подая \textbf{ему} оставление грехов, и прощение согрешений, вольных же и невольных, уврачуй \textbf{его} язвы неисцельныя, всякий же недуг и всякую язю, даруй ему душевное исцеление (Там же, с. 339).
    
    6. Презри яко непамятозлобивый Бог согрешения \textbf{его} вся, свободи \textbf{его} от вечныя муки, уста \textbf{его} Твоего хваления исполни, устне \textbf{его} отверзи к славословию имене Твоего, руце \textbf{его} простри к деланию заповедей Твоих, нозе \textbf{его} к течению благовествования Твоего исправи, вся \textbf{его} уды и мысль Твоею укрепляя благодатию (Там же, с. 345--346).

                \subsubsection[Склонение притяжательных местоимений]{Склонение притяжательных местоимений {\slv{мо́й}} ({\slv{тво́й}}, {\slv{сво́й}}) и {\slv{на́шъ}} ({\slv{ва́шъ}})}
    
    \begin{center}
%        {\tabcaptsize Склонение притяжательных местоимений {\slv{мо́й}} ({\slv{тво́й}}, {\slv{сво́й}}) и {\slv{на́шъ}} ({\slv{ва́шъ}})}
        \renewcommand*{\arraystretch}{1.4}
        \footnotesize\begin{tabular}[c]{|c|c|c|c|c|c|c|c|}
            \hline
            
            ~
            & \makecell{Па-\\деж}
            & м. р.
            & ж. р.
            & ср. р.
            & м. р.
            & ж. р.
            & ср. р.
            \\\hline
            
            \multirow{6}{*}{\spheading[10em]{Единственное число}}
            & И.
            & {\slv{мо́й}}
            & {\slv{моѧ̀}}
            & {\slv{моѐ}}
            & {\slv{на́шъ}}
            & {\slv{на́ша}}
            & {\slv{на́ше}}
            \\\cline{2-8}
            
            & Р.
            & {\slv{моегѡ̀}}
            & {\slv{моеѧ̀}}
            & {\slv{моегѡ̀}}
            & {\slv{на́шегѡ}}
            & {\slv{на́шеѧ}}
            & {\slv{на́шегѡ}}
            \\\cline{2-8}
            
            & Д.
            & {\slv{моемꙋ̀}}
            & {\slv{мое́й}}
            & {\slv{моемꙋ̀}}
            & {\slv{на́шемꙋ}}
            & {\slv{на́шей}}
            & {\slv{на́шемꙋ}}
            \\\cline{2-8}
            
            & В.
            & \makecell{{\slv{моего̀,}}\\{\slv{мо́й}}}
            & {\slv{мою̀}}
            & {\slv{моѐ}}
            & \makecell{{\slv{на́шего,}}\\{\slv{на́шъ}}}
            & {\slv{на́шꙋ}}
            & {\slv{на́ше}}
            \\\cline{2-8}
            
            & Т.
            & {\slv{мои́мъ}}
            & {\slv{мое́ю}}
            & {\slv{мои́мъ}}
            & {\slv{на́шим}}
            & {\slv{на́шею}}
            & {\slv{на́шим}}
            \\\cline{2-8}
            
            & П.
            & {\slv{ѡ҆ мое́мъ}}
            & {\slv{ѡ҆ мое́й}}
            & {\slv{ѡ҆ мое́мъ}}
            & {\slv{ѡ҆ на́шемъ}}
            & {\slv{ѡ҆ на́шей}}
            & {\slv{ѡ҆ на́шемъ}}
            \\\hline
            
            \multirow{4}{*}{\spheading[4.5em]{Дв. число}}
            & \makecell{И.}
            & {\slv{моѧ̑}}
            & \multicolumn{2}{c|}{{\slv{мои̑}}}
            & {\slv{на̑ши}}
            & \multicolumn{2}{c|}{{\slv{на̑ша}}}
            \\\cline{2-8}
            
            & Р. П.
            & \multicolumn{3}{c|}{{\slv{моє́ю}}}
            & \multicolumn{3}{c|}{{\slv{на́шєю}}}
            \\\cline{2-8}
            
            & Д. Т.
            & \multicolumn{3}{c|}{{\slv{мои́ма}}}
            & \multicolumn{3}{c|}{{\slv{на́шима}}}
            \\\cline{2-8}
            
            & В.
            & \multicolumn{3}{c|}{{\slv{моѧ̑}}}
            & {\slv{на̑ши}}
            & \multicolumn{2}{c|}{{\slv{на̑ша}}}
            \\\hline
            
            \multirow{6}{*}{\spheading[10em]{Множественное число}}
            & И.
            & {\slv{моѝ}}
            & \multicolumn{2}{c|}{{\slv{моѧ̑}}}
            & {\slv{на́ши}}
            & {\slv{на́шѧ}}
            & {\slv{на̑ша}}
            \\\cline{2-8}
            
            & Р.
            & \multicolumn{3}{c|}{{\slv{мои́хъ}}}
            & \multicolumn{3}{c|}{{\slv{на́шихъ}}}
            \\\cline{2-8}
            
            & Д.
            & \multicolumn{3}{c|}{{\slv{мои̑мъ}}}
            & \multicolumn{3}{c|}{{\slv{на́шымъ}}}
            \\\cline{2-8}
            
            & В.
            & \makecell{{\slv{мои́хъ,}}\\{\slv{моѧ̑}}}
            & \multicolumn{2}{c|}{{\slv{моѧ̑}}}
            & \makecell{{\slv{на́шихъ,}}\\{\slv{на́шѧ}}}
            & {\slv{на́шѧ}}
            & {\slv{на̑ша}}
            \\\cline{2-8}
            
            & Т.
            & \multicolumn{3}{c|}{{\slv{мои́ми}}}
            & \multicolumn{3}{c|}{{\slv{на́шими}}}
            \\\cline{2-8}
            
            & П.
            & \multicolumn{3}{c|}{{\slv{ѡ҆ мои́хъ}}}
            & \multicolumn{3}{c|}{{\slv{ѡ҆ на́шихъ}}}
            \\\hline
            
        \end{tabular}
    \end{center}

    Притяжательные местоимения {\slv{мо́й, тво́й, сво́й}} к своим основам {\slv{мо-, тво-, сво-}} присоединяют в косвенных падежах полностью падежные формы личного местоимения 3-го лица {\slv{ѻ҆́нъ}}.
    
    Притяжательные местоимения {\slv{на́шъ}} и {\slv{ва́шъ}} к своим основам {\slv{наш-}} и {\slv{ваш-}} также присоединяют в косвенных падежах формы личного местоимения 3-го лица {\slv{ѻ҆́нъ}}. Кроме того, для различения созвучных падежей употребляются буквы {\slv{а}} и {\slv{ѧ}}, {\slv{и}} и {\slv{ы}}, а также облегченное ударение.

                \subsubsection{Склонение определительного местоимения {\slv{ве́сь}}}
    
    \begin{placedtabular}[%
%        caption={\tabcaptsize Склонение определительного местоимения {\slv{ве́сь}}}
        font=\shrunkensize
        ]{|c|c|c|c|c|c|c|c|}
        \hline
        
        \multicolumn{4}{|c|}{Единственное число}
        & \multicolumn{4}{c|}{Множественное число}
        \\\hline
        
        \makecell{Па-\\деж}
        & м. р.
        & ж. р.
        & ср. р.
        & \makecell{Па-\\деж}
        & м. р.
        & ж. р.
        & ср. р.
        \\\hline
        
        И.
        & {\slv{ве́сь}}
        & {\slv{всѧ̀}}
        & {\slv{всѐ}}
        & И.        
        & {\slv{всѝ}}
        & \multicolumn{2}{c|}{{\slv{всѧ̑}}}
        \\\hline
        
        Р.
        & {\slv{всегѡ̀}}
        & {\slv{всеѧ̀}}
        & {\slv{всегѡ̀}}
        & Р.        
        & \multicolumn{3}{c|}{{\slv{всѣ́хъ}}}
        \\\hline
        
        Д.
        & {\slv{всемꙋ̀}}
        & {\slv{все́й}}
        & {\slv{всемꙋ̀}}
        & Д.        
        & \multicolumn{3}{c|}{{\slv{всѣ̑мъ}}}
        \\\hline
        
        В.
        & {\slv{всего̀, ве́сь}}
        & {\slv{всю̀}}
        & {\slv{всѐ}}
        & В.        
        & {\slv{всѣ́хъ, всѧ̑}}
        & \multicolumn{2}{c|}{{\slv{всѧ̑}}}
        \\\hline
        
        Т.
        & {\slv{всѣ́мъ}}
        & {\slv{все́ю}}
        & {\slv{всѣ́мъ}}
        & Т.        
        & \multicolumn{3}{c|}{{\slv{всѣ́ми}}}
        \\\hline
        
        П.
        & {\slv{ѡ҆ все́мъ}}
        & {\slv{ѡ҆ все́й}}
        & {\slv{ѡ҆ все́мъ}}
        & П.        
        & \multicolumn{3}{c|}{{\slv{ѡ҆ всѣ́хъ}}}
        \\\hline
        
    \end{placedtabular}

    \bigskip
    Местоимение {\slv{ве́сь}} двойственного числа не может иметь по своему значению.

                \subsubsection{Склонение указательного местоимения {\slv{се́й}}}
    
    \begin{placedtabular}[%
%        caption={\tabcaptsize Склонение указательного местоимения {\slv{се́й}}}
        ]{|c|c|c|c|c|c|c|c|}
        \hline
        
        \multicolumn{4}{|c|}{\mkcellb{Единственное число}}
        & \multicolumn{4}{c|}{\mkcellb{Множественное число}}
        \\\hline
        
        \mkcellb{Па-\\деж}
        & \mkcellb{м. р.}
        & \mkcellb{ж. р.}
        & \mkcellb{ср. р.}
        & \mkcellb{Па-\\деж}
        & \mkcellb{м. р.}
        & \mkcellb{ж. р.}
        & \mkcellb{ср. р.}
        \\\hline
        
        И.
        & {\slv{се́й, сі́й}}
        & {\slv{сїѧ̀}}
        & {\slv{сїѐ, сѐ}}
        & И.        
        & {\slv{сі́и}}
        & \multicolumn{2}{c|}{{\slv{сїѧ̑}}}
        \\\hline
        
        Р.
        & {\slv{сегѡ̀}}
        & {\slv{сеѧ̀}}
        & {\slv{сегѡ̀}}
        & Р.        
        & \multicolumn{3}{c|}{{\slv{си́хъ}}}
        \\\hline
        
        Д.
        & {\slv{семꙋ̀}}
        & {\slv{се́й}}
        & {\slv{семꙋ̀}}
        & Д.        
        & \multicolumn{3}{c|}{{\slv{си̑мъ}}}
        \\\hline
        
        В.
        & {\slv{сего̑, се́й}}
        & {\slv{сїю̀}}
        & {\slv{сїѐ, сѐ}}
        & В.        
        & {\slv{си́хъ, сїѧ̑}}
        & \multicolumn{2}{c|}{{\slv{сїѧ̑}}}
        \\\hline
        
        Т.
        & {\slv{си́мъ}}
        & {\slv{се́ю}}
        & {\slv{си́мъ}}
        & Т.        
        & \multicolumn{3}{c|}{{\slv{си́ми}}}
        \\\hline
        
        П.
        & {\slv{ѡ҆ се́мъ}}
        & {\slv{ѡ҆ се́й}}
        & {\slv{ѡ҆ се́мъ}}
        & П.        
        & \multicolumn{3}{c|}{{\slv{ѡ҆ си́хъ}}}
        \\\hline
        
        \multicolumn{4}{|c|}{\mkcellb{Двойственное число}}
        \\\cline{1-4}
        
        \makecell{И. В.}
        & {\slv{сїѧ̑}}
        & {\slv{сі̑и}}
        & {\slv{сїи̑}}
        \\\cline{1-4}
        
        \makecell{Р. П.}
        & {\slv{сею̀}}
        & {\slv{сїю̑}}
        & {\slv{сею̀}}
        \\\cline{1-4}
        
        \makecell{Д. Т.}
        & \multicolumn{3}{c|}{{\slv{си́ма}}}
        \\\cline{1-4}
        
    \end{placedtabular}

                \subsubsection{Склонение вспомогательных местоимений {\slv{кто̀}} и {\slv{что̀}}}

    Все до сего времени рассмотренные образцы склонений имели свои падежные формы личного местоимения 3-го лица {\slv{ѻ҆́нъ}}. Такие местоимения относятся к мягкому склонению.
    
    Неродовые вспомогательные местоимения {\slv{кто̀}} и {\slv{что̀}} относятся уже к твердому склонению, отличительной чертой которых является в родительном падеже окончание {\slv{-огѡ}}, в дательном~---~{\slv{-омꙋ}} и в винительном {\slv{-ого}}. Рассмотрим это склонение, заметив при этом, что местоимение {\slv{что̀}} имеет в некоторых падежах двойные и даже тройные формы, несколько уклоняющиеся от твердого склонения и более примыкащие к мягкому.
    
    Местоимения {\slv{кто̀}} и {\slv{что̀}} двойственного и множественного чисел не имеют.
    
    \bigskip\autorows[-1pt]{l}{3}{l}{
        \hspca{\footnotesize И.}, {{\slv{кто̀}}}, {{\slv{что̀}}},
        \hspca{\footnotesize Р.}, {{\slv{когѡ̀}}}, {{\slv{чегѡ̀, чесѡ̀, чесогѡ̀}}},
        \hspca{\footnotesize Д.}, {{\slv{комꙋ̀}}}, {{\slv{чемꙋ̀, чесомꙋ̀}}},
        \hspca{\footnotesize В.}, {{\slv{кого̀}}}, {{\slv{что̀, чесо̀}}},
        \hspca{\footnotesize Т.}, {{\slv{ки́мъ}}}, {{\slv{чи́мъ}}},
        \hspca{\footnotesize П.}, {{\slv{ѡ҆ ко́мъ}}}, {{\slv{ѡ҆ че́мъ, ѡ҆ чесо́мъ}}}
    }

    По образцу склонения этих местоимений склоняются неродовые местоимения: {\slv{никто̀, ничто̀, никто́же, ничто́же, кто̀-ли́бо}} и пр.
    
    При склонении местоимений {\slv{никто́же}} и {\slv{ничто́же}} неизменяемая частица {\slv{же}} присоединяется к падежному окончанию, например, р.п.~---~{\slv{никогѡ́же, ничесѡ́же}}; д.п.~---~{\slv{никомꙋ́же, ничемꙋ́же}} и т.д.
    
    При склонении местоимений {\slv{никто̀}} и {\slv{ничто̀}}, а также {\slv{никто́же}} и {\slv{ничто́же}}, предлог, относящийся к этим местоимениям, ставится в падежах между слитной частицей {\slv{ни}} и местоимением. Например: {\slv{ни во что̀, ни ᲂу҆ когѡ̀, ни на чесо́мъ, ни ѡ҆ ко́мже}} и т.п.
    \pagebreak
    
                \subsubsection{Склонение указательного местоимения {\slv{то́й}} (тот)}

    Указательное местоимение {\slv{то́й}} и некоторые другие также относятся к твердому склонению, причем в некоторых падежах множественного числа местоимение {\slv{то́й}} имеет многообразные формы, а в винительном падеже во всех родах и числах и энклитические формы.

    \medskip\begin{placedtabular}[%
%        caption={\tabcaptsize Склонение указательного местоимения {\slv{то́й}}}
        font=\shrunkensize
        ]{|c|c|c|c|c|c|c|c|}
        \hline
        
        \multicolumn{4}{|c|}{\mkcellb{Единственное число}}
        & \multicolumn{4}{c|}{\mkcellb{Множественное число}}
        \\\hline
        
        \mkcellb{Па-\\деж}
        & \mkcellb{м. р.}
        & \mkcellb{ж. р.}
        & \mkcellb{ср. р.}
        & \mkcellb{Па-\\деж}
        & \mkcellb{м. р.}
        & \mkcellb{ж. р.}
        & \mkcellb{ср. р.}
        \\\hline
        
        И.
        & \makecell{{\slv{то́й}}}
        & \makecell{{\slv{та́ѧ, та̀}}}
        & \makecell{{\slv{то́е, то̀}}}
        & И.        
        & \makecell{{\slv{ті́и}}}
        & \makecell{{\slv{ты́ѧ}}}
        & \makecell{{\slv{та̑ѧ, та̑}}}
        \\\hline
        
        Р.
        & \makecell{{\slv{тогѡ̀}}}
        & \makecell{{\slv{тоѧ̀}}}
        & \makecell{{\slv{тогѡ̀}}}
        & Р.        
        & \multicolumn{3}{c|}{{\slv{тѣ́хъ}}}
        \\\hline
        
        Д.
        & \makecell{{\slv{томꙋ̀}}}
        & \makecell{{\slv{то́й}}}
        & \makecell{{\slv{томꙋ̀}}}
        & Д.        
        & \multicolumn{3}{c|}{{\slv{тѣ̑мъ, ты́мъ}}}
        \\\hline
        
        В.
        & \makecell{{\slv{того̀,}}\\{\slv{то́й}}}
        & \makecell{{\slv{тꙋ́ю, тꙋ̀}}}
        & \makecell{{\slv{то́е, то̀}}}
        & В.        
        & \makecell{{\slv{тѣ́хъ,}}\\{\slv{ты́ѧ, ты̑}}}
        & \makecell{{\slv{ты҆́ѧ}}}
        & \makecell{{\slv{та̑ѧ, та̑}}}
        \\\hline
        
        Т.
        & \makecell{{\slv{тѣ́мъ}}}
        & \makecell{{\slv{то́ю}}}
        & \makecell{{\slv{тѣ́мъ}}}
        & Т.        
        & \multicolumn{3}{c|}{{\slv{тѣ́ми}}}
        \\\hline
        
        П.
        & \makecell{{\slv{ѡ҆ то́мъ}}}
        & \makecell{{\slv{ѡ҆ то́й}}}
        & \makecell{{\slv{ѡ҆ то́мъ}}}
        & П.        
        & \multicolumn{3}{c|}{{\slv{ѡ҆ тѣ́хъ, ѡ҆ ты́хъ}}}
        \\\hline
        
        \multicolumn{4}{|c|}{\mkcellb{Двойственное число}}
        \\\cline{1-4}
        
        \makecell{И. В.}
        & \makecell{{\slv{та̑}}}
        & \makecell{{\slv{тѣ̀}}}
        & \makecell{{\slv{та̑, тѣ̀}}}
        \\\cline{1-4}
        
        \makecell{Р. П.}
        & \multicolumn{3}{c|}{{\slv{тѡ́ю}}}
        \\\cline{1-4}
        
        \makecell{Д. Т.}
        & \multicolumn{3}{c|}{{\slv{тѣ́ма}}}
        \\\cline{1-4}
        
    \end{placedtabular}

    \bigskip
    По образцу склонения местоимения {\slv{то́й}} склоняется и местоимение {\slv{то́йжде}}, причем частица {\slv{-жде}} присоединяется без изменения непосредственно к падежным окончаниям.
    
    По этому же образцу склоняется местоимение {\slv{и҆ны́й}}, причем это местоимение в родительном падеже единственного числа женского рода имеет окончание {\slv{-ыѧ}}, например:
    
    \bigskip\autorows{l}{1}{l}{
        \hspca{\slv{Не и҆́мамы и҆ны́ѧ по́мощи}} (Кондак Пресвятой Богородице).
    }
    \pagebreak
    
                \subsubsection{Склонение указательного местоимения {\slv{ѻ҆́нъ, ѻ҆́ный}}}

    Указательное местоимение {\slv{ѻ҆́ный}} в своей краткой форме {\slv{ѻ҆́нъ}} в именительном падеже всех родов заменило впоследствии первоначальную форму  местоимения 3-го лица в этом же падеже {\slv{и҆̀, ꙗ҆̀, є҆̀}}.
    
    По своим падежным окончаниям это местоимение принадлежит также к твердому склонению, но имеет и некоторые особенности. Вот образец этого склонения.
    
    \begin{placedtabular}[%
%        caption={\tabcaptsize Склонение указательного местоимения {\slv{ѻ҆́нъ, ѻ҆́ный}}}
        font=\shrunkensize
        ]{|c|c|c|c|c|c|c|c|}
        \hline
        \multirow{2}{*}{\makecell{Па-\\деж}}
        & \multicolumn{3}{c|}{Единственное число}
        & \multicolumn{3}{c|}{Множественное число}
        \\\cline{2-7}
        
        & м. р.
        & ж. р.
        & ср. р.
        & м. р.
        & ж. р.
        & ср. р.
        \\\hline
        
        И.
        & \makecell{{\slv{ѻ҆́нъ,}}\\{\slv{ѻ҆́ный}}}
        & \makecell{{\slv{ѻ҆́на,}}\\{\slv{ѻ҆́наѧ}}}
        & \makecell{{\slv{ѻ҆́но}}\\{\slv{ѻ҆́ное}}}
        & \makecell{{\slv{ѻ҆́ни, ѻ҆́ны,}}\\{\slv{ѻ҆́нїи}}}
        & \makecell{{\slv{ѻ҆́ны,}}\\{\slv{ѻ҆̀ныѧ}}}
        & \makecell{{\slv{ѻ҆̀на,}}\\{\slv{ѻ҆̀наѧ}}}
        \\\hline
        
        Р.
        & \makecell{{\slv{ѻ҆́нагѡ}}}
        & \makecell{{\slv{ѻ҆́ноѧ,}}\\{\slv{ѻ҆́ныѧ}}}
        & \makecell{{\slv{ѻ҆́нагѡ}}}
        & \multicolumn{3}{c|}{{\slv{ѻ҆́нѣхъ}}}
        \\\hline
        
        Д.
        & \makecell{{\slv{ѻ҆́номꙋ}}}
        & \makecell{{\slv{ѻ҆́ной,}}\\{\slv{ѻ҆́нѣй}}}
        & \makecell{{\slv{ѻ҆́номꙋ}}}
        & \multicolumn{3}{c|}{{\slv{ѻ҆̀нымъ, ѻ҆́нѣмъ}}}
        \\\hline
        
        В.
        & \makecell{{\slv{ѻ҆́нъ, ѻ҆́на,}}\\{\slv{ѻ҆́наго}}}
        & \makecell{{\slv{ѻ҆́нꙋ,}}\\{\slv{ѻ҆́нꙋю}}}
        & \makecell{{\slv{ѻ҆́но,}}\\{\slv{ѻ҆́ное}}}
        & \makecell{{\slv{ѻ҆́ны, ѻ҆̀ныѧ,}}\\{\slv{ѻ҆́нѣхъ, ѻ҆́ныхъ}}}
        & \makecell{{\slv{ѻ҆́ны,}}\\{\slv{ѻ҆̀ныѧ}}}
        & \makecell{{\slv{ѻ҆̀на,}}\\{\slv{ѻ҆̀наѧ}}}
        \\\hline
        
        Т.
        & \makecell{{\slv{ѻ҆́нымъ,}}\\{\slv{ѻ҆́нѣмъ}}}
        & \makecell{{\slv{ѻ҆́ною}}}
        & \makecell{{\slv{ѻ҆́нымъ}}\\{\slv{ѻ҆́нѣмъ}}}
        & \multicolumn{3}{c|}{{\slv{ѻ҆́нѣми}}}
        \\\hline
        
        П.
        & \makecell{{\slv{ѡ҆ ѻ҆́номъ}}}
        & \makecell{{\slv{ѡ҆ ѻ҆́ной,}}\\{\slv{ѡ҆ ѻ҆́нѣй}}}
        & \makecell{{\slv{ѡ҆ ѻ҆́номъ}}}
        & \multicolumn{3}{c|}{{\slv{ѡ҆ ѻ҆́нѣхъ}}}
        \\\hline
        
        \multicolumn{4}{|c|}{Двойственное число}
        \\\cline{1-4}
        
        \makecell{И. В.}
        & \makecell{{\slv{ѻ҆̀на}}}
        & \makecell{{\slv{ѻ҆́нѣ}}}
        & \makecell{{\slv{ѻ҆̀на}}}
        \\\cline{1-4}
        
        \makecell{Р. П.}
        & \multicolumn{3}{c|}{{\slv{ѻ҆̀нꙋ}}}
        \\\cline{1-4}
        
        \makecell{Д. Т.}
        & \multicolumn{3}{c|}{{\slv{ѻ҆́нѣма}}}
        \\\cline{1-4}
        
    \end{placedtabular}
    
    \bigskip
    По этому образцу склоняется местоимение {\slv{є҆ли́кїй}} (в краткой форме {\slv{є҆ли́къ}}). В краткой форме гортанный {\slv{к}} смягчается перед соответствующими гласными в {\slv{ц}} ({\slv{є҆ли́цы}}), а в родительном падеже единственного числа женского рода в своей полной форме имеет {\slv{є҆ли́кїѧ}}.
    
    В значении 3-го лица в церковнославянском языке часто употребляются указательные местоимения {\slv{се́й, то́й}}, например: {\slv{Се́й прїи́де во свидѣ́тельство}} (Ин. 1, 7); {\slv{Не бѣ̀ то́й свѣ́тъ}} (Ин. 1, 8). В русском переводе в обоих текстах стоит <<он>>.

            \subsection{Имя прилагательное}
                \subsubsection{Склонение полных имен прилагательных}

    Полные имена прилагательные образовались из кратких через присоединение к их окончаниям личного местоимения 3-го лица в своей первоначальной форме {\slv{и҆̀, ꙗ҆̀, є҆̀}}. Вот примеры таких образований: 1) с твердым окончанием ({\slv{до́брый}}) и 2) с мягким окончанием ({\slv{си́нїй}}).
    
    \bigskip\autorows{l}{1}{l}{
        \hspca{{\slv{до́бръ}} + {\slv{и҆̀}} = {\slv{до́бро}} + {\slv{и҆̀}} = {\slv{до́бр}} + {\slv{ый}} = {\slv{до́брый}}},
        
        \hspca{{\slv{до́бра}} + {\slv{є҆гѡ̀}} = {\slv{до́бра}} + {\slv{а҆гѡ̀}} = {\slv{до́бр}} + {\slv{а}} + {\slv{гѡ}} = {\slv{до́брагѡ}} и т.д.},
        
        \hspca{{\slv{си́нь}} + {\slv{и҆̀}} = {\slv{си́ни}} + {\slv{и҆̀}} = {\slv{си́н}} + {\slv{ий}} = {\slv{си́нїй}}},
        
        \hspca{{\slv{си́нѧ}} + {\slv{є҆гѡ̀}} = {\slv{си́нѧ}} + {\slv{ѧ҆гѡ̀}} = {\slv{си́н}} + {\slv{ѧ}} + {\slv{гѡ}} = {\slv{си́нѧгѡ}} и т.д.}
    }

    Рассмотрим склонение этих прилагательных в их полной форме.
    
    \begin{center}
%        {\tabcaptsize Склонение имен прилагательных, образованных из кратких, в их полной форме}
        \renewcommand*{\arraystretch}{1.4}
        \footnotesize\begin{tabular}[c]{|c|c|c|c|c|c|c|c|}
            \hline
            
            ~
            & \makecell{Па-\\деж}
            & м. р.
            & ж. р.
            & ср. р.
            & м. р.
            & ж. р.
            & ср. р.
            \\\hline
            
            \multirow{6}{*}{\spheading[10em]{Единственное число}}
            & И. З.
            & {\slv{до́брый}}
            & {\slv{до́браѧ}}
            & {\slv{до́брое}}
            & {\slv{си́нїй}}
            & {\slv{си́нѧѧ}}
            & {\slv{си́нее}}
            \\\cline{2-8}
            
            & Р.
            & {\slv{до́брагѡ}}
            & {\slv{до́брыѧ}}
            & {\slv{до́брагѡ}}
            & {\slv{си́нѧгѡ}}
            & {\slv{си́нїѧ}}
            & {\slv{си́нѧгѡ}}
            \\\cline{2-8}
            
            & Д.
            & {\slv{до́бромꙋ}}
            & {\slv{до́брѣй}}
            & {\slv{до́бромꙋ}}
            & {\slv{си́немꙋ}}
            & {\slv{си́ней}}
            & {\slv{си́немꙋ}}
            \\\cline{2-8}
            
            & В.
            & \makecell{{\slv{до́браго,}}\\{\slv{до́брый}}}
            & {\slv{до́брꙋю}}
            & {\slv{до́брое}}
            & \makecell{{\slv{си́нѧго,}}\\{\slv{си́нїй}}}
            & {\slv{си́нюю}}
            & {\slv{си́нее}}
            \\\cline{2-8}
            
            & Т.
            & {\slv{до́брым}}
            & {\slv{до́брою}}
            & {\slv{до́брымъ}}
            & {\slv{си́нимъ}}
            & {\slv{си́нею}}
            & {\slv{си́нимъ}}
            \\\cline{2-8}
            
            & П.
            & {\slv{ѡ҆ до́брѣмъ}}
            & {\slv{ѡ҆ до́брѣй}}
            & {\slv{ѡ҆ до́брѣмъ}}
            & {\slv{ѡ҆ си́нѣмъ}}
            & {\slv{ѡ҆ си́нѣй}}
            & {\slv{ѡ҆ си́нѣмъ}}
            \\\hline
            
            \multirow{4}{*}{\spheading[4.3em]{Дв. число}}
            & \makecell{И. В.\\З.}
            & {\slv{дѡ́браѧ}}
            & {\slv{до́брѣи}}
            & {\slv{до́брѣи}}
            & {\slv{си̑нѧѧ}}
            & \multicolumn{2}{c|}{{\slv{си̑нїи}}}
            \\\cline{2-8}
            
            & Р. П.
            & \multicolumn{3}{c|}{{\slv{дѡ́брꙋю}}}
            & \multicolumn{3}{c|}{{\slv{си̑нюю}}}
            \\\cline{2-8}
            
            & Д. Т.
            & \multicolumn{3}{c|}{{\slv{до́брыма}}}
            & \multicolumn{3}{c|}{{\slv{си́нима}}}
            \\\hline
            
            \multirow{6}{*}{\spheading[10.5em]{Множественное число}}
            & И. З.
            & {\slv{до́брїи}}
            & {\slv{дѡ́брыѧ}}
            & {\slv{дѡ́браѧ}}
            & {\slv{си́нїи}}
            & {\slv{си̑нїѧ}}
            & {\slv{си̑нѧѧ}}
            \\\cline{2-8}
            
            & Р.
            & \multicolumn{3}{c|}{{\slv{до́брыхъ}}}
            & \multicolumn{3}{c|}{{\slv{си́нихъ}}}
            \\\cline{2-8}
            
            & Д.
            & \multicolumn{3}{c|}{{\slv{дѡ́брымъ}}}
            & \multicolumn{3}{c|}{{\slv{си̑нимъ}}}
            \\\cline{2-8}
            
            & В.
            & \makecell{{\slv{до́брыхъ,}}\\{\slv{дѡ́брыѧ}}}
            & {\slv{дѡ́брыѧ}}
            & {\slv{дѡ́браѧ}}
            & \makecell{{\slv{си́нихъ,}}\\{\slv{си̑нїѧ}}}
            & {\slv{си̑нїѧ}}
            & {\slv{си̑нѧѧ}}
            \\\cline{2-8}
            
            & Т.
            & \multicolumn{3}{c|}{{\slv{до́брыми}}}
            & \multicolumn{3}{c|}{{\slv{си́ними}}}
            \\\cline{2-8}
            
            & П.
            & \multicolumn{3}{c|}{{\slv{ѡ҆ до́брыхъ}}}
            & \multicolumn{3}{c|}{{\slv{ѡ҆ си́нихъ}}}
            \\\hline
            
        \end{tabular}
    \end{center}


                \subsubsection{Общие замечания к склонению полных имен прилагательных}

    1. При склонении полных имен прилагательных гортанные звуки в основе смягчаются перед {\slv{ѣ}} и {\slv{\large и}} на общем основании, например:
    
    \bigskip\autorows{l}{1}{l}{
        \hspca{{\slv{бла{\large г}і́й}}~---~{\slv{бла́{\large з}ѣй}}~---~{\slv{бла́{\large з}ѣмъ}}}
    }

    2. Имя прилагательное {\slv{мно́гїй}} при склонении имеет следущие особенности:
        
    \bigskip\autorows[-1pt]{l}{2}{l}{
        \hspca творительный падеж един. числа муж. и ср. рода, {\slv{мно́земъ}},
        \hspca родительный падеж множ. числа во всех родах, {\slv{мно́зехъ}},
        \hspca дательный падеж множ числа во всех родах, {\slv{мнѡ́земъ}},
        \hspca предложный падеж множ. числа во всех родах, {\slv{ѡ҆ мно́зѣхъ}}
    }

    3. Полные имена прилагательные с основой на шипящие склоняются по следующему образцу ({\slv{то́щїй}}~---~напрасный, тщетный).
    
    \medskip\begin{placedtabular}[%
        caption={\tabcaptsize Полное имя прилагательное с основой на шипящий звук {\slv{то́щїй}}}
        ,font=\shrunkensize
        ]{|c|c|c|c|c|c|c|c|}
        \hline
        \multirow{2}{*}{\makecell{Па-\\деж}}
        & \multicolumn{3}{c|}{Единственное число}
        & \multicolumn{3}{c|}{Множественное число}
        \\\cline{2-7}
        
        & м. р.
        & ж. р.
        & ср. р.
        & м. р.
        & ж. р.
        & ср. р.
        \\\hline
        
        И. З.
        & \makecell{{\slv{то́щїй}}}
        & \makecell{{\slv{то́щаѧ}}}
        & \makecell{{\slv{то́щее}}}
        & \makecell{{\slv{то́щїи}}}
        & \makecell{{\slv{то́щыѧ}}}
        & \makecell{{\slv{тѡ́щаѧ}}}
        \\\hline
        
        Р.
        & \makecell{{\slv{то́щагѡ}}}
        & \makecell{{\slv{то́щїѧ}}}
        & \makecell{{\slv{то́щагѡ}}}
        & \multicolumn{3}{c|}{{\slv{то́щихъ}}}
        \\\hline
        
        Д.
        & \makecell{{\slv{то́щемꙋ}}}
        & \makecell{{\slv{то́щей}}}
        & \makecell{{\slv{то́щемꙋ}}}
        & \multicolumn{3}{c|}{{\slv{то́щымъ}}}
        \\\hline
        
        В.
        & \makecell{{\slv{то́щаго,}}\\{\slv{то́щїй}}}
        & \makecell{{\slv{то́щꙋю}}}
        & \makecell{{\slv{то́щее}}}
        & \makecell{{\slv{то́щихъ,}}\\{\slv{то́щыѧ}}}
        & \makecell{{\slv{то́щыѧ}}}
        & \makecell{{\slv{тѡ́щаѧ}}}
        \\\hline
        
        Т.
        & \makecell{{\slv{то́щїимъ}}}
        & \makecell{{\slv{то́щею}}}
        & \makecell{{\slv{то́щїимъ}}}
        & \multicolumn{3}{c|}{{\slv{то́щими}}}
        \\\hline
        
        П.
        & \makecell{{\slv{ѡ҆ то́щемъ}}}
        & \makecell{{\slv{ѡ҆ то́щей}}}
        & \makecell{{\slv{ѡ҆ то́щемъ}}}
        & \multicolumn{3}{c|}{{\slv{ѡ҆ то́щихъ}}}
        \\\hline
        
        \multicolumn{4}{|c|}{Двойственное число}
        \\\cline{1-4}
        
        \makecell{И. В.\\З.}
        & \makecell{{\slv{тѡ́щаѧ}}}
        & \multicolumn{2}{c|}{{\slv{тѡ́щїи}}}
        \\\cline{1-4}
        
        \makecell{Р. П.}
        & \multicolumn{3}{c|}{{\slv{тѡ́щꙋю}}}
        \\\cline{1-4}
        
        \makecell{Д. Т.}
        & \multicolumn{3}{c|}{{\slv{то́щима}}}
        \\\cline{1-4}
        
    \end{placedtabular}

    \bigskip
    4. Имена прилагательные {\slv{бж҃їй, ве́лїй}} и подобные им склоняются с небольшими отступлениями. Приведем для образца склонение имени прилагательного {\slv{ве́лїй}}.
    
    \begin{placedtabular}[%
%        caption={\tabcaptsize Полное имя прилагательное {\slv{ве́лїй}}}
        font=\shrunkensize
        ]{|c|c|c|c|c|c|c|}
        \hline
        \multirow{2}{*}{\makecell{Па-\\деж}}
        & \multicolumn{3}{c|}{Единственное число}
        & \multicolumn{2}{c|}{Множественное число}
        \\\cline{2-6}
        
        & м. р.
        & ж. р.
        & ср. р.
        & м. р.
        & ж. и ср. р.
        \\\hline
        
        И. З.
        & \makecell{{\slv{ве́лїй}}}
        & \makecell{{\slv{ве́лїѧ}}}
        & \makecell{{\slv{ве́лїе}}}
        & \makecell{{\slv{ве́лїи}}}
        & \makecell{{\slv{вє́лїѧ}}}
        \\\hline
        
        Р.
        & \makecell{{\slv{ве́лїѧгѡ}}}
        & \makecell{{\slv{ве́лїѧ}}}
        & \makecell{{\slv{ве́ліѧгѡ}}}
        & \multicolumn{2}{c|}{{\slv{ве́лїихъ}}}
        \\\hline
        
        Д.
        & \makecell{{\slv{ве́лїемꙋ}}}
        & \makecell{{\slv{ве́лїей}}}
        & \makecell{{\slv{ве́лїемꙋ}}}
        & \multicolumn{2}{c|}{{\slv{вє́лїимъ}}}
        \\\hline
        
        В.
        & \makecell{{\slv{ве́лїѧго,}}\\{\slv{ве́лїй}}}
        & \makecell{{\slv{ве́лїю}}}
        & \makecell{{\slv{ве́лїе}}}
        & \makecell{{\slv{ве́лїихъ,}}\\{\slv{вє́лїѧ}}}
        & \makecell{{\slv{вє́лїѧ}}}
        \\\hline
        
        Т.
        & \makecell{{\slv{ве́лїимъ}}}
        & \makecell{{\slv{ве́лїею}}}
        & \makecell{{\slv{ве́лїимъ}}}
        & \multicolumn{2}{c|}{{\slv{ве́лїими}}}
        \\\hline
        
        П.
        & \makecell{{\slv{ѡ҆ ве́лїемъ}}}
        & \makecell{{\slv{ѡ҆ ве́лїей}}}
        & \makecell{{\slv{ѡ҆ ве́лїемъ}}}
        & \multicolumn{2}{c|}{{\slv{ѡ҆ ве́лїихъ, ѡ҆ ве́лїѧхъ}}}
        \\\hline
        
        \multicolumn{4}{|c|}{Двойственное число}
        \\\cline{1-4}
        
        \makecell{И. В.\\З.}
        & \makecell{{\slv{вє́лїѧ}}}
        & \multicolumn{2}{c|}{{\slv{вє́лїи}}}
        \\\cline{1-4}
        
        \makecell{Р. П.}
        & \multicolumn{3}{c|}{{\slv{вє́лїю}}}
        \\\cline{1-4}
        
        \makecell{Д. Т.}
        & \multicolumn{3}{c|}{{\slv{ве́лїима}}}
        \\\cline{1-4}
        
    \end{placedtabular}

    \bigskip
    5. Если в полном имени прилагательном имеется суффикс {\slv{-ск-}}, то он изменяется в некоторых падежах в {\slv{-ст-}}, например: {\slv{лю́дскїй}}~---~{\slv{людсті́и}}.

                \subsubsection{Понятие о степенях сравнения имен прилагательных}

    Качественные имена прилагательные имеют \textbf{степени сравнения}.
    
    Если качественное прилагательное обозначает такое количество, которое у предмета или явления может быть в большой и в меньшей мере, то такое прилагательное может иметь сравнительную степень и превосходную степень.
    
    \emph{Сравнительная} степень показывает, что в одном предмете или явлении какого-нибудь качества больше, чем в другом.
    
    \emph{Превосходная} степень показывает, что в одном предмете или явлении какого-нибудь качества больше, чем во всех других предметах или явлениях, то есть, иначе говоря, показывает наивысшую меру качества сравнительно с другими однородными предметами или явлениями.
    
    В соответствии этим двум степеням, первоначальная форма имени прилагательного, от которой образуются степени сравнения и превосходная, называется \emph{положительной} степенью. Например:
    
    \bigskip\autorows[-1pt]{l}{3}{l}{
        \hspca{\slv{свѧты́й}}, ~--~, степень положительная,
        \hspca{\slv{свѧтѣ́е}}, ~--~, степень сравнительная,
        \hspca{\slv{свѧтѣ́йшїй}}, ~--~, степень превосходная
    }

                \subsubsection{Сравнительная степень имен прилагательных}

    Сравнительная степень имен прилагательных образуется из положительной через замену полного окончания прилагательных на {\slv{-шїй}} (м. р.), {\slv{-шаѧ}} (ж. р.), {\slv{-шее}} (ср. р.). Например:

    \bigskip\autorows{l}{1}{l}{
        \hspca{{\slv{чи́ст-ый}}~---~{\slv{чи́ст-шїй}}},
        \hspca{{\slv{чи́ст-аѧ}}~---~{\slv{чи́ст-шаѧ}}},
        \hspca{{\slv{чи́ст-ое}}~---~{\slv{чи́ст-шее}}}
    }

    или на {\slv{-ѣй}} (м. р.), {\slv{-ѣйши}} (ж. р.), {\slv{-ѣе}} (ср. р). Например:

    \bigskip\autorows{l}{1}{l}{
        \hspca{{\slv{честн-ы́й}}~---~{\slv{честн-ѣ́й}}},
        \hspca{{\slv{честн-а́ѧ}}~---~{\slv{честн-ѣ́йши}}},
        \hspca{{\slv{честн-о́е}}~---~{\slv{честн-ѣ́е}}}
    }

    Если корень слова оказывается на губной согласной {\slv{п}} или {\slv{б}} или на зубной {\slv{д}}, то эти согласные смягчаются вставкой соответственно или плавного звука {\slv{л}}, или шипящго {\slv{ж}}, причем если в слове имеются суффиксы {\slv{-к-}}, {\slv{-ок-}}, {\slv{-ек-}}, то они выпадают. Например:

    \bigskip\autorows{l}{1}{l}{
        \hspca{{\slv{крѣ́пкїй}}~---~{\slv{крѣ́плшїй}}},
        \hspca{{\slv{глꙋбо́кїй}}~---~{\slv{глꙋ́блшїй}}},
        \hspca{{\slv{сла́дкїй}}~---~{\slv{сла́ждшїй}}}
    }
    
    В словах с основой на гортанные происходит по известным правилам смягчение, причем {\slv{ѣ}} в окончании переходит в {\slv{а}}. Например:
    
    \bigskip\autorows{l}{1}{l}{
        \hspca{{\slv{мно́гїй}}~---~{\slv{мно́жшїй}}},
        \hspca{{\slv{ти́хїй}}~---~{\slv{ти́шшїй}}},
        \hspca{{\slv{высо́кїй}}~---~{\slv{высоча́й}}}
    }

    Качественные наречия имеют сравнительную степень, сходную со средним родом сравнительной степени имен прилагательных. Например:

    \bigskip\autorows{l}{1}{l}{
        \hspca{{\slv{вельмѝ}}~---~{\slv{вѧ́щше, па́че, ли́шше}}},
        \hspca{{\slv{ѕлѣ̀}}~---~{\slv{ѕлѣ́е}}},
        \hspca{{\slv{ле́гцѣ}}~---~{\slv{легча́е}}},
        \hspca{{\slv{ско́рѡ}}~---~{\slv{скорѧ́е}} и т.д.}
    }

    Сравнительная степень имен прилагательных склоняется по образцу склонения прилагательных в положительной степени с основой на шипящие.

                \subsubsection{Превосходная степень имен прилагательных}

    Превосходная степень имен прилагательных образуется через замену их полного окончания на {\slv{-ѣйшїй}} (м. р.), {\slv{-ѣйшаѧ}} (ж. р.), {\slv{-ѣйшее}} (ср. р), например:

    \bigskip\autorows{l}{1}{l}{
        \hspca{{\slv{чи́ст-ый}}~---~{\slv{чист-ѣ́йшїй}}},
        \hspca{{\slv{чи́ст-аѧ}}~---~{\slv{чист-ѣ́йшаѧ}}},
        \hspca{{\slv{чи́ст-ое}}~---~{\slv{чист-ѣ́йшее}}}
    }
    
    или, когда основа прилагательных оканчивается на гортанный, то со смягчением на {\slv{-айшїй}} (м. р.), {\slv{-айшаѧ}} (ж. р.), {\slv{-айшее}} (ср. р), например:
    
    \bigskip\autorows{l}{1}{l}{
        \hspca{{\slv{мно́г-їй}}~---~{\slv{множ-а́йшїй}}},
        \hspca{{\slv{высо́к-аѧ}}~---~{\slv{высоч-а́йшаѧ}}},
        \hspca{{\slv{ти́х-ое}}~---~{\slv{тиш-а́йшее}}}
    }

    Превосходная степень иногда выражается присоединением к положительной степени местоимения {\slv{са́мый}} или наречий: {\slv{ѕѣлѡ̀, вельмѝ, па́че всегѡ̀}}. Например: {\slv{са́мый бли́зкїй, ѕѣлѡ̀ до́бръ, вельмѝ кра́сенъ, па́че всегѡ̀ мꙋ́дръ}}.
    
    Превосходная степень также получается иногда через присоединение к положительной степени приставок {\slv{все-, пре-, веле-, три-, тре-}} и др. Например: {\slv{всеблагі́й, пресвѧты́й, велегла́сный, трисвѧты́й, трегꙋ́бый}} и т.п.
    
    Превосходная степень, выражающая высшую степень качества предмета с указанием на круг предметов, по отношению к которому происходит выделение обладателя качества,~---~называется \emph{соотносительной} (superlativus). Такие формы превосходной степени сочетаются со следущими словами: {\slv{всѣ́хъ, ѿ, въ, ме́ждꙋ}}. Например: {\slv{Ѕмі́й же бѣ̀ мꙋдрѣ́йшїй всѣ́хъ ѕвѣре́й. Мꙋ́жества про́ситъ ѿ тогѡ̀, є҆́же всѣ́хъ є҆́сть немощнѣ́йшее}} (Прем. 13, 18).
    \pagebreak

                \subsubsection{Неправильные степени сравнения}

    Они происходят от следущих прилагательных:
    
    \begin{placedtabular}[%
%        caption={\tabcaptsize Неправильные степени сравнения}
        ]{|c|c|c|}
        
        \hline
        \mkcella{Положительная степень}
        & \mkcella{Сравнительная степень}
        & \mkcella{Превосходная степень}
        \\\hline
        
        {\slv{благі́й}}
        & {\slv{ᲂу҆́нїй, ᲂу҆́ншїй}}
        & {\slv{лꙋ́чшїй}} ({\slva{преблагі́й}})
        \\\hline
        
        {\slv{ве́лїй}} ({\slva{вели́кїй}})
        & {\slv{бо́лѣе, бо́лїй, бо́льшїй}}
        & {\slv{вѧ́щшїй}} ({\slva{велича́йшїй}})
        \\\hline
        
        {\slv{высо́кїй}}
        & {\slv{вы́шшїй}} ({\slva{высоча́й}})
        & {\slv{вы́шнїй}} ({\slva{высоча́йшїй}})
        \\\hline
        
        {\slv{ѕлы́й}}
        & {\slv{го́ршїй, го́рше, горѣ́е}}
        & {\slv{ѕлѣ́йшїй}}
        \\\hline
        
        {\slv{ма́лый}}
        & {\slv{мні́й, мнѣ́е}}
        & {\slv{ме́ньшїй}} ({\slva{малѣ́йшїй}})
        \\\hline

    \end{placedtabular}

            \subsection{Имя числительное}
                \subsubsection{Понятие об имени числительном}

    \textbf{Именем числительным} называется часть речи, обозначающая или количество предметов, или их порядок.
    
    В первом случае имена числительные называтся \emph{количественными} и отвечают на вопрос: <<сколько?>> ({\slv{два̀, пѧ́ть}}).
    
    Во втором случае имена числительные называются \emph{порядковыми} и отвечают на вопрос: <<на который?>> ({\slv{вторы́й, пѧ́тый}}).
    
    Как количественные, так и порядковые имена числительные звательного падежа не имеют.

                \subsubsection{Имена числительные количественные}

    Имена числительные количественные в пределе первого десятка произносятся так: {\slv{є҆ди́нъ, два̀, трїѐ, четы́ре, пѧ́ть, ше́сть, се́дмь, ѻ҆́смь, де́вѧть, де́сѧть}}.
    
    Имена числительные количественные в пределе второго десятка образуются через прибавление десятка к единицам с помощью предлога {\slv{на}}: {\slv{є҆дин̾на́десѧть, четырена́десѧть, ѻ҆смьна́десѧть}} и т.д.
    
    Круглые десятки образуются таким же образом, но без соединительного предлога {\slv{на}}: {\slv{два́десѧть, четы́редесѧть, де́вѧтьдесѧть}} и т.п.
    
    Круглые сотни, кроме числительного {\slv{с҃}} ({\slv{двѣ́сте}}), произносятся как и по-русски: {\slv{сто̀, три́ста, пѧтьсѡ́тъ, ѻ҆смьсѡ́тъ}} и т.д.
    
    Имена числительные, состоящие из десятков и единиц или из сотен, десятков и единиц, соединяются друг с другом или непосредственно, например: {\slv{сто̀ пѧтьдесѧ́ть}}, или посредством союза {\slv{и҆}}: {\slv{сто̀ и҆ пѧтьдесѧ́ть и҆ трѝ}}.
    
    В предложении имена числительные количественные могут быть подлежащим или дополнением. Например: {\slv{Є҆ди́нъ ѿ ва́съ преда́стъ мѧ̀}} (Мф. 26, 21); {\slv{Два́десѧти ла́кѡтъ широта̀ хра́ма}} (3 Цар. 6, 2)
    \pagebreak

                \subsubsection{Имена числительные порядковые}

    Имена числительные порядковые имеют твердое окончание полных имен прилагательных:
    
    \bigskip\autorows{l}{1}{l}{
        \hspca{{\slv{втор-ы́й, втор-а́ѧ, втор-о́е; пѧ́т-ый, пѧ́таѧ, пѧ́т-ое}}}
    }
    
    \noindent
    и склоняются они правильно по их образцу; только числительное {\slv{тре́тїй}} склоняется по образу полного прилагательного {\slv{ве́лїй}}.
    
    Порядковые круглые десятки образуются присоединением единиц к десяткам: {\slv{двадесѧ́тый, тридесѧ́тый, четыредесѧ́тый}}. Но, начиная с {\slv{н҃}}, количество единиц ставится в родительном падеже: {\slv{пѧтидесѧ́тый}} и т.д.
    
    Порядковое составное число может иметь различные формы. Оно может состоять из последнего только порядкового числительного при первых числительных количественных, например: {\slv{сто̀ пѧтьдесѧ́ть тре́тїй}}, или состоит из всех порядковых числительных, например: {\slv{четырехсо́тое пѧ́тое лѣ́то}}.
    
    Имена числительные порядковые в предложении бывают большей частью определением. Например: {\slv{И҆ въ тре́тїй де́нь воста́ти}} (Мф. 16, 21).
    
                \subsubsection{Склонение количественных числительных}

    Имена числительные количественные склоняются вообще как имена существительные, причем по окончанию определяется их склонение. Рассмотрим склонение количественных числительных: {\slv{два̀}} ({\slv{ѻ҆́ба}}), {\slv{трїѐ}}, {\slv{четы́ре}} и {\slv{пѧ́ть}}, как имеющие в своем склонении некоторые особенности.
    
    Имена числительные {\slv{два̀}} и {\slv{ѻ҆́ба}} по своему значению имеют только двойственное число.
    
    \medskip\begin{placedtabular}[%
%        caption={\tabcaptsize Числительные {\slv{два̀}} и {\slv{ѻ҆́ба}}}
        ]{|c|c|c|c|c|c|}
        \hline
        \multirow{2}{*}{\mkcellb{Па-\\деж}}
        & \multicolumn{5}{c|}{\mkcellb{Двойственное число}}
        \\\cline{2-6}
        
        & \mkcellb{м. р.}
        & \mkcellb{ж. и ср. р.}
        & \mkcellb{м. р.}
        & \mkcellb{ж. р.}
        & \mkcellb{ср. р.}
        \\\hline
        
        И. В.
        & \makecell{{\slv{два̀}}}
        & \makecell{{\slv{двѣ̀}}}
        & \makecell{{\slv{ѻ҆́ба}}}
        & \makecell{{\slv{ѻ҆́бѣ}}}
        & \makecell{{\slv{ѻ҆́ба}}}
        \\\hline
        
        Р. П.
        & \multicolumn{2}{c|}{{\slv{двою̀, двꙋ̀}}}
        & \makecell{{\slv{ѻ҆бою̀}}}
        & \makecell{{\slv{ѻ҆бѣю̀}}}
        & \makecell{{\slv{ѻ҆бою̀}}}
        \\\hline
        
        Д. Т.
        & \multicolumn{2}{c|}{{\slv{двѣма̀}}}
        & \multicolumn{3}{c|}{{\slv{ѻ҆бѣ́ма}}}
        \\\hline
        
    \end{placedtabular}
    \pagebreak

    \bigskip
    По своему значению числительное {\slv{трїѐ}} ({\slv{трѝ}}) имеет только множественное число, но с многообразными формами.
    
    \begin{placedtabular}[%
%        caption={\tabcaptsize Числительное {\slv{трїѐ}}}
        ]{|c|c|c|c|}
        \hline
        \multirow{2}{*}{\mkcellb{Па-\\деж}}
        & \multicolumn{3}{c|}{\mkcellb{Множественное число}}
        \\\cline{2-4}
        
        & \mkcellb{м. р.}
        & \mkcellb{ж. р.}
        & \mkcellb{ср. р.}
        \\\hline
        
        И.
        & {\slv{трїѐ, трѝ}}
        & \multicolumn{2}{c|}{{\slv{трѝ}}}
        \\\hline
        
        Р.
        & \multicolumn{3}{c|}{{\slv{трїе́хъ, тре́хъ, трі́й}}}
        \\\hline
        
        Д.
        & \multicolumn{3}{c|}{{\slv{трїе́мъ, тре́мъ}}}
        \\\hline
        
        В.
        & \multicolumn{2}{c|}{{\slv{трїе́хъ, тре́хъ, трѝ}}}
        & {\slv{трѝ}}
        \\\hline

        Т.
        & \multicolumn{3}{c|}{{\slv{трїе́ми, тремѝ}}}
        \\\hline

        П.
        & \multicolumn{3}{c|}{{\slv{ѡ҆ трїе́хъ, ѡ҆ тре́хъ}}}
        \\\hline

    \end{placedtabular}

    \bigskip
    Имя числительное {\slv{четы́ре}} по своему значению имеет только множественное число.

    \begin{placedtabular}[%
%        caption={\tabcaptsize Числительное {\slv{четы́ре}}}
        ]{|c|c|c|}
        \hline
        \multirow{2}{*}{\mkcellb{Па-\\деж}}
        & \multicolumn{2}{c|}{\mkcellb{Множественное число}}
        \\\cline{2-3}
        
        & \mkcellb{м. р.}
        & \mkcellb{ж. и ср. р.}
        \\\hline
        
        И. В.
        & {\slv{четы́ре}}
        & {\slv{четы́ри}}
        \\\hline
        
        Р.
        & \multicolumn{2}{c|}{{\slv{четы́рь, четыре́хъ}}}
        \\\hline
        
        Д.
        & \multicolumn{2}{c|}{{\slv{четы́ремъ}}}
        \\\hline
        
        Т.
        & \multicolumn{2}{c|}{{\slv{четы́рьми}}}
        \\\hline
        
        П.
        & \multicolumn{2}{c|}{{\slv{ѡ҆ четыре́хъ}}}
        \\\hline
        
    \end{placedtabular}

    \bigskip
    Имя числительное {\slv{пѧ́ть}} не имеет единственного и двойственного чисел, а также не имеет родов. По образцу склонения числительного {\slv{пѧ́ть}} склоняются числительные: {\slv{ше́сть, се́дмь, ѻ҆́смь, де́вѧть}} и {\slv{де́сѧть}}.

    \begin{placedtabular}[%
%        caption={\tabcaptsize Числительное {\slv{пѧ́ть}}}
        ]{|c|c|}
        \hline
        \multicolumn{2}{|c|}{\mkcellb{Множественное число}}
        \\\hline

        И. В.
        & {\slv{пѧ́ть}}
        \\\hline
        
        Р.
        & {\slv{пѧтѝ, пѧти́хъ}}
        \\\hline
        
        Д.
        & {\slv{пѧтѝ, пѧти́мъ}}
        \\\hline
        
        Т.
        & {\slv{пѧтьмѝ, пѧтїю̀}}
        \\\hline
        
        П.
        & {\slv{ѡ҆ пѧти́хъ}}
        \\\hline
        
    \end{placedtabular}

        \section{Глагол}
                \subsubsection{Понятие о причастии}

    \textbf{Причастием} называется такая форма глагола, которая, с одной стороны, имеет признаки глагола~---~время, а с другой стороны, имеет признаки имени прилагательного~---~род и падежные окончания.
    
    Поэтому эти формы причастны глаголу и имени прилагательному, почему они и названы \emph{причастиями}.
    
    В церковнославянском языке есть две категории причастий:
    
    \medskip\autorows{l}{1}{l}{
        \hspca{1) несклоняемое (или спрягаемое),},
        \hspca{2) склоняемое.}
    }

                \subsubsection{Несклоняемое (спрягаемое) причастие}

    Несклоняемое причастие образуется от неопределенной формы глагола через перемену окончания {\slv{-ти}} на {\slv{-лъ}}, при этом, если основа глагола оканчивается на зубные {\slv{д-}} или {\slv{т-}}, то эти согласные опускаются. Например:
    
    \bigskip\autorows{l}{1}{l}{
        \hspca{{\slv{ѡ҆долѣва́-ти}}~---~{\slv{ѡ҆долѣва́-лъ}}},
        \hspca{{\slv{нес-тѝ}}~---~{\slv{не́слъ}}},
        \hspca{{\slv{вес-тѝ}} ({\slv{вед-тѝ}})~---~{\slv{ве́-лъ}} (вместо: {\slv{ве́д-лъ}})},
        \hspca{{\slv{плес-тѝ}} ({\slv{плет-тѝ}})~---~{\slv{пле́-лъ}} (вместо: {\slv{пле́т-лъ}})},
        \hspca{{\slv{мо-щѝ}} ({\slv{мог-тѝ}})~---~{\slv{мо́г-лъ}}},
        \hspca{{\slv{ре-щѝ}} ({\slv{рек-тѝ}})~---~{\slv{ре́к-лъ}}}
    }

    Причастие это не склоняется. Оно называется иногда спрягаемым, так как служит для образования сложных времен и наклонений глагола, а потому имеет три рода и три числа.

    \begin{quote}
        Так, в \textbf{единственном} числе это причастие имеет окончание {\slv{-лъ}} для мужского рода ({\slv{твори́-лъ}}), {\slv{-ла}} для женского рода ({\slv{твори́-ла}}), {\slv{-ло}} для среднего рода ({\slv{твори́-ло}});
        
        в \textbf{двойственном} числе: {\slv{-ла}} для мужского рода ({\slv{твори̑-ла}}), {\slv{-ли}} для женского и среднего рода ({\slv{твори̑-ли}}), причем для различения этих форм от созвучных форм единственного и множественного чисел ставится облегченное ударение или, когда возможно, иное начертание одной и той же буквы, а в исключительных случаях, когда ударение падает на первый открытый союз~---~апострофь;
        
        наконец, во \textbf{множественном} числе для всех родов окончанием является {\slv{-ли}} ({\slv{твори́-ли}}).
    \end{quote}
    \pagebreak

    Спрягаемое причастие от вспомогательного глагола {\slv{бы́ти}} имеет следущие формы:
    
    \begin{placedtabular}[%
%        caption={\tabcaptsize Спрягаемое причастие от вспомогательного глагола {\slv{бы́ти}}}
        ]{|c|c|c|c|}
        
        \hline
        \mkcellb{Число}
        & \mkcellb{м. р.}
        & \mkcellb{ж. р.}
        & \mkcellb{ср. р.}
        \\\hline
        
        \mkcellb{Ед.}
        & {\slv{бы́лъ}}
        & {\slv{бы́ла}}
        & {\slv{бы́ло}}
        \\\hline
        
        \mkcellb{Дв.}
        & {\slv{бы́ла}}
        & \multicolumn{2}{c|}{{\slv{бы̑ли}}}
        \\\hline
        
        \mkcellb{Множ.}
        & \multicolumn{3}{c|}{{\slv{бы́ли}}}
        \\\hline

    \end{placedtabular}

    \bigskip
    Под влиянием русского языка спрягаемое причастие употребляется иногда и отдельно, само по себе, тогда оно имеет значение русского прошедшего времени.

                \subsubsection{Прошедшее совершенное время глаголов}

    Прошедшим совершенным временем называется такое прошедшее время глаголов, которое выражает действие не только вполне законченное, но и определенно совершившееся. Например:
    
    \bigskip\autorows{l}{1}{l}{
        \hspca{{\slv{Возлюби́лъ є҆сѝ пра́вдꙋ}} (Пс. 44, 8)}
    }

    Прошедшее совершенное время образуется из спрягаемого причастия данного глагола и вспомогательного глагола {\slv{бы́ти}} в настоящем времени, при этом причастие изменяется по родам и числам, а глагол~---~по числам и лицам.
    
    Вот образец спряжения вспомогательного глагола {\slv{бы́ти}} и тематического глагола {\slv{твори́ти}} в прошедшем совершенном времени.

    \medskip\begin{placedtabular}[%
%        caption={\tabcaptsize Спряжение вспомогательного глагола {\slv{бы́ти}} и тематического глагола {\slv{твори́ти}} в прошедшем совершенном времени}
        ]{|c|c|c|}
        
        \hline
        \mkcella{Лицо}
        & \mkcella{Единственное число}
        & \mkcella{Множественное число}
        \\\hline
        
        \mkcella{1}
        & \makecell[l]{{\slv{бы́лъ, -а̀, -о, є҆́смь}}\\{\slv{твори́лъ, -а, -о, є҆́смь}}}
        & \makecell[l]{{\slv{бы́ли є҆смы̀}}\\{\slv{твори́ли є҆смы̀}}}
        \\\hline
        
        \mkcella{2}
        & \makecell[l]{{\slv{бы́лъ, -а̀, -о, є҆сѝ}}\\{\slv{твори́лъ, -а̀, -о, є҆сѝ}}}
        & \makecell[l]{{\slv{бы́ли є҆стѐ}}\\{\slv{твори́ли є҆стѐ}}}
        \\\hline

        \mkcella{3}
        & \makecell[l]{{\slv{бы́лъ, -а̀, -о, є҆́сть}}\\{\slv{твори́л, -а̀, -о, є҆́сть}}}
        & \makecell[l]{{\slv{бы́ли сꙋ́ть}}\\{\slv{твори́ли сꙋ́ть}}}
        \\\hline
        
        \multicolumn{3}{|c|}{\mkcella{Двойственное число}}
        \\\hline
        
        \multicolumn{2}{|c|}{\mkcella{Мужской род}}
        & \multicolumn{1}{c|}{\mkcella{Жен. и сред. род}}
        \\\hline
        
        \mkcella{1}
        & \makecell[l]{{\slv{бы́ла є҆сва̀}} ({\slv{є҆сма̀}})\\{\slv{твори̑ла є҆сва̀}} ({\slv{є҆сма̀}})}
        & \makecell[l]{{\slv{бы̑ли є҆свѣ̀}}\\{\slv{твори̑ли е҆свѣ̀}}}
        \\\hline
        
        \mkcella{2}
        & \multirow{2}{*}{\makecell[l]{{\slv{бы́ла є҆ста̀}}\\{\slv{твори̑ла̀ є҆ста̀}}}}
        & \multirow{2}{*}{\makecell[l]{{\slv{бы̑ли є҆стѣ̀}}\\{\slv{твори̑ли є҆стѣ̀}}}}
        \\\cline{1-1}
        
        \mkcella{3}
        &
        &
        \\\hline

    \end{placedtabular}

                \subsubsection{Давнопрошедшее время глаголов}
                
    Давнопрошедшим временем называется такое прошедшее время глаголов, которое выражает действие или состояние давно совершившееся и окончившееся прежде другого прошедшего действия или состояния. Например:
    
    \bigskip\autorows{l}{1}{l}{
        \hspca{{\slv{Се́й ме́ртвъ бѣ̀ и҆ ѡ҆живѐ}} (Лк. 15, 24)}
    }
    
    Давнопрошедшее время образуется из спрягаемого причастия данного глагола и вспомогательного глагола {\slv{бы́ти}} в преходящем времени, причем причастие изменяется по родам и числам, а глагол~---~по числам и лицам.
    
    При спряжении изменение причастия на {\slv{-лъ}} и связки (глагола {\slv{бы́ти}}) происходит подобным образом, как и у перфекта (прошедшее совершенное время глаголов). Примеры: {\slv{и҆зги́блъ бѣ̀, и҆ ѡ҆брѣ́тесѧ}} (Лк. 15, 24); {\slv{мно́зи ѿ і҆ꙋдє́й бѧ́хꙋ пришлѝ къ ма́рѳѣ и҆ марі́и}} (Ин. 11, 19); {\slv{и҆ тьма̀ а҆́бїе сы́сть, и҆ не ᲂу҆̀ бѣ̀ прише́лъ къ ни̑мъ і҆и҃съ}} (Ин. 6, 17).
    
    Возьмем за образцы спряжения те же глаголы {\slv{бы́ти}} и {\slv{твори́ти}}.
    
    \medskip\begin{placedtabular}[%
%        caption={\tabcaptsize Давнопрошедшее время глаголов {\slv{бы́ти}} и {\slv{твори́ти}}}
        ]{|c|c|c|}
        
        \hline
        \mkcella{Лицо}
        & \mkcella{Единственное число}
        & \mkcella{Множественное число}
        \\\hline
        
        \mkcella{1}
        & \makecell[l]{{\slv{бы́лъ, -а̀, -о бѧ́хъ, бѣ́хъ}}\\{\slv{твори́лъ, -а, -о бѧ́хъ, бѣ́хъ}}}
        & \makecell[l]{{\slv{бы́ли бѧ́хомъ, бѣ́хомъ}}\\{\slv{твори́ли бѧ́хомъ, бѣ́хомъ}}}
        \\\hline
        
        \mkcella{2}
        & \makecell[l]{{\slv{бы́лъ, -а̀, -о бѣ̀}}\\{\slv{твори́лъ, -а̀, -о бѣ̀}}}
        & \makecell[l]{{\slv{бы́ли бѧ́сте, бѣ́сте}}\\{\slv{твори́ли бѧ́сте, бѣ́сте}}}
        \\\hline
        
        \mkcella{3}
        & \makecell[l]{{\slv{бы́лъ, -а̀, -о бѧ́ше, бѣ̀}}\\{\slv{твори́л, -а̀, -о бѧ́ше, бѣ̀}}}
        & \makecell[l]{{\slv{бы́ли бѧ́хꙋ, бѣ́хꙋ, бѣ́ша}}\\{\slv{твори́ли бѧ́хꙋ, бѣ́хꙋ, бѣша̀}}}
        \\\hline
        
        \multicolumn{3}{|c|}{\mkcella{Двойственное число}}
        \\\hline
        
        \multicolumn{2}{|c|}{\mkcella{Мужской род}}
        & \multicolumn{1}{c|}{\mkcella{Жен. и сред. род}}
        \\\hline
        
        \mkcella{1}
        & \makecell[l]{{\slv{бы́ла бѧ́хова, бѣ́хова}}\\{\slv{твори̑ла бѧ́хова, бѣ́хова}}}
        & \makecell[l]{{\slv{бы̑ли бѧ́ховѣ, бѣ́ховѣ}}\\{\slv{твори̑ли бѧ́ховѣ, бѣ́ховѣ}}}
        \\\hline
        
        \mkcella{2}
        & \multirow{2}{*}{\makecell[l]{{\slv{бы́ла бѧ́ста, бѣ́ста}}\\{\slv{твори̑ла бѧ́ста, бѣ́ста}}}}
        & \multirow{2}{*}{\makecell[l]{{\slv{бы̑ли бѧ́стѣ, бѣ́стѣ}}\\{\slv{твори̑ла бѧ́стѣ, бѣ́стѣ}}}}
        \\\cline{1-1}
        
        \mkcella{3}
        &
        &
        \\\hline
        
    \end{placedtabular}

                \subsubsection{Условное наклонение глаголов}

    Кроме рассмотренных нами наклонений глагола: изъявительного, желательного и повелительного, в церковнославянском языке имеется еще \textbf{условное наклонение}.
    
    Условное наклонение глагола выражает действие, возможное только при каком-либо условии или обстоятельстве, от которого зависит это действие. Например:
    
    \bigskip\autorows{l}{1}{l}{
        \hspca{{\slv{Ка́кѡ смѣ́ли бы́хомъ, сп҃се, пѣ́ти тѧ̀;}} (Вел. повечерие)}
    }
    
    Это наклонение имеет одно прошедшее время, которое так и называется \emph{прошедшим}.
    
    Прошедшее время условного наклонения составляется из спрягаемого причастия данного глагола и аориста вспомогательного глагола {\slv{бы́ти}}, причем причастие изменяется только по родам и числам, а глагол {\slv{бы́ти}}~---~по числам и лицам.
    
    Возьмем для спряжения те же глаголы {\slv{бы́ти}} и {\slv{твори́ти}}.
    
    \medskip\begin{placedtabular}[%
%        caption={\tabcaptsize Условное наклонение глаголов {\slv{бы́ти}} и {\slv{твори́ти}}}
        ]{|c|c|c|}
        
        \hline
        \mkcella{Лицо}
        & \mkcella{Единственное число}
        & \mkcella{Множественное число}
        \\\hline
        
        \mkcella{1}
        & \makecell[l]{{\slv{бы́лъ, -а̀, -о бы́хъ}}\\{\slv{твори́лъ, -а, -о бы́хъ}}}
        & \makecell[l]{{\slv{бы́ли бы́хомъ}}\\{\slv{твори́ли бы́хомъ}}}
        \\\hline
        
        \mkcella{2}
        & \makecell[l]{{\slv{бы́лъ, -а̀, -о бы̀}}\\{\slv{твори́лъ, -а̀, -о бы̀}}}
        & \makecell[l]{{\slv{бы́ли бы́сте}}\\{\slv{твори́ли бы́сте}}}
        \\\hline
        
        \mkcella{3}
        & \makecell[l]{{\slv{бы́лъ, -а̀, -о бы̀}}\\{\slv{твори́л, -а̀, -о бы̀}}}
        & \makecell[l]{{\slv{бы́ли бы́ша}}\\{\slv{твори́ли бы́ша}}}
        \\\hline
        
        \multicolumn{3}{|c|}{\mkcella{Двойственное число}}
        \\\hline
        
        \multicolumn{2}{|c|}{\mkcella{Мужской род}}
        & \multicolumn{1}{c|}{\mkcella{Жен. и сред. род}}
        \\\hline
        
        \mkcella{1}
        & \makecell[l]{{\slv{бы́ла бы́хова, бы́сва}}\\{\slv{твори̑ла бы́хова, бы́сва}}}
        & \makecell[l]{{\slv{бы̑ли бы́ховѣ, бы́свѣ}}\\{\slv{твори̑ла бы́ховѣ, бы́свѣ}}}
        \\\hline
        
        \mkcella{2}
        & \multirow{2}{*}{\makecell[l]{{\slv{бы́ла бы́ста}}\\{\slv{твори̑ла бы́ста}}}}
        & \multirow{2}{*}{\makecell[l]{{\slv{бы̑ли бы́стѣ}}\\{\slv{твори̑ла бы́стѣ}}}}
        \\\cline{1-1}
        
        \mkcella{3}
        &
        &
        \\\hline
        
    \end{placedtabular}

    \bigskip
    Иногда вместо аориста вспомогательного глагола {\slv{бы́ти}} во 2-м лице единственного числа присоединяется тот же глагол в настоящем времени этого лица {\slv{є҆сѝ}}. Тогда форма аориста {\slv{бы̀}} обращается в частицу {\slv{бы}}. Что же касается 3-го лица единственного числа, то в этом случае очень часто форма аориста {\slv{бы̀}} заменяется просто частицей {\slv{бы}} (т.е. пишется без ударения).
    
    Частица {\slv{бы}} может стоять как после глагола, так и перед ним, а также может быть отделена от глагола другими словами.
    \pagebreak
    
                \subsubsection{Безличные глаголы}

    \textbf{Безличными глаголами} называются такие, которые обозначают действие или состояние без действующего предмета или явления, а как бы сами собою. Например: {\slv{Подоба́етъ бо тлѣ́нномꙋ семꙋ̀ ѡ҆блещи́сѧ въ нетлѣ́нїи}} (1 Кор. 15, 53); {\slv{Не вѣ́сте, что̀ ᲂу҆́трѣ слꙋчи́тсѧ}} (Иак. 4, 14).
    
    Безличные глаголы не изменяются ни по числам, ни по лицам; они употребляются только в форме 3-го лица единственного числа настоящего, или преходящего, или прошедшего совершенного времени. Например:
    
    \medskip\begin{adjustwidth}{\hstbb}{0cm}
        \begin{tabular}[t]{l|l|l}
            
            \mkcella[tc]{Настоящее время}
            & \mkcella[tc]{Преходящее время}
            & \mkcella[tc]{Прошедшее совершенное\\время}
            \\
            
            {\slv{мни́тсѧ}} {\tiny (думается, кажется)}
            & {\slv{мнѧ́шесѧ}}
            & {\slv{мни́лосѧ є҆́сть}}
            \\
            
            {\slv{подоба́етъ}} {\tiny (надлежит)}
            & {\slv{подоба́ше}}
            & {\slv{подоба́ло є҆́сть}}
            \\
            
            {\slv{слꙋча́етсѧ}}
            & {\slv{слꙋча́шесѧ}}
            & {\slv{слꙋча́лосѧ є҆́сть}}
            \\
            
        \end{tabular}
    \end{adjustwidth}

    \medskip
    \noindent
    а также в аористе и будущем времени, если глаголы однократного вида:
    
    \medskip\begin{adjustwidth}{\hstbb}{0cm}
        \begin{tabular}[t]{l|l}
            
            \mkcella[tc]{Аорист}
            & \mkcella[tc]{Будущее время}
            \\
            
            {\slv{возмнѣ́сѧ}}
            & {\slv{возмни́тсѧ}}
            \\
            
            {\slv{восподоба̀}}
            & {\slv{восподоба́етъ}}
            \\
            
            {\slv{слꙋчи́сѧ}}
            & {\slv{слꙋчи́тсѧ}}
            \\
            
        \end{tabular}
    \end{adjustwidth}

    \medskip
    В качестве безличного глагола употребляется иногда в аористе глагол {\slv{бы́ти}}. Например:
    
    \medskip\autorows{l}{1}{l}{
        \hspca{{\slv{Бы́сть же во дни̑ ты̑ѧ}} (Лк. 2, 1)}
    }

    В этом случае форма {\slv{бы́сть}} переводится по-русски словами: <<было>>, <<случилось>>.
    
    В составных безличных глаголах, каковы: {\slv{лѣ́сть є҆́сть}} (можно, позволено), {\slv{мо́щно є҆́сть}} (возможно), {\slv{ꙗ҆́вѣ є҆́сть}} (явно, открыто), {\slv{го́дѣ є҆́сть}} (угодно, приятно), {\slv{досто́йно є҆́сть}} (пристойно, прилично) и некоторые другие, спрягается только в указанных временах вспомогательный, например:
    
    \bigskip\autorows{l}{2}{l}{
        \hspca{наст. вр.: {\slv{мо́щно є҆́сть}}}, {аорист: {\slv{мо́щно бы́сть}}},
        \hspca{преход. вр.: {\slv{мо́щно бѣ̀}}}, {будущ. вр.: {\slv{мо́щно бꙋ́детъ}}}
    }
    
    Безличные глаголы в предложении бывают сказуемыми, но при них не может быть подлежащего; оно только как бы каким-то неопределенным образом подразумевается, причем всегда как бы в среднем роде, поэтому и безличный глагол согласуется с подразумеваемым подлежащим в этом же, т.е. среднем роде, что особенно ярко выражается в формах прошедшего времени, например: {\slv{мни́лосѧ є҆́сть, подоба́ло є҆́сть}} и т.д.
    
    Иногда в качестве безличного глагола употребляется глагол {\slv{нѣ́сть}}. Например:
    
    \medskip\autorows{l}{1}{l}{
        \hspca{{\slv{И҆́стины нѣ́сть въ на́съ}} (1 Ин. 1, 8)}
    }

                \subsubsection{Возвратная форма глагола}

    Возвратной формой глагола называется такая его форма, которая показывает, что действие, выражаемое глаголом, относится к самому деятелю, а не переходит на другой предмет. Например:
    
    \bigskip\autorows{l}{1}{l}{
        \hspca{{\slv{Что̀ хва́лишисѧ во ѕло́бѣ, си́льне;}} (Пс. 51, 3)}
    }

    Здесь {\slv{хва́лиши сѧ̀}}, т.е. <<хвалишь себя>>.
    
    Возвратная форма глагола и характеризуется всегда конечным слогом {\slv{-сѧ}}. Этот слог {\slv{-сѧ}} представляет собой энклитическую форму винительного падежа  ({\slv{сѧ̀}}) возвратного местоимения {\slv{себє̀}}, слившуюся с глагольным окончанием.
    
    Иногда этот конечный слог {\slv{-сѧ}} отрывается от глагола и ставится впереди его. Например:
    
    \medskip\autorows{l}{1}{l}{
        \hspca{{\slv{Гдⷭ҇и, что̀ {\large сѧ ᲂу҆мно́жиша} стꙋжа́ющїи мѝ;}} (Пс. 3, 1)}
    }
    
    В некоторых случаях между {\slv{сѧ}} и глаголом могут помещаться в энклитической форме личные местоимения {\slv{мѝ}}, {\slv{тѝ}}, служащие в предложении дополнением, а также союз {\slv{бо}} и частица {\slv{же}}. Например:
    
    \bigskip\autorows{l}{1}{l}{
        \hspca{{\slv{Насы́щꙋсѧ, внегда̀ {\large ꙗ҆ви́ти ми сѧ} сла́вѣ твое́й}} (Пс. 16, 15)},
        \hspca{{\slv{Человѣ́че, {\large ѡ҆ставлѧ́ютъ ти сѧ} грѣсѝ твоѝ}} (Лк. 5, 20)},
        \hspca{{\slv{{\large Чермнꙋ̀етъ бо сѧ} не́бо}} (Мф. 16, 2)},
        \hspca{{\slv{{\large Возвесели́ти же сѧ} подоба́ше}} (Лк. 15, 32)}
    }

    В этих случаях над личными местоимениями {\slv{мѝ}} и {\slv{тѝ}} ударение не ставится, а иногда даже наблюдается их слитное правописание с глаголами. Например: {\slv{ꙗ҆ви́ти{\large ми}сѧ, мо́лим{\large ти}сѧ}} и т.п.
    
    Изредка между {\slv{сѧ}} и глаголом могут помещаться слова и в том случае когда {\slv{сѧ}} стоит впереди глагола. Например:
    
    
    \bigskip\autorows{l}{1}{l}{
        \hspca{{\slv{Что̀ сѧ ва́мъ {\large мни́тъ};}} (Мф. 21, 28)}
    }

    Если два возвратных глагола фигурируют в одном предложении, то при втором глаголе {\slv{-сѧ}} обычно опускается. Например:

    \bigskip\autorows{l}{1}{l}{
        \hspca{{\slv{Да не {\large смꙋща́етсѧ} се́рдце ва́ше, ни {\large ᲂу҆страша́етъ}}}},
        \hspca{(вместо: {\slv{ᲂу҆страша́етсѧ}}) (Ин. 14, 24)}
    }

                \subsubsection{Страдательная форма глагола}

    \textbf{Страдательной формой} глагола называется такая его форма, которая выражает действие, испытываемое предметом от других деятелей. Например:
    
    \bigskip\autorows{l}{1}{l}{
        \hspca{{\slv{Всѧ́чєскаѧ тѣ́мъ}} ({\slv{і҆и҃сомъ}}) {\slv{{\large созда́шасѧ}}} (Кол. 1, 16)}
    }

    Страдательная форма глагола характеризуется также конечным слогом {\slv{-сѧ}}, присоединяемым к глагольному окончанию, но отличается от возвратной формы тем, что этот слог {\slv{-сѧ}} уже не фигурирует здесь в качестве возвратного местоимения {\slv{сѧ̀}}, т.е. не придает глаголу возвратного значения, а только представляет собой частицу.
    
    Глагол в страдательной форме требует после себя дополнения в творительном падеже. Так, в приведенном примере:
    
    \medskip\autorows{l}{1}{l}{
        \hspca{{\slv{Всѧ́чєскаѧ созда́шасѧ}} (Кем?)~---~{\slv{{\large тѣ́мъ}}} (тв. па. <<Им>>)}
    }
    
    Иногда творительный падеж дополнения можно заменить родительным с предлогом {\slv{ѿ}}. В этом случае наш пример может принять такую форму:
    
    \bigskip\autorows{l}{1}{l}{
        \hspca{{\slv{Всѧ́чєскаѧ}} (от Кого?)~---~{\slv{{\large ѿ тогѡ̀} созда́шасѧ}}}
    }
    
    Очень часто страдательная форма глагола выражается сложной формой, состоящей из причастия страдательной формы в соединении со вспомогательным глаголом {\slv{бы́ти}}, как будет указано далее.

                \subsubsection{Склоняемые причастия}

    Склоняемые причастия разделяются на краткие и полные, те и другие бывают настоящего и прошедшего времени; кроме того, они могут быть дйствитльными и страдательными.
    
    \emph{Действительные} причастия обозначают признак предмета, который сам действует или действовал. Например:
    
    \bigskip\autorows{l}{1}{l}{
        \hspca{{\slv{Возвеселѧ́тсѧ всѝ, ᲂу҆пова́ющїи на гдⷭ҇а}} (Пс. 5, 12)},
        \hspca{{\slv{Сохрани́ мѧ, гдⷭ҇и, ѿ лица̀ нечести́выхъ, ѡ҆стра́стшихъ мѧ̀}}},
        \hspca{(Пс. 16, 8--9)}
    }
    
    \emph{Страдательные} причастия показывают признаки предмета, который испытывает или испытывал действие со стороны другого предмета. Например:
    
    \bigskip\autorows{l}{1}{l}{
        \hspca{{\slv{Хрⷭ҇те бж҃е, спрославлѧ́емый о҆ц҃ꙋ и҆ ст҃о́мꙋ дх҃ꙋ, спасѝ на́съ}} (Изобраз.)},
        \hspca{{\slv{Оу҆ничиже́нъ є҆́сть пред̾ ни́мъ лꙋка́внꙋѧй}} (Пс. 14, 4)}
    }

                \subsubsection{Краткие действительные причастия настоящего времени}

    Краткие действительные причастия настоящего времени образуются через присоединение к основе глаголов настоящего времени окончаний:
    
    \medskip\begin{adjustwidth}{\hstbb}{0cm}
        \begin{tabular}[c]{lll}
            
            {\tiny для м. р.:} {\slv{-а}} ({\slv{-ѧ}}), {\slv{-ы}}
            & {\slv{-ꙋщъ}} ({\slv{-ющъ}}) {\tiny (для 1-го спр.)}
            & {\slv{-ащъ}} ({\slv{-ѧщъ}}) {\tiny (для 2-го спр.)}
            \\
            
            {\tiny для ж. р.:}
            & {\slv{-ꙋщи}} ({\slv{-ющи}}) {\tiny (для 1-го спр.)}
            & {\slv{-ащи}} ({\slv{-ѧщи}}) {\tiny (для 2-го спр.)}
            \\
            
            {\tiny для ср. р.:}
            & {\slv{-ꙋще}} ({\slv{-юще}}) {\tiny (для 1-го спр.)}
            & {\slv{-аще}} ({\slv{-ѧще}}) {\tiny (для 2-го спр.)}
            \\

        \end{tabular}
    \end{adjustwidth}

    \medskip\begin{placedtabular}[%
        ]{|c|c|c|c|}
        \hline
        \multicolumn{2}{|c|}{\mkcella{Мужской род}}
        & \mkcella{Женский род}
        & \mkcella{Средний род}
        \\\hline
        
        {\slv{нес{\large ы̀}}}
        & {\slv{нес{\large ꙋ́щъ}}}
        & {\slv{нес{\large ꙋ́щи}}}
        & {\slv{нес{\large ꙋ́ще}}}
        \\\hline
        
        {\slv{бї{\large ѧ̀}}}
        & {\slv{бї{\large ю́щъ}}}
        & {\slv{бї{\large ю́щи}}}
        & {\slv{бї{\large ю́ще}}}
        \\\hline
        
                    
        {\slv{слы́ш{\large а}}}
        & {\slv{слы́ш{\large ащъ}}}
        & {\slv{слы́ш{\large ащи}}}
        & {\slv{слы́ш{\large аще}}}
        \\\hline
        
                    
        {\slv{хва́л{\large а}}}
        & {\slv{хва́л{\large ѧщъ}}}
        & {\slv{хва́л{\large ѧщи}}}
        & {\slv{хва́л{\large ѧще}}}
        \\\hline

    \end{placedtabular}
    
    \bigskip
    Как видно из приведенной таблицы, краткое причастие мужского рода имеет две формы; одна из них, с окончанием на {\slv{-а}} ({\slv{-ѧ}}), {\slv{ы}}, перешла в русский язык, получив название \emph{деепричастия}, причем окончание {\slv{-ы}} изменилось в \textbf{я} ({\slv{несы̀}} = неся).
    
    От глаголов совершенного вида, как не имеющих форм настоящего времени, не может быть и причастий настоящего времени (как кратких, так и полных).
    
    Краткие действительные причастия настоящего времени склоняются по образцу кратких имен прилагательных с основой на шипящие, причем именительный падеж множественного числа будет иметь окончания: {\slv{-ще}} (м. р.), {\slv{-щы}} (ж. р.), {\slv{-ща}} (ср. р.). Например: {\slv{творѧ̑ще}} (м. р.), {\slv{творѧ́щы}} (ж. р.), {\slv{творѧ̑ща}} (ср. р.).
    
                \subsubsection{Полные действительные причастия настоящего времени}

    Полные действительные причастия настоящего времени образуются из кратких того же времени через замену кратких окончаний на полные, а именно:
    
    \medskip\begin{adjustwidth}{\hstbb}{0cm}
        \begin{tabular}[l]{ll}
            
            {\small окончания м. р.:}
            & {\slv{-ай}} ({\slv{-ѧй}}), {\slv{-ый, -ꙋщїй}} ({\slv{-ющїй}}), {\slv{-ащїй}} ({\slv{-ѧщий}})
            \\
            
            {\small окончания ж. р.:}
            & {\slv{-ущая}} ({\slv{-ющая}}), {\slv{-ащая}} ({\slv{-ящая}})
            \\
            
            {\small окончания ср. р.:}
            & {\slv{-ущее}} ({\slv{-ющее}}), {\slv{-ащее}} ({\slv{-ящее}})
            \\
            
        \end{tabular}
    \end{adjustwidth}
    
    \medskip\begin{placedtabular}[%
        ]{|c|c|c|c|}
        \hline
        \multicolumn{2}{|c|}{\mkcella{Мужской род}}
        & \mkcella{Женский род}
        & \mkcella{Средний род}
        \\\hline
        
        {\slv{нес{\large ы́й}}}
        & {\slv{нес{\large ꙋ́щий}}}
        & {\slv{нес{\large ꙋ́щаѧ}}}
        & {\slv{нес{\large ꙋ́щее}}}
        \\\hline
        
        {\slv{бї{\large ѧ́й}}}
        & {\slv{бї{\large ю́щїй}}}
        & {\slv{бї{\large ю́щаѧ}}}
        & {\slv{бї{\large ю́щее}}}
        \\\hline
        
        
        {\slv{слы́ш{\large ай}}}
        & {\slv{слы́ш{\large ащїй}}}
        & {\slv{слы́ш{\large ащаѧ}}}
        & {\slv{слы́ш{\large ащее}}}
        \\\hline
        
        
        {\slv{хва́л{\large ѧй}}}
        & {\slv{хва́л{\large ѧщїй}}}
        & {\slv{хва́л{\large ѧщаѧ}}}
        & {\slv{хва́л{\large ѧщее}}}
        \\\hline
        
    \end{placedtabular}

    \bigskip
    Полные причастия настоящего времени склоняются по образцу полных имен прилагательных с основой на шипящие, причем именительный падеж множественного числа будет иметь окончания {\slv{-щїи}} (м. р.), {\slv{-щыѧ}} (ж. р.), {\slv{-щаѧ}} (ср. р.). Например: {\slv{творѧ́щїи}} (м. р.), {\slv{творѧ́щыѧ}} (ж. р.), {\slv{творѧ̑щаѧ}} (ср. р.).
    
    У действительных причастий настоящего времени кратких и полных, образованных от возвратных глаголов, ко всем падежным окончаниям присоединяется частица {\slv{-сѧ}}: {\slv{креща́сѧ, креща́йсѧ, креща́ющїйсѧ}}.

                \subsubsection{Краткие действительные причастия прошедшего времени}

    Краткие действительные причастия прошедшего времени образуются от основы неопределенной формы глагола, причем, если основа оканчивается на \textbf{согласный}, то к ней непосредственно присоединятся окончания: {\slv{-ъ}} или {\slv{-шъ}} (м. р.), {\slv{-ши}} (ж. р.), {\slv{-ше}} (ср. р.), а если основа оканчивается на \textbf{гласный}, то к ней непосредственно присоединяются окончания: {\slv{-въ}} или {\slv{-вшъ}} (м. р.), {\slv{-вши}} (ж. р.), {\slv{-вше}} (ср. р.).

    \begin{placedtabular}[%
%        caption={\tabcaptsize Краткие действительные причастия прошедшего времени}
        ]{|c|c|c|c|}
        \hline
        \multicolumn{2}{|c|}{\mkcella{Мужской род}}
        & \mkcella{Женский род}
        & \mkcella{Средний род}
        \\\hline
        
        {\slv{не́с{\large ъ}}}
        & {\slv{не́с{\large шъ}}}
        & {\slv{не́с{\large ши}}}
        & {\slv{не́с{\large ше}}}
        \\\hline
        
        {\slv{слы́ша{\large въ}}}
        & {\slv{слы́ша{\large вшъ}}}
        & {\slv{слы́ша{\large вши}}}
        & {\slv{слы́ша{\large вше}}}
        \\\hline
        
        
        {\slv{хвали́{\large въ}}}
        & {\slv{хвали́{\large вшъ}}}
        & {\slv{хвали́{\large вши}}}
        & {\slv{хвали́{\large вше}}}
        \\\hline
        
    \end{placedtabular}

    \bigskip
    Краткие действительные причастия прошедшего времени (как и настоящего времени) с окончанием {\slv{-ъ}} ({\slv{-въ}}) в мужском роде и с окончанием {\slv{-ши}} ({\slv{-вши}}) в женском роде перешли в русский язык в качестве деепричастий прошедшего времени.
    
    Краткие причастия прошедшего вреени от глаголов 2-го спряжения (с тематической основой {\slv{и}}) часто сокращают окончание {\slv{-въ}} на {\slv{-ь}} со смягчением предыдущего согласного. Например: {\slv{сотвори́въ}}~---~{\slv{сотво́рь}}, {\slv{ѡ҆ста́вивъ}}~---~{\slv{ѡ҆ста́вль}}, {\slv{и҆спо́лнивъ}}~---~{\slv{и҆спо́лнь}}.
    
    Глагол {\slv{бы́ти}}, в отличие от тематических и даже прочих архаических глаголов, имеет причастие (как краткое, так и полное) будущего времени: {\slv{бꙋ́дꙋщъ}} (м. р.), {\slv{бꙋ́дꙋщи}} (ж. р.), {\slv{бꙋ́дꙋще}} (ср. р.), причем форма {\slv{бꙋ́дꙋщи}} (ж. р.) перешла в русский язык в качестве деепричастия <<будучи>>.
    
    Склоняются краткие действительные причастия прошедшего времени по образцу склонений кратких имен прилагательных с основой на шипящую, причем именительный падеж множественного числа будет иметь окончания: {\slv{-вше}} (м. р.), {\slv{-вшы}} (ж. р.), {\slv{-вша}} (ср. р.). Например: {\slv{твори̑вше}} (м. р.), {\slv{твори́вшы}} (ж. р.), {\slv{твори̑вша}} (ср. р.).

                \subsubsection{Полные действительные причастия прошедшего времени}

    Полные действительные причастия прошедшего времени образуются из кратких того же времени через замену кратких окончаний на полные, а именно:
    
    \medskip\begin{adjustwidth}{\hstbb}{0cm}
        \begin{tabular}[l]{ll}
            
            {\small м. р. имеет окончания:}
            & {\slv{-шїй, -вшїй}} ({\slv{-вый}})
            \\
            
            {\small ж. р. имеет окончания:}
            & {\slv{-шаѧ, -вшаѧ}}
            \\
            
            {\small ср. р. имеет окончания:}
            & {\slv{-шее, -вшее}}
            \\
            
        \end{tabular}
    \end{adjustwidth}
    
    \begin{placedtabular}[%
%        caption={\tabcaptsize Полные действительные причастия прошедшего времени}
        ]{|c|c|c|c|}
        \hline
        \multicolumn{2}{|c|}{\mkcella{Мужской род}}
        & \mkcella{Женский род}
        & \mkcella{Средний род}
        \\\hline
        
        \multicolumn{2}{|c|}{\slv{не́с{\large шїй}}}
        & {\slv{не́с{\large шаѧ}}}
        & {\slv{не́с{\large шее}}}
        \\\hline
        
        {\slv{хвали́{\large вшїй}}}
        & {\slv{хвали́{\large вый}}}
        & {\slv{хвали́{\large вшаѧ}}}
        & {\slv{хвали́{\large вшее}}}
        \\\hline

        {\slv{твори́{\large вшїй}}}
        & {\slv{твори́{\large вый}}}
        & {\slv{твори́{\large вшаѧ}}}
        & {\slv{твори́{\large вшее}}}
        \\\hline
        
    \end{placedtabular}

    \bigskip
    В глаголах совершенного вида вместо причастного окончания {\slv{-вый}} встречается окончание {\slv{-ей}} со смягчением предшествовавшего согласного. Например: {\slv{просвѣти́вый}}~---~{\slv{просвѣще́й}}, {\slv{ꙗ҆ви́выйсѧ}}~---~{\slv{ꙗ҆вле́й}}.
    
    Склоняются рассмотренные причастия по образцу полных прилагательных с основой на шипящие, причем именительный падеж множественного числа будет иметь окончания:
    
    \medskip\begin{adjustwidth}{\hstbb}{0cm}
        \begin{tabular}[l]{ll}
            
            {\small для мужского рода:}
            & {\slv{-шїи}} ({\slv{-вшїи}}): {\slv{не́сшїи, твори́вшїи}}
            \\
            
            {\small для женского рода:}
            & {\slv{-шыѧ}} ({\slv{-вшыѧ}}): {\slv{не́сшыѧ, твори́вшыѧ}}
            \\
            
            {\small для среднего рода:}
            & {\slv{-шаѧ}} ({\slv{-вшаѧ}}): {\slv{нє́сшаѧ, твори̑вшаѧ}}
            \\
            
        \end{tabular}
    \end{adjustwidth}

    \medskip
    У действительных причастий прошедшего времени кратких и полных, образованных от возвратных глаголов, ко всем падежным окончаниям присоединяется частица {\slv{-сѧ}}: {\slv{крести́всѧ, крести́выйсѧ}}.

                \subsubsection{Краткие страдательные причастия настоящего времени}

    Краткие страдательные причастия настоящего времени имеют одинаковую форму с первым лицом множественного числа настоящего времени изъявительного наклонения при условии принятия трех родов.
    
    Например, 1-е лицо мн. ч. наст. вр. будет: ({\slv{мы̀}}) {\slv{твори́мъ}}, а причастия будут: муж. рода~---~{\slv{твори́мъ}} ({\slv{мь}}), жен. рода~---~{\slv{твори́ма}}, а сред. рода~---~{\slv{твори́мо}}.
    
    В мужском роде иногда, для различения формы причастия от 1-го лица мн. ч. наст. времени принимается окончание {\slv{-мь}} вместо {\slv{-мъ}}, а иногда переносится ударение, например: {\slv{лю́бимъ}} (1-е лицо мн. ч. наст. вр.), {\slv{люби́мъ}} (краткое причастие).
    
    Родовые окончания (совместно с суффиксом {\slv{-м-}}) краткого причастия {\slv{-мъ, -ма, -мо}} присоединяются к основе глагола настоящего времени через тематические гласные {\slv{-е-}} и {\slv{-и-}}. Например:
    
    \bigskip\autorows[-1pt]{l}{2}{l}{
        \hspca{{\slv{велича́-ю}}:}, {{\slv{велича́-е-ма, велча́-е-мо}}},
        \hspca{{\slv{хвал-ю̀}}:}, {{\slv{хвал-и́-мъ, хвал-и́-ма, хвал-и́-мо}}}
    }

    Но эти же родовые окончания могут иногда присоединяться и через соединительный гласный {\slv{-о-}}, если глаголы первообразные. Например:
    
    \bigskip\autorows[-1pt]{l}{2}{l}{
        \hspca{{\slv{нес-тѝ}}:}, {{\slv{нес-о́-мъ, нес-о́-ма, нес-о́-мо}}}
    }

    Склоняются краткие страдательные причастия настоящего времени по образцу прилагательных с кратким окончанием ({\slv{мла́дъ, млада̀, мла́до}}).

                \subsubsection{Полные страдательные причастия настоящего времени}

    Они образуются из кратких через перемену окончаний:
    
    \bigskip\autorows[-1pt]{l}{5}{l}{
        \hspca{мужского рода}, {{\slv{-мъ}}}, {на}, {{\slv{-мый}}:}, {{\slv{твори́мый}}},
        \hspca{женского рода}, {{\slv{-ма}}}, {на}, {{\slv{-маѧ}}:}, {{\slv{твори́маѧ}}},
        \hspca{среднего рода}, {{\slv{-мо}}}, {на}, {{\slv{-мое}}:}, {{\slv{твори́мое}}}
    }

    Полные страдательные причастия настоящего времени почти переходят уже в разряд полных имен прилагательных, а потому и склоняются без всяких особенностей по их образцу с окончанием на {\slv{-ый}} ({\slv{до́брый, мꙋ́дрый}}) и т.п.

                \subsubsection{Краткие страдательные причастия прошедшего времени}

    Такие причастия образуются от неопределенной формы глагола через замену окончания {\slv{-ти}} на окончания причастий:
    
    \bigskip\autorows[-1pt]{l}{9}{l}{
        \hspca{м. р.}, {{\slv{-нъ}}:}, {{\slv{глаго́ланъ}},}, {или}, {{\slv{-енъ}}:}, {{\slv{несе́нъ}}}, {или}, {{\slv{-тъ}}:}, {{\slv{покры́тъ}}}, 
        \hspca{ж. р.}, {{\slv{-на}}:}, {{\slv{глаго́лана}},}, {или}, {{\slv{-ена}}:}, {{\slv{несе́на}}}, {или}, {{\slv{-та}}:}, {{\slv{покры́та}}}, 
        \hspca{ср. р.}, {{\slv{-но}}:}, {{\slv{глаго́лано}},}, {или}, {{\slv{-ено}}:}, {{\slv{несе́но}}}, {или}, {{\slv{-то}}:}, {{\slv{покры́то}}}
    }

    Если основа глагола оканчивается на гортанные, зубные ({\slv{д, с, т}}) и губные, то при образовании причастий происходит по соответствующим законам смягчение. Например:
    
    \bigskip\autorows[-1pt]{l}{4}{l}{
        \hspca{{\slv{рещѝ}} ({\slv{рек-тѝ}}):}, {{\slv{рече́нъ}}}, {{\slv{рече́на}}}, {{\slv{рече́но}}},
        \hspca{{\slv{ѡ҆свѣти́ти}}:}, {{\slv{ѡ҆свѣще́нъ}}}, {{\slv{ѡ҆свѣще́на}}}, {{\slv{ѡ҆свѣще́но}}},
        \hspca{{\slv{роди́ти}}:}, {{\slv{рожде́нъ}}}, {{\slv{рожде́на}}}, {{\slv{рожде́но}}},
        \hspca{{\slv{возлюби́ти}}:}, {{\slv{возлю́бленъ}}}, {{\slv{возлю́блена}}}, {{\slv{возлю́блено}}}
    }

    В последнем случае (относительно смягчения губных) составляют исключения причастия: {\slv{ᲂу҆ѧ́звенъ}} (и {\slv{ᲂу҆ѧзвле́нъ}}) и {\slv{благослове́нъ}}.
    
    Все краткие страдательные причастия прошедшего времени склоняются, как соответствующие им имена прилагательные с кратким окончанием.
    \pagebreak

                \subsubsection{Полные страдательные причастия прошедшего времени}

    Они образуются также из кратких того же времени через перемену окончаний:
    
    \bigskip\autorows[-1pt]{l}{9}{l}{
        \hspca{м. р.}, {{\slv{-нъ}}}, {на}, {{\slv{-нный}}:}, {{\slv{глаго́ланный}},}, {{\slv{-ен}}}, {на}, {{\slv{-енный}}:}, {{\slv{несе́нный}},},
        \hspca{}, {{\slv{-тъ}}}, {на}, {{\slv{-тый}}:}, {{\slv{покры́тый}}}, {}, {}, {}, {},
        \hspca{ж. р.}, {{\slv{-на}}}, {на}, {{\slv{-ннаѧ}}:}, {{\slv{глаго́ланнаѧ}},}, {{\slv{-ена}}}, {на}, {{\slv{-енная}}:}, {{\slv{несе́ннаѧ}},},
        \hspca{}, {{\slv{-та}}}, {на}, {{\slv{-таѧ}}:}, {{\slv{покры́таѧ}}}, {}, {}, {}, {},
        \hspca{ср. р.}, {{\slv{-но}}}, {на}, {{\slv{-нное}}:}, {{\slv{глаго́ланное}},}, {{\slv{-ено}}}, {на}, {{\slv{-енное}}:}, {{\slv{несе́нное}},},
        \hspca{}, {{\slv{-то}}}, {на}, {{\slv{-тое}}:}, {{\slv{покры́тое}}.}, {}, {}, {}, {}
    }

    Все полные страдательные причастия прошедшего времени склоняются, как соответствующие им имена прилагательные с полным окончанием.

                \subsubsection{Сложная страдательная форма глаголов}

    Сложная страдательная форма глаголов, как уже было замечено, состоит из страдательного причастия спрягаемого глагола и вспомогательного глагола {\slv{бы́ти}}. В этом случае страдательное причастие (обыкновенно к краткой форме настоящего или прошедшего времени) изменяется по родам и числам, а глагол {\slv{бы́ти}}~---~по наклонениям, временам, лицам и числам.
    
    Необходимо заметить, что страдательная неопределенная форма в своем сложном виде изменяется по родам причастия спрягаемого глагола, причем это причастие берется в дательном падеже единственного числа настоящего или прошедшего времени. Например:
    
    \bigskip\autorows[-1pt]{l}{3}{l}{
        \hspca{муж. и ср. род:}, {{\slv{несо́мꙋ}} ({\slv{несе́нꙋ}}) {\slv{бы́ти}};}, {{\slv{хвали́мꙋ}} ({\slv{хвале́нꙋ}}) {\slv{бы́ти}}},
        \hspca{жен. род:}, {{\slv{несо́мѣ}} ({\slv{несе́нѣ}}) {\slv{бы́ти}};}, {{\slv{хвали́мѣ}} ({\slv{хвале́нѣ}}) {\slv{бы́ти}}}
    }

    В настоящем времени изъявительного наклонения причастие спрягаемого глагола большей частью берется только в причастном настоящем времени, а в прочих временах и наклонениях (кроме прошедшего совершенного времени) оно может быть как в причастном настоящем, так и в причастном прошедшем.
    
    Мы коснемся спряжения глаголов сложной страдательной формы лишь в простых временах изъявительного наклонения, а также в повелительном наклонении, как наиболее употребляемых в церковнославянских текстах. Для примера возьмем глагол {\slv{твори́мꙋ бы́ти}}.
    
    \bigskip
    Неопределенная форма:
    
    \bigskip\autorows[-1pt]{l}{2}{l}{
        \hspca{муж. и ср. род}, {{\slv{твори́мꙋ}} ({\slv{творе́нꙋ}}) {\slv{бы́ти}}},
        \hspca{жен. род}, {{\slv{твори́мѣ}} ({\slv{творе́нѣ}}) {\slv{бы́ти}}}
    }
    \pagebreak
    
    Изъявительное наклонение:

    \bigskip\autorows[-1pt]{l}{2}{l}{
        \hspca{настоящее время:}, {{\slv{твори́мь, твори́ма, твори́мо є҆́смь}} и т.д.},
        \hspca{преходящее время:}, {{\slv{твори́мь}} ({\slv{творе́нъ}}){\slv{, -а, -о бѧ́хъ}} ({\slv{бѣ́хъ}})  и т.д.},
        \hspca{аорист:}, {{\slv{твори́мь}} ({\slv{творе́нъ}}){\slv{, -а, -о бы́хъ}} и т.д.},
        \hspca{будущее время:}, {{\slv{твори́мь}} ({\slv{творе́нъ}}){\slv{, -а, -о бꙋ́дꙋ}} и т.д.}
    }
    
    Повелительное наклонение:

    \bigskip\autorows{l}{1}{l}{
        \hspca{{\slv{твори́мь}} ({\slv{творе́нъ}}){\slv{, -а, -о бꙋ́ди}} и т.д.}
    }

        \section{Неизменяемые части речи}
                \subsubsection{Наречие}

    \textbf{Наречием} называется неизменяемая часть речи, употребляемая в предложении как обстоятельственное слово и служащая для указания качества, количества, времени, места и образа действия предмета.
    
    Наречия, выражающие обстоятельство \emph{качества}, например:

    \bigskip\autorows{l}{1}{l}{
        \hspca{{\slv{Прпⷣбне, моли {\large непреста́ннѡ} ѡ҆ всѣ́хъ на́съ}}}
    }

    \noindent
    происходят большей частью  от качественных имен прилагательных, преимущественно кратких, среднего рода, причем отличаются от них окончаниями {\slv{-ѡ}} или {\slv{-ѣ}}. Например:

    \medskip\begin{adjustwidth}{\hstbb}{0cm}
        \begin{tabular}[l]{ll}
            Имена прилагательные & Наречия           \\
            {\slv{добро̀}}        & {\slv{до́брѣ}}     \\
            {\slv{ѕло̀}}          & {\slv{ѕлѣ̀}}       \\
            {\slv{непоро́чно}}    & {\slv{непоро́чнѡ}} \\
        \end{tabular}
    \end{adjustwidth}

    \medskip
    \emph{Количественные} наречия выражают какое-либо количество. Например: {\slv{стори́цею, сꙋгꙋ́бѡ, толи́кѡ, семи́жды}} и др.
    
    \bigskip\autorows[-1pt]{l}{2}{l}{
        \hspca{Наречия \emph{времени}:}, {{\slv{вы́нꙋ}} (всегда){\slv{, дне́сь, заꙋ́тра}} и др.},
        \hspca{Наречия \emph{места}:}, {{\slv{вспѧ́ть}} (назад){\slv{, ѡ҆деснꙋ́ю}} (справа) и др.},
        \hspca{Наречия \emph{образа действия}:}, {{\slv{а҆́бїе}} (тотчас){\slv{, вкꙋ́пѣ}} (вместе),},
        \hspca{}, {{\slv{всꙋ́е}} (попусту) и др.}
    }

    Некоторые наречия обозначают различные вопросы, например: {\slv{ка́мѡ;}} (куда?), {\slv{доко́лѣ;}} (до каких пор?), {\slv{вскꙋ́ю;}} (зачем?) и др.

                \subsubsection{Предлог}

    \textbf{Предлогом} называется служебная часть речи, стоящая перед другим словом и указывающая на отношение между предметами или явлениями. Например:

    \bigskip\autorows{l}{1}{l}{
        \hspca{{\slv{Ми́лость гдⷭ҇нѧ {\large ѿ} вѣ́ка и҆ {\Large до} вѣ́ка {\Large на} боѧ́щихсѧ є҆гѡ̀}} (Пс. 102, 17)}
    }

    Необходимо отметить местоположение предлога {\slv{ра́ди}} (<<ради>>, <<для>>), который в церковнославянском предложении ставится после управляемого слова. Например: {\slv{На́съ ра́ди человѣ̑къ}} (Символ веры) (по-русски: <<ради \emph{нас} людей>>).
    
    Предлог {\slv{вмѣ́стѡ}} иногда разделяется на свои составные части {\slv{в}}({\slv{ъ}}) и {\slv{мѣ́сто}} и между этими частями может быть вставлено какое-либо слово; окончание же части {\slv{-мѣ́сто}} будет уже {\slv{-о}}, а не {\slv{-ѡ}}. Например:

    \bigskip\autorows{l}{1}{l}{
        \hspca{{\slv{Ра́дость же є҆́ѵѣ въ печа́ли мѣ́сто}} (т.е. {\slv{вмѣ́сто печа́ли}})},
        \hspca{{\slv{подала̀ є҆сѝ}} (Воскрес. непорочны)}
    }
    
    Кроме того, нужно заметить, что предлоги {\slv{ѿ}} и {\slv{и҆з̾}} требуют после себя всегда родительного падежа. Например:
    
    \bigskip\autorows{l}{1}{l}{
        \hspca{{\slv{Ненави́дѧщїи сїѡ́на посрами́тесѧ ѿ гдⷭ҇а}} (Пс. 128, 5)},
        \hspca{{\slv{И҆зведѝ и҆з̾ темни́цы дꙋ́шꙋ мою̀}} (Пс. 141, 8)}
    }

                \subsubsection{Союз}

    \textbf{Союзом} называется служебная часть речи, показывающая связь или отношение между членами предложения или между самими предложениями.
    
    В церковнославянском языке, в некоторых случаях, один и тот же союз может иметь несколько значений. Рассмотрим значения и особенности употребительнейших союзов: {\slv{бо, ᲂу҆́бѡ, ᲂу҆̀бо, ꙗ҆́кѡ}}.
    
    \bigskip\mockitem{1. Особенности союза {\slv{бо}}}
    \medskip
    
    Союз {\slv{бо}} (<<так>>, <<потому что>>) ставится после слова, к которому относится, и означает \emph{причину} или \emph{цель}: {\slv{Не ᲂу҆́мре бо, но спи́тъ}} (Лк. 8, 52).
    
    Когда союз {\slv{бо}} стоит после отрицания {\slv{не}}, то над ним ставится ударение, чтобы при сочетании слов {\slv{не}} и {\slv{бо}} и при слитном их произношении нельзя было смешать это сочетание со словом {\slv{не́бо}}: {\slv{Не бо̀ врагѡ́мъ твои̑мъ та́йнꙋ повѣ́мъ}} (Молитва ко св. причащению).
    \pagebreak
        
    \bigskip\mockitem{2. Значение союза {\slv{ᲂу҆́бѡ}}}
    \medskip
    
    Союз {\slv{ᲂу҆́бѡ}}~---~уступительный (<<хотя>>) и иногда~---~утвердительный (<<подлинно>>, <<действительно>>). Например:
        
    \bigskip\autorows{l}{1}{l}{
        \hspca{{\slv{Всѝ ᲂу҆́бѡ текꙋ́тъ, є҆ди́нъ же прїе́млетъ по́честь}} (1 Кор. 9, 24)},
        \hspca{(\emph{Хотя} все бегут [на состязаниях], но один принимает награду)},
        \hspca{{\slv{Дре́вле ᲂу҆́бѡ ѿ не сꙋ́щихъ созда́вый мѧ̀}} (Воскрес. непорочны)},
        \hspca{(Ты, \emph{подлинно} создавший меня в древности из не существующих},
        \hspca{[тварей])}
    }
    
    \mockitem{3. Значение союза {\slv{ᲂу҆̀бо}}}
    \medskip
    
    Союз {\slv{ᲂу҆̀бо}}, созвучный по произношению с союзом {\slv{ᲂу҆́бѡ}}, но различный по начертанию, есть союз заключительный (<<итак>>, <<следовательно>>) и иногда~---~вопросительный (соответствующий русским частицам <<а>> или <<же>>).
    
    \bigskip\autorows{l}{1}{l}{
        \hspca{{\slv{Бꙋ́дите ᲂу҆̀бо вы соверше́ни}} (Мф. 5, 48). (\emph{Итак}, будьте вы совершенны)},
        \hspca{{\slv{Оу҆̀бо разꙋмѣ́еши ли, ꙗ҆̀же чте́ши;}} (Деян. 8, 30)},
        \hspca{(\emph{А} понимаешь ли, что читаешь?)},
        \hspca{{\slv{Что̀ ᲂу҆̀бо сотвори́мъ;}} (Лк. 3, 10). (Что \emph{же} мы должны делать?)}
    }
        
    \mockitem{4. Значение союза {\slv{ꙗ҆́кѡ}}}
    \medskip
    
    Союз {\slv{ꙗ҆́кѡ}} имеет разнообразные значения, но в большинстве случаев он обозначает русский союз <<потому что>> или сравнительную частицу <<как>>. Например:

    \bigskip\autorows{l}{1}{l}{
        \hspca{{\slv{Досто́йно є҆́сть ꙗ҆́кѡ вои́стиннꙋ}} (потому что справедливо)},
        \hspca{{\slv{бл҃жи́ти тѧ̀ бцⷣꙋ}}},
        \hspca{{\slv{Бꙋ́дите мꙋ́дри ꙗ҆́кѡ}} (как) {\slv{ѕмїѧ̑}} (Мф. 10, 16)}
    }

    Но иногда союз {\slv{ꙗ҆́кѡ}} имеет значение <<приблизительно>>, <<около>>:
    
    \bigskip\autorows{l}{1}{l}{
        \hspca{{\slv{Пребы́сть же марїа́мъ съ не́ю ꙗ҆́кѡ}} (около) {\slv{трѝ мѣ́сѧцы}} (Лк. 1, 56)}
    }

    Если же союз {\slv{ꙗ҆́кѡ}} стоит в сочетании со словом {\slv{что̀}}, то в таком сочетании он переводится по-русски созом <<зачем>>:
    
    \bigskip\autorows{l}{1}{l}{
        \hspca{{\slv{Что̀ ꙗ҆́кѡ}} (зачем) {\slv{съ мытари̑ и҆ грѣ̑шники ꙗ҆́стъ и҆ пїе́тъ;}} (Мк. 2, 16)}
    }


                \subsubsection{Частицы}

    \textbf{Частицей} называется служебное слово, которое придает словам различные смысловые оттенки.
    
    Таковы частицы: {\slv{є҆да̀, є҆́й, ᲂу҆жѐ, си́це, нижѐ, да, ли, же, не, ни}} и др.
    
    Иногда в значении частиц могут быть даже другие части речи, и тогда они теряют свой прежний смысл; таковы слова, выполняющие роль частиц: {\slv{и҆, а҆, а҆́ще, бо, что̀, сѧ̀}} и др.
    
    Некоторые частицы образуют грамматические формы слов.
    
    Частица {\slv{ни}} образует отрицательные местоимения и наречия: {\slv{никто̀, ничто̀, николи́же}} и др.
    
    Если частица {\slv{ни}} относится к глаголу, то не всегда ставится непосредственно перед ним; иногда между этой частицей и глаголом помещаются другие слова. Например:

    \bigskip\autorows{l}{1}{l}{
        \hspca{{\slv{Ни ли сегѡ̀ члѝ є҆стѐ}} (Лк. 6, 3)}
    }

    Частица {\slv{да}} образует желательное наклонение глаголов: {\slv{да повели́ши, да сотворю̀}}.
    
    Частица {\slv{се}} (<<вот>>) служит в предложении нередко вводным словом и, как таковое, отделяется от последующих слов запятой:

    \bigskip\autorows{l}{1}{l}{
        \hspca{{\slv{И҆ сѐ, ѿверзо́шасѧ є҆мꙋ̀ нб҃са̀}} (Мф. 3, 16)}
    }

                \subsubsection{Междометия}

    \textbf{Междометием} называется часть речи, выражающая душевные волнения, чувствования и переживания.
    
    В церковнославянском языке встречаются междометия, выражающие удивление ({\slv{Ѽ! Ѽле!}}), поношение ({\slv{ᲂу҆а̀}}), скорбь ({\slv{ᲂу҆вы̀}}), радость ({\slv{Бла́го мѝ!}}), благоговение ({\slv{Ѽ ка́ко!}}) и некоторые другие.
    
    Междометие {\slv{є҆́й}} имеет также значение утвердительной частицы \emph{да}:

    \bigskip\autorows{l}{1}{l}{
        \hspca{{\slv{Бꙋ́ди же сло́во ва́ше, є҆́й, є҆́й: нѝ, нѝ}} (Мф. 5, 37)},
        \hspca{{\slv{И҆лѝ і҆ꙋде́євъ бг҃ъ то́кмѡ, а҆ не и҆ ꙗ҆зы́кѡвъ; є҆́й, и҆ ꙗ҆зы́кѡвъ}}},
        \hspca{(Рим. 3, 29)}
    }

    Междометие в предложении не связано ни с каким его членом и само не является членом предложения.
    \pagebreak
    
        \section{Некоторые особенности церковнославянского синтаксиса}
                \subsubsection{Двойной винительный падеж}

    В церковнославянском языке после глаголов {\slv{нарица́ти, твори́ти, положи́ти, поста́вити, ви́дѣти, ѡ҆брѣстѝ}} и некоторых других ставятся иногда два винительных падежа, причем один из них принадлежит лицу, а другой~---~званию, должности или назначению. Например:

    \bigskip\autorows{l}{1}{l}{
        \hspca{{\slv{Кто́ тѧ поста́ви нача́льника и҆ сꙋдїю̀;}} (Деян. 7, 35)}
    }
    
    При переводе на русский язык второй винительный падеж ставится в творительном падеже. Данное предложение в русском переводе примет такой вид: <<Кто \emph{тебя} поставил \emph{начальником} и \emph{судьей?}>>.
    
    Если при упомянутых глаголах находится отрицание {\slv{}} или {\slv{}}, то это отрицание требует после себя двух родительных падежей:
    
    \bigskip\autorows{l}{1}{l}{
        \hspca{{\slv{Не твори́те до́мꙋ о҆ц҃а моегѡ̀ до́мꙋ кꙋ́пленагѡ}} (Ин. 2, 16)}
    }
    
    В этом случае первый родительный падеж переводится на русский язык родительным же падежом, а второй родительный падеж~---~творительным падежом. Русский перевод этого предложения таков: <<Не делайте \emph{дома} Моего Отца \emph{домом} торговли>>.
    
                \subsubsection{Падеж родительный-разделительный}

    В предложении при местоимениях и именах числительных играет большую роль падеж так называемый родительный-разделительный, когда нужно указать часть от целого.
    
    При местоимении падеж родительный-разделительный ставится или с предлогом {\slv{ѿ}} (в значении <<из>>), или без него. Например:

    \bigskip\autorows{l}{1}{l}{
        \hspca{{\slv{Є҆ди́нъ ѿ ва́съ преда́стъ мѧ̀. Нача́ша глаго́лати є҆мꙋ̀ єди́нъ}}},
        \hspca{{\slv{кі́йждо и҆́хъ: є҆да̀ а҆́зъ є҆́смь;}} (Мф. 26, 21--22)}
    }
    
    При именах прилагательных падеж родительный-разделительный ставится большей частью при сравнительной и превосходной степени:
    
    \bigskip\autorows{l}{1}{l}{
        \hspca{{\slv{Ѕмі́й же бѣ̀ мꙋдрѣ́йшїй всѣ́хъ ѕвѣре́й, сꙋ́щихъ на землѝ}} (Быт. 3, 1)}
    }

    При именах числительных падеж родительный-разделительный ставится с предлогом {\slv{ѿ}}, когда нужно выделить один или два предмета из ряда однородных предметов. Например:
    
    \bigskip\autorows{l}{1}{l}{
        \hspca{{\slv{И҆ посла̀ два̀ ѿ ᲂу҆чени̑къ свои́хъ}} (Мк. 14, 13)}
    }
    
                \subsubsection{Замена притяжательных местоимений личными и возвратными в дательном падеже}

    Притяжательные местоимения: {\slv{мо́й, тво́й, сво́й,}} являющиеся в предложении определениями, заменяются иногда личными и возвратными местоимениями в энклитической форме дательного падежа: {\slv{мѝ, тѝ, сѝ}}. Например:
    
    \bigskip\autorows{l}{1}{l}{
        \hspca{{\slv{Помѧни́ мѧ, гдⷭ҇и, є҆гда̀ прїи́деши во црⷭ҇твїи сѝ}} (Лк. 23, 42)}
    }
    
    Отсюда оборот речи {\slv{во своѧ̑ си}}. Например:
    
    \bigskip\autorows{l}{1}{l}{
        \hspca{{\slv{Ѻ҆ни́ же возврати́шасѧ во своѧ̑ си}} (Деян. 21, 6)}
    }

                \subsubsection{Употребление местоимения и имени прилагательного в смысле имени существительного}

    Местоимения, употребляемые в смысле имени существительного, бывают вместо него определяемым словом и ставятся во множественном числе среднего рода, если под ним подразумевается неопределенное количество или что-либо вообще многое. Определения же их согласуются с ними, как с определенными существительными. Например:
    
    \bigskip\autorows{l}{1}{l}{
        \hspca{{\slv{Моѧ̑ всѧ̑ твоѧ̑ сꙋ́ть и҆ твоѧ̑ моѧ̑}} (Ин. 17, 11)}
    }
    
    В русском языке в этом случае ставится единственное число среднего рода. Приведенный стих по-русски переводится так: <<Все Мое~---~Твое, и Твое~---~Мое>>.
    
    Подобно местоимениям и имена прилагательные (а также причастия), когда они в предложении заменяют собою имена существительные, ставятся в среднем роде единственного и множественного числа:
    
    \bigskip\autorows{l}{1}{l}{
        \hspca{{\slv{Лꙋка́вое пред̾ тобо́ю сотвори́хъ}} (Пс. 50, 6)},
        \hspca{{\slv{Безвѣ̑стнаѧ и҆ та̑йнаѧ премꙋ́дрости твоеѧ̀ ꙗ҆ви́лъ мѝ є҆сѝ}} (Пс. 50, 8)},
        \hspca{{\slv{Ви́дѣхъ неразꙋмѣва́ющыѧ и҆ и҆ста́ѧхъ}} (Пс. 118, 158)}
    }
    \pagebreak

                \subsubsection{Отсутствие отрицания при глаголе при наличии местоимений {\slv{никто̀, ничто̀}}}

    При местоимениях {\slv{никто̀, ничто̀}} ({\slv{никто́же, ничто́же}}) отрицательная частица {\slv{}}, которая ставится в русском языке перед глагольным сказуемым в церковнославянском языке обычно опускается. Например:
    
    \bigskip\autorows{l}{1}{l}{
        \hspca{{\slv{Бг҃а никто́же ви́дѣ}} (Ин. 1, 18)}
    }

    По-русски это предложение переводится так: <<Бога никто \emph{не} видел>>, т.е. по-русски перед глагольным сказуемым частица \textbf{не} не опускается.
    
    \bigskip\autorows{l}{1}{l}{
        \hspca{{\slv{Никомꙋ́же ничесо́же рцы̀}} (Мк. 1, 44)}
    }

    Перевод: <<Никому ничего \emph{не} говори>>.
    
                \subsubsection{Славянский член и его употребление}

    В славянском предложении иногда перед подлежащим и второстепенными членами нередко встречаются относительные местоимения: {\slv{и҆́же, ꙗ҆́же, є҆́же,}} которые, по-видимому, излишни, так как они не переводятся на русский язык. В подобных случаях эти местоимения не являются уже собственно местоимениями, а служат в качестве \emph{славянского члена} по образцу греческого члена (\textgreek{ὁ, ἡ, τὸ}), ничем не переводимого.
    
    Употребление члена в церковнославянском языке служит для оттенения и усиления той части предложения, которая имеет особо важное значение среди других его частей. Так, например, во 2-м члене Символа веры слова {\slv{и҆́же ѿ о҆ц҃а рожде́ннаго}} стоят с членом {\slv{и҆́же}}, потому что они главные во 2-м члене и из-за них, главным образом, собирался Первый Вселенский Собор.
    
    Славянский член падежей не имеет, а изменяется только по родам. Употребление рода члена зависит от рода того слова, к которому он относится. Чтобы в славянском тексте отличить славянский член от относительных местоимений {\slv{и҆́же, ꙗ҆́же, є҆́же,}} необходимо заметить, что славянский член употребляется с причастной формой:
    
    \bigskip\autorows{l}{1}{l}{
        \hspca{{\slv{А҆́зъ є҆́смь хлѣ́бъ живо́тный, и҆́же сше́дый съ нб҃сѐ}} (Ин. 6, 51)},
        \hspca{{\slv{И҆́же вездѣ̀ сы́й}} (Молитва Св. Духу)},
        \hspca{{\slv{Є҆́же ѡ҆ на́съ испо́лнивъ смотре́нїе, и҆ ꙗ҆́же на землѝ соедини́въ}}},
        \hspca{{\slv{нбⷭ҇нымъ, возне́слсѧ є҆сѝ во сла́вѣ, хрⷭ҇те бж҃е на́шъ}}},
        \hspca{(Тропарь Вознесению Господню)}
    }

                \subsubsection[Особенности славянского придаточного предложения]{Особенности славянского придаточного предложения с союзным словом {\slv{ꙗ҆́кѡ}}}

    Союз {\slv{ꙗ҆́кѡ}}, связывающий придаточное предложение с главным, при переводе на русский язык может принимать смысл различных союзных слов. Возьмем несколько примеров:
    
    \bigskip\autorows{l}{1}{l}{
        \hspca{{\slv{Не вѣ́сте ли, ꙗ҆́кѡ хра́мъ бж҃їй є҆стѐ;}} (1 Кор. 3, 16)}
    }

    Здесь придаточное предложение переводится по-русски предаточным дополнительным, а союз {\slv{ꙗ҆́кѡ}}~---~союзным словом <<что>>: <<Разве не знаете, \emph{что} вы~---~храм Божий?>>.
    
    \bigskip\autorows{l}{1}{l}{
        \hspca{{\slv{Вкꙋси́те и҆ ви́дите, ꙗ҆́кѡ бл҃гъ гдⷭ҇ь}} (Пс. 33, 9)}
    }
    
    Здесь придаточное предложение есть сравнительное, и союз {\slv{ꙗ҆́кѡ}} переводится союзным словом <<как>>: <<Испытайте и увидите, \emph{как} благ Господь!>>.
    
    \bigskip\autorows{l}{1}{l}{
        \hspca{{\slv{Да не ѿстꙋ́пиши ѿ менє̀, ꙗ҆́кѡ ско́рбь бли́з̾}} (Пс. 21, 12)}
    }
    
    Здесь придаточное предложение обстоятельства времени; союз {\slv{ꙗ҆́кѡ}} переводится словом <<когда>>: <<Ты не покинешь меня, \emph{когда} приблизится горе (ко мне)>>.
    
    \bigskip\autorows{l}{1}{l}{
        \hspca{{\slv{Ꙗ҆́кѡ да сбꙋ́детсѧ рече́нное прⷪ҇рѡ́ки}} (Мф. 2, 23)}
    }

    Здесь союз {\slv{ꙗ҆́кѡ}} находится в сочетании с частицей {\slv{да}}; в таком соединении он переводится союзным словом <<чтобы>>: <<\emph{Чтобы} сбылось сказанное пророками>>.

                \subsubsection{Особенности славянских придаточных предложений с союзными словами {\slv{є҆́же, во є҆́же, ѡ є҆́же}}}

    Очень часто местоимение {\slv{є҆́же}} стоит в сочетании с неопределенной формой глагола, который служит сказуемым придаточного предложения. В таком сочетании придаточное предложение переводится на русский язык придаточным дополнительным посредством союзного слова <<чтобы>>:
    
    \bigskip\autorows{l}{1}{l}{
        \hspca{{\slv{Лицѐ гдⷭ҇не на творѧ́щыѧ ѕла̑ѧ, є҆́же потреби́ти ѿ землѝ}}},
        \hspca{{\slv{па́мѧть и҆́хъ}} (Лк. 33, 17)}
    }
    
    Следует заметить, что при переводе неопределенная форма глагола заменяется иногда прошедшим временем. Вот перевод приведенного стиха: <<Лицо Господне (обращено) на делающих злое, \emph{чтобы} была истреблена на земле их память>>.
    
    Такая структура перевода относится и к тому случаю, когда местоимение {\slv{є҆́же}} находится в сочетании с предлогом {\slv{во}}:
    
    \bigskip\autorows{l}{1}{l}{
        \hspca{{\slv{Оу҆стнѣ̀ моѝ ѿве́рзи, во є҆́же пѣ́ти тѧ̀, ст҃а́ѧ трⷪ҇це}}},
        \hspca{(Утренние молитвы)}
    }
    
    Перевод: <<Открой мои уста, \emph{чтобы} я воспевал Тебя, Святая Троица!>>.
    
    Если местоимение {\slv{}} находится в сочетании с предлогом {\slv{}}, то такое сочетание переводится по-русски: <<о том>> или <<о том, чтобы>>.
    
    \bigskip\autorows{l}{1}{l}{
        \hspca{{\slv{Сказа́нїе ѡ҆ є҆́же каковꙋ̀ подоба́етъ бы́ти дх҃вникꙋ̀}} (Требник)}
    }
    
    Перевод: <<Указание \emph{о том}, каким должен быть духовник>>.
    
    \bigskip\autorows{l}{1}{l}{
        \hspca{{\slv{Ѡ҆ є҆́же прости́тисѧ и҆̀мъ всѧ́комꙋ прегрѣше́нїю}} (Ектения)}
    }
    
    Перевод: <<\emph{О том, чтобы} простилось им всякое согрешение>>.\medskip

                \subsubsection[Особенности славянского придаточного предложения]{Особенности славянского придаточного предложения с союзным словом {\slv{да}}}

    Возьмем такое предложение:
    
    \bigskip\autorows{l}{1}{l}{
        \hspca{{\slv{Хощꙋ̀, да то́й пребыва́етъ}} (Ин. 21, 32)}
    }
    
    Здесь, при сочетании частицы {\slv{да}} со сказуемым придаточного предложения в настоящем времени, частица {\slv{да}} переводится союзным словм <<чтобы>>, а глагольное сказуемое~---~прошедшим временем. Русский перевод этого предложения таков: <<Хочу, \emph{чтобы} он \emph{пребывал}>>.
    
                \subsubsection{Оборот <<Дательный самостоятельный>>}

    Дательный самостоятельный~---~это такой оборот, где в придаточном предложении имеется подлежащее, независимое от подлежащего главного предложения, стоящее в дательном падеже и согласованное с глагольным сказуемым, выраженным кратким причастием, стоящим в том же падеже.
    
    В этом случае дательный падеж как подлежащего, так и сказуемого не управляется другими словами, является как бы самостоятельным, а потому и называется \emph{дательным самостоятельным}.
    
    Возьмем такое предложение:
    
    \bigskip\autorows{l}{1}{l}{
        \hspca{{\slv{Се́рдцꙋ веселѧ́щꙋсѧ, лицѐ цвѣте́тъ}} (Притч. 15, 30)}
    }
    
    Здесь главное предложение~---~{\slv{лицѐ цвѣте́тъ}}, а придаточное~---~{\slv{се́рдцꙋ веселѧ́щꙋсѧ}}. В придаточном предложении подлежащее и сказуемое поставлены в дательном падеже и не зависят ни от каких других слов. При переводе этого оборота на русский язык он принимает форму придаточного предложения времени, причем подлежащее ставится уже в именительном падеже, а сказуемое~---~в том времени изъявительного наклонения, которое указывается временем причастия.
    
    Приведенный пример в русском переводе примет такую форму: <<Когда сердце веселится, то цветет лицо>>.
    
    Иногда в этом обороте встречается пропуск существительного, определяемого причастием. Например:
    
    \bigskip\autorows{l}{1}{l}{
        \hspca{{\slv{Сꙋ́щꙋ по́здѣ въ де́нь то́й}} (Ин. 20, 19)}
    }
    
    Пропущено {\slv{сꙋ́щꙋ вре́мени}}.
    
    \bigskip\autorows{l}{1}{l}{
        \hspca{{\slv{То́ликꙋ сꙋ́щꙋ}} (т.е. {\slv{мно́жествꙋ ры́бъ}}) {\slv{не прото́ржесѧ мре́жа}} (Ин. 21, 11)}
    }
    
    При таких пропусках запятая, отделяющая данный оборот от главного предложения, обычно опускается.

                \subsubsection{Оборот <<Дательный с неопределенным>>}

    Дательный с неопределенным~---~это такой оборот, где в придаточном предложении после союзов {\slv{}} а иногда глагола {\slv{}} (в значении <<случилось>> или без этого значения, но тогда в соединении с союзами {\slv{}} или {\slv{}}) подлежащее ставится в дательном падеже, а сказуемое~---~в неопределенной форме глагола.
    
    Возьмем предложение:
    
    \bigskip\autorows{l}{1}{l}{
        \hspca{{\slv{Во́лны влива́хꙋсѧ въ кора́бль, ꙗ҆́кѡ ᲂу҆жѐ погрꙋжа́тисѧ є҆мꙋ̀}}},
        \hspca{(Мк. 4, 37)}
    }

    Здесь подлежащее {\slv{є҆мꙋ̀}} стоит в дательном падеже, сказуемое {\slv{погрꙋжа́тисѧ}}~---~в неопределенной форме глагола, причем подлежащее и сказуемое стоят после союза {\slv{ꙗ҆́кѡ}}. При переводе этого оборота на русский язык он принимает форму придаточного предложения дополнительного или обстоятельного Приведенный пример в русском переводе примет такой вид: <<Волны вливались в корабль (так), что \emph{он} уже \emph{погружался}>>.
    
                \subsubsection{Оборот <<Винительный с неопределенным>>}

    Винительный с неопределенным~---~это такой оборот, где в придаточном предложении подлежащее ставится в винительном падеже, а сказуемое~---~в неопределенной форме глагола.
    
    Возьмем такое предложение:
    
    \bigskip\autorows{l}{1}{l}{
        \hspca{{\slv{Хощꙋ̀ ва́съ безпеча́льныхъ бы́ти}} (1 Кор. 7, 32)}
    }
    
    Здесь подлежащее {\slv{}} стоит в винительном падеже, а сказуемое {\slv{}}~---~в неопределенной форме глагола. При переводе на русский язык этот оборот принимает форму придаточного дополнительного предложения. Приведенный пример в русском переводе примет такой вид: <<Хочу, чтобы \emph{вы были} беспечальны>>.
    
    \begin{flushright}
        {\tiny{\slv{Ны́нѣ ѿпꙋща́еши}}, \DTMnow}
    \end{flushright}

\end{document}
