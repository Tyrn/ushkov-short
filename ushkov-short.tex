\documentclass[11pt,a4paper,oneside]{memoir}

\pagestyle{plain}

%\usepackage[hypcap=false,format=hang]{caption}
%\usepackage{caption}
%\captionsetup{labelformat=empty}

\usepackage{longtable}
\usepackage{makecell}
\usepackage{multirow}
\usepackage[usenames,dvipsnames,svgnames,table]{xcolor}
\usepackage{fontspec}
\usepackage{xltxtra}
% code borrowed from Polyglossia documentation — Thanks!
\definecolor{myblue}{rgb}{0.02,0.04,0.48}
\definecolor{lightblue}{rgb}{0.61,.8,.8}
\definecolor{myred}{rgb}{0.65,0.04,0.07}

\usepackage{polyglossia}
\setmainlanguage{russian}
\setotherlanguages{churchslavonic,english}
\usepackage{churchslavonic}
\usepackage{lettrine}

\setmainfont[Mapping=tex-text]{Liberation Serif}
\setsansfont[Mapping=tex-text]{Liberation Sans}
\setmonofont[Mapping=tex-text]{Liberation Mono}

\newfontfamily\churchslavonicfont[Script=Cyrillic,Ligatures=TeX,Scale=1.33333333,HyphenChar="005F]{PonomarUnicode.otf} 
\newfontfamily{\slv}[Scale=MatchLowercase]{Ponomar Unicode TT}
\newfontfamily{\ust}[Scale=MatchLowercase]{Menaion Unicode TT}
\newfontfamily{\ind}{Indiction Unicode TT}

\usepackage{indentfirst}
\frenchspacing
\clubpenalty=10000
\widowpenalty=10000

\sloppy

%% Ensure sequential numbering of subsubsections.
\setsecnumdepth{subsubsection}
\counterwithout{subsubsection}{subsection}
\renewcommand{\thechapter}{\arabic{chapter}}
\renewcommand{\thesection}{\arabic{section}}
\renewcommand{\thesubsection}{\Roman{subsection}}
\renewcommand{\thesubsubsection}{\S\arabic{subsubsection}}

\setcounter{tocdepth}{3} % Must not precede the above
\usepackage[hidelinks]{hyperref}

%% Suppress \hline on demand
\newcommand{\hln}{}

%% Borrowed from titlepages by Peter Wilson
%\usepackage{pst-text}

%%%% Additional font macros
\makeatletter
%%%% light series
%% e.g., s:12
\DeclareRobustCommand\ltseries
{\not@math@alphabet\ltseries\relax
    \fontseries\ltdefault\selectfont}
%% e.g., t:32
\newcommand{\ltdefault}{l}
%% e.g., v:19
\DeclareTextFontCommand{\textlt}{\ltseries}

% heavy(bold) series
\DeclareRobustCommand\hbseries
{\not@math@alphabet\hbseries\relax
    \fontseries\hbdefault\selectfont}
%% e.g., t:32
\newcommand{\hbdefault}{hb}
%% e.g., v:19
\DeclareTextFontCommand{\texthb}{\hbseries}
\makeatother

\newcommand*{\isbn}{{\small\textsc{ISBN}}}

%%% for the Web-O-Mints fonts
\newcommand*{\wb}[2]{\fontsize{#1}{#2}\usefont{U}{webo}{xl}{n}}
%\renewcommand*{\wb}[2]{}%    probably kills Web-O-Mints (and some layouts?)
%%% for the Fontsite 500 fonts
\newcommand*{\FSfont}[1]{%
    \fontencoding{T1}\fontfamily{#1}\selectfont}
%\renewcommand*{\FSfont}[1]{}%    kills special font selections

\newcommand*{\labelit}[1]{\phantomsection\label{#1}}
\newcommand*{\refit}[1]{(graphic on page~\pageref{#1})}

\chapterstyle{dash}\renewcommand*{\chaptitlefont}{\normalfont\itshape\LARGE}
\setlength{\beforechapskip}{2\onelineskip}
\setsecheadstyle{\normalfont\Large\raggedright}
\makeindex
\renewcommand*{\indexname}{Index of Designers}
\makeatletter
\newcommand*{\boxminipage}{%
    \@ifnextchar [%]
    \@ibxminipage
    {\@iiibxminipage c\relax[s]}}
\def\@ibxminipage[#1]{%
    \@ifnextchar [%]
    {\@iibxminipage{#1}}%
    {\@iiibxminipage{#1}\relax[s]}}
\def\@iibxminipage#1[#2]{%
    \@ifnextchar [%]
    {\@iiibxminipage{#1}{#2}}%
    {\@iiibxminipage{#1}{#2}[#1]}}
\let\@bxminto\@empty
\def\@iiibxminipage#1#2[#3]#4{%
    \ifx\relax#2\else
    \setlength\@tempdimb{#2}%
    \def\@bxminto{to\@tempdimb}%
    \fi
    \leavevmode
    \@pboxswfalse
    \if #1b\vbox
    \else
    \if #1t\vtop
    \else
    \ifmmode \vcenter
    \else \@pboxswtrue $\vcenter
    \fi
    \fi
    \fi
    %  \@bxminto
    \bgroup%          outermost vbox
    \hsize #4
    \hrule\@height\fboxrule
    \hbox\bgroup%   inner hbox
    \vrule\@width\fboxrule \hskip\fboxsep 
    \vbox \@bxminto
    \bgroup% innermost vbox
    \vskip\fboxsep
    \advance\hsize -2\fboxrule \advance\hsize -2\fboxsep
    \textwidth\hsize \columnwidth\hsize
    \@parboxrestore
    \def\@mpfn{mpfootnote}\def\thempfn{\thempfootnote}\c@mpfootnote\z@
    \let\@footnotetext\@mpfootnotetext
    \let\@listdepth\@mplistdepth \@mplistdepth\z@
    \@minipagerestore\@minipagetrue
    \everypar{\global\@minipagefalse\everypar{}}}

\def\endboxminipage{%
    \par\vskip-\lastskip
    \ifvoid\@mpfootins\else
    \vskip\skip\@mpfootins\footnoterule\unvbox\@mpfootins\fi
    \vskip\fboxsep
    \egroup%    end innermost vbox
    \hskip\fboxsep \vrule\@width\fboxrule
    \egroup%    end hbox
    \hrule\@height\fboxrule
    \egroup%    end outermost vbox
    \if@pboxsw $\fi}
\makeatother

\DeclareRobustCommand{\cs}[1]{\texttt{\char`\\#1}}
\newlength{\tpheight}\setlength{\tpheight}{0.9\textheight}
\newlength{\txtheight}\setlength{\txtheight}{0.9\tpheight}
\newlength{\tpwidth}\setlength{\tpwidth}{0.9\textwidth}
\newlength{\txtwidth}\setlength{\txtwidth}{0.9\tpwidth}
\newlength{\drop}

\newenvironment{showtitle}{%
    \begin{boxminipage}[c][\tpheight]{\tpwidth}
        \centering\begin{vplace}\begin{minipage}[c][\txtheight]{\txtwidth}}%
            {\end{minipage}\end{vplace}\end{boxminipage}}

\newcommand*{\titleCC}{\begingroup% City of Cambridge
    \drop=0.1\txtheight
    \vspace*{\drop}
    \centering 
    {\Large\itshape КРАТКИЙ УЧЕБНИК}\\[0.5\drop]
    {\textcolor{Red}{\HUGE\bfseries {\slv{✠}}}}\par
    \vspace{\drop}
    {\LARGE\itshape ЦЕРКОВНОСЛАВЯНСКОГО ЯЗЫКА}\par
    \vfill
    {\Large УШКОВ А.В.}\par
    \vfill
%    {\plogo}\\[0.5\baselineskip]
    {\itshape Веркола}\par
    {\scshape 2017}\par
    %\vfill
    \vspace*{\drop}
    \endgroup}
%% From titlepages by Peter Wilson, end

\begin{document}
    \begin{titlingpage}
        %% Borrowed from titlepages by Peter Wilson
        \begin{showtitle}
            \titleCC
        \end{showtitle}
        \labelit{CC}
        {
            \par\vspace{0.2\baselineskip}
            %        \footnotesize \verb?\titleCC?
        }
        %% From titlepages by Peter Wilson, end
    \end{titlingpage}
    
    \tableofcontents
    
    \chapter*{}
    \markboth{}{}
        \section*{Благолепие церковнославянского языка}
        \label{sec:blago}
        \addcontentsline{toc}{section}{\nameref{sec:blago}}

    Прошло уже более тысячи лет со времени Крещения Руси. Воспринятое ею Православие, благодаря трудам святых Мефодия и Кирилла, совершает свои благоговейные богослужения на церковнославянском языке. Он по своей структуре наиболее близок к греческому, и это позволило сохранить существующий в Греческой церкви строй богослужения, что явилось для многочисленных славянских народов великим благом и живительным источником благочестия и совершенствования нравственного в духе истины правой веры, а также руководством в земных делах: образования, иконографии, храмо- и градостроительства. Наша Русская Православная Церковь восприняла в неповрежденности и чистоте от Греческой Матери Церкви все догматическое, литургическое, духовное, святоотеческое наследие и предание, связав все это, с помощью славянского языка, с духом народа христианского на Руси, с жизнью и бытом и просвещением, ибо на славянский язык были переведены все богослужебные книги и писания греческих отцов, прием сохранялась точность, ясность и истинность, хотя перевод и являлся творческим делом.
    
    Славянский язык помогал памятовать о Христе, хранить Его живой образ и Его учение в сердце и душе и приносить плоды живого творческого восприятия слова Божия. В Церкви шло становление и образование самого славянского языка, его внутренняя христианизация и воцерковление, преображение самой стихии славянской мысли и слова, славянского голоса, души народа, На долю русского народа выпало редкое счастье принять христианство тогда, когда прогремели громом своим Вселенские Соборы и утверждено в чистоте апостольское Православие, и в Церкви Греческой утвердились и литургические, и догматические, и нравственные истины и самый язык славянский, ничем не запятнанный и чистый, воспринял для своего народа сосуд благодати, и в благодати Христовой народ верный обновляется, и просвещается, и спасается, и становится причастным вечной Истине, и славянский язык богатство веры и культуры передал многим полудиким племенам и народам, и в Церкви они нашли единение и братство Христово как дети Божии. На Руси потрудились многие ревнители духовного просвещения: и князь Владимир~---~просветитель, и св. Андрей Боголюбский, и свв. Александр Невский, Даниил Московский, святитель Петр, митрополит Алексий, преподобный Сергий Радонежский и многие другие подвижники благочестия. Созидались монастыри, храмы, школы. Создавались библиотеки, делались переводы на славянский язык, составлялись азбуки для малых и полудиких народностей, писались летописи.
    
    Славянский язык хранил церковное единство и православные традиции. 2 августа 1581 г. впервые была напечатана полная Библия на славянском языке в г. Остроге, что способствовало глубокому изучению Священного Писания. Русская Церковь в XVI в., когда протестанты превращали христианство в систему философских построений, как плод похи Возрождения язычества, начала вести длительную и тяжелую борьбу за сохранение чистоты веры, церковнославянского языка, церковно-национальной культуры.
    
    Славянская Библия занимает самое первое место по своей полноте и точности воспроизведения библейских древних писаний, о чем свидетельствуют и списки Кумранских рукописей, написанных за 200---300 лет до Р.Х., когда не было нужды искажать тексты Священного Писания. Древние рукописи своим текстом совпадают с острожской славянской Библией. В славянской Библии точно сохранено летоисчисление~---~5508 лет от сотворения мира до Рождества Христова, и все мессианские места. Известно, что текст 70 толковников, взятый в основу славянской Библии, долгое время считался в александрийских синагогах у евреев священным переводом, и его читали на греческом языке 200---300 лет. Масоретский же текст Библии идет от Акибы (около II века после Р.Х.).
    
    Кирилло-Мефодианский перевод Евангелия на славянский язык отличается точностью, верностью подлиннику и ясностью, чистотою выражения христианских понятий.
    
    Славянский текст Священного Писания и православного богослужения~---~свидетель первого тысячелетия, то есть неразделенного христианства, и в то же время он является драгоценным наследием равноапостольских трудов свв. Кирилла и Мефодия, на котором воспитываются чада Русской Православной Церкви.
    
    Церковнославянский богослужебный текст является важным и святым. Христиане разных конфессий стремятся осмыслить свои традиции в сравнении с традициями других исповеданий, ведь священные книги Нового Завета переведены на 1050 языков, и среди этих переводов самое важное место занимает славянский. Поскольку перевод Писания был сделан с греческого языка, а греческая текстология устанавливала единый печатный греческий текст XVII в., потом в XX веке его вытеснили из употребления, то в вопросе изысканий и оценки конечных результатов многих реконструкций текстов важную роль играет славянский текст как свидетель текстовой традиции первого тысячелетия, то есть времени неразделенного христианства.
    
    Острожская Библия 1581 года на славянском языке ни по форме, ни по содержанию не изменилась и является эталоном, по которому можно проследить реконструкцию греческого текста по времени.
    
    Анализ и сопоставление текстов показывает, что славянский текст стоит на твердой, нерушимой основе древней богослужебной практики Православия, а западное христианство отошло от древних традиций.
    
    Святые равноапостольные братья Кирилл и Мефодий использовали для перевода церковный византийский текст, который не искажался, был под контролем и оберегался всегда и всюду в Церкви, как нечто постоянное, неизменное и непреложное, как сама Истина, как церковно-текстуальное предание, уходящее в апостольскую древность. Это предание Церкви святое, соборное, апостольское, как поток, исходит из свежего родника живой воды евангельской, течет по чистому руслу под ясным небом и при тихой погоде, и его прозрачные воды менее доступны для загрязнения человеческими мудрованиями во имя кичащегося ума и в ущерб созидающей любви.
    
    В текстах встречаются несколько десятков слов труднопонятных, как тектон~---~дровосек, усмен~---~кожан, измена~---~выкуп, выну~---~всегда и др., но здравый человек за неделю их запомнит, ведь даже иностранные языки учат, да не один, а два-три языка, заучивая 40-45 слов в день, а язык, когда-то бывший родным, разговорным языком наших предков, язык, освященный веками, надо любить и хранить.
    
    Славянский язык в Церкви на Руси освящен непрерывным богослужение, и внешняя гармония, чистота, сердечность, молитвенность, мир Христов и утешение благодатное водворяются в душе православного христианина, как плод веры и любви.
    
    Большое значение славянскому языку для православного богослужения придает и Грузинская Церковь. Архимандрит Рафаил (Карелин) из Тбилиси выразил то так: <<В настоящее время в литургической жизни Грузинской Автокефальной Церкви славянский язык является по своему значению вторым, после древнегрузинского, богослужебным языком. Церковная славистика имеет для нас практическую важность и теоретический интерес>>.
    
    В связи с этим необходимо остановиться на нескольких вопросах, касающихся значения славянского языка для современного православного богослужения, рассматривая при этом славянский язык обобщенно и условно, как единую структурную и знаковую систему.
    
    В данном докладе славянский язык рассматривается обобщенно и условно, как единая структурная и знаковая система.
    
    Славянский язык насчитывает более чем тысячелетнюю литературную историю. Он не стал мертвым языком, не сделался достоянием одних археологов и лингвистов; славянский язык звучит на всех континентах, на нем совершается богослужение в православных церквах, на нем миллионы людей воссылают молитвы к Богу.
    
    В последнее время неоднократно поднимался вопрос: не является ли славянский язык литургическим архаизмом, преградой к пониманию текстов, и не следует ли перевести все богослужебные книги на современные языки?
    
    Такие попытки предпринимались в России и других странах.
    
    В России языковые литургические реформы не встретили поддержки основной массы верующих, и на сегодняшний день богослужебным языком остается церковнославянский. Можно ли объяснить это явление только одной приверженностью к традиции, или же оно имеет еще другие основания?
    
    Полагаем, что эти основания следующие.
    
    I. Славянскому языку, как и другим древним языкам (имеется в виду древнегреческий и древнегрузинский), присуща особая динамическая структура; он лучше передает пульс религиозной жизни, глубже выражает молитвенные чувства, чем современные языки.
    
    Древние языки более подходят к цельному синтетическому восприятию, новые~---~к аналитическому, дробному; древние~---~к созерцанию, новые~---~к логизированию; древние языки полны энергии и эмоций, новые, в сравнении с ними, носят рационалистический, описательный характер. Древние языки дают большую возможность соприкоснуться с глубиной явлений, с духовными субстанциями, сделать человека участником событий, современные~---~изложить явления в определенной системе и дать их анализ.
    
    Сравнивая лексическую и грамматическую ткань древних и новых языков, мы видим, что древние языки располагают меньшим по объему словарным фондом, но их грамматический строй отличается большим многообразием, пластичностью и совершенством.
    
    Мобильность и экспрессивность глагольных форм, богатство аффиксов и флексий, не имеющих аналогии в современных языках, лаконизм и динамика синтаксических структур, этимологическая глубина лексики создают неповторимую выразительность и, по словам патриарха Пимена, <<особую красоту>> славянского языка. Эти возможности славянского языка позволяют выразить в богослужебных текстах многоплановость событий, синергизм и параллелизм действий, объединить различные хронологические периоды в единые смысловые циклы (как бы преодолев линейную протяженность времени), проявить и усилить подтекст каждого предложения.
    
    Сложность и гибкость глагольных форм славянского языка делают одну и ту же по словарному составу фразу то ажурно-легкой и полетной, то~---~до физической ощутимости тяжелой и твердой, как бы изваянной из мрамора. Лаконизм, внутренняя чеканность и в то же время как бы внешняя незавершенность и обрывистость предложений, часто без последовательных переходов от одного предмета к другому, наделяют смысловые паузы эмоциональным содержанием, подчеркивая, что богослужебный текст~---~это не монолог, а диалог, таинственная беседа души с Божеством, что в священных событиях присутствует, как их постоянный субъект, личность каждого молящегося.
    
    Говоря образно, динамика древних языков созвучна динамике света. Богослужение~---~это симфония из лучей Божественных энергий. Священная история изображается в этой Божественной симфонии на светящемся фоне вечности, земные реалии~---~в их логоистическом преображении.
    
    Следует отметить, что в богослужебных текстах мастерски использован точно и строго очертанный круг изобразительных средств. Это делалось древними гимнографами для того, чтобы не подменить духовных переживаний эстетическими, т.е. душевными, не обременить ума молящегося художественной информацией и, тем самым, не оземлить молитву.
    
    Древние гимнографы мыслили Церковь как идею единства в Ее реальном и мистическом воплощении. Унификация изобразительных средств~---~одно из условий общности молитвы, духовного познания, эмоций и сопереживаний верующих~---~тела Церкви.
    
    Динамичные структуры, внутренние смысловые емкости и скульптурная пластичность славянского языка помогают воспринимать традиционные символы и метафористические образы в разных контекстах как вечно новые и неповторимые.
    
    Одним словом, древние языки более приспособлены для выражения явлений и динамики духовной жизни. Это первая и главная причина их сохранения в православном богослужении.
    
    II. Второе основание~---~трудность самого перевода.
    
    Богослужебные тексты~---~это шедевры священной поэзии особого типа и порядка. Переводчик должен хорошо знать святоотеческое богословие, именно святоотеческое, а не схоластическое, чтобы понять глубокие по догматическому содержанию литургические тексты. Далее, переводчик должен обладать одновременно глубокой филологической подготовкой и незаурядным поэтическим дарованием. Без этого перевод богослужебных книг будет казаться обесцвеченным и прозаическим пересказом великих поэм.
    
    Православные церковные службы называют опоэтизированным, иконографическим, поющим богословием.
    
    И, наконец, самое главное, переводчик должен иметь живую веру, церковность и религиозную интуицию, которая дала бы возможность почувствовать, пережить, оценить каждую фразу и слово переводимого текста.
    
    Интересно, что Петербургский митрополит Гавриил и Московский митрополит Филарет, руководившие работами по переводу святоотеческих произведений с греческого языка на славянский и русский, считали нужным давать рукописи переводов для проверки простым старцам-монахам, которые не знали греческого языка, но, обладая общностью внутреннего духовного опыта с древними писателями-подвижниками и находясь с ними, так сказать, на одной духовной волне, интуитивно чувствовали всякую фальшь и ошибку перевода.
    
    III. Третье основание~---~традиция. Традиция~---~это кристаллизация истории. Это актуальное бытие прошлого в настоящем. Любовь и уважение к прошлому сохранили до наших дней старинные соборы и храмы в их первозданном величии, иконы и фрески изумительной духовной глубины. Живая традиция сохранила нам дивное, неповторимое православное богослужение.
    
    Само слово <<Православие>> значит правое, правильное богопочитание, основным признаком которого является верность апостольскому преданию.
    
    Церковная традиция возникла из общности духовного опыта христиан, опыта emph{богообщения}. emph{Одинаковое религиозное чувство создало} одинаковые или очень схожие формы. Церковное богослужение и храмовое устройство~---~это синтез жизни Церкви в эпоху ее духовного расцвета. Содержание создает формы, но и формы хранят и оберегают содержание. В данном случае древние языки имеют вспомогательное, но важное значение для сохранения в чистоте и внутренней целостности одного из видов церковного предания~---~богослужебного канона.
    
    Чем устойчивее и тверже канон, тем глубже и чище он выражает общечеловеческую и духовную потребность: каноническое есть церковное, церковное~---~соборное, соборное же~---~всечеловеческое.
    
    Славянский язык, наряду с другими древними языками, стал сакральным, священным языком Церкви.
    
    Вопрос о ясности богослужебных славянских текстов для верующих приобретает все большее значение.
    
    Следует обратить особое внимание на научную сторону новых изданий богослужебных книг, снабдить их справочным материалом, в том числе словарем, где наряду с буквальным переводом давалась бы этимология славянской лексики. Не только Священное Писание, но и следующий за ним духовный пласт~---~богослужебный канон~---~также нуждается в толкованиях и комментариях. Перед нами стоит задача составления литургической экзегезы, отвечающей интеллектуальному уровню и духовным запросам современных верующих.
    
    Славянское православное богослужение~---~это бесценное сокровище мистического гносиса, огромный потенциал духовных сил и энергии, который мы должны сохранить не только для себя, но и для будущих поколений.
    
    Велико значение церковнославянского языка для православного богослужения и православного пастыря. Но церковнославянский язык имеет очень важное значение и для православного хистианина. На этом языке совершается богослужение в нашей Православной Русской Церкви и написаны наши священные и богослужебные книги. По своему возвышенному характеру, по своей силе и звучности церковнославянский язык является наиболее совершенным средством для выражения религиозных настроений православного русского человека. Высшие стремления души и глубокие чувствования, отрешенные от земного и направленные к небесному, чистому и вечному, получают наиболее соответствующее выражение в этом языке, далеком от всего обычного, житейского. Но для этого необходимо, чтобы все читаемое в храме Божием достигало своей цели, т.е. доходило бы до сознания души христианина, научая ее истинному благочестию и указывая ей путь спасения. А потому нужно, чтобы те, которые употребляют язык Церкви~---~чтецы, певцы церковно- и священнослужители~---~в совершенстве понимали то, что произносят их уста. Кроме того, без полного знания языка Матери Церкви нельзя дать ответа вопрошающим о недоуменных словах и выражениях церковного языка. А таких слов и выражений в церковнославянском языке достаточно.
    
    Итак, все, что читается и поется в храмах Божиих, должно быть истолковано людьми, стоящими на высоте своего призвания, а такими людьми, прежде всего, должны быть православные пастыри. Только тогда у православного русского человека укрепится сознательное и вместе с тем благоговейное отношение к богослужению нашей Православной Церкви, и само его мировоззрение обогатится достаточным запасом религиозно-церковных понятий
    
    Славянский язык назван именно славянским потому, что на нем говорили наши предки славяне.
    
    Славяне, обитавшие первоначально около Карпат, долгое время были непросвещенным языческим народом. Очень часто им случалось соприкасаться с культурными народами Византийской и Римской империй. Мало-помалу славяне стали подчиняться влиянию этих уже просвещенных христианством народов. От греков и римлян к славянам проникло и быстро у них распространилось христианство. Но, слушая богослужения на чуждых им греческом или латинском языках, славяне не могли себе вполне усвоить истины нового для них учения, и потому многие из них были христианами лишь по имени, а по существу сохранили прежнюю грубость нравов и держались старых языческих суеверий. Так было до IX века.
    
    В половине IX века святые братья Кирилл (в миру Константин) и Мефодий, родом греки, задумали облегчить славянам понимание христианского богослужения и перевели для них богослужебные книги с греческого языка на славянский. Так как в то время у славян не было еще письменности, то святые братья сами составили славянскую азбуку, взяв за образец греческую азбуку и дополнив ее недостающими буквами. Таким образом, богослужебные книги были переведены ими на славянский язык, а сами они стали апостолами~---~просветителями славян.
    
    Святые братья Кирилл (827---869 гг.) и Мефодий (\dag 885 г.) были детьми знатного вельможи и родились в Солуни (Фессалониках)~---~главном городе Македонии, страны, населенной по преимуществу славянами. Старший брат, Мефодий, после домашнего воспитания занимал сначала военные, а потом административные должности, но впоследствии принял монашество, поселившись на горе Олимп. Младший, Кирилл, отличался блестящими способностями к учению, обучался словесным, философским и математическим наукам. Его ожидали почести в свете, но он не прельстился этим, а принял сан священника и место библиотекаря в библиотеке при храме св. Софии. Впоследствии св. Кирилл удалился к брату на Олимп. Предполагают, что в это время жизни на Олимпе святые братья и начали переводить богослужебные книги на славянский язык.
    
    Около 862 года, через посредничество Византийского императора Михаила III, святые братья были приглашены Моравским князем Ростиславом в Моравию для проповеди христианства. Здесь начинается главная деятельность свв. Кирилла и Мефодия. В княжествах Моравском и Панионском (где ныне Венгрия и Чехия) христианство было уже проповедано среди тамошних славян немецким, латинским духовенством. Но латинское духовенство совершало богослужение на чуждом славянам латинском языке. А так как латинский язык был непонятен славянам, то, конечно, проповедь немецкого духовенства оказалась безуспешной.
    
    Святые братья, начав учить славян Христовой вере, дали им священные книги на славянском языке. На славянский язык были переведены и богослужебные книги. На славянском языке стали они совершеть богослужение, а для обучения славян письменности стали заводить и школы. Христианство между славянами начало быстро распространяться.
    
    Немецкое духовенство, видя успехи святых братьев в деле распространения ими христианства среди славян, из зависти подало на них жалобу в Рим. Римский папа Николай I вызвал к себе святых братьев на суд (в то время не было еще разделения Церкви, все были православные, и только что начинались несогласия между Римом и Византией). Святые братья отправились в Рим, но уже не застали папу Николая I в живых, а преемник его Андриан очень благосклонно отнесся к свв. Кириллу и Мефодию. Кирилл представил папе Евангелие на славянском языке, и папа положил славянский перевод в соборе св. Петра, где святые братья даже отслужили торжественную литургию на славянском языке. Св. Кирилл в Риме заболел и скончался. Умирая, он убеждал св. Мефодия не покидать дела проповеди и распространения веры среди славян. Папа Андриан рукоположил Мефодия в епископа Моравии, отпустил его с честью и дал грамоту, в которой признал богослужение на славянском языке целесообразным.
    
    По возвращении св. Мефодия к Моравийским славянам враги не оставили его в покое. Они восстановили против него немецкого императора Людовика и добились того, что Мефодий был заточен в темницу, где и пробыл в заточении два с половиной года. Когда ему возвращена была свобода, то его вновь оклеветали перед папой. Пришлось Мефодию опять отправляться в Рим и защищать себя. Оправданный папой, св. Мефодий продолжал свое великое дело.
    
    Когда св. Мефодий скончался, то на учеников его было воздвигнуто гонение. Они, изгнанные из Моравии, ушли в другие славянские земли, главным образом, в Болгарию. Болгария заботливо хранила труды святых просветителей и потом, когда в конце X века, при св. князе Владимире, Русь приняла христианство, списки этих книг передала нашему отечеству. В те древние времена славянские наречия очень мало отличались друг от друга, и книги одного славянского племени были совершенно пригодны для богослужения и чтения другого.
    
    О высоком значении церковнославянского языка убедительно говорит Соколов Г.А. в своем <<Слове в защиту церковнославянского языка>> (Астрахань, 1910).
    
    <<Славянское слово имеет определенную силу как фактор религиозно-нравственного развития, вот уже 1000 лет просвещающего Русь Православную. Его воспитательное значение сохраняется и до сих пор, ибо те же чувства, то же настроение господствует среди чад Русской Православной Церкви и непосредственная сила религиозного наставления, выраженная в формах славянской речи, особенно чувствуется ими>>.
    
    Содержание церковных книг неотделимо от формы. Церковнославянский язык является в данном случае идеальной формой. Отделить форму от идеи~---~значило бы ослабить ее, лишить ее всей полноты ее силы и значения. Сбросить эту форму без ущерба для содержания невозможно.
    
    Слово церковное слилось с жизнью Православной России.
    
    Церковнославянский язык~---~это драгоценный сосуд, созданный для высокого содержания, вполне ему по основным своим свойствам соответствующий. Ведь не напрасно историки пришли к выводу, что труд переводов Мефодия и Кирилла, <<их великое дело было пережито нами в собственной истории: в русской мысли, в русском чувстве, в русском подвиге>>.
    
    Насколько вера искреннее и глубже, настолько острее и потребность в форме именно славянской речи, в простом и величавом ее течении, в ее дивной методичности и музыкальности.
    
    Сила мысли и сила чувства находятся в связи с силой церковнославянской речи.
    
    Быть в церкви, возноситься душою к Богу, поучаться стало не только потребностью, но и влечением, свойством души, природой.
    
    Славянское слово исполнено веры и высшей духовной правды.
    
    Церковнославянская речь не только своим духом, исполненным непосредственной веры и умиления, но даже видом своих письмен влечет в горний мир возвышенных созерцаний и настроений, христианского подвига и нравственного совершенства.
    
    Церковнославянский язык как бы впитал в себя те свойства, какие принадлежат самому тексту, и самим своим видом и особенностями способствует восприятию текста.
    
    История и жизнь убеждают нас, что славянский язык~---~это сокровище, которое нужно хранить, которым нужно дорожить.
    
    Благодать Божия поставила славянский язык в один ряд с тремя превосходными языками: греческим, латинским и еврейским.
    
    Некоторая оторванность языка литургического от разговорной речи, от обыденной жизни,~---~составляет достоинство молитвенной речи. Иметь для религиозных целей язык понятный и в то же время торжественный, далекий от того, в котором встречаются вульгарности (а в наше время вульгарность~---~уже принятый тон, потому что речь изобилует грубостью и грязью), и даже от слов и оборотов обыденного языка~---~это общечеловеческая потребность.
    
    Как первые детские впечатления кладут самые глубокие основы нравственной и душевной жизни человека и остаются на всю жизнь милы и дороги, так и первоначальные священные и богослужебные тексты навсегда остаются дороги и священны для народа, т.к. от их впечатлений и влияний возникли первые начала самой живой, именно религиозной стороны духовной жизни народа.
    
    Славянский язык необходим для обнаружения религиозных стремлений.
    
    Славянский язык~---~язык Церкви и религиозной письменности.
    
    Славянский язык~---~лучшее средство, чтобы привести к крепкой вере и жизни по вере.
    
    До принятия христианства на Руси не было устойчивого языческого мировоззрения, твердо выработанной языческой космогонии.
    
    Расшатать и разорвать связь между религиозными и нравственными стремлениями и славянским языком~---~не значило бы поколебать те основы духовных устремлений, которые были заложены еще 1000 лет тому назад.
    
    Учение веры и нравственности отлилось в формы славянской речи.
    
    Славянский язык принес неоценимые религиозно-просветительные блага русскому народу. Славянский язык дал те привычки, которые у народа русского стали второй природой. В чем особенности этой привычки? Своей искренностью, непосредственностью, глубоко проникнутая верой в Бога и Его промыслительные действия, речь на этом языке самим складом своим, самим выбором слов и выражений, тем духом, который, ее проникает, производит теперь это уже привычное влияние на душу верующего. Далее, язык церковнославянский, уже много веков имея единственной задачей выражать Божественное учение и повествования о домостроительстве Божием в направлении жизни людей, так сроднился с тем, для выражения чего он служит, что воспитанный на нем народ особенно восприимчив именно к тому, что он выражает, и в самом его звуковом составе, в движении речи ощущает этот высший смысл, слышит голос Бога, сам к Нему на этом языке обращается и веками усвоил то убеждение, что славянский язык есть орудие, средство общения на нем с Богом.
    
    Там, где у нас на Руси для целей религиозных вместо славянского языка употребляют русский, там уже нет Православия. Есть у нас секты, где на собраниях, имеющих характер богослужения, читают и поют по-русски. Но кто же их назовет православными? И вот те именно причины, которые отделили их от Православия и сделали их по мировоззрению и религиозным понятиям иными~---~они-то привели их к предпочтению русского языка языку славянскому. Когда под влиянием новых учений и критического отношения к вере отцов изменилось настроение, прекратилась восприимчивость к живым и непосредственным воздействиям славянской речи, тогда православное богослужение и обряды потеряли для них свое значение. Дальнейшим результатом всего этого явилось стремление проникнутый духом православной веры язык славянский заменить языком живой, разговорной речи, более удовлетворяющим требованиям сознания. Но, несомненно, он уже не то говорит воображению и чувству. Когда явилось стремление к замене для религиозных нужд славянской речи речью русской, там, очевидно, уже произошли духовные перемены, ведущие к измене Православию. Склонность к тому или иному языку есть только симптом и результат этих внутренних и глубоких перемен в вере и мировоззрении. Лишь перемены в верованиях ведут за собой измену языку прежних верований.
    
    Православные, пока они православные, не могут желать этой перемены уже по складу своей духовной природы; и чем тверже, искреннее и глубже вера их, тем дальше они от этих желаний. Они непосредственно чувствуют, что перемена языка была бы связана с переменой в самом духе православной религии. В славянской речи они слышат молитвенный голос своих отцов и дедов и сливаются с ними в единстве многовековой истории, в единстве молитвы всей России, всех славян.
    
    Те, кто желает перевода богослужебных текстов на русский язык, желают этого не в силу своих собственных религиозных потребностей. Они почти все равнодушны и к богослужению, и к языку Церкви. Они желают только в силу теоретических соображений о большей доступности текстов богослужебного языка, о более сознательном отношении к читаемому и т.д. Но те, кто живут религиозной жизнью, кто нуждается в Церкви и службах ее, те именно и не желают перевода, да и не могут желать, не изменившись в своих отношениях к религии и к тем чувствам, которые вызываются славянским языком.
    
    Заменить один язык другим~---~значит изменить свойства и особенности содержания, придавши ему новый характер, несоответствующий тому, какой вложили духоносные переводчики в славянские тексты; это значило бы принизить его содержание, приблизить к мирской жизни, а нередко придать и характер вульгарности.
    
    Всякое слово православного на славянском языке знакомо, близко, сродни всякому человеку, воспитанному в Церкви.
    
    Церковнославянский язык далек от утилитарности, он возбуждает чувство великого, возвышенного и вековечного. Он не колеблется и не изменяется~---~сообразно с тем, как неизменна и устойчива вера православная.
    
    Перевести на частный язык, каким является русский, Божественные истины~---~значило бы поставить их в общий поток жизни, внося в перевод перемены в языке, вызванные нуждами жизни, т.е. обусловленные социальными движениями, изменениями в искусствах, науке, торговле и вообще во всех сферах деятельности, обусловливающих развитие языка. Но как вера устойчива и неизменна в своем существе, так должен быть устойчив и язык религии, не меняясь с каждым поколением, согласно движению общественной жизни. Язык общеупотребительный в житейских делах с каждым поколением уже не тот, и одно столетие жизни вносит в него значительные перемены. Неужели и язык Церкви с каждым поколением должен подновляться и изменяться, внося новые и новые перемены вопреки неизменяемости Кафолической Церкви?
    
    Когда на Русь была перенесена церковнославянская письменность, то на этом языке совершалось богослужение, читались священные книги, писались жития русских святых, летописи, писались проповеди и другие религиозного и нравственно-назидательного содержания творения. И только акты государственные, договоры и другие официальные бумаги писались по-русски.
    
    В то время между языками теми существовало значительное сходство. Языки влияли один на другой вплоть до времени Петра I, когда возмогли возникать сомнения, считать ли главный отдел литературы (летописи, сказания, хождения и т.д.) высшим стилем русской речи или низшим стилем речи церковнославянской.
    
    Церковнославянский язык дает созерцание высоких религиозных идей и образов, влияет на сердце, возбуждает нравственные чувства, влечет к улучшению жизни на началах любви, самопожертвования и служения своим близким. Он имеет свои особенные задачи. Не принижаясь, не ослабевая, религиозная вера не может заковываться в формы, свойственные знанию.
    
    Церковнославянский язык достоин сохранения и как лучшая форма выражения великой идеи христианства в том виде, в каком оно сложилось в Православной Церкови и осуществилось в душе русского народа.
    
    Мысль о богослужении на русском языке родилась у людей, стоящих вдали от Православия, равнодушных к Церкви, не переживавших на себе влияния славянской речи.
    
    Славянский язык мы защищаем не по праву его давности, а лишь потому, что он имеет благотворное значение и свои преимущества перед русской речью.
    
    Различие во взглядах на славянский язык в конце концов связывается с различием мировоззрения и отношения к религии. Со стороны отрицателей славянского языка сказывается и отрицательное отношение к религии. Не является ли отрицание славянского языка в богослужении посягательством на самую религию, а не только на формы ее выражения?
    
    Нерасположенные к славянскому языку относятся не сочувственно и к религии.
    
    \section*{Краткий учебник церковнославянского языка\\
        \emph{предисловие}}
    \label{sec:brief}
    \addcontentsline{toc}{section}{\nameref{sec:brief}}
    
    Составитель сего учебника доброй памяти emph{профессор Анатолий Васильевич Ушков}.
    
    Анатолий Васильевич Ушков родился 7 августа 1894 года в г. Самаре в семье сотрудника Самарской Духовной Консистории. По окончании четырех классов Самарской Духовной семинарии, он в 1912 году успешно сдал вступительные экзамены в Казанский университет и был зачислен на физико-математический факультет. После университета, в 1916 году, А. Ушков был мобилизован на военную службу; проходил обучение в Киевском артиллерийском училище. С 1918 по 1945 г. он преподавал математику и физику в средних и специальных учебных заведениях в городах Самаре, Красноярске и Москве. Заведовал учебной частью и был руководителем методических совещаний по вопросам преподавания физики и математики. С 1939 по 1943 г. заочно учился в Московском государственном педагогическом институте на факультете русского языка и литературы, закончив который, некоторое время, наряду с преподаванием математики, вел курс литературы в средних учебных заведениях г. Москвы.
    
    В 1945 г. он осуществил свою давнишнюю мечту~---~послужить Святой Церкви и поступил в Московскую Духовную Академию. По окончании академии, в 1949 г., со званием кандидата богословия, полученным за сочинение <<Душа и ее бессмертие по христианскому учению>>, он преподавал катехизис, а затем церковнославянский язык в духовной семинарии. В 1964 г., по прочтении двух пробных лекций <<Древний мир перед пришествием Христа Спасителя>> и <<Основные положения формальной логики>>, А.В. Ушкову было присвоено звание доцента и поручено чтение лекций по логике в академии. Им составлен ряд учебных пособий для духовных школ: <<Краткая пасхалия>>, <<Логика в курсе академического образования>>, <<Краткая пасхалия в общедоступном изложении>>, <<Астрономический справочник>>, <<Краткое описание солнечной системы>>, <<Календарный счет времени>>, <<Счет и мера с древнейших времен до наших дней>>. Анатолием Васильевичем был подготовлен учебный курс церковнославянского языка, за который Совет академии присвоил ему 15 декабря 1969 г. степень магистра богословия и утвердил в должности профессора.
    
    Анатолий Васильевич отличался большой скромностью, чутким, внимательным отношением к людям; поражала его необычайная работоспособность, величайшее трудолюбие. Жажда проповеди слова Божия была его духовной потребностью. Он произнес 65 проповедей в праздничные и воскресные дни в академическом храме, в них~---~яркое выражение его глубоких христианских чувств и безграничной, сердечной любви к Богу и людям. В начале 1971---1972 учебного года Анатолий Васильевич по болезни вынужден был оставить работу в академии и перейти на пенсию.
    
    Скончался А.В. Ушков 14 января 1972 г., на 78-м году жизни, приобщившись Святых Христовых Тайн. Похоронен на кладбище в Сергиевом Посаде.
    
    Подготовил к печати доцент-иерей В. Москвич.
    
    \chapter*{1-й класс}
    \label{ch:firstgrade}
    \addcontentsline{toc}{chapter}{\nameref{ch:firstgrade}}
        \section*{Введение}
        \label{sec:intro}
        \addcontentsline{toc}{section}{\nameref{sec:intro}}
                \subsubsection[Значение церковнославянского языка]{Значение церковнославянского языка для православного богослужения и православного пастыря}
                
    Церковнославянский язык имеет очень важное значение для православного христианина. На этом языке совершается богослужение в нашей Православной Русской Церкви и написаны наши священные и богослужебные книги. По своему возвышенному характеру, по своей силе и звучности церковнославянский язык является наиболее совершенным средством для выражения религиозных настроений православного русского человека. Высшие стремления духа и глубокие чувствования, отрешенные от земного и направленные к небесному, чистому и вечному, получают наиболее соответствующее выражение в этом языке, далеком от обычного, житейского.
    
    Но для этого необходимо, чтобы все читаемое в храме Божием достигало своей цели, т.е. доходило бы до сознания души христианина, научая ее истинному благочестию и указывая ей путь спасения. А поэтому нужно, чтобы те, которые употребляют язык Церкви~---~чтецы, певцы, церковно- и священнослужители~---~в совершенстве понимали то, что произносят их уста. Да кроме того, без полного изучения языка матери-Церкви нельзя дать ответа вопрошающим о недоуменных словах и выражениях церковного языка. А таких слов и выражений в церковнославянском языке достаточно. Взять хотя бы такие слова: \emph{абие, амо, аще, выну, вкупе, ей, ктому, нань, леть, паче, паки, рамо, сице, тацы, убо, уне, яко}; или такие выражения: \emph{Змий сей, его же создал еси ругахуся ему; Чермнует бо ся небо; И бяху выну в церкви; Обыде нас последняя бездна; Любовию же, Дево, песни ткати спротяженно сложенныя неудобно есть; Противодышущу росоносному Духу, со огнем сущу пояху} и многие другие.
    
    Итак, все что читается и поется в храмах Божиих, должно быть истолковано людьми, стоящими на высоте своего призвания, а такими людьми, прежде всего, должны быть православные пастыри. Только тогда у православного русского человека укрепится сознательное и вместе с тем благоговейное отношение к богослужению нашей Православной Церкви и самое его мировоззрение обогатится достаточным запасом религиозно-церковных понятий.
    
                \subsubsection{Возникновение письменности у славян}
                
    Славянский язык назван именно \emph{славянским} потому, что на нем говорили наши предки славяне.
    
    Славяне, обитавшие первоначально около Карпат, долгое время были непросвещенным языческим народом. Очень часто им случалось соприкасаться с культурными народами Византийской и Римской империй. Мало-помалу славяне стали подчиняться влиянию этих уже просвещенных христианских народов. От греков и римлян к славянам проникло и быстро у них распространилось христианство. Но, слушая богослужение на чуждом им греческом или латинском языке, славяне не могли себе вполне усвоить истины нового для них учения, и потому многие из них были христианами лишь по имени, а по существу сохраняли прежнюю грубость нравов и держались старых языческих суеверий. Так было до половины IX века.
    
    В Половине IX века святые братья Кирилл (в миру Константин) и Мефодий, родом греки, задумали облегчить славянам понимание христианского богослужения~---~и перевели для них богослужебные книги с греческого языка на славянский. Так как у славян в то время не было еще письменности, то святые братья сами составили славянскую азбуку, взяв за образец греческую азбуку и дополнив ее недостающими буквами. Таким образом, богослужебные книги были переведены ими на славянский язык, а сами они стали апостолами-просветителями славян.
    
                \subsubsection{Деятельность святых братьев Кирилла и Мефодия}

    Святые братья Кирилл (827---869 гг.) и Мефодий (\dag 885 г.) были детьми знатного вельможи и родились в Солуни (Фессалониках)~---~главном городе Македонии, страны, населенной по преимуществу славянами. Старший брат, Мефодий, после домашнего воспитания занимал сначала военные, а потом административные должности, но впоследствии принял монашество, поселившись на горе Олимп. Младший, Кирилл, отличался блестящими способностями к учению, обучался словесным, философским и математическим наукам. Его ожидали почести в свете, но он не прельстился этим, а принял сан священника и место библиотекаря в библиотеке при храме св. Софии. Впоследствии св. Кирилл удалился к брату на Олимп. Предполагают, что в это время жизни на Олимпе святые братья и начали переводить богослужебные книги на славянский язык.

    Около 862 года, через посредничество Византийского императора Михаила III, святые братья были приглашены Моравским князем Ростиславом в Моравию для проповеди христианства. Здесь начинается главная деятельность свв. Кирилла и Мефодия. В княжествах Моравском и Панионском (где ныне Венгрия и Чехия) христианство было уже проповедано среди тамошних славян немецким, латинским духовенством. Но латинское духовенство совершало богослужение на чуждом славянам латинском языке. А так как латинский язык был непонятен славянам, то, конечно, проповедь немецкого духовенства оказалась безуспешной.

    Святые братья, начав учить славян Христовой вере, дали им священные книги на славянском языке. На славянский язык были переведены и богослужебные книги. На славянском языке стали они совершать богослужение, а для обучения славян письменности стали заводить и школы. Христианство между славянами начало быстро распространяться.

    Немецкое духовенство, видя успехи святых братьев в деле распространения ими христианства среди славян, из зависти подало на них жалобу в Рим. Римский папа Николай I вызвал к себе святых братьев на суд (в то время не было еще разделения Церкви, все были православные, и только что начинались несогласия между Римом и Византией). Святые братья отправились в Рим, но уже не застали папу Николая I в живых, а преемник его Андриан очень благосклонно отнесся к свв. Кириллу и Мефодию. Кирилл представил папе Евангелие на славянском языке, и папа положил славянский перевод в соборе св. Петра, где святые братья даже отслужили торжественную литургию на славянском языке. Св. Кирилл в Риме заболел и скончался. Умирая, он убеждал св. Мефодия не покидать дела проповеди и распространения веры среди славян. Папа Андриан рукоположил Мефодия в епископа Моравии, отпустил его с честью и дал грамоту, в которой признал богослужение на славянском языке целесообразным.

    По возвращении св. Мефодия к Моравийским славянам враги не оставили его в покое. Они восстановили против него немецкого императора Людовика и добились того, что Мефодий был заточен в темницу, где и пробыл в заточении два с половиной года. Когда ему возвращена была свобода, то его вновь оклеветали перед папой. Пришлось Мефодию опять отправляться в Рим и защищать себя. Оправданный папой, св. Мефодий продолжал свое великое дело.

    Когда св. Мефодий скончался, то на учеников его было воздвигнуто гонение. Они, изгнанные из Моравии, ушли в другие славянские земли, главным образом, в Болгарию. Болгария заботливо хранила труды святых просветителей и потом, когда в конце X века, при св. князе Владимире, Русь приняла христианство, списки этих книг передала нашему отечеству. В те древние времена славянские наречия очень мало отличались друг от друга, и книги одного славянского племени были совершенно пригодны для богослужения и чтения другого.


        \section{Церковнославянское буквоначертание}
                \subsubsection{Церковнославянская азбука}

    Устная наша речь состоит из членораздельных \textbf{звуков}.

    Изображения звуков речи условными знаками в письме или в печати называются \textbf{буквами}.

    Собрание всех букв языка, расположенных в общепринятом порядке, называется \textbf{азбукой} (от названия двух первых славянских букв <<аз>> и <<буки>>). Церковнославянская азбука состоит из 39 букв.


    \begin{center}
        \renewcommand*{\arraystretch}{1.4}    \begin{longtable}{|c|c|c|c|}
            \caption*{Буквы церковнославянской азбуки}\\
            \hline
            
            Прописная буква
            & Малая буква
            & Название
            & Произношение
            \\
            
            \hline
            \endfirsthead
            
            \multicolumn{4}{l}
            {
%                \tablename\ \thetable\ -- 
                \footnotesize\textit{Начало на предыдущей странице}
            }
            \\
            
            \hline
            
            Прописная буква
            & Малая буква
            & Название
            & Произношение
            \\
                
            \hline
            \endhead
            \hline
            
            \multicolumn{4}{r}{\footnotesize\textit{Продолжение на следующей странице}}
            \\
            
            \endfoot
            \hline
            \endlastfoot
            
            {\slv{А}}     & {\slv{а}}       & аз      & а\\\hln
            {\slv{Б}}     & {\slv{б}}       & буки    & б\\\hln
            {\slv{В}}     & {\slv{в}}       & веди    & в\\\hln
            {\slv{Г}}     & {\slv{г}}       & глаголь & г\\\hln
            {\slv{Д}}     & {\slv{д}}       & добро   & д\\\hln
            {\slv{Е}}     & {\slv{є, е}}    & есть    & е\\\hln
            {\slv{Ж}}     & {\slv{ж}}       & живете  & ж\\\hln
            {\slv{Ѕ}}     & {\slv{ѕ}}       & зело    & з\\\hln
            {\slv{З}}     & {\slv{з}}       & земля   & з\\\hln
            {\slv{И}}     & {\slv{и}}       & иже     & и\\\hln
            {\slv{І}}     & {\slv{ї}}       & и       & и\\\hln
            {\slv{К}}     & {\slv{к}}       & како    & к\\\hln
            {\slv{Л}}     & {\slv{л}}       & люди    & л\\\hln
            {\slv{М}}     & {\slv{м}}       & мыслете & м\\\hln
            {\slv{Н}}     & {\slv{н}}       & наш     & н\\\hln
            {\slv{Ѻ, Ѽ}} & {\slv{ѻ, о}}    & он      & о\\\hln
            {\slv{П}}     & {\slv{п}}       & покой   & п\\\hln
            {\slv{Р}}     & {\slv{р}}       & рцы     & р\\\hln
            {\slv{С}}     & {\slv{с}}       & слово   & с\\\hln
            {\slv{Т}}     & {\slv{т}}       & твердо  & т\\\hln
            {\slv{Оу}}    & {\slv{ᲂу, ꙋ, у}} & ук      & у\\\hln
            {\slv{Ф}}     & {\slv{ф}}       & ферт    & ф\\\hln
            {\slv{Х}}     & {\slv{х}}       & хер     & х\\\hln
            {\slv{Ц}}     & {\slv{ц}}       & цы      & ц\\\hln
            {\slv{Ч}}     & {\slv{ч}}       & червь   & ч\\\hln
            {\slv{Ш}}     & {\slv{ш}}       & ша      & ш\\\hln
            {\slv{Щ}}     & {\slv{щ}}       & ща      & щ\\\hln
            {\slv{Ъ}}     & {\slv{ъ}}       & ер      & --\\\hln
            {\slv{Ы}}     & {\slv{ы}}       & ы       & ы\\\hln
            {\slv{Ь}}     & {\slv{ь}}       & ерь     & --\\\hln
            {\slv{Ѣ}}     & {\slv{ѣ}}       & ять     & е\\\hln
            {\slv{Ю}}     & {\slv{ю}}       & ю       & ю\\\hln
            {\slv{Ꙗ, Ѧ}} & {\slv{ꙗ, ѧ}}    & я       & я\\\hln
            {\slv{Ѡ}}     & {\slv{ѡ}}       & омега   & о\\\hln
            {\slv{Ѿ}}     & {\slv{ѿ}}       & от      & от\\\hln
            {\slv{Ѯ}}     & {\slv{ѯ}}       & кси     & кс\\\hln
            {\slv{Ѱ}}     & {\slv{ѱ}}       & пси     & пс\\\hln
            {\slv{Ѳ}}     & {\slv{ѳ}}       & фита    & ф\\\hln
            {\slv{Ѵ}}     & {\slv{ѵ}}       & ижица   & и, в\\\hln

        \end{longtable}
    \end{center}

    \begin{flushright}
        \emph{Упражнение 1}
    \end{flushright}
    \medskip
    
    \textbf{Славянские или церковные буквы, сходные с русскими}
    
    {\slv{а б в г д е ж з и к л м н о п р с т ф х ц ч ш щ ъ ы ь ю}}
    \medskip
    
    \begin{center}
        \begin{slv}
            Бо́гъ.~\textemdash~Спа́съ.~\textemdash~Сы́нъ.~\textemdash~Ма́-ти.~\textemdash~Не́-бо.~\textemdash~Ча̑-да.~\textemdash~Лю́-ди.~\textemdash~По-сты̀.~\textemdash~Ѻ҆-те́цъ.~\textemdash~Це́р-ковь.~\textemdash~Гос-по́дь.~\textemdash~Мо-ли̑т-вы.~\textemdash~Пра́зд-ни-цы.~\textemdash~Спа-си́-тель.~\textemdash~Бла-го-сло-ве́нъ Бо́гъ на́шъ.~\textemdash~Бо́гъ є҆-динъ є҆́сть.~\textemdash~Го́с-по-ди, бла-го-сло-вѝ.~\textemdash~Въ це́рк-вахъ бла-го-сло-ви́-те Бо́-га.~\textemdash~Гла́съ бы́сть съ не-бе-сѐ: Ты̀ є҆-сѝ Сы́нъ Мо́й воз-лю́б-лен-ный.~\textemdash~Бо́гъ на́шъ на не-бе-сѝ и на зе-млѝ.~\textemdash~Воз-лю́-би-ши Го́с-по-да Бо́-га тво-е-го̀.~\textemdash~Ще́дръ и ми́-лос-тивъ Гос-по́дь.~\textemdash~Бра́т-ство воз-лю-би́-те.~\textemdash~Чтѝ ѻ҆т-ца̀ тво-е-го̀ и ма́-терь тво-ю̀.~\textemdash~Сы́нъ не по-кор-ли́-вый в по-ги́-бель.
        \end{slv}
    \end{center}


                \subsubsection{Основные правила церковнославянского чтения и письма}
                
    При церковнославянском чтении нужно соблюдать следующие основные правила:
    
    1. Читать нужно внятно и произносить слова так, как они напечатаны в книге. Например, слово {\slv{є҆го̀}} нужно читать <<его>> (а не <<ево>>), {\slv{моли́тва}} (а не <<малитва>>), {\slv{єди́наго}} (а не <<единова>>).
    
    2. Звука <<ё>> в церковнославянском языке совсем нет, а потому букву {\slv{є}} или {\slv{е}} нужно всегда произносить как <<е>>, а не как <<ё>>. Например, {\slv{твоѐ}} (а не <<твоё>>), {\slv{моѐ}} (а не <<моё>>), {\slv{тве́рдый}} (а не <<твёрдый>>).
    
    3. В церковнославянских словах на одном из слогов ставится \textbf{ударение}, чтобы показать, что этот слог нужно усилить голосом или, как говорят, сделать на нем ударение. Ударение обозначается знаком~~{\slv{́}}, или~~{\slv{̀}}, или~~{\slv{̑}}, например: {\slv{влады́ко, хвала̀, сїѧ̑}}. Если же слово начинается с гласного звука, то над буквой этого звука ставится \textbf{придыхание}, которое изображается знаком~~{\slv{҆}}, например: {\slv{о҆де́жда. є҆сѝ}}.
    
    4. В церковнославянском языке начальное слово предложения (после точки) пишется с прописной (большой) буквы, например: {\slv{Въ нача́лѣ сотворѝ бг҃ъ не́бо и҆ зе́млю. Землѧ̀ же бѣ неви́дима и҆ неꙋстро́ена}} (Быт. 1, 1--2). Прописная буква пишется иногда в начале каждого стиха, даже если стих начинается после запятой или другого какого-нибудь знака помимо точки.
    
    5. Священные имена лиц, собственные имена, т.е. имена, названия городов, стран, гор, рек, морей и пр., если они стоят в середине или в конце предложение (т.е. не после точки), пишутся с малой буквы, например: {\slv{Рече́ же гдⷭ҇ь къ мѡѷсе́ю и҆ а҆арѡ́нꙋ въ землѝ є҆гѵ́петстѣй}} (Исх. 12, 1).


                \subsubsection{Особенности правописания и произношения некоторых букв в церковнославянском языке}

    Буква {\slv{Г}} (<<глаголь>>) перед {\slv{г, к, х}} произносится как звук <<и>>, например: {\slv{сѷгклі́тъ}} (<<синклит>>), {\slv{а҆сѷгкрі́тъ}} (<<асинкрит>>). Исключаются из этого правила слова: {\slv{а҆гге́й}} (<<Аггей>>) и {\slv{а҆́гге́лъ}} (<<аггел>>, в значении злого духа).
    
    Буквы {\slv{є, е}}. Буква {\slv{є}} (так называемое <<удлиненное есть>>) пишется всегда в начале слова; в середине же и на конце слова обычно пишется буква {\slv{е}} (<<простое есть>>), например: {\slv{є҆́зеро, є҆ле́нь, се́рдце}}. Но иногда буква {\slv{є}} пишется и в середине слова и даже в конце слова. Эти случаи будут рассмотрены впоследствии.
    
    Буква {\slv{ѕ}} (<<зело>>) произносится как русское <<з>> и пишется только в семи коренных словах: {\slv{ѕвѣзда̀, ѕвѣ́рь, ѕе́лїе, ѕла́къ, ѕмі́й, ѕло̀, ѕѣлѡ̀}}, а также в производных от этих слов: {\slv{ѕлы́й, ѡ҆ѕло́бити, ѕла́чный, ѕвѣ́рскїй}} и др.
    
    Буква {\slv{ї}} (<<и>>) пишется пред гласными, например: {\slv{воскресе́нїе}}. Но в словах иностранных (еврейских и греческих) эта буква пишется и перед согласными, например: {\slv{і҆́долъ, вїно̀}}. Слово <<мир>> в значении <<вселенная>> в славянском языке пишется через {\slv{ї}}, например: {\slv{ненави́дитъ ва́съ мі́ръ}} (вселенная) (Ин. 15, 19), а это же самое слово, но в значении <<покой>>, <<тишина>> пишется через {\slv{и}} (<<иже>>), например: {\slv{ми́ръ мо́й даю̀ ва́мъ}} (т.е. даю благодатное умиротворение, покой) (Ин. 14, 27).
    
    Буквы {\slv{ѻ, о, ѽ}}. Буква {\slv{ѻ}} (<<он польск\'ое>>) пишется в начале слова, например: {\slv{ѻ҆́гнь, ѻ҆́ко}}. Исключение составляет слово {\slv{і҆ѻрда́нъ}}. Буква {\slv{о}} (<<он простое>>) пишется в середине и в конце слова, например: {\slv{со́нмъ, не́бо}}. Буква {\slv{ѽ}} употребляется только в качестве междометия.
    
    Буквы {\slv{ᲂу, ꙋ, у}} (<<ук>>) изображают звук <<у>>, причем буква {\slv{ᲂу}} пишется в начале слова, а {\slv{ꙋ}}~---~в середине и на конце слова, например: {\slv{ᲂу҆че́нїе, ᲂу҆́бѡ, дꙋша̀, рꙋкꙋ̀}}. Буква {\slv{у}} (без буквы {\slv{о}}) в словах совсем не пишется, а употребляется в значении цифры, о чем будет сказано далее.
    
    Буква {\slv{ѣ}} (<<ять>>) первоначально изображала собою звук как бы средний между нашими звуками <<е>> и <<я>> (почему и названа <<ять>>). Но с утратой этого звука в произношении буква {\slv{ѣ}} в последствии в церковнославянском языке стала означать звук <<е>>.
    
    Буквы {\slv{ъ}} (<<ер>>) и {\slv{ь}} (<<ерь>>). Буква {\slv{ъ}} первоначально обозначала краткий звук <<о>>, а буква {\slv{ь}}~---~краткий звук <<е>>. Эти звуки были потому краткие, что произносились более кратко и слабо, чем другие согласные звуки. В некоторых случаях они были особенно слабы и стали совсем исчезать из произношения. В письме сохранилась постановка буквы {\slv{ъ}} в конце слов, которая теперь показывает твердое произношение конечного согласного звука, например: {\slv{хра́мъ, пра́ведникъ}}. Иногда в этом случае букву {\slv{ъ}} заменяет особый знак <<ерок>>, или <<ерик>>, который обозначается~~{\slv{̾}} и ставится над последней буквой слова. Как правило, ерок ставится в односложных предлогах, оканчивающихся на согласный звук, как-то: {\slv{ѡ҆б̾, и҆з̾, без̾, над̾, под̾}} и др. Над этими предлогами (кроме предлога {\slv{бли́з̾}}) также не ставят ударений. Когда эти предлоги соединяются в качестве приставок с другим словом, начинающимся с гласного звука, в одно слово, то ерок, смотря по произношению, удерживается (в качестве разделительного знака), например: {\slv{ѡ҆б̾ѧтїѧ, и҆з̾ѧсни́ти}}, или опускается, например: {\slv{и҆зы́де}}. Предлоги, состоящие только из согласных звуков: {\slv{въ, къ, съ}}, при раздельном их написании всегда пишутся с буквой {\slv{ъ}}, а не с ероком. Буква {\slv{ь}}, исчезнувшая как гласный звук из произношения, осталась лишь знаком мягкости предыдущего согласного звука, а потому сохранила свою постановку как в середине, так и на конце слов.

                \subsubsection{Правописание и произношение букв, перешедших в церковнославянский язык из греческого языка}

    Буквы, перешедшие с греческого языка в церковнославянский язык, следующие: {\slv{ѡ, ѿ, ѯ, ѱ, ѳ, ѵ}}.

    Буква {\slv{ѡ}} (<<омега>>) пишется в предлогах {\slv{ѡ҆}} и {\slv{ѡ҆б̾}} и в словах, начинающихся этими предлогами (в качестве приставок), например: {\slv{ѡ҆чище́нїе, ѡ҆бличи́ти}}. Если слово, начинающееся буквой {\slv{ѡ}}, соединяется с приставкой в одно слово, то в полученном составном слове буква {\slv{ѡ}} почти всегда удерживается, например: {\slv{преѡбраже́нїе}}. Буква {\slv{ѡ}} пишется также в словах, заимствованных из других языков (греческого и еврейского), например: {\slv{ѡ҆са́нна, і҆кѡ́на, канѡ́нъ, і҆ѡа́ннъ}} и др. Наконец, буква {\slv{ѡ}} пишется в некоторых наречиях, оканчивающихся звуком <<о>>, союзах, а также и в других случаях, о чем будет сказано впоследствии.

    Буква {\slv{ѿ}} (<<от>>), как уже показывает и ее название, представляет соединение двух букв {\slv{ѡ}} и {\slv{т}}, что соответствует и ее начертанию. Эта составная буква употребляется в качестве предлога {\slv{ѿ}}, а также в словах, начинающихся этим предлогом (в качестве приставки), например: {\slv{ѿра́да, ѿмще́нїе}}.

    Буквы {\slv{ѯ}} (<<кси>>), {\slv{ѱ}} (<<пси>>), {\slv{ѳ}} (<<фита>>) употребляются только в словах, заимствованных с греческого языка, например: {\slv{а҆леѯа́ндръ, ѱалти́рь, ѳома̀}}.

    Буква {\slv{ѵ}} (<<ижица>>) пишется в словах, взятых с греческого языка, причем произносится или как звук <<и>>, или как звук <<в>>. Если буква {\slv{ѵ}} стоит в начале слова или после согласной буквы, то она произносится как наш гласный звук <<и>>. В этом случае над нею ставятся две наклонные черточки, например: {\slv{сѷно́дъ}}. Эти черточки заменяются или придыханием, когда с буквы {\slv{ѵ}} начинается слово, или ударением, когда оно падает на слог, содержащий {\slv{ѵ}}, например: {\slv{ѵ҆ссѡ́пъ, тѵ́хѡнъ}}. После гласного звука буква {\slv{ѵ}} произносится как согласный звук <<в>>, например: {\slv{є҆́ѵа, лаѵре́нтїй, леѵі́тъ}}. Исключение составляют некоторые собственные имена, например: {\slv{мѡѷсе́й}}, а также составные слова, например: {\slv{триѷпоста́сный}}. Слово {\slv{мѵ́ро}} пишется через {\slv{ѵ}} для различения его в косвенных падежах от слов {\slv{мі́ръ}} и {\slv{ми́ръ}}.


                \subsubsection{Правописание букв {\large\slv{ꙗ}} и {\large\slv{ѧ}}}

    Буква {\slv{ꙗ}} представляет собою соединение двух звуков: {\slv{ї}} и {\slv{а}}. Прежде буква {\slv{ї}} обозначала тот звук, который изображается в русском языке иногда буквой <<й>> (<<и краткое>>), например в словах <<мой>>, <<лейка>>. Этот звук (а не буква) называется <<йотом>>, а гласные, перед которыми он слышится называются <<йотированными>>. При соединении звука <<йота>> со звуком {\slv{а}} получалось созвучие {\slv{їа}} (<<йа>>), которое и стали изображать буквой {\slv{ꙗ}}, а произносить: <<я>>. Обычно йотированная буква {\slv{ꙗ}} пишется только в начале слова, например: {\slv{ꙗ҆́рость, ꙗ҆́кѡ}}.
    
    Буква {\slv{ѧ}} первоначально изображала носовой звук, произносимый как <<ем>> или <<ен>>. Впоследствии этот носовой звук, называемый в древности <<малым юсом>>, перешел в звук <<я>>, но только не йотированный, т.е. без соединения с ним звука <<йота>>. Но нейотированный звук <<я>> может произноситься только после согласного звука, а потому буква {\slv{ѧ}} пишется в середине и конце слова, например: {\slv{мѧте́жъ, стезѧ̀}}.
    
    Но в слове <<язык>>, которое в славянском языке имеет много значений, как-то: член тела, орган речи, наречие или говор, народ, племя, то условились, что если слово <<язык>> обозначает народ или племя, писать букву {\slv{ꙗ}}, например: {\slv{ꙗ҆зы́къ самарі́йскїй}} (т.е. народ самарийский). Если же слово <<язык>> обозначает член тела, орган речи, наречие (говор), то условились писать букву{\slv{ѧ}}, например: {\slv{Прильпнѝ ѧ҆зы́къ мо́й горта́ни моемꙋ̀}} (Пс. 136, 6); {\slv{церко́вно-славѧ́нскїй ѧ҆зы́къ}}.
    
    \begin{flushright}
    \emph{Упражнение 2}
    \end{flushright}
    \medskip
    
    \textbf{Славянские или церковные буквы, не сходные с русскими}
    \medskip
    
    \begin{slv}
        Оу҆́хо.~\textemdash~Оу҆́тро.~\textemdash~Оу҆ста̀.~\textemdash~Оу҆́тренѧ.~\textemdash~Оу҆спе́ние.~\textemdash~Оу҆пова́нїе.~\textemdash~Дꙋ́хъ.~\textemdash~Дꙋша̀.~\textemdash~Слꙋ́жба.~\textemdash~І҆исꙋ́съ.~\textemdash~Ꙗ҆́сли.~\textemdash~Ꙗ҆зы́къ.~\textemdash~И҆́мѧ.~\textemdash~Сѣ́ьѧ.~\textemdash~Марі́ѧ.~\textemdash~Клѧ́тва.~\textemdash~Вече́рнѧ.~\textemdash~Лїтꙋргі́ѧ.~\textemdash~І҆ѡ́на.~\textemdash~І҆ѡа́ннъ.~\textemdash~І҆ѡ́сифъ.~\textemdash~І҆ѻрда́нъ.~\textemdash~І҆ѡанні́кїй.
        
        Го́споди, поми́лꙋй.~\textemdash~Приклонѝ ᲂу҆́хо Твоѐ мнѣ̀ и҆ ᲂуслы́ши глаго́лы моѧ̀.~\textemdash~Ѕаꙋ́тра ᲂу҆слы́ши гла́съ мо́й, Царю̀ мо́й и҆ Бо́же мо́й.~\textemdash~Оу҆слы́ши, Го́споди, пра́вдꙋ мою̀, вонмѝ моле́нїю моемꙋ̀.~\textemdash~Да возра́дꙋетсѧ дꙋша̀ моѧ̀ ѡ҆ Го́сподѣ.~\textemdash~Положѝ, Го́споди, хране́нїе ᲂу҆стѡ́мъ мои̑мъ.~\textemdash~Сла́ва Тебѣ̀ Бо́гꙋ Благода́телю на́шемꙋ во вѣ́ки вѣко́въ.~\textemdash~Сла́ва Ѻ҆тцꙋ̀ и҆ Сы́нꙋ и҆ Свѧто́мꙋ Дꙋ́хꙋ.~\textemdash~Бо́же, ми́лостивъ бꙋ́ди мнѣ̀ грѣ́шникꙋ.
        
        Ꙗ҆вѝ нам, Го́споди, ми́лость Твою̀ и҆ спасе́нїе Твоѐ да́ждь на́мъ.~\textemdash~И҆сповѣ́дайтесѧ Го́споди, ꙗ҆́кѡ благ: ꙗ҆́кѡ въ вѣ́къ ми́лость є҆гѡ́.~\textemdash~Да бꙋ́детъ во́лѧ твоѧ̀, ꙗ҆́кѡ на небесѝ и҆ на землѝ.~\textemdash~А҆́нгели, ᲂу҆спѣ́нїе Пречи́стыѧ ви́дѣвше, ᲂу҆диви́шасѧ.
        
        Благословѝ, дꙋшѐ моѧ̀, Го́спода.~\textemdash~Го́споди, воззва́хъ къ Тебѣ̀, ᲂу҆слы́ши мѧ̀.~\textemdash~Поми́лꙋй мѧ̀, Бо́же, поми́лꙋй мѧ̀.~\textemdash~Пресвѧта́ѧ Тро́ице, поми́лꙋй на́съ.~\textemdash~Свѧ́тъ, Свѧ́тъ, Свѧ́тъ є҆сѝ Бо́же, Богоро́дицею поми́лꙋй на́съ.
        
        Го́споди, ѡ҆чи́сти грѣхѝ на́шѧ.~\textemdash~Моли́ Бо́га ѡ҆ на́съ, свѧти́телю ѻ҆́тче Нїко́лае, ꙗ҆́кѡ мы̀ ᲂу҆се́рднѡ къ Тебѣ̀ прибега́емъ.~\textemdash~Ѽ, Всепѣ́таѧ Ма́ти, ро́ждшаѧ всѣ́хъ свѧты́хъ Свѧтѣ́йшее Сло́во!~\textemdash~Ѽ, пресла́дкїй и҆ всеще́дрый І҆исꙋ́се, прїимѝ ны́нѣ ма́лое моле́нїе сїѐ на́ше, ꙗ҆́коже прїѧ́лъ є҆сѝ вдови́цы два̀ ле́пта.
    \end{slv}

                \subsubsection{Надстрочные знаки}

    В церковнославянском языке употребляются следующие надстрочные знаки: \textbf{ударения}, \textbf{придыхание} и \textbf{титла}.

    \medskip
    \underline{1. Ударения}    
    \medskip

    Ударения ставятся над гласными буквами для усиления слога. Первый слог слова, если он начинается с гласного звука, и последний слог слова, если он оканчивается гласным звуком, называются \emph{открытыми} слогами; все же остальные слоги слова называются \emph{закрытыми} слогами. Ударение ставится также над односложными словами.
    
    Ударения в церковнославянском языке бывают: \textbf{острое} ({\slv{~́}}), \textbf{тупое} ({\slv{~̀}}) и \textbf{облегченное} ({\slv{~̑}}).
    
    Острое ударение может стоять на любом слоге слова, кроме последнего открытого слога, например: {\slv{си́ла, мо́ре, любо́вь, горта́нь, є҆диномꙋ́дрие}}.
    
    Тупое ударение ставится всегда только на последнем открытом слоге, например: {\slv{рꙋка̀, млеко̀, сотворѝ, рцы̀}}.
    
    Облегченное ударение может стоять на всех слогах слова, кроме первого открытого слога. Это ударение вообще употребляется для различения созвучных грамматических форм, о чем будет сказано в надлежащем месте.
    
    Если за словом с ударением на последнем открытом слоге следует одно из односложных слов, оканчивающихся на гласный звук, т.е. состоящих из одного открытого слога, а именно: {\slv{бо̀, мѧ̀, мѝ, тѧ̀, тѝ, сѧ̀, сѝ, жѐ}} (но не предлоги {\slv{во, на, по, при}} и проч.), то ударение над этими односложными словами опускается, а предшествующее им слово меняет тупое ударение на острое. Например: {\slv{Сꙋди́ ми, бж҃е}} (Пс. 42, 1); {\slv{Бѣ́ же во дни̑ во це́ркви ᲂу҆чѧ̀}} (Лк. 21, 37).
    
    Односложные предлоги и частицы {\slv{не, ни, но, на, по, да, же, при, пре, бо, во, ко, со, за}} и др. не имеют над собою никакого ударения. Исключение из этих правил составляет слово {\slv{сѐ}} (всю), которое сохраняет над собою ударение, и иногда частица {\slv{ни}}, когда она употребляется в значении русского слова <<нет>>, например: {\slv{Бꙋ́ди же сло́во ва́ше: є҆́й, є҆́й: нѝ, нѝ}} (Мф. 5, 37).
    
    \medskip
    \underline{2. Придыхание}
    \medskip
    
    Когда слово начинается с открытого слога, то над гласной буквой этого слога ставится придыхание ({\slv{~҆}}), например: {\slv{є҆ле́й, и҆мѣ́ти, ᲂу҆чени́къ, ꙗ҆зы́къ}}.
    
    В словах, где ударение падает на начальный открытый слог, то над гласной буквой ставится придыхание вместе с острым ударением, например: {\slv{ѻ҆́ко, а҆́ще, и҆́мѧ, є҆́зеро}}. Такое соединение придыхания с острым ударением называется словом {\slv{и҆́со}}.
    
    Союзы {\slv{и҆}} и {\slv{а҆}}, а также предлоги {\slv{ᲂу҆}} и {\slv{ѡ҆}} пишутся только с одним придыханием (без ударения).
    
    Иногда бывает соединение придыхания с тупым ударением. Такое соединение называется словом {\slv{а҆́построфь}}.
    
    Апострофь ставится на начальном открытом ударном слоге для различения смысла некоторых созвучных слов. Например: {\slv{и҆}} (союз <<и>>) и {\slv{и҆̀}} (он, его), {\slv{ᲂу҆}} (предлог <<у>>) и {\slv{ᲂу҆̀}} (еще), {\slv{и҆́же}} (который) и {\slv{и҆̀же}} (которые) и т.п.

    Эти апострофорные слова сохраняют свое ударение и после слов, оканчивающихся ударным открытым слогом. Например: {\slv{И̑ а҆́бїе воззва̀ ѧ҆̀}} (Мк. 1, 20).
    
    \medskip
    \underline{3. Титла}
    \medskip
    
    В церковнославянском языке некоторые слова пишутся сокращенно: с пропуском одной или нескольких букв, а иногда и целых слогов в середине слова. Сокращение это обозначается особыми знаками \textbf{титлами}.
    
    Различают простое титло и буквенные титла.
    
    \emph{Простое} титло изображается знаком {\slv{~҃}}, который ставится вверху слова над той буквой, перед которой или после которой имеется пропуск одной или нескольких букв. Например: {\slv{бг҃ъ}} (Бог), {\slv{ѻ҆ц҃ъ}} (Отец), {\slv{бл҃года́ть}} (благодать).
    
    \emph{Буквенное} титло представляет собою дугообразный титловой знак {\slv{~҇}} с какой-либо подписной буквой, именно с той, какая буква пропущена или находится среди пропущенных (если их несколько). Буквенные титла следующие:
    \medskip
    
    {\slv{~ⷢ҇}}~---~глаголь-титло: {\slv{є҆ѵⷢ҇лїе}} (Евангелие),

    {\slv{~ⷪ҇}}~---~он-титло: {\slv{прⷪ҇ро́къ}} (пророк),

    {\slv{~ⷬ҇}}~---~рцы-титло: {\slv{и҆мⷬ҇къ}} (имярек),

    {\slv{~ⷭ҇}}~---~слово-титло: {\slv{гдⷭ҇ь}} (Господь).
    \medskip
    
    К буквенным титлам еще относится добро-титло, которое изображается без титлового знака {\slv{~҇}}, а просто ставится буква <<добро>> вверху слова. Например: {\slv{бцⷣа}} (Богородица), {\slv{влⷣко}} (Владыко).
    
    Общее правило постановки титл таково. Под титло подводится слово тогда, когда оно относится к лицам и предметам священным, почитаемым; наоборот, те же самые слова, но относящиеся к обыкновенным именам, предметам и понятиям, в особенности же к языческим, пишутся без титла. Например:
    \medskip
    
    {\slv{бг҃ъ}} (истинный Бог) и {\slv{бо́гъ}} (бог языческий)

    {\slv{і҆и҃съ}} (имя Господа) и {\slv{і҆исꙋ́съ}} (имя человека)

    {\slv{мр҃і́а}} (Пресвятая Дева Мария) и {\slv{марі́а}} (имя женщины)

    {\slv{мт҃и}} (Матерь Господа) и {\slv{ма́ти}} (земная мать)

    {\slv{ѻ҆ц҃ъ}} (Отец Небесный) и {\slv{ѻ҆те́цъ}} (земной отец)

    {\slv{дв҃а}} (Дева Пресвятая) и {\slv{дѣ́ва}} (девица)

    {\slv{дх҃ъ}} (Дух Святый) и {\slv{дꙋ́хъ}} (дыхание)

    {\slv{а҆́гг҃лъ}} (Ангел) и {\slv{а҆́ггелъ}} (аггел, или злой дух)
    \medskip

    Изредка под титлами пишутся и обыкновенные слова, имеющие в отдельных случаях священное значение или священную память:
    
    {\slv{мцⷭ҇ъ}} (месяц), {\slv{чл҃вѣ́къ}} (человек), {\slv{нн҃ѣ}} (ныне), {\slv{срⷣце}} (сердце), {\slv{сл҃нце}} (солнце), {\slv{дн҃ь}} (день).
    
    Есть сокращения на особые случаи:
    
    1. В венце образа Христа Спасителя встречаются надписи: {\slv{о҆ ѡ҆́н}}. Это греческое слово (), написанное славянскими буквами и означающее {\slv{сы́й}}, т.е. сущий.
    
    2. В венце образа распятого на кресте Христа Спасителя имеются такие буквы: {\slv{І҆.Н҃.Ц҃.І҆.}}, или {\slv{І҆Н҃Ц҃І҆}}. Это означает: <<Иисус Назарянин Царь Иудейский>>.
    
    3. На образе Божией Матери встречается надпись {\slv{мр҃ ѳꙋ҃}}, т.е. по-гречески ()~---~Матерь Бога.
        
                \subsubsection{Буквы как числовые знаки (цифры)}

    Буквы церковнославянской азбуки служат одновременно и цифрами. Чтобы показать, что данная буква является цифрой, над буквой ставится простое титло {\slv{~҃}}.

    \begin{center}
        \renewcommand*{\arraystretch}{1.4}    \begin{longtable}{|c|c|c|c|}
%            \caption*{Числа}\\

            \hline
            
            \emph{Единицы}
            & \emph{Десятки}
            & \emph{Сотни}
            & \emph{Тысячи}
            \\
            
            \scriptsize\makecell{изображаются\\следующими\\буквами}
            & \scriptsize\makecell{изображаются\\следующими\\буквами}
            & \scriptsize\makecell{изображаются\\следующими\\буквами}
            & \scriptsize\makecell{изображаются теми же буквами,\\какими единицы,\\десятки и сотни,\\но с прибавлением впереди\\буквы значка}
            \\
        
            &
            &
            & \huge{\slv{҂}}
            \\
            
            \hline
            \endfirsthead
            
            \multicolumn{4}{l}%
            {
                %                \tablename\ \thetable\ -- 
                \footnotesize\textit{Начало на предыдущей странице}
            }
            \\
            
%            \hline
%            Единицы & Десятки & Сотни & Тысячи\\ % This is the "next" header.
            \hline
            \endhead
            \hline
            
            \multicolumn{4}{r}{\footnotesize\textit{Продолжение на следующей странице}}
            \\
            
            \endfoot
            \hline
            \endlastfoot
            
            {\slv{а҃}} (1) & {\slv{і҃}} (10) & {\slv{р҃}} (100) & {\slv{҂а҃}} (1000) \\\hln
            {\slv{в҃}} (2) & {\slv{к҃}} (20) & {\slv{с҃}} (200) & {\slv{҂в҃}} (2000) \\\hln
            {\slv{г҃}} (3) & {\slv{л҃}} (30) & {\slv{т҃}} (300) & {\slv{҂г҃}} (3000) \\\hln
            {\slv{д҃}} (4) & {\slv{м҃}} (40) & {\slv{у҃}} (400) & {\slv{҂і҃}} (10000) \\\hln
            {\slv{є҃}} (5) & {\slv{н҃}} (50) & {\slv{ф҃}} (500) & {\slv{҂к҃}} (20000) \\\hln
            {\slv{ѕ҃}} (6) & {\slv{ѯ҃}} (60) & {\slv{х҃}} (600) & {\slv{҂л҃}} (30000) \\\hln
            {\slv{з҃}} (7) & {\slv{ѻ҃}} (70) & {\slv{ⷬѱ҃}} (700) & {\slv{҂р҃}} (100000) \\\hln
            {\slv{и҃}} (8) & {\slv{п҃}} (80) & {\slv{ѿ}} (800) & {\slv{҂с҃}} (200000) \\\hln
            {\slv{ѳ҃}} (9) & {\slv{ч҃}} (90) & {\slv{ц҃}} (900) & {\slv{҂҂а҃}} (1000000) \\\hln
            
        \end{longtable}
    \end{center}

    Десять тысяч ({\slv{҂і҃}}) произносится по-славянски {\slv{тьма̀}}, двадцать тысяч ({\slv{҂к҃}})~---~{\slv{двѣ̀ тьмы̀}}, тридцать тысяч ({\slv{҂л҃}})~---~{\slv{трѝ тьмы̀}} и т.д. Сто тысяч ({\slv{҂р҃}}) называется по-славянски {\slv{легеѡ́нъ}}, двести тысяч ({\slv{҂с҃}})~---~{\slv{два̀ легеѡ́на}} и т.д.
    
    Числа от 10 до 20 составляются следующим образом: сначала ставят единицы и справа приписывают десятки, причем титло ставится над единицами. Вот нумерация чисел второго десятка: {\slv{а҃і}} (11), {\slv{в҃і}} (12), {\slv{г҃і}} (13), {\slv{д҃і}} (14), {\slv{є҃і}} (15), {\slv{ѕ҃і}} (16), {\slv{з҃і}} (17), {\slv{и҃і}} (18), {\slv{ѳ҃і}} (19).
    
    Числа от 20 до 100 составляются порядком, т.е. сперва пишут десятки, а справа приписывают единицы; титло ставится над десятками: {\slv{к҃а}} (21), {\slv{л҃в}} (32), {\slv{м҃д}} (44), {\slv{н҃ѕ}} (56), {\slv{ѯ҃г}} (63), {\slv{ѻ҃ѳ}} (79).
    
    Таким же обычным порядком составляются числа и от 100 до 1000 и далее; титло во всех этих числах ставится всегда на предпоследней цифре (второй цифре справа): {\slv{сл҃є}} (235), {\slv{тн҃з}} (357), {\slv{фп҃ѕ}} (586), {\slv{хч҃д}} (694), {\slv{ѱм҃ф}} (749), {\slv{ѿо҃ⷢⷢг}} (873), {\slv{цк҃в}} (922), {\slv{҂ацн҃з}} (1957), {\slv{҂аѿп҃ѕ}} (1886), {\slv{҂аѿч҃д}} (1894), {\slv{҂зтч҃а}} (7391), {\slv{҂кцг҃і}} (20913), {\slv{҂ѻєу҃ѕ}} (75406).
    \bigskip

    \textbf{Упражнения в чтении употребительнейших слов под титлами}
    
    \begin{flushright}
    \emph{Упражнение 3}
    \end{flushright}
    \medskip
    
    \begin{slv}
        Бг҃ъ (Бо́гъ), Бг҃а (Бо́га), Бг҃ꙋ (Бо́гꙋ), Бг҃ови (Бо́гови), Бг҃омъ (Бо́гомъ), Бж҃е (Бо́же), ѡ҆ Бз҃ѣ (Бо́зѣ).
        
        Бг҃ъ любы̀ є҆́сть.~\textemdash~Бг҃ъ застꙋ́прикъ мо́й.~\textemdash~По́йте Бг҃ꙋ на́шемꙋ, по́йте!~\textemdash~А҆́ще кто̀ рече́тъ, ꙗ҆́кѡ люблю̀ Бг҃а, а҆ бра́та своего̀ ненави́дитъ, ло́жъ є҆́сть.~\textemdash~Бж҃е, ѡ҆чи́сти мѧ̀ грѣ́шнаго.~\textemdash~Ѡ҆ Бз҃ѣ спасе́нїе моѐ и҆ сла́ва моѧ̀.~\textemdash~Бж҃їѧ никто́же вѣ́сть.
        
        Гдⷭ҇ь (Госпо́дь), Гдⷭ҇а (Го́спода), Гдⷭ҇ꙋ (Го́сподꙋ), Гдⷭ҇ви (Го́сподеви), Гдⷭ҇и (Го́споди), ѡ҆ Гдⷭ҇ѣ (Го́споде).
        
        Гдⷭ҇ь мнѣ̀ помо́щникъ.~\textemdash~Не во́змеши и҆́мене Гдⷭ҇а Бг҃а твоегѡ̀ всꙋ́е.~\textemdash~Гдⷭ҇ꙋ помо́лимсѧ! Гдⷭ҇и, поми́лꙋй.~\textemdash~Гдⷭ҇и, не введѝ менѐ въ напа́сть.~\textemdash~Да возра́дꙋетсѧ дꙋша̀ моѧ̀ ѡ҆ Гдⷭ҇ѣ.
        
        І҆и҃съ (І҆исꙋ̀съ) Хрⷭ҇то́съ (Хрїсто́съ), І҆и҃са Хрⷭ҇та̀, І҆и҃сомъ Хрⷭ҇то́мъ, ѡ҆ І҆и҃сѣ Хрⷭ҇тѣ̀, Сп҃съ (Спа́съ), Сп҃си́тель (Спаси́тель), Хрⷭ҇то́въ (Христо́въ).
        
        Воскр҃съ І҆и҃съ ѿ гро́ба, ꙗ҆́кѡже проречѐ.~\textemdash~Вѣ́рꙋю во Є҆ди́наго Гдⷭ҇а І҆и҃са Хрⷭ҇та̀.~\textemdash~І҆и҃се, Сы́не Бж҃їй, поми́лꙋй мѧ̀.~\textemdash~І҆и҃се, Сп҃се мо́й, спасѝ мѧ҃!~\textemdash~Возра́довасѧ дх҃ъ мо́й ѡ҆ Бз҃ѣ Сп҃сѣ мое́мъ.~\textemdash~Дрꙋ́къ дрꙋ́га тѧгѡты̀ носи́те, и҆ та́кѡ и҆спо́лните зако́нъ Хрⷭ҇то́въ.
    \end{slv}

    \begin{flushright}
        \emph{Упражнение 4}
    \end{flushright}
    \medskip
    
    \begin{slv}
        Ст҃ъ (Свѧ́тъ), Ст҃ы́й (Свѧты́й), Ст҃а́гѡ, Ст҃о́мꙋ, Ст҃и, Ст҃а́ѧ, Ст҃ы̑мъ.
        
        Ст҃ъ Гдⷭ҇ь Бг҃ъ на́шъ.~\textemdash~Ст҃ъ, Ст҃ъ, Ст҃ъ Гдⷭ҇ь Вседержи́тель.~\textemdash~Поклони́мсѧ Ст҃о́мꙋ Гдⷭ҇ꙋ І҆и҃сꙋ, Є҆ди́номꙋ безгрѣ́шномꙋ.~\textemdash~Ст҃ы́й Бж҃е, Ст҃ы́й Крѣ́пкїй, Ст҃ы́й Безсме́ртный поми́лꙋй на́съ!~\textemdash~Свѧ́ти бꙋ́дите, ꙗ҆́кѡ А҆́зъ ст҃ъ (є҆́смь) Гдⷭ҇ь Бг҃ъ ва́шъ.~\textemdash~Ст҃а̑ѧ ст҃ы̑мъ.
        
        Ѻ҆цъ (Ѻ҆те́цъ), Сн҃ъ (Сы́нъ), Дх҃ъ (Дꙋ́хъ), Дш҃а̀ (Дꙋша̀), Трⷪ҇ца (Тро́ица), Нн҃ѣ (Ны́нѣ), Прⷭ҇нѡ (При́снѡ).
        
        Ѻ҆ц҃ъ, Сн҃ъ и҆ Ст҃ы́й Дх҃ъ Трⷪ҇ца Ст҃а́ѧ.~\textemdash~Оу҆пова́нїе моѐ Ѻ҆ц҃ъ, прибѣ́жище моѐ Сн҃ъ, покро́въ мо́й Дх҃ъ Ст҃ы́й, Трⷪ҇ца Ст҃а́ѧ, сла́ва Тебѣ̀.~\textemdash~Во и҆́мѧ Ѻ҆ц҃а̀ и҆ Сн҃а и҆ Ст҃а́гѡ Дх҃а. А҆ми́нь.~\textemdash~Сла́ва Ѻ҆ц҃ꙋ̀ и҆ Сн҃ꙋ и҆ Ст҃о́мꙋ Дх҃ꙋ, и҆ нн҃ѣ и҆ прⷭ҇нѡ, и҆ во вѣ́ки вѣкѡ́въ. А҆ми́нь.~\textemdash~Вѣ́рꙋю и҆ въ Дх҃а Ст҃а́го, Гдⷭ҇а животворѧ́щаго, и҆́же со Ѻ҆ц҃емъ и҆ Сн҃омъ спокланѧ́ема и҆ ссла́вима.~\textemdash~Прест҃а́ѧ Трⷪ҇це, Бж҃е на́шъ, сла́ва Тебѣ̀.~\textemdash~Ѿ сна̀ воста́въ, бл҃годарю̀ Тѧ, Ст҃а́ѧ Трⷪ҇це.~\textemdash~Да возра́дꙋетсѧ дш҃а҃ моѧ̀ ѡ҆ Гдⷭ҇ѣ, ѡ҆блече́ бо мѧ̀ въ ри́зꙋ спасе́нїѧ.
        
        Бцⷣа (Богоро́дица), Дв҃а (Дѣ́ва), Мр҃і́ѧ (Марі́ѧ), Мт҃и (Ма́ти), Прⷭ҇нодв҃а (Приснодѣ́ва).
        
        Мл҃твами Бцⷣы, млⷭ҇тиве, ѡ҆чи́сти мно́жество согрѣше́нїй на́шихъ.~\textemdash~Пребл҃гослове́нна є҆сѝ Бцⷣе Дв҃о.~\textemdash~Ра́дꙋйсѧ, бл҃года́тнаѧ Мр҃і́е.~\textemdash~Спасѝ ѿ бѣ́дъ рабы̑ твоѧ̑, Бцⷣе.~\textemdash~Пресла́внаѧ Прⷭ҇нодв҃о, Мт҃и Хрⷭ҇та̀ Бг҃а, принесѝ на́шꙋ мл҃твꙋ Сн҃ꙋ Твоемꙋ̀ и҆ Бг҃ꙋ на́шемꙋ, да спасе́тъ Тобо́ю дш҃ы на́ша.
    \end{slv}
    \medskip

    \begin{flushright}
    \emph{Упражнение 5}
    \end{flushright}

    \begin{slv}
        А҆́гг҃лъ (А҆́нгелъ), Влⷣко (Влады́ко), Влⷣчца (Влады́чица), Бл҃гъ (Бла́гъ), Цр҃ь (Ца́рь), Цр҃ица (Цари́ца).
        
        Ѡ҆полчи́тсѧ А҆́гг҃лъ Гдⷭ҇ень ѡ҆́крестъ боѧ́щихсѧ є҆гѡ̀, и҆ и҆зба́витъ и҆̀хъ.~\textemdash~Ст҃і́и а҆рха́г҃гли, моли́те Бг҃а ѡ҆ на́съ.~\textemdash~А҆́гг҃лѡмъ свои̑мъ заповѣ́сть ѡ҆ тебѣ̀, сохрани́ти тѧ̀ во всѣ́хъ пꙋте́хъ твои́хъ.~\textemdash~По́йте Бг҃ꙋ на́шемꙋ, по́йте Цр҃е́ви на́шемꙋ, по́йте: ꙗ҆́кѡ цр҃ь всеѧ̀ землѝ Бг҃ъ.~\textemdash~А҆́гг҃ле, храни́телю мо́й ст҃ы́й.~\textemdash~Бл҃гословѝ, Влⷣка, ст҃ы́й вхо́ди.~\textemdash~Прест҃а́ѧ Влⷣчце Бцⷣе, молѝ ѡ҆ на́съ грѣ́шныхъ.~\textemdash~Гдⷭ҇и и҆ Влⷣко живота́ моегѡ̀, дꙋ́хъ пра́здности, ᲂу҆ны́нїѧ, любонача́лїѧ, и҆ праздносло́вїѧ не да́ждь мѝ.~\textemdash~Бл҃гъ мнѣ̀ зако́нъ ᲂу҆́стъ Твои́хъ, па́че ты́сѧщъ зла́та и сребра̀.~\textemdash~Црⷭ҇тво Твоѐ црⷭ҇тво всѣ́хъ вѣкѡ́въ.
        
        Є҆ѵⷢ҇лі́стъ (Є҆ѵангелі́стъ), Є҆пⷭ҇кпъ (Є҆пі́скопъ), І҆и҃ль (І҆зра́иль), Крⷭ҇тъ (Кре́стъ), І҆ерⷭ҇ли́мъ (І҆ерꙋсали́мъ).
        
        Велича́емъ тѧ̀, а҆пⷭ҇ле Хрⷭ҇то́въ и҆ є҆ѵⷢ҇лі́сте І҆ѡа́нне Бг҃осло́ве.~\textemdash~Крⷭ҇тꙋ̀ Твоемꙋ̀ поклонѧ́емсѧ, Влⷣко!~\textemdash~Сла́ва, Гдⷭ҇и, крⷭ҇тꙋ̀ Твоемꙋ̀ чтⷭ҇но́мꙋ.~\textemdash~И̑сповѣ́дꙋю є҆ди́но кр҃ще́нїе во ѡ҆ставле́нїе грѣхѡ́въ.~\textemdash~Ѡ̑гради́ мѧ, Гдⷭ҇и, си́лою чтⷭ҇на́гѡ и҆ животворѧ́щагѡ Твоегѡ̀ крⷭ҇та̀, и҆ сохрани́ мѧ ѿ всѧ́кагѡ ѕла̀.
    \end{slv}
    \medskip

    \begin{flushright}
        \emph{Упражнение 6}
    \end{flushright}

    \begin{slv}
        Мрⷣость (Мꙋ̀дрость), Премⷣръ (Премꙋ́дръ), Млⷭ҇ть (Ми́лость), Мл҃тва (Моли́тва), Прⷭ҇то́лъ (Престо́лъ).
        
        И҆́стинꙋ возлюби́лъ є҆сѝ, безвѣ́стнаѧ и҆ та́йнаѧ премⷣрости Твоеѧ̀ ꙗ҆ви́лъ мѝ є҆сѝ.~\textemdash~Оу҆тверди́лъ є҆́сть Гдⷭ҇ь млⷭ҇ть Свою̀ на боѧ́щихсѧ Є҆гѡ́.~\textemdash~Прⷭ҇то́лъ Тво́й, Бж҃е, въ вѣ́къ вѣ́ка.
        
        Всѧ̑ нбⷭ҇ныѧ си̑лы ст҃ы́хъ а҆́гг҃лъ и҆ а҆рха̑гг҃лъ, моли́те ѡ҆ на́съ грѣ́шныхъ.~\textemdash~Пра́ведницы просвѣтѧ́тсѧ ꙗ҆́кѡ со́лнце, въ Ца́рствїи Ѻ҆ц҃а̀ и҆́хъ.~\textemdash~Сщ҃е́нницы Твоѝ ѡ҆блекꙋ̀тсѧ пра́вдою, и҆ пре́пбнїи Твоѝ возра́дꙋютсѧ.~\textemdash~Ржⷭ҇тво̀ Твоѐ, Хрⷭ҇те Бж҃е нашъ, возсїѧ̀ мі́рови свѣ́тъ ра́зꙋма.~\textemdash~Нб҃о прⷭ҇то́лъ Мо́й, землѧ̀ же подно́жїе но́гъ Мои́хъ.~\textemdash~Па́мѧть прⷣвныхъ съ похвала́ми: и҆́мѧ же нечести́выхъ ᲂу҆гаса́етъ.~\textemdash~И̑з̾ ᲂу҆́стъ младє́нецъ и҆ ссꙋ́щихъ соверши́лъ є҆сѝ хвалꙋ̀.~\textemdash~Совѣ́тъ Гдⷭ҇ень во вѣ́къ пребыва́етъ, помышлє҆нїѧ срⷣца Є҆гѡ́ въ ро́дъ и҆ ро́дъ.~\textemdash~Въ нача́лѣ сотворѝ Бг҃ъ не́бо и҆ зе́млю.~\textemdash~Ѻ҆́чи Гдⷭ҇ни на прⷣвныѧ, и҆ ᲂу҆́ши Є҆гѡ́ въ моли́твꙋ и҆̀хъ.~\textemdash~Млⷭ҇рдїѧ двє́ри ѿве́рзи на́мъ, Бл҃гослове́ннаѧ Бцⷣе!~\textemdash~Не клени́тесѧ ни нб҃ом, ни земле́ч, ни и҆но́ю ко́ею клѧ́твою.~\textemdash~Во Црⷭ҇твїи Твое́мъ помѧнѝ на́съ, Гдⷭ҇и, є҆гда̀ прїи́деши во Црⷭ҇твїи Твое́мъ.
    \end{slv}


        \section[Общие понятия о грамматических формах]{Общие понятия о грамматических формах церковнославянского языка}
                \subsubsection{Разделение звуков}

    \medskip
    Звуки церковнославянского языка, как и в русском языке, разделяются на \textbf{гласные} и \textbf{согласные}.
    \medskip
    
    Гласные звуки бывают:
    \medskip
    
    \emph{твердые:} {\slv{а, о, ꙋ, ы}};

    \emph{мягкие:} {\slv{ꙗ}} ({\slv{ѧ}}){\slv{, е, ю, и}} ({\slv{ї, ѵ}}){\slv{, ѣ}}.
    \medskip
    
    Согласные звуки делятся на:
    \medskip
    
    \emph{губные:} {\slv{б, в, м, п, ф}} ({\slv{ѳ}}){\slv{, ѵ}};
    
    \emph{гортанные:} {\slv{г, к, х}};
    
    \emph{зубные:} {\slv{д, з}} ({\slv{ѕ}}){\slv{, с, т, ц}}~---~эти звуки, кроме {\slv{д}} и {\slv{т}}, 
    
    называются еще свистящими;
    
    \emph{шипящие:} {\slv{ж, ч, ш, щ}};
    
    \emph{плавные:} {\slv{л, м, н, р}}; плавные звуки {\slv{м}} и {\slv{н}} 
    
    называются еще носовыми.
    \medskip

                \subsubsection{Чередование звуков}

    При изменении слов гласные и согласные звуки могут чередоваться ({\slv{берꙋ̀~\textemdash~соб{\Large и}ра́ю, сꙋ́хъ~\textemdash~сꙋ{\Large ш}и́ти}}). Особенно часто чередуются твердые согласные звуки с более мягкими. Это чередование происходит или посредством замены твердых согласных мягкими, или посредством вставок мягкого согласного перед твердым или после него. Такое чередование согласных носит название \emph{смягчение согласных звуков}.
    
    Вот основные случаи смягчения:

    \bigskip
    \underline{1. Смягчение гортанных звуков}
    \medskip
    
    Гортанные звуки {\slv{г, к, х}} перед гласными звуками {\slv{е}} и {\slv{и}} смягчаются посредством замены их шипящими, а именно:

    \begin{flushleft}
        \renewcommand*{\arraystretch}{1.2}
        \begin{tabular}[l]{clcl}

            ~~~~~
            & {\slv{г~\textemdash~ж:}}
            & ~~~~~
            & {\slv{дрꙋ́гъ~\textemdash~дрꙋ́же~\textemdash~дрꙋжи́ти}}
            \\
            
            ~~~~~
            & {\slv{к~\textemdash~ч:}}
            & ~~~~~
            & {\slv{ре́къ~\textemdash~речѐ~\textemdash~рѣ́чи}}
            \\
            
            ~~~~~
            & {\slv{х~\textemdash~ш:}}
            & ~~~~~
            & {\slv{дꙋ́хъ~\textemdash~дꙋ́ше}}
            \\
            
        \end{tabular}
    \end{flushleft}

    Но эти же самые гортанные звуки, находясь перед падежными окончаниями {\slv{ѣ}} и {\slv{и}} ({\slv{ы}}), а также перед {\slv{и}} ({\slv{ы}}) в некоторых глагольных формах,~---~смягчаются несколько иначе, посредством замены их свистящими:

    \begin{flushleft}
        \renewcommand*{\arraystretch}{1.2}
        \begin{tabular}[l]{clcl}
            
            ~~~~~
            & {\slv{г~\textemdash~з:}}
            & ~~~~~
            & {\slv{дрꙋ́гъ~\textemdash~дрꙋ́зи~\textemdash~ѡ҆ дрꙋзѣ́хъ}}
            \\
            
            ~~~~~
            & {\slv{к~\textemdash~ц:}}
            & ~~~~~
            & {\slv{ре́къ~\textemdash~рцы̀}}
            \\
            
            ~~~~~
            & {\slv{х~\textemdash~с:}}
            & ~~~~~
            & {\slv{лꙋ́хъ~\textemdash~лꙋ́си~\textemdash~ѡ҆ лꙋ́сѣх}}
            \\
            
        \end{tabular}
    \end{flushleft}

    Итак, гортанные звуки могут смягчаться или шипящими (1-й закон смягчения), или свистящими (2-й закон смягчения.)
    
    \bigskip
    \underline{2. Смягчение зубных звуков}
    \medskip
    
    Зубные звуки смягчаются шипящими, а именно:

    \begin{flushleft}
        \renewcommand*{\arraystretch}{1.2}
        \begin{tabular}[l]{clcl}
            
            ~~~~~
            & {\slv{з~\textemdash~ж:}}
            & ~~~~~
            & {\slv{вѧза́ти~\textemdash~вѧжꙋ̀}}
            \\
            
            ~~~~~
            & {\slv{с~\textemdash~ш:}}
            & ~~~~~
            & {\slv{писа́ти~\textemdash~пишꙋ̀}}
            \\
            
            ~~~~~
            & {\slv{т~\textemdash~щ:}}
            & ~~~~~
            & {\slv{свѣ́тъ~\textemdash~свѣщꙋ̀~\textemdash~свѣща̀}}
            \\
            
            ~~~~~
            & {\slv{ц~\textemdash~ч:}}
            & ~~~~~
            & {\slv{ѻ҆те́цъ~\textemdash~ѻ҆те́чествїе}}
            \\
            
        \end{tabular}
    \end{flushleft}

    Зубной звук {\slv{д}} смягчается вставкой перед ним звука {\slv{ж}}:
    
    \begin{flushleft}
        \renewcommand*{\arraystretch}{1.2}
        \begin{tabular}[l]{cl}
            
            ~~~~~
            & {\slv{ходи́ти~\textemdash~хождꙋ̀~\textemdash~хожде́нїе}}
            \\
            
        \end{tabular}
    \end{flushleft}
    
    \medskip
    \underline{3. Смягчение губных звуков}
    \medskip

    Губные звуки {\slv{б, в, м, п}} перед некоторыми гласными смягчаются вставкой после губных плавного звука {\slv{л}}:
    
    \begin{flushleft}
        \renewcommand*{\arraystretch}{1.2}
        \begin{tabular}[l]{cl}
            
            ~~~~~ & {\slv{люби́ти~\textemdash~люблю̀}}\\
            ~~~~~ & {\slv{ꙗзви́ти~\textemdash~ꙗ҆звлю̀}}\\
            ~~~~~ & {\slv{ломи́ти~\textemdash~ломлю̀}}\\
            ~~~~~ & {\slv{спа́ти~\textemdash~сплю̀}}\\
            
        \end{tabular}
    \end{flushleft}

                \subsubsection{Состав слова}

    Слова образуются через сочетания отдельных звуков.
    
    Соединение гласного звука с одним или несколькими согласными составляет \textbf{слог}. Очевидно, что в слове столько слогов, сколько в нем гласных звуков, а поэтому слова могут быть \emph{односложные, двухсложные} и вообще \emph{многосложные}.
    
    Одни из звуков, из которых образуются слова, являются неизменными, другие же могут изменяться. Так, например, в словах: {\slv{да́ти, дарова́нїе, воздаѧ́нїе}} и т.д. Звуки {\slv{да}} во всех этих словах не изменяются, остальные же звуки: {\slv{-ти, -рованїе, воз-, -ѧнїе}}~---~представляют собою различные сочетания.
    
    Неизменяемые звуки в слове составляют корень слова. Существенным признаком корня слова служит его \emph{односложность}.
    
    Часть слова, стоящая перед корнем, называется \textbf{приставкой}, например: в слове {\slv{воздаѧ́нїе}} слог {\slv{воз-}}, стоящий перед корнем {\slv{-да-}}, есть приставка.
    
    Часть слова, стоящая после корня и указывающая на определенный падеж в именах или на лицо в глаголах, называется \textbf{окончанием}, например в словах {\slv{сы́нъ, сы́на, сы́нꙋ}} или {\slv{нестѝ, несꙋ́тъ}} звуки {\slv{-ъ, -а, -ꙋ, -ти, -ꙋтъ}} являются окончаниями.
    
    Иногда между корнем слова и окончанием вставляется один или несколько звуков, которые образуют вставочные слоги в слове. Такие вставочные слоги в слове называются \textbf{суффиксами}; так, например, в слове {\slv{дарова́нїе}} слоги {\slv{-ро-, -ва-, -нї-}} являются суффиксами.
    
    Окончание слова меняется в зависимости от падежей и чисел в именах или в зависимости от времени, лица, числа в глаголах. Остальные же части слова (корень, приставка, суффиксы) в этих случаях не меняются и в своей совокупности называются \textbf{основой} слова. Например, в слове {\slv{воздаѧ́нїе}} основой слова будет {\slv{воздаѧ́нї}}, а окончанием~---~{\slv{е}}.

                \subsubsection{Слова простые и сложные}

    Слова бывают простые и сложные.
    
    Если в состав слова входит только один корень, то такое слово называется \textbf{простым}. Например: {\slv{вѣ́ра, ѡ҆бходи́ти}}.
    
    Если же слово произошло от соединения двух или более простых слов, то такое слово называется \textbf{сложным}. Возьмем, например, два таких простых слова: {\slv{смире́нїе}} и {\slv{мꙋ̀дрость}}. Из этих двух слов мы можем составить одно сложное слово: {\slv{смиренномꙋ́дрїе}}. Ясно, что в сложном слове столько корней, сколько в состав этого слова входит простых слов. В приведенном сложном слове {\slv{смиренномꙋ́дрїе}} два корня: {\slv{-мир-}} и {\slv{-мꙋдр-}}.
    
    Простые слова соединяются в сложные посредством так называемой \textbf{соединительной гласной}. Такими соединительными гласными служат звуки {\slv{е}} или {\slv{о}}, например: {\slv{пꙋт{\Large е}ше́ствие, благ{\Large о}лѣ́пїе}}.

                \subsubsection{Славянское неполногласие}

    \textbf{Славянским неполногласием} называется наличие в словах некоторых звуковых сочетаний между согласными. Эти звуковые сочетания следующие: {\slv{ра, ла, ре, ле}}.
    
    Неполногласием эти звуковые сочетания названы в противоположность соответствующим звуковым сочетаниям: {\slv{оро, оло, ере, еле}}, которые называются \textbf{русским полногласием}.
    
    Вот несколько примеров славянского неполногласия:
    
    \begin{flushleft}
        \renewcommand*{\arraystretch}{1.2}
        \begin{tabular}[l]{clclcl}
            
            ~~~~~
            & {\slv{гра́дъ}}~---~город
            & ~~~~~
            & {\slv{сребро̀}}~---~серебро
            & ~~~~~
            & {\slv{брада̀}}~---~борода
            \\
            
            ~~~~~
            & {\slv{зла́то}}~---~золото
            & ~~~~~
            & {\slv{млеко̀}}~---~молоко
            & ~~~~~
            & {\slv{глава̀}}~---~голова
            \\
            
            ~~~~~
            & {\slv{дре́во}}~---~дерево
            & ~~~~~
            & {\slv{кла́съ}}~---~колос
            & ~~~~~
            & {\slv{врата̀}}~---~ворота
            \\
            
        \end{tabular}
    \end{flushleft}

    

                \subsubsection{Части речи и их грамматические формы}

    В церковнославянском языке, как и в русском, десять частей речи. Все части речи разделяются на изменяемые и неизменяемые.

    К \textbf{изменяемым частям речи} относятся имена существительные, прилагательные, числительные, местоимения и глаголы. Это слова самостоятельные, обозначающие предметы, признаки, количество, действия и т.д.

    \underline{Имя существительное} бывает \emph{трех родов}:
    
    \begin{flushleft}
        \renewcommand*{\arraystretch}{1.2}
        \begin{tabular}[l]{clclclcl}
            
            ~~~~~
            & мужского:
            & ~~~~~
            & {\slv{человѣ́къ}}
            & ~~~~~
            & {\slv{ѻ҆лта́рь}}
            & ~~~~~
            & {\slv{хра́мъ}}
            \\
            
            ~~~~~
            & женского:
            & ~~~~~
            & {\slv{жена̀}}
            & ~~~~~
            & {\slv{це́рковь}}
            & ~~~~~
            & {\slv{си́ла}}
            \\
            
            ~~~~~
            & среднего:
            & ~~~~~
            & {\slv{вре́мѧ}}
            & ~~~~~
            & {\slv{село̀}}
            & ~~~~~
            & {\slv{благоволе́нїе}}
            \\
            
        \end{tabular}
    \end{flushleft}

    \underline{Имя прилагательное} и некоторые имена числительные, а также некоторые местоимения в свою очередь сами могут изменяться по родам. Например:
    
    \begin{flushleft}
        \renewcommand*{\arraystretch}{1.2}
        \begin{tabular}[l]{clclclcl}
            
            ~~~~~
            & мужской род:
            & ~~~~~
            & {\slv{благі́й}}
            & ~~~~~
            & {\slv{пѧ́тый}}
            & ~~~~~
            & {\slv{ѻ҆́нъ}}
            \\
            
            ~~~~~
            & женский род:
            & ~~~~~
            & {\slv{блага́ѧ}}
            & ~~~~~
            & {\slv{пѧ́таѧ}}
            & ~~~~~
            & {\slv{ѻна̀}}
            \\
            
            ~~~~~
            & средний род:
            & ~~~~~
            & {\slv{благо́е}}
            & ~~~~~
            & {\slv{пѧ́тое}}
            & ~~~~~
            & {\slv{ѻ҆но̀}}
            \\
            
        \end{tabular}
    \end{flushleft}
    
    Эти четыре части речи изменяются также по числам и падежам.

    \emph{Чисел} в церковнославянском языке три: единственное, двойственное, множественное.
    
    \emph{Падежей} в церковнославянском языке семь: именительный, родительный, дательный, винительный, звательный, творительный, предложный.
    
    Звательный падеж служит обращением к лицу или предмету и во множественном числе почти всегда сходен с именительным (имена числительные и местоимения звательного падежа не имеют).
    
    В двойственном числе во всех именах (кроме некоторых числительных и местоимений) сходны между собою следующие падежи:
    
    \begin{flushleft}
        \renewcommand*{\arraystretch}{1.2}
        \begin{tabular}[l]{clcl}
            
            ~~~~~ & 1) & ~ & именительный, винительный, звательный\\
            ~~~~~ & 2) & ~ & родительный, предложный\\
            ~~~~~ & 3) & ~ & дательный, творительный\\
            
        \end{tabular}
    \end{flushleft}

    Изменение слов по падежам называется \emph{склонением}.
    
    Все четыре перечисленные части речи называются \emph{склоняемыми}.
    
    \underline{Глагол} изменяется по временам, лицам и числам.
    
    Действия или состояния предметов, выражаемые глаголами, совершаются \emph{во времени}. Они могут происходить в настоящем времени, прошедшем времени и будущем времени.
    
    Глагол имеет \emph{три лица}: 1-е, 2-е и 3-е.
    
    \emph{Чисел} церковнославянский глагол имеет три: единственное, двойственное, множественное.
    
    Двойственное число глагола имеет особенность, заключающуюся в том, что это число изменяется по родам: у женского и среднего рода окончания сходны, но различаются от окончаний мужского рода; кроме того, 2-е и 3-е лица в каждом роде одинаковы по окончаниям.
    
    Изменения глагола по временам, лицам и числам, в двойственном числе и по родам называется \emph{спряжением}.
    
    Глагол имеет еще особые формы, о которых будет сказано впоследствии.
    
    К \textbf{неизменным частям речи} относятся наречия, предлоги, союзы, частицы и междометия. Из них только наречия являются самостоятельными словами, а предлоги, союзы и частицы представляют собою так называемые \emph{служебные слова}, которые придают самостоятельным словам надлежащие взаимные соотношения или различные оттенки смысла. Что же касается междометий, то они занимают совершенно отдельное место, не являются сами по себе ни самостоятельными, ни служебными словами, а выражают только различные чувства и переживания.

                \subsubsection{Понятие о предложении}

    Мысль, выраженная словами, называется \textbf{предложением}.
    
    Слова, входящие в состав предложения, бывают связаны между собою известными сочетаниями. Возьмем, например, такое предложение:
    
    \medskip
    {\slv{Небеса̀ повѣ́даютъ сла́вꙋ бж҃їю}} (Пс. 18, 2)
    \medskip
    
    Здесь слово {\slv{небеса̀}} связано (или, как говорят, согласовано) со словом {\slv{повѣ́даютъ}}, а слово {\slv{сла́вꙋ}}~---~со словом {\slv{бж҃їю}}.
    
    Каждое слово в предложении (если это слово не является служебным словом или междометием) отвечает на какой-то вопрос; так, в приведенном предложении
    
    \begin{flushleft}
        \renewcommand*{\arraystretch}{1.2}
        \begin{tabular}[l]{cllcl}
            
            ~~~~~
            & слово
            & {\slv{небеса̀}}
            & отвечает на вопрос
            & \emph{что?} (имен. падеж)
            \\
            
            ~~~~~
            &
            & {\slv{повѣ́даютъ}}
            & --
            & \emph{что делают?}
            \\
            
            ~~~~~
            &
            & {\slv{сла́вꙋ}}
            & --
            & \emph{что?} (винит. падеж)
            \\
            
            ~~~~~
            &
            & {\slv{бж҃їю}}
            & --
            & \emph{чью?}
            \\
            
        \end{tabular}
    \end{flushleft}

    Самостоятельные слова, входящие в состав предложения и отвечающие на какой-нибудь вопрос, называются \textbf{членами предложения}.
    
    В каждом предложении говорится о ком- или о чем-либо.
    
    То, о чем говорится в предложении, называется \emph{подлежащим}. Подлежащее всегда отвечает на вопрос именительного падежа: {\large кто}? или {\large что}? В данном предложении слово {\slv{небеса́}} является подлежащим.
    
    То, что говорится о подлежащем, называется \emph{сказуемым}. Сказуемое отвечает на вопрос: {\large что делает подлежащее}? или {\large что с ним делается}? или {\large что оно такое}? В приведенном предложении сказуемым является слово {\slv{повѣ́даютъ}}.
    
    Подлежащее и сказуемое могут иметь при себе объяснительные слова. В предложении слова {\slv{сла́вꙋ бж҃їю}} являются объяснительными словами.
    
    Подлежащее и сказуемое называются \emph{главными членами}, а объяснительные слова~---~\emph{второстепенными членами}.
    
    Из второстепенных членов различают дополнения, определения и обстоятельства.
    
    \emph{Дополнением} в предложении называется слово, относящееся к сказуемому и отвечающее на вопросы косвенных падежей, т.е. всех падежей, кроме именительного и звательного. В том же предложении дополнением служит слово {\slv{сла́вꙋ}} ({\large что}?~---~винит. падеж).
    
    \emph{Определение} может отвечать на вопросы: {\large какой}? {\large чей}? {\large который}? {\large сколько}? В данном предложении определением служит слово {\slv{бж҃їю}} ({\large чью}? {\large какую}?).
    
    \emph{Обстоятельства} выражают различные условия действия или состояния подлежащего. Они могут обозначать место действия, время, образ, цель и причину действия, отвечая на соответствующие вопросы.

                \subsubsection{Пунктуация}

    \emph{Пунктуацией} называется правило расстановки знаков препинания между словами в предложении. В церковнославянской письменности знаки препинания следующие: точка, запятая, двоеточие, знак вопроса, из которых точка и запятая ставятся в тех же случаях, как и в русском языке.
    
    Двоеточие ({\slv{:}}) употребляется перед собственной речью, или когда последующее предложение объясняет предыдущее. Но в церковнославянском тексте двоеточие употребляется и в том случае, когда по-русски должна стоять точка с запятой, например:
    
    \begin{center}
        {\slv{Воскли́кнемъ бг҃ꙋ сп҃си́телю на́шемꙋ: предвари́мъ лицѐ є҆гѡ̀\\ во и҆сповѣ́данїи}} (Пс. 94, 1--2)
    \end{center}

    Кроме того, двоеточие употребляется вместо русского многоточия,  когда указывается только начало церковного чтения или песнопения, например:
    
    \medskip
    {\slv{Сла́ва, и҆ ны́нѣ: Вѣ́рꙋю во є҆ди́наго бг҃а ѻ҆ц҃а̀: ({\footnotesize листъ} г҃). трис҃тое. Прес҃та́ѧ трⷪ҇це: По ѻ҆́ч҃е на́шъ: {\footnotesize свѧще҆́нникъ:} Ꙗ҆́кѡ твоѐ є҆́сть црⷭ҇тво:}}
    \medskip
    
    Для изображения знака вопрос в церковнославянском тексте употребляется знак <<точка с запятой>> ({\slv{;}}). Например:

    \begin{center}
        {\slv{А҆́зъ тре́бꙋю тобо́ю крⷭ҇ти́тисѧ, и҆ ты́ ли\\грѧде́ши ко мнѣ̀;}} (Мф. 3, 14)
    \end{center}

    Иногда знак вопроса в церковнославянском тексте обозначает знак восклицания ({\slv{!}}). Например:

    \begin{center}
        {\slv{Гдⷭ҇и, что́ сѧ ᲂу҆мно́жиша стꙋжа́ющїи мѝ;}} (Пс. 3, 2)\\(Господи, как умножились враждующие против меня!)
    \end{center}

        \section{Глагол}
                \subsubsection{Понятие о глаголе}

    Глагол есть самая основная часть речи во всех языках. В предложении глагол фигурирует почти всегда в качестве главного члена предложения~---~сказуемого. Поэтому-то, прежде изучения других самостоятельных частей речи церковнославянского языка, рекомендуется предварительно изучить простейшие церковнославянские глагольные формы.
    
    \textbf{Глаголом} называется часть речи, выражающая действие или состояние предмета или явления. Например:

    \begin{center}
        {\slv{И̑щи́те и҆ ѡбрѧ́щете}} (Мф. 7, 7)\\{\slv{А҆́зъ ᲂуснꙋ́хъ и҆ спа́хъ, воста́хъ}} (Пс. 3, 6)
    \end{center}


                \subsubsection{Глаголы архаические}

    Глаголы подчиняются известным законам образования и спряжения. Но в церковнославянском языке существует несколько глаголов, которые не подчиняются общим законам образования и спряжения. У этих глаголов сохранились еще древние первоначальные формы славянского языка, а потому такие глаголы получили название \emph{архаических}, т.е. древних, старинных. К таким глаголам относятся следующие пять: {\slv{бы́ти, вѣ́дѣти, ꙗ҆́сти, да́ти, и҆мѣ́ти}}.
    
    Из этих архаических глаголов глагол {\slv{бы́ти}} имеет особо важное значение. В силу такого исключительного значения архаического глагола {\slv{бы́ти}} его принято называть \emph{вспомогательным глаголом}.

                \subsubsection{Неопределенная форма глаголов}

    У всех глаголов есть такая форма, которая не спрягается и служит лишь указанием действия или состояния. Например: {\slv{писа́ти, хвали́ти, нестѝ}}.
    
    Такая глагольная форма называется \emph{неопределенной формой} глагола и считается \emph{началом глагола}.
    
    Неопределенная форма глагола всегда имеет окончание {\slv{-ти}}. Если иногда неопределенная форма глагола и имеет окончание {\slv{-щи}} ({\slv{мощѝ, рещѝ}}), то такое окончание не представляет собою никакого исключения: это то же окончание {\slv{-ти}}, но только с предшествующим гортанным {\slv{-г-}} или {\slv{-к-}} ({\slv{мог-тѝ, рек-тѝ}}), а эти гортанные ({\slv{-г-}} и {\slv{-к-}}), соединяясь с {\slv{-т-}}, переходят, по закону смягчения, вместе с ним в {\slv{-щ-}}.
    
    Таким образом, вместо {\slv{могтѝ}} получилось {\slv{мощѝ}} ({\slv{г}} + {\slv{т}} = {\slv{щ}}), а вместо {\slv{ректѝ}}~---~{\slv{рещѝ}} ({\slv{к}} + {\slv{т}} = {\slv{щ}}).
    
%    \pagebreak

                \subsubsection{Спряжение в настоящем времени вспомогательного глагола {\slv{бы҆́ти}}}

    Глагол в настоящем времени служит для выражения действия или состояния в данный момент. Например:

    \begin{center}
        {\slv{Гдⷭ҇ь пасе́тъ мѧ̀}} (Пс. 22, 1)
    \end{center}

    Важное значение глагола {\slv{бы́ти}} среди других глаголов, как глагола вспомогательного, заставляет изучать его спряжение прежде спряжения других глаголов.

    \begin{center}
        Спряжение глагола {\slv{бы́ти}} в настоящем времени
        \renewcommand*{\arraystretch}{1.2}
        \begin{tabular}[c]{|c|c|c|c|c|}
            \hline
            
            \multirow{2}{*}{\scriptsize\makecell{Лицо}}
            & \multirow{2}{*}{\scriptsize\makecell{Единственное\\число}}
            & \multicolumn{2}{c|}{\scriptsize\makecell{Двойственное число}}
            & \multirow{2}{*}{\scriptsize\makecell{Множественное\\число}}
            \\
            
            \cline{3-4}
            
            &
            & \scriptsize\makecell{Мужской род}
            & \scriptsize\makecell{Жен. и сред. род}
            &
            \\
            
            \hline
            
            1-е
            & \makecell{{\footnotesize\slv{а҆́зъ}} {\slv{є҆́смь}}}
            & \makecell{{\footnotesize\slv{мы̀}} {\slv{є҆сва́}}\\({\scriptsize или} {\slv{є҆сма̀}})}
            & \makecell{{\footnotesize\slv{мы̀}} {\slv{є҆свѣ̀}}}
            & \makecell{{\footnotesize\slv{мы̀}} {\slv{є҆смы̀}}}
            \\\hline
            
            2-е
            & \makecell{{\footnotesize\slv{ты̀}} {\slv{є҆сѝ}}}
            & \makecell{{\footnotesize\slv{вы̀}} {\slv{є҆ста̀}}}
            & \makecell{{\footnotesize\slv{вы̀}} {\slv{є҆стѣ̀}}}
            & \makecell{{\footnotesize\slv{вы̀}} {\slv{є҆стѐ}}}
            \\\hline
            
            3-е
            & \makecell{{\footnotesize\slv{ѻ҆́нъ, ѻ҆на̀, ѻ҆но̀}}\\{\slv{є҆́сть}}}
            & \makecell{{\footnotesize\slv{ѻ҆́на}} {\slv{є҆ста̀}}}
            & \makecell{{\footnotesize\slv{ѻ҆́нѣ}} {\slv{є҆стѣ̀}}}
            & \makecell{{\footnotesize\slv{ѻ҆нѝ, ѻ҆нѣ̀, ѻ҆нѝ}}\\{\slv{сꙋ́ть}}}
            \\\hline
        
        \end{tabular}
    \end{center}

    Если при глаголе {\slv{бы́ти}} находится отрицание {\slv{не}},то в настоящем времени {\slv{не}} и {\slv{є҆́смь}} сливаются. Происходит так называемое \emph{стяжание} двух звуков {\slv{е}}. При стяжании двойной звук {\slv{ее}} переходит в {\slv{ѣ}}, и образуется слово {\slv{нѣсмь}}. В таком положении глагол {\slv{бы́ти}} с отрицанием {\slv{не}} и спрягается в настоящем времени во всех лицах и числах: {\slv{нѣ́смь, нѣ́си, нѣ́сть}}\ldots~{\slv{нѣ́смы, нѣ́сте}}\ldots~Исключение из этого правила составляет только 3-е лицо множественного числа, где отрицание {\slv{не}} пишется раздельно: {\slv{не сꙋ́ть}}.

    \begin{flushright}
        \emph{Упражение 7}
    \end{flushright}

    Поставьте приведенные в скобках слова в соответствующую форму.
    
    \begin{flushleft}
        \renewcommand*{\arraystretch}{1.2}
        \begin{tabular}[l]{cll}
            
            ~~~~~
            & \emph{Образец:}
            & \makecell[l]{В начале ты, Господи, основал землю,\\и небеса~---~({\slv{бы́ти}}) дело рук Твоих (Пс. 101, 26).}
            \\
            
            ~~~~~
            &
            &
            \\
            
            ~~~~~
            & \emph{Ответ:}
            & \makecell[l]{В начале ты, Господи, основал землю,\\и небеса~---~{\slv{сꙋ́ть}} дело рук Твоих.}
            \\
            
        \end{tabular}
    \end{flushleft}

    Правильность ответа проверьте в церковнославянском тексте Нового Завета.
    
    1. Петр же поднял его, говоря: встань; я тоже _____ ({\slv{бы́ти}}) человек (Деян. 10, 26).
    
    2. Грех не должен над вами господствовать, ибо вы _____ ({\slv{бы́ти}}) не под законом, но под благодатью (Рим. 6, 14).
    
    3. Но злой дух сказал в ответ: Иисуса знаю, и павел мне известен, а вы кто _____ ({\slv{бы́ти}})? (Деян. 19, 15).
    
    4. Итак, смотрите, поступайте осторожно, не как неразумные, потому что дни лукавы _____ ({\slv{бы́ти}}) (Еф. 5, 15).
    
    5. Ибо теперь мы живы _____ ({\slv{бы́ти}}), когда вы стоите в Господе (1 Фес. 3, 8).
    
    6. Итак, неизвинителен ты _____ ({\slv{бы́ти}}), всякий человек (Рим. 2, 1).
    
    7. Потому что немудрое Божие премудрее человеков _____ ({\slv{бы́ти}}), и немощное Божие сильнее человеков _____ ({\slv{бы́ти}}) (1 Кор. 1, 25).
    
    8. Что же? станем ли грешить, потому что мы _____ ({\slv{бы́ти}}) не под законом, а под благодатью? (Рим. 6, 15).
    
    9. Ибо нет _____ ({\slv{бы́ти}}) лицеприятия у Бога (Рим. 2, 11).
    
    10. Ибо все вы _____ ({\slv{бы́ти}}) сыны света и сыны дня: мы не _____ ({\slv{бы́ти}}) сыны ночи, ни тьмы (1 Фес. 5, 5).

                \subsubsection{Спряжение в настоящем времени прочих глаголов}

    Для образца спряжения прочих глаголов в настоящем времени возьмем глаголы {\slv{нестѝ}} и {\slv{хвали́ти}}.
    
    \begin{center}
        \renewcommand*{\arraystretch}{1.2}
        \begin{tabular}[c]{|c|c|c|}
            \hline
            
            \makecell{Лицо}
            & \makecell{Единственное число}
            & \makecell{Множественное число}
            \\\hline
            
            \makecell{1-е}
            & \makecell{{\slv{несꙋ̀, хвалю̀}}}
            & \makecell{{\slv{несе́мъ, хва́лимъ}}}
            \\\hln
            
            \makecell{2-е}
            & \makecell{{\slv{несе́ши, хва́лиши}}}
            & \makecell{{\slv{несе́те, хва́лите}}}
            \\\hln
            
                        
            \makecell{3-е}
            & \makecell{{\slv{несе́тъ, хва́литъ}}}
            & \makecell{{\slv{несꙋ́тъ, хва́лѧтъ}}}
            \\\hline

            
            \multicolumn{3}{|c|}{\makecell{Двойственное число}}
            \\\hline

            & \makecell{Мужской род}
            & \makecell{Женский и средний род}
            \\\hline
            
            \makecell{1-е}
            & \makecell{{\slv{несе́ва, хва́лива}}}
            & \makecell{{\slv{несе́вѣ, хва́ливѣ}}}
            \\\hln
            
            \makecell{2-е}
            & \makecell{{\slv{несе́та, хва́лита}}}
            & \makecell{{\slv{несе́тѣ, хва́литѣ}}}
            \\\hln
            
            \makecell{3-е}
            & \makecell{{\slv{несе́та, хва́лита}}}
            & \makecell{{\slv{несе́тѣ, хва́литѣ}}}
            \\\hline

        \end{tabular}
    \end{center}

    Необходимо заметить:

%                \subsubsection{Глагольные основы}
%                \subsubsection{Глаголы тематические и разделение их на два спряжения}
%                \subsubsection{Преходящее время вспомогательного глагола{\slv{бы҆́ти}}}
%                \subsubsection{Аорист вспомогательного глагола {\slv{бы҆́ти}}}
%                \subsubsection{Преходящее время тематических глаголов}
%                \subsubsection{Аорист тематических глаголов}
%                \subsubsection{Глаголы вида совершенного и несовершенного}
%                \subsubsection{Будущее время глаголов}
%                \subsubsection{Понятие о наклонениях глагола}
%                \subsubsection{Желательное наклонение глаголов}
%                \subsubsection{Повелительное наклонение глаголов}
%                \subsubsection{Спряжение архаических глаголов}
%                \subsubsection{Будущее время описательное}
%        \section{Склоняемые части речи}
%            \subsection{Имя существительное}
%                \subsubsection{Родовые окончания имен существительных}
%                \subsubsection{Первое склонение имен существительных}
%                \subsubsection{Общие замечания к первому склонению имен существительных}
%                \subsubsection{Имена существительные первого склонения с основой на шипящие и на{\slv{ц}}}
%                \subsubsection{Второе склонение имен существительных}
%                \subsubsection{Общие замечания ко второму склонению имен существительных}
%                \subsubsection{Имена существительные второго склонения с основой на шипящие и на {\slv{ц}}}
%                \subsubsection{Третье склонение имен существительных}
%            \subsection{Имя прилагательное}
%                \subsubsection{Разделение имен прилагательных}
%                \subsubsection{Краткие имена прилагательные}
%                \subsubsection{Склонение кратких имен прилагательных с твердым окончанием}
%                \subsubsection{Склонение кратких имен прилагательных с мягким окончанием}
%                \subsubsection{Склонение кратких имен прилагательных с основой на шипящие}
%            \subsection{Местоимение}
%                \subsubsection{Понятие о местоимении}
%                \subsubsection{Склонение личных местоимений {\slv{а҆́зъ}} и {\slv{ты̀}} и возвратного {\slv{себѐ}}}
%                \subsubsection{Особенности глагольного сказуемого в предложении}
%                \subsubsection{Согласование слов в предложении}
%                \subsubsection{Управление слов в предложении}
%                \subsubsection{Обращение}
%    \chapter*{2-й класс}
%    \label{ch:secondgrade}
%    \addcontentsline{toc}{chapter}{\nameref{ch:secondgrade}}
%    \setcounter{section}{0}
%    \setcounter{subsubsection}{0}
%        \section{Склоняемые части речи}
%            \subsection{Имя существительное}
%                \subsubsection{Имена существительные неравносложные}
%                \subsubsection{Имена существительные разносклоняемые}
%            \subsection{Местоимение}
%                \subsubsection{Склонение личного местоимения 3-го лица {\slv{ѻ҆́нъ}}}
%                \subsubsection{Склонение притяжательных местоимений}
%                \subsubsection{Склонение определительного местоимения {\slv{ве́сь}}}
%                \subsubsection{Склонение указательного местоимения {\slv{се́й}}}
%                \subsubsection{Склонение вспомогательных местоимений {\slv{кто̀}} и {\chs{что̀}}}
%                \subsubsection{Склонение указательного местоимения {\slv{то́й}} (тот)}
%                \subsubsection{Склонение указательного местоимения {\slv{ѻ҆́нъ, ѻ҆́ный}}}
%            \subsection{Имя прилагательное}
%                \subsubsection{Склонение полных имен прилагательных}
%                \subsubsection{Общие замечания к склонению полных имен прилагательных}
%                \subsubsection{Понятие о степенях сравнения имен прилагательных}
%                \subsubsection{Сравнительная степень имен прилагательных}
%                \subsubsection{Превосходная степень имен прилагательных}
%                \subsubsection{Неправильные степени сравнения}
%            \subsection{Имя числительное}
%                \subsubsection{Понятие об имени числительном}
%                \subsubsection{Имена числительные количественные}
%                \subsubsection{Имена числительные порядковые}
%                \subsubsection{Склонение количественных числительных}
%        \section{Глагол}
%                \subsubsection{Понятие о причастии}
%                \subsubsection{Несклоняемое (спрягаемое) причастие}
%                \subsubsection{Прошедшее совершенное время глаголов}
%                \subsubsection{Давнопрошедшее время глаголов}
%                \subsubsection{Условное наклонение глаголов}
%                \subsubsection{Безличные глаголы}
%                \subsubsection{Возвратная форма глагола}
%                \subsubsection{Страдательная форма глагола}
%                \subsubsection{Склоняемые причастия}
%                \subsubsection{Краткие действительные причастия настоящего времени}
%                \subsubsection{Полные действительные причастия настоящего времени}
%                \subsubsection{Краткие действительные причастия прошедшего времени}
%                \subsubsection{Полные действительные причастия прошедшего времени}
%                \subsubsection{Краткие страдательные причастия настоящего времени}
%                \subsubsection{Полные страдательные причастия настоящего времени}
%                \subsubsection{Краткие страдательные причастия прошедшего времени}
%                \subsubsection{Полные страдательные причастия прошедшего времени}
%                \subsubsection{Сложная страдательная форма глаголов}
%        \section{Неизменяемые части речи}
%                \subsubsection{Наречие}
%                \subsubsection{Предлог}
%                \subsubsection{Союз}
%                \subsubsection{Частицы}
%                \subsubsection{Междометия}
%        \section{Некоторые особенности церковнославянского синтаксиса}
%                \subsubsection{Двойной винительный падеж}
%                \subsubsection{Падеж родительный-разделительный}
%                \subsubsection{Замена притяжательных местоимений личными и возвратными в дательном падеже}
%                \subsubsection{Употребление местоимения и имени прилагательного в смысле имени существительного}
%                \subsubsection{Отсутствие отрицания при глаголе при наличии местоимений{\slv{никто̀, ничто̀}}}
%                \subsubsection{Славянский член и его употребление}
%                \subsubsection{Особенности славянского придаточного предложения}
%                \subsubsection{Особенности славянских придаточных предложений с союзными словами {\slv{є҆́же, во є҆́же, ѡ є҆́же}}}
%                \subsubsection{Особенности славянского придаточного предложения}
%                \subsubsection{Оборот <<Дательный самостоятельный>>}
%                \subsubsection{Оборот <<Дательный с неопределенным>>}
%                \subsubsection{Оборот <<Винительный с неопределенным>>}

\end{document}
