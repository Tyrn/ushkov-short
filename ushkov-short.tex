\documentclass[11pt,a4paper]{memoir}

\usepackage[usenames,dvipsnames,svgnames,table]{xcolor}
\usepackage{fontspec}
\usepackage{xltxtra}
% code borrowed from Polyglossia documentation — Thanks!
\definecolor{myblue}{rgb}{0.02,0.04,0.48}
\definecolor{lightblue}{rgb}{0.61,.8,.8}
\definecolor{myred}{rgb}{0.65,0.04,0.07}

\usepackage{polyglossia}
\setmainlanguage{russian}
\setotherlanguages{churchslavonic,english}
\usepackage{churchslavonic}
\usepackage{lettrine}

\setmainfont[Mapping=tex-text]{Liberation Serif}
\setsansfont[Mapping=tex-text]{Liberation Sans}
\setmonofont[Mapping=tex-text]{Liberation Mono}

\newfontfamily\churchslavonicfont[Script=Cyrillic,Ligatures=TeX,Scale=1.33333333,HyphenChar="005F]{PonomarUnicode.otf} 
\newfontfamily{\slv}[Scale=MatchLowercase]{Ponomar Unicode TT}
\newfontfamily{\ust}[Scale=MatchLowercase]{Menaion Unicode TT}
\newfontfamily{\ind}{Indiction Unicode TT}

\usepackage{indentfirst}
\frenchspacing
\clubpenalty=10000
\widowpenalty=10000

\sloppy

%% Insert Church Slavonic snippet into (Russian) text
\newcommand{\chs}[2][slv]{
    \begin{#1}\ignorespaces
        #2
    \ignorespacesafterend\end{#1}
}

\begin{document}
    
    \chapter*{\small БЛАГОЛЕПИЕ ЦЕРКОВНОСЛАВЯНСКОГО ЯЗЫКА}
    Прошло уже более тысячи лет со времени Крещения Руси. Воспринятое ею Православие, благодаря трудам святых Мефодия и Кирилла, совершает свои благоговейные богослужения на церковнославянском языке. Он по своей структуре наиболее близок к греческому, и это позволило сохранить существующий в Греческой церкви строй богослужения, что явилось для многочисленных славянских народов великим благом и живительным источником благочестия и совершенствования нравственного в духе истины правой веры, а также руководством в земных делах: образования, иконографии, храмо- и градостроительства.

\end{document}