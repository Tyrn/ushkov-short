\documentclass[11pt,a4paper,oneside]{memoir}

\pagestyle{plain}

\usepackage[explicit]{titlesec}

\usepackage{longtable}
\usepackage{makecell}
\usepackage{multirow}
\usepackage[usenames,dvipsnames,svgnames,table]{xcolor}
\usepackage{fontspec}
\usepackage{xltxtra}
% code borrowed from Polyglossia documentation — Thanks!
\definecolor{myblue}{rgb}{0.02,0.04,0.48}
\definecolor{lightblue}{rgb}{0.61,.8,.8}
\definecolor{myred}{rgb}{0.65,0.04,0.07}

\usepackage{polyglossia}
\setmainlanguage{russian}
\setotherlanguages{churchslavonic,english}
\usepackage{churchslavonic}
\usepackage{lettrine}

\setmainfont[Mapping=tex-text]{Liberation Serif}
\setsansfont[Mapping=tex-text]{Liberation Sans}
\setmonofont[Mapping=tex-text]{Liberation Mono}

\newfontfamily\churchslavonicfont[Script=Cyrillic,Ligatures=TeX,Scale=1.33333333,HyphenChar="005F]{PonomarUnicode.otf} 
\newfontfamily{\slv}[Scale=MatchLowercase]{Ponomar Unicode TT}
\newfontfamily{\ust}[Scale=MatchLowercase]{Menaion Unicode TT}
\newfontfamily{\ind}{Indiction Unicode TT}

\usepackage{indentfirst}
\frenchspacing
\clubpenalty=10000
\widowpenalty=10000

\sloppy

%% Ensure sequential numbering of subsubsections and paragraphs.
\setsecnumdepth{paragraph}
\counterwithout{paragraph}{subsubsection}
\counterwithout{subsubsection}{subsection}
\renewcommand{\thechapter}{\arabic{chapter}}
\renewcommand{\thesection}{\arabic{section}}
\renewcommand{\thesubsection}{\Roman{subsection}}
\renewcommand{\thesubsubsection}{\S\arabic{subsubsection}}
\setcounter{tocdepth}{3} % Must not precede the above

%\setparaheadstyle{\normalsize\itshape}
%\setafterparaskip{1ex}

%\titleformat{<command>}[<shape>]{<format>}{<label>}{<sep>}{<before>}[<after>]
\titleformat{\paragraph}[hang]{\raggedleft\normalfont\small\itshape\bfseries}{}{0pt}{#1 \arabic{paragraph}}

%% Suppress \hline on demand
\newcommand{\hln}{}

%% Rotate cell
\newcommand{\spheading}[2][10em]{% \spheading[<width>]{<stuff>}
    \rotatebox{90}{\parbox{#1}{\raggedright #2}}}

%% Borrowed from titlepages by Peter Wilson
%\usepackage{pst-text}

%%%% Additional font macros
\makeatletter
%%%% light series
%% e.g., s:12
\DeclareRobustCommand\ltseries
{\not@math@alphabet\ltseries\relax
    \fontseries\ltdefault\selectfont}
%% e.g., t:32
\newcommand{\ltdefault}{l}
%% e.g., v:19
\DeclareTextFontCommand{\textlt}{\ltseries}

% heavy(bold) series
\DeclareRobustCommand\hbseries
{\not@math@alphabet\hbseries\relax
    \fontseries\hbdefault\selectfont}
%% e.g., t:32
\newcommand{\hbdefault}{hb}
%% e.g., v:19
\DeclareTextFontCommand{\texthb}{\hbseries}
\makeatother

\newcommand*{\isbn}{{\small\textsc{ISBN}}}

%%% for the Web-O-Mints fonts
\newcommand*{\wb}[2]{\fontsize{#1}{#2}\usefont{U}{webo}{xl}{n}}
%\renewcommand*{\wb}[2]{}%    probably kills Web-O-Mints (and some layouts?)
%%% for the Fontsite 500 fonts
\newcommand*{\FSfont}[1]{%
    \fontencoding{T1}\fontfamily{#1}\selectfont}
%\renewcommand*{\FSfont}[1]{}%    kills special font selections

\newcommand*{\labelit}[1]{\phantomsection\label{#1}}
\newcommand*{\refit}[1]{(graphic on page~\pageref{#1})}

\chapterstyle{dash}\renewcommand*{\chaptitlefont}{\normalfont\itshape\LARGE}
\setlength{\beforechapskip}{2\onelineskip}
\setsecheadstyle{\normalfont\Large\raggedright}
\makeindex
\renewcommand*{\indexname}{Index of Designers}
\makeatletter
\newcommand*{\boxminipage}{%
    \@ifnextchar [%]
    \@ibxminipage
    {\@iiibxminipage c\relax[s]}}
\def\@ibxminipage[#1]{%
    \@ifnextchar [%]
    {\@iibxminipage{#1}}%
    {\@iiibxminipage{#1}\relax[s]}}
\def\@iibxminipage#1[#2]{%
    \@ifnextchar [%]
    {\@iiibxminipage{#1}{#2}}%
    {\@iiibxminipage{#1}{#2}[#1]}}
\let\@bxminto\@empty
\def\@iiibxminipage#1#2[#3]#4{%
    \ifx\relax#2\else
    \setlength\@tempdimb{#2}%
    \def\@bxminto{to\@tempdimb}%
    \fi
    \leavevmode
    \@pboxswfalse
    \if #1b\vbox
    \else
    \if #1t\vtop
    \else
    \ifmmode \vcenter
    \else \@pboxswtrue $\vcenter
    \fi
    \fi
    \fi
    %  \@bxminto
    \bgroup%          outermost vbox
    \hsize #4
    \hrule\@height\fboxrule
    \hbox\bgroup%   inner hbox
    \vrule\@width\fboxrule \hskip\fboxsep 
    \vbox \@bxminto
    \bgroup% innermost vbox
    \vskip\fboxsep
    \advance\hsize -2\fboxrule \advance\hsize -2\fboxsep
    \textwidth\hsize \columnwidth\hsize
    \@parboxrestore
    \def\@mpfn{mpfootnote}\def\thempfn{\thempfootnote}\c@mpfootnote\z@
    \let\@footnotetext\@mpfootnotetext
    \let\@listdepth\@mplistdepth \@mplistdepth\z@
    \@minipagerestore\@minipagetrue
    \everypar{\global\@minipagefalse\everypar{}}}

\def\endboxminipage{%
    \par\vskip-\lastskip
    \ifvoid\@mpfootins\else
    \vskip\skip\@mpfootins\footnoterule\unvbox\@mpfootins\fi
    \vskip\fboxsep
    \egroup%    end innermost vbox
    \hskip\fboxsep \vrule\@width\fboxrule
    \egroup%    end hbox
    \hrule\@height\fboxrule
    \egroup%    end outermost vbox
    \if@pboxsw $\fi}
\makeatother

\DeclareRobustCommand{\cs}[1]{\texttt{\char`\\#1}}
\newlength{\tpheight}\setlength{\tpheight}{0.9\textheight}
\newlength{\txtheight}\setlength{\txtheight}{0.9\tpheight}
\newlength{\tpwidth}\setlength{\tpwidth}{0.9\textwidth}
\newlength{\txtwidth}\setlength{\txtwidth}{0.9\tpwidth}
\newlength{\drop}

\newenvironment{showtitle}{%
    \begin{boxminipage}[c][\tpheight]{\tpwidth}
        \centering\begin{vplace}\begin{minipage}[c][\txtheight]{\txtwidth}}%
            {\end{minipage}\end{vplace}\end{boxminipage}}

\newcommand*{\titleCC}{\begingroup% City of Cambridge
    \drop=0.1\txtheight
    \vspace*{\drop}
    \centering 
    {\Large\itshape КРАТКИЙ УЧЕБНИК}\\[0.5\drop]
    {\textcolor{Red}{\HUGE\bfseries {\slv{✠}}}}\par
    \vspace{\drop}
    {\LARGE\itshape ЦЕРКОВНОСЛАВЯНСКОГО ЯЗЫКА}\par
    \vfill
    {\Large УШКОВ А.В.}\par
    \vfill
%    {\plogo}\\[0.5\baselineskip]
    {\itshape Веркола}\par
    {\scshape 2017}\par
    %\vfill
    \vspace*{\drop}
    \endgroup}
%% From titlepages by Peter Wilson, end

\usepackage[hidelinks]{hyperref}

\begin{document}
    \begin{titlingpage}
        %% Borrowed from titlepages by Peter Wilson
        \begin{showtitle}
            \titleCC
        \end{showtitle}
        \labelit{CC}
        {
            \par\vspace{0.2\baselineskip}
            %        \footnotesize \verb?\titleCC?
        }
        %% From titlepages by Peter Wilson, end
    \end{titlingpage}
    
    \tableofcontents*
    
    \chapter*{}
%    \markboth{}{}
        \section*{Благолепие церковнославянского языка}
        \label{sec:blago}
%        \phantomsection\addcontentsline{toc}{section}{\nameref{sec:blago}}

%% \nameref{} is not quite reliable in this context

        \phantomsection\addcontentsline{toc}{section}{Благолепие церковнославянского языка}

    Прошло уже более тысячи лет со времени Крещения Руси. Воспринятое ею Православие, благодаря трудам святых Мефодия и Кирилла, совершает свои благоговейные богослужения на церковнославянском языке. Он по своей структуре наиболее близок к греческому, и это позволило сохранить существующий в Греческой церкви строй богослужения, что явилось для многочисленных славянских народов великим благом и живительным источником благочестия и совершенствования нравственного в духе истины правой веры, а также руководством в земных делах: образования, иконографии, храмо- и градостроительства. Наша Русская Православная Церковь восприняла в неповрежденности и чистоте от Греческой Матери Церкви все догматическое, литургическое, духовное, святоотеческое наследие и предание, связав все это, с помощью славянского языка, с духом народа христианского на Руси, с жизнью и бытом и просвещением, ибо на славянский язык были переведены все богослужебные книги и писания греческих отцов, прием сохранялась точность, ясность и истинность, хотя перевод и являлся творческим делом.
    
    Славянский язык помогал памятовать о Христе, хранить Его живой образ и Его учение в сердце и душе и приносить плоды живого творческого восприятия слова Божия. В Церкви шло становление и образование самого славянского языка, его внутренняя христианизация и воцерковление, преображение самой стихии славянской мысли и слова, славянского голоса, души народа, На долю русского народа выпало редкое счастье принять христианство тогда, когда прогремели громом своим Вселенские Соборы и утверждено в чистоте апостольское Православие, и в Церкви Греческой утвердились и литургические, и догматические, и нравственные истины и самый язык славянский, ничем не запятнанный и чистый, воспринял для своего народа сосуд благодати, и в благодати Христовой народ верный обновляется, и просвещается, и спасается, и становится причастным вечной Истине, и славянский язык богатство веры и культуры передал многим полудиким племенам и народам, и в Церкви они нашли единение и братство Христово как дети Божии. На Руси потрудились многие ревнители духовного просвещения: и князь Владимир~---~просветитель, и св. Андрей Боголюбский, и свв. Александр Невский, Даниил Московский, святитель Петр, митрополит Алексий, преподобный Сергий Радонежский и многие другие подвижники благочестия. Созидались монастыри, храмы, школы. Создавались библиотеки, делались переводы на славянский язык, составлялись азбуки для малых и полудиких народностей, писались летописи.
    
    Славянский язык хранил церковное единство и православные традиции. 2 августа 1581 г. впервые была напечатана полная Библия на славянском языке в г. Остроге, что способствовало глубокому изучению Священного Писания. Русская Церковь в XVI в., когда протестанты превращали христианство в систему философских построений, как плод похи Возрождения язычества, начала вести длительную и тяжелую борьбу за сохранение чистоты веры, церковнославянского языка, церковно-национальной культуры.
    
    Славянская Библия занимает самое первое место по своей полноте и точности воспроизведения библейских древних писаний, о чем свидетельствуют и списки Кумранских рукописей, написанных за 200--300 лет до Р.Х., когда не было нужды искажать тексты Священного Писания. Древние рукописи своим текстом совпадают с острожской славянской Библией. В славянской Библии точно сохранено летоисчисление~---~5508 лет от сотворения мира до Рождества Христова, и все мессианские места. Известно, что текст 70 толковников, взятый в основу славянской Библии, долгое время считался в александрийских синагогах у евреев священным переводом, и его читали на греческом языке 200--300 лет. Масоретский же текст Библии идет от Акибы (около II века после Р.Х.).
    
    Кирилло-Мефодианский перевод Евангелия на славянский язык отличается точностью, верностью подлиннику и ясностью, чистотою выражения христианских понятий.
    
    Славянский текст Священного Писания и православного богослужения~---~свидетель первого тысячелетия, то есть неразделенного христианства, и в то же время он является драгоценным наследием равноапостольских трудов свв. Кирилла и Мефодия, на котором воспитываются чада Русской Православной Церкви.
    
    Церковнославянский богослужебный текст является важным и святым. Христиане разных конфессий стремятся осмыслить свои традиции в сравнении с традициями других исповеданий, ведь священные книги Нового Завета переведены на 1050 языков, и среди этих переводов самое важное место занимает славянский. Поскольку перевод Писания был сделан с греческого языка, а греческая текстология устанавливала единый печатный греческий текст XVII в., потом в XX веке его вытеснили из употребления, то в вопросе изысканий и оценки конечных результатов многих реконструкций текстов важную роль играет славянский текст как свидетель текстовой традиции первого тысячелетия, то есть времени неразделенного христианства.
    
    Острожская Библия 1581 года на славянском языке ни по форме, ни по содержанию не изменилась и является эталоном, по которому можно проследить реконструкцию греческого текста по времени.
    
    Анализ и сопоставление текстов показывает, что славянский текст стоит на твердой, нерушимой основе древней богослужебной практики Православия, а западное христианство отошло от древних традиций.
    
    Святые равноапостольные братья Кирилл и Мефодий использовали для перевода церковный византийский текст, который не искажался, был под контролем и оберегался всегда и всюду в Церкви, как нечто постоянное, неизменное и непреложное, как сама Истина, как церковно-текстуальное предание, уходящее в апостольскую древность. Это предание Церкви святое, соборное, апостольское, как поток, исходит из свежего родника живой воды евангельской, течет по чистому руслу под ясным небом и при тихой погоде, и его прозрачные воды менее доступны для загрязнения человеческими мудрованиями во имя кичащегося ума и в ущерб созидающей любви.
    
    В текстах встречаются несколько десятков слов труднопонятных, как тектон~---~дровосек, усмен~---~кожан, измена~---~выкуп, выну~---~всегда и др., но здравый человек за неделю их запомнит, ведь даже иностранные языки учат, да не один, а два-три языка, заучивая 40-45 слов в день, а язык, когда-то бывший родным, разговорным языком наших предков, язык, освященный веками, надо любить и хранить.
    
    Славянский язык в Церкви на Руси освящен непрерывным богослужение, и внешняя гармония, чистота, сердечность, молитвенность, мир Христов и утешение благодатное водворяются в душе православного христианина, как плод веры и любви.
    
    Большое значение славянскому языку для православного богослужения придает и Грузинская Церковь. Архимандрит Рафаил (Карелин) из Тбилиси выразил то так: <<В настоящее время в литургической жизни Грузинской Автокефальной Церкви славянский язык является по своему значению вторым, после древнегрузинского, богослужебным языком. Церковная славистика имеет для нас практическую важность и теоретический интерес>>.
    
    В связи с этим необходимо остановиться на нескольких вопросах, касающихся значения славянского языка для современного православного богослужения, рассматривая при этом славянский язык обобщенно и условно, как единую структурную и знаковую систему.
    
    В данном докладе славянский язык рассматривается обобщенно и условно, как единая структурная и знаковая система.
    
    Славянский язык насчитывает более чем тысячелетнюю литературную историю. Он не стал мертвым языком, не сделался достоянием одних археологов и лингвистов; славянский язык звучит на всех континентах, на нем совершается богослужение в православных церквах, на нем миллионы людей воссылают молитвы к Богу.
    
    В последнее время неоднократно поднимался вопрос: не является ли славянский язык литургическим архаизмом, преградой к пониманию текстов, и не следует ли перевести все богослужебные книги на современные языки?
    
    Такие попытки предпринимались в России и других странах.
    
    В России языковые литургические реформы не встретили поддержки основной массы верующих, и на сегодняшний день богослужебным языком остается церковнославянский. Можно ли объяснить это явление только одной приверженностью к традиции, или же оно имеет еще другие основания?
    
    Полагаем, что эти основания следующие.
    
    I. Славянскому языку, как и другим древним языкам (имеется в виду древнегреческий и древнегрузинский), присуща особая динамическая структура; он лучше передает пульс религиозной жизни, глубже выражает молитвенные чувства, чем современные языки.
    
    Древние языки более подходят к цельному синтетическому восприятию, новые~---~к аналитическому, дробному; древние~---~к созерцанию, новые~---~к логизированию; древние языки полны энергии и эмоций, новые, в сравнении с ними, носят рационалистический, описательный характер. Древние языки дают большую возможность соприкоснуться с глубиной явлений, с духовными субстанциями, сделать человека участником событий, современные~---~изложить явления в определенной системе и дать их анализ.
    
    Сравнивая лексическую и грамматическую ткань древних и новых языков, мы видим, что древние языки располагают меньшим по объему словарным фондом, но их грамматический строй отличается большим многообразием, пластичностью и совершенством.
    
    Мобильность и экспрессивность глагольных форм, богатство аффиксов и флексий, не имеющих аналогии в современных языках, лаконизм и динамика синтаксических структур, этимологическая глубина лексики создают неповторимую выразительность и, по словам патриарха Пимена, <<особую красоту>> славянского языка. Эти возможности славянского языка позволяют выразить в богослужебных текстах многоплановость событий, синергизм и параллелизм действий, объединить различные хронологические периоды в единые смысловые циклы (как бы преодолев линейную протяженность времени), проявить и усилить подтекст каждого предложения.
    
    Сложность и гибкость глагольных форм славянского языка делают одну и ту же по словарному составу фразу то ажурно-легкой и полетной, то~---~до физической ощутимости тяжелой и твердой, как бы изваянной из мрамора. Лаконизм, внутренняя чеканность и в то же время как бы внешняя незавершенность и обрывистость предложений, часто без последовательных переходов от одного предмета к другому, наделяют смысловые паузы эмоциональным содержанием, подчеркивая, что богослужебный текст~---~это не монолог, а диалог, таинственная беседа души с Божеством, что в священных событиях присутствует, как их постоянный субъект, личность каждого молящегося.
    
    Говоря образно, динамика древних языков созвучна динамике света. Богослужение~---~это симфония из лучей Божественных энергий. Священная история изображается в этой Божественной симфонии на светящемся фоне вечности, земные реалии~---~в их логоистическом преображении.
    
    Следует отметить, что в богослужебных текстах мастерски использован точно и строго очертанный круг изобразительных средств. Это делалось древними гимнографами для того, чтобы не подменить духовных переживаний эстетическими, т.е. душевными, не обременить ума молящегося художественной информацией и, тем самым, не оземлить молитву.
    
    Древние гимнографы мыслили Церковь как идею единства в Ее реальном и мистическом воплощении. Унификация изобразительных средств~---~одно из условий общности молитвы, духовного познания, эмоций и сопереживаний верующих~---~тела Церкви.
    
    Динамичные структуры, внутренние смысловые емкости и скульптурная пластичность славянского языка помогают воспринимать традиционные символы и метафористические образы в разных контекстах как вечно новые и неповторимые.
    
    Одним словом, древние языки более приспособлены для выражения явлений и динамики духовной жизни. Это первая и главная причина их сохранения в православном богослужении.
    
    II. Второе основание~---~трудность самого перевода.
    
    Богослужебные тексты~---~это шедевры священной поэзии особого типа и порядка. Переводчик должен хорошо знать святоотеческое богословие, именно святоотеческое, а не схоластическое, чтобы понять глубокие по догматическому содержанию литургические тексты. Далее, переводчик должен обладать одновременно глубокой филологической подготовкой и незаурядным поэтическим дарованием. Без этого перевод богослужебных книг будет казаться обесцвеченным и прозаическим пересказом великих поэм.
    
    Православные церковные службы называют опоэтизированным, иконографическим, поющим богословием.
    
    И, наконец, самое главное, переводчик должен иметь живую веру, церковность и религиозную интуицию, которая дала бы возможность почувствовать, пережить, оценить каждую фразу и слово переводимого текста.
    
    Интересно, что Петербургский митрополит Гавриил и Московский митрополит Филарет, руководившие работами по переводу святоотеческих произведений с греческого языка на славянский и русский, считали нужным давать рукописи переводов для проверки простым старцам-монахам, которые не знали греческого языка, но, обладая общностью внутреннего духовного опыта с древними писателями-подвижниками и находясь с ними, так сказать, на одной духовной волне, интуитивно чувствовали всякую фальшь и ошибку перевода.
    
    III. Третье основание~---~традиция. Традиция~---~это кристаллизация истории. Это актуальное бытие прошлого в настоящем. Любовь и уважение к прошлому сохранили до наших дней старинные соборы и храмы в их первозданном величии, иконы и фрески изумительной духовной глубины. Живая традиция сохранила нам дивное, неповторимое православное богослужение.
    
    Само слово <<Православие>> значит правое, правильное богопочитание, основным признаком которого является верность апостольскому преданию.
    
    Церковная традиция возникла из общности духовного опыта христиан, опыта emph{богообщения}. emph{Одинаковое религиозное чувство создало} одинаковые или очень схожие формы. Церковное богослужение и храмовое устройство~---~это синтез жизни Церкви в эпоху ее духовного расцвета. Содержание создает формы, но и формы хранят и оберегают содержание. В данном случае древние языки имеют вспомогательное, но важное значение для сохранения в чистоте и внутренней целостности одного из видов церковного предания~---~богослужебного канона.
    
    Чем устойчивее и тверже канон, тем глубже и чище он выражает общечеловеческую и духовную потребность: каноническое есть церковное, церковное~---~соборное, соборное же~---~всечеловеческое.
    
    Славянский язык, наряду с другими древними языками, стал сакральным, священным языком Церкви.
    
    Вопрос о ясности богослужебных славянских текстов для верующих приобретает все большее значение.
    
    Следует обратить особое внимание на научную сторону новых изданий богослужебных книг, снабдить их справочным материалом, в том числе словарем, где наряду с буквальным переводом давалась бы этимология славянской лексики. Не только Священное Писание, но и следующий за ним духовный пласт~---~богослужебный канон~---~также нуждается в толкованиях и комментариях. Перед нами стоит задача составления литургической экзегезы, отвечающей интеллектуальному уровню и духовным запросам современных верующих.
    
    Славянское православное богослужение~---~это бесценное сокровище мистического гносиса, огромный потенциал духовных сил и энергии, который мы должны сохранить не только для себя, но и для будущих поколений.
    
    Велико значение церковнославянского языка для православного богослужения и православного пастыря. Но церковнославянский язык имеет очень важное значение и для православного хистианина. На этом языке совершается богослужение в нашей Православной Русской Церкви и написаны наши священные и богослужебные книги. По своему возвышенному характеру, по своей силе и звучности церковнославянский язык является наиболее совершенным средством для выражения религиозных настроений православного русского человека. Высшие стремления души и глубокие чувствования, отрешенные от земного и направленные к небесному, чистому и вечному, получают наиболее соответствующее выражение в этом языке, далеком от всего обычного, житейского. Но для этого необходимо, чтобы все читаемое в храме Божием достигало своей цели, т.е. доходило бы до сознания души христианина, научая ее истинному благочестию и указывая ей путь спасения. А потому нужно, чтобы те, которые употребляют язык Церкви~---~чтецы, певцы церковно- и священнослужители~---~в совершенстве понимали то, что произносят их уста. Кроме того, без полного знания языка Матери Церкви нельзя дать ответа вопрошающим о недоуменных словах и выражениях церковного языка. А таких слов и выражений в церковнославянском языке достаточно.
    
    Итак, все, что читается и поется в храмах Божиих, должно быть истолковано людьми, стоящими на высоте своего призвания, а такими людьми, прежде всего, должны быть православные пастыри. Только тогда у православного русского человека укрепится сознательное и вместе с тем благоговейное отношение к богослужению нашей Православной Церкви, и само его мировоззрение обогатится достаточным запасом религиозно-церковных понятий
    
    Славянский язык назван именно славянским потому, что на нем говорили наши предки славяне.
    
    Славяне, обитавшие первоначально около Карпат, долгое время были непросвещенным языческим народом. Очень часто им случалось соприкасаться с культурными народами Византийской и Римской империй. Мало-помалу славяне стали подчиняться влиянию этих уже просвещенных христианством народов. От греков и римлян к славянам проникло и быстро у них распространилось христианство. Но, слушая богослужения на чуждых им греческом или латинском языках, славяне не могли себе вполне усвоить истины нового для них учения, и потому многие из них были христианами лишь по имени, а по существу сохранили прежнюю грубость нравов и держались старых языческих суеверий. Так было до IX века.
    
    В половине IX века святые братья Кирилл (в миру Константин) и Мефодий, родом греки, задумали облегчить славянам понимание христианского богослужения и перевели для них богослужебные книги с греческого языка на славянский. Так как в то время у славян не было еще письменности, то святые братья сами составили славянскую азбуку, взяв за образец греческую азбуку и дополнив ее недостающими буквами. Таким образом, богослужебные книги были переведены ими на славянский язык, а сами они стали апостолами~---~просветителями славян.
    
    Святые братья Кирилл (827---869 гг.) и Мефодий (\dag 885 г.) были детьми знатного вельможи и родились в Солуни (Фессалониках)~---~главном городе Македонии, страны, населенной по преимуществу славянами. Старший брат, Мефодий, после домашнего воспитания занимал сначала военные, а потом административные должности, но впоследствии принял монашество, поселившись на горе Олимп. Младший, Кирилл, отличался блестящими способностями к учению, обучался словесным, философским и математическим наукам. Его ожидали почести в свете, но он не прельстился этим, а принял сан священника и место библиотекаря в библиотеке при храме св. Софии. Впоследствии св. Кирилл удалился к брату на Олимп. Предполагают, что в это время жизни на Олимпе святые братья и начали переводить богослужебные книги на славянский язык.
    
    Около 862 года, через посредничество Византийского императора Михаила III, святые братья были приглашены Моравским князем Ростиславом в Моравию для проповеди христианства. Здесь начинается главная деятельность свв. Кирилла и Мефодия. В княжествах Моравском и Панионском (где ныне Венгрия и Чехия) христианство было уже проповедано среди тамошних славян немецким, латинским духовенством. Но латинское духовенство совершало богослужение на чуждом славянам латинском языке. А так как латинский язык был непонятен славянам, то, конечно, проповедь немецкого духовенства оказалась безуспешной.
    
    Святые братья, начав учить славян Христовой вере, дали им священные книги на славянском языке. На славянский язык были переведены и богослужебные книги. На славянском языке стали они совершеть богослужение, а для обучения славян письменности стали заводить и школы. Христианство между славянами начало быстро распространяться.
    
    Немецкое духовенство, видя успехи святых братьев в деле распространения ими христианства среди славян, из зависти подало на них жалобу в Рим. Римский папа Николай I вызвал к себе святых братьев на суд (в то время не было еще разделения Церкви, все были православные, и только что начинались несогласия между Римом и Византией). Святые братья отправились в Рим, но уже не застали папу Николая I в живых, а преемник его Андриан очень благосклонно отнесся к свв. Кириллу и Мефодию. Кирилл представил папе Евангелие на славянском языке, и папа положил славянский перевод в соборе св. Петра, где святые братья даже отслужили торжественную литургию на славянском языке. Св. Кирилл в Риме заболел и скончался. Умирая, он убеждал св. Мефодия не покидать дела проповеди и распространения веры среди славян. Папа Андриан рукоположил Мефодия в епископа Моравии, отпустил его с честью и дал грамоту, в которой признал богослужение на славянском языке целесообразным.
    
    По возвращении св. Мефодия к Моравийским славянам враги не оставили его в покое. Они восстановили против него немецкого императора Людовика и добились того, что Мефодий был заточен в темницу, где и пробыл в заточении два с половиной года. Когда ему возвращена была свобода, то его вновь оклеветали перед папой. Пришлось Мефодию опять отправляться в Рим и защищать себя. Оправданный папой, св. Мефодий продолжал свое великое дело.
    
    Когда св. Мефодий скончался, то на учеников его было воздвигнуто гонение. Они, изгнанные из Моравии, ушли в другие славянские земли, главным образом, в Болгарию. Болгария заботливо хранила труды святых просветителей и потом, когда в конце X века, при св. князе Владимире, Русь приняла христианство, списки этих книг передала нашему отечеству. В те древние времена славянские наречия очень мало отличались друг от друга, и книги одного славянского племени были совершенно пригодны для богослужения и чтения другого.
    
    О высоком значении церковнославянского языка убедительно говорит Соколов Г.А. в своем <<Слове в защиту церковнославянского языка>> (Астрахань, 1910).
    
    <<Славянское слово имеет определенную силу как фактор религиозно-нравственного развития, вот уже 1000 лет просвещающего Русь Православную. Его воспитательное значение сохраняется и до сих пор, ибо те же чувства, то же настроение господствует среди чад Русской Православной Церкви и непосредственная сила религиозного наставления, выраженная в формах славянской речи, особенно чувствуется ими>>.
    
    Содержание церковных книг неотделимо от формы. Церковнославянский язык является в данном случае идеальной формой. Отделить форму от идеи~---~значило бы ослабить ее, лишить ее всей полноты ее силы и значения. Сбросить эту форму без ущерба для содержания невозможно.
    
    Слово церковное слилось с жизнью Православной России.
    
    Церковнославянский язык~---~это драгоценный сосуд, созданный для высокого содержания, вполне ему по основным своим свойствам соответствующий. Ведь не напрасно историки пришли к выводу, что труд переводов Мефодия и Кирилла, <<их великое дело было пережито нами в собственной истории: в русской мысли, в русском чувстве, в русском подвиге>>.
    
    Насколько вера искреннее и глубже, настолько острее и потребность в форме именно славянской речи, в простом и величавом ее течении, в ее дивной методичности и музыкальности.
    
    Сила мысли и сила чувства находятся в связи с силой церковнославянской речи.
    
    Быть в церкви, возноситься душою к Богу, поучаться стало не только потребностью, но и влечением, свойством души, природой.
    
    Славянское слово исполнено веры и высшей духовной правды.
    
    Церковнославянская речь не только своим духом, исполненным непосредственной веры и умиления, но даже видом своих письмен влечет в горний мир возвышенных созерцаний и настроений, христианского подвига и нравственного совершенства.
    
    Церковнославянский язык как бы впитал в себя те свойства, какие принадлежат самому тексту, и самим своим видом и особенностями способствует восприятию текста.
    
    История и жизнь убеждают нас, что славянский язык~---~это сокровище, которое нужно хранить, которым нужно дорожить.
    
    Благодать Божия поставила славянский язык в один ряд с тремя превосходными языками: греческим, латинским и еврейским.
    
    Некоторая оторванность языка литургического от разговорной речи, от обыденной жизни,~---~составляет достоинство молитвенной речи. Иметь для религиозных целей язык понятный и в то же время торжественный, далекий от того, в котором встречаются вульгарности (а в наше время вульгарность~---~уже принятый тон, потому что речь изобилует грубостью и грязью), и даже от слов и оборотов обыденного языка~---~это общечеловеческая потребность.
    
    Как первые детские впечатления кладут самые глубокие основы нравственной и душевной жизни человека и остаются на всю жизнь милы и дороги, так и первоначальные священные и богослужебные тексты навсегда остаются дороги и священны для народа, т.к. от их впечатлений и влияний возникли первые начала самой живой, именно религиозной стороны духовной жизни народа.
    
    Славянский язык необходим для обнаружения религиозных стремлений.
    
    Славянский язык~---~язык Церкви и религиозной письменности.
    
    Славянский язык~---~лучшее средство, чтобы привести к крепкой вере и жизни по вере.
    
    До принятия христианства на Руси не было устойчивого языческого мировоззрения, твердо выработанной языческой космогонии.
    
    Расшатать и разорвать связь между религиозными и нравственными стремлениями и славянским языком~---~не значило бы поколебать те основы духовных устремлений, которые были заложены еще 1000 лет тому назад.
    
    Учение веры и нравственности отлилось в формы славянской речи.
    
    Славянский язык принес неоценимые религиозно-просветительные блага русскому народу. Славянский язык дал те привычки, которые у народа русского стали второй природой. В чем особенности этой привычки? Своей искренностью, непосредственностью, глубоко проникнутая верой в Бога и Его промыслительные действия, речь на этом языке самим складом своим, самим выбором слов и выражений, тем духом, который, ее проникает, производит теперь это уже привычное влияние на душу верующего. Далее, язык церковнославянский, уже много веков имея единственной задачей выражать Божественное учение и повествования о домостроительстве Божием в направлении жизни людей, так сроднился с тем, для выражения чего он служит, что воспитанный на нем народ особенно восприимчив именно к тому, что он выражает, и в самом его звуковом составе, в движении речи ощущает этот высший смысл, слышит голос Бога, сам к Нему на этом языке обращается и веками усвоил то убеждение, что славянский язык есть орудие, средство общения на нем с Богом.
    
    Там, где у нас на Руси для целей религиозных вместо славянского языка употребляют русский, там уже нет Православия. Есть у нас секты, где на собраниях, имеющих характер богослужения, читают и поют по-русски. Но кто же их назовет православными? И вот те именно причины, которые отделили их от Православия и сделали их по мировоззрению и религиозным понятиям иными~---~они-то привели их к предпочтению русского языка языку славянскому. Когда под влиянием новых учений и критического отношения к вере отцов изменилось настроение, прекратилась восприимчивость к живым и непосредственным воздействиям славянской речи, тогда православное богослужение и обряды потеряли для них свое значение. Дальнейшим результатом всего этого явилось стремление проникнутый духом православной веры язык славянский заменить языком живой, разговорной речи, более удовлетворяющим требованиям сознания. Но, несомненно, он уже не то говорит воображению и чувству. Когда явилось стремление к замене для религиозных нужд славянской речи речью русской, там, очевидно, уже произошли духовные перемены, ведущие к измене Православию. Склонность к тому или иному языку есть только симптом и результат этих внутренних и глубоких перемен в вере и мировоззрении. Лишь перемены в верованиях ведут за собой измену языку прежних верований.
    
    Православные, пока они православные, не могут желать этой перемены уже по складу своей духовной природы; и чем тверже, искреннее и глубже вера их, тем дальше они от этих желаний. Они непосредственно чувствуют, что перемена языка была бы связана с переменой в самом духе православной религии. В славянской речи они слышат молитвенный голос своих отцов и дедов и сливаются с ними в единстве многовековой истории, в единстве молитвы всей России, всех славян.
    
    Те, кто желает перевода богослужебных текстов на русский язык, желают этого не в силу своих собственных религиозных потребностей. Они почти все равнодушны и к богослужению, и к языку Церкви. Они желают только в силу теоретических соображений о большей доступности текстов богослужебного языка, о более сознательном отношении к читаемому и т.д. Но те, кто живут религиозной жизнью, кто нуждается в Церкви и службах ее, те именно и не желают перевода, да и не могут желать, не изменившись в своих отношениях к религии и к тем чувствам, которые вызываются славянским языком.
    
    Заменить один язык другим~---~значит изменить свойства и особенности содержания, придавши ему новый характер, несоответствующий тому, какой вложили духоносные переводчики в славянские тексты; это значило бы принизить его содержание, приблизить к мирской жизни, а нередко придать и характер вульгарности.
    
    Всякое слово православного на славянском языке знакомо, близко, сродни всякому человеку, воспитанному в Церкви.
    
    Церковнославянский язык далек от утилитарности, он возбуждает чувство великого, возвышенного и вековечного. Он не колеблется и не изменяется~---~сообразно с тем, как неизменна и устойчива вера православная.
    
    Перевести на частный язык, каким является русский, Божественные истины~---~значило бы поставить их в общий поток жизни, внося в перевод перемены в языке, вызванные нуждами жизни, т.е. обусловленные социальными движениями, изменениями в искусствах, науке, торговле и вообще во всех сферах деятельности, обусловливающих развитие языка. Но как вера устойчива и неизменна в своем существе, так должен быть устойчив и язык религии, не меняясь с каждым поколением, согласно движению общественной жизни. Язык общеупотребительный в житейских делах с каждым поколением уже не тот, и одно столетие жизни вносит в него значительные перемены. Неужели и язык Церкви с каждым поколением должен подновляться и изменяться, внося новые и новые перемены вопреки неизменяемости Кафолической Церкви?
    
    Когда на Русь была перенесена церковнославянская письменность, то на этом языке совершалось богослужение, читались священные книги, писались жития русских святых, летописи, писались проповеди и другие религиозного и нравственно-назидательного содержания творения. И только акты государственные, договоры и другие официальные бумаги писались по-русски.
    
    В то время между языками теми существовало значительное сходство. Языки влияли один на другой вплоть до времени Петра I, когда возмогли возникать сомнения, считать ли главный отдел литературы (летописи, сказания, хождения и т.д.) высшим стилем русской речи или низшим стилем речи церковнославянской.
    
    Церковнославянский язык дает созерцание высоких религиозных идей и образов, влияет на сердце, возбуждает нравственные чувства, влечет к улучшению жизни на началах любви, самопожертвования и служения своим близким. Он имеет свои особенные задачи. Не принижаясь, не ослабевая, религиозная вера не может заковываться в формы, свойственные знанию.
    
    Церковнославянский язык достоин сохранения и как лучшая форма выражения великой идеи христианства в том виде, в каком оно сложилось в Православной Церкови и осуществилось в душе русского народа.
    
    Мысль о богослужении на русском языке родилась у людей, стоящих вдали от Православия, равнодушных к Церкви, не переживавших на себе влияния славянской речи.
    
    Славянский язык мы защищаем не по праву его давности, а лишь потому, что он имеет благотворное значение и свои преимущества перед русской речью.
    
    Различие во взглядах на славянский язык в конце концов связывается с различием мировоззрения и отношения к религии. Со стороны отрицателей славянского языка сказывается и отрицательное отношение к религии. Не является ли отрицание славянского языка в богослужении посягательством на самую религию, а не только на формы ее выражения?
    
    Нерасположенные к славянскому языку относятся не сочувственно и к религии.
    
    \section*{Краткий учебник церковнославянского языка\\\emph{предисловие}}
    \label{sec:brief}
%    \addcontentsline{toc}{section}{\nameref{sec:brief}}
    \phantomsection\addcontentsline{toc}{section}{Краткий учебник церковнославянского языка\\\emph{предисловие}}
    
    Составитель сего учебника доброй памяти emph{профессор Анатолий Васильевич Ушков}.
    
    Анатолий Васильевич Ушков родился 7 августа 1894 года в г. Самаре в семье сотрудника Самарской Духовной Консистории. По окончании четырех классов Самарской Духовной семинарии, он в 1912 году успешно сдал вступительные экзамены в Казанский университет и был зачислен на физико-математический факультет. После университета, в 1916 году, А. Ушков был мобилизован на военную службу; проходил обучение в Киевском артиллерийском училище. С 1918 по 1945 г. он преподавал математику и физику в средних и специальных учебных заведениях в городах Самаре, Красноярске и Москве. Заведовал учебной частью и был руководителем методических совещаний по вопросам преподавания физики и математики. С 1939 по 1943 г. заочно учился в Московском государственном педагогическом институте на факультете русского языка и литературы, закончив который, некоторое время, наряду с преподаванием математики, вел курс литературы в средних учебных заведениях г. Москвы.
    
    В 1945 г. он осуществил свою давнишнюю мечту~---~послужить Святой Церкви и поступил в Московскую Духовную Академию. По окончании академии, в 1949 г., со званием кандидата богословия, полученным за сочинение <<Душа и ее бессмертие по христианскому учению>>, он преподавал катехизис, а затем церковнославянский язык в духовной семинарии. В 1964 г., по прочтении двух пробных лекций <<Древний мир перед пришествием Христа Спасителя>> и <<Основные положения формальной логики>>, А.В. Ушкову было присвоено звание доцента и поручено чтение лекций по логике в академии. Им составлен ряд учебных пособий для духовных школ: <<Краткая пасхалия>>, <<Логика в курсе академического образования>>, <<Краткая пасхалия в общедоступном изложении>>, <<Астрономический справочник>>, <<Краткое описание солнечной системы>>, <<Календарный счет времени>>, <<Счет и мера с древнейших времен до наших дней>>. Анатолием Васильевичем был подготовлен учебный курс церковнославянского языка, за который Совет академии присвоил ему 15 декабря 1969 г. степень магистра богословия и утвердил в должности профессора.
    
    Анатолий Васильевич отличался большой скромностью, чутким, внимательным отношением к людям; поражала его необычайная работоспособность, величайшее трудолюбие. Жажда проповеди слова Божия была его духовной потребностью. Он произнес 65 проповедей в праздничные и воскресные дни в академическом храме, в них~---~яркое выражение его глубоких христианских чувств и безграничной, сердечной любви к Богу и людям. В начале 1971--1972 учебного года Анатолий Васильевич по болезни вынужден был оставить работу в академии и перейти на пенсию.
    
    Скончался А.В. Ушков 14 января 1972 г., на 78-м году жизни, приобщившись Святых Христовых Тайн. Похоронен на кладбище в Сергиевом Посаде.
    
    \medskip
    Подготовил к печати доцент-иерей В. Москвич.
    
    \chapter*{1-й класс}
    \label{ch:firstgrade}
    \addcontentsline{toc}{chapter}{\nameref{ch:firstgrade}}
        \section*{Введение}
        \label{sec:intro}
%        \phantomsection\addcontentsline{toc}{section}{\nameref{sec:intro}}
        \phantomsection\addcontentsline{toc}{section}{Введение}
                \subsubsection[Значение церковнославянского языка]{Значение церковнославянского языка для православного богослужения и православного пастыря}
                
    Церковнославянский язык имеет очень важное значение для православного христианина. На этом языке совершается богослужение в нашей Православной Русской Церкви и написаны наши священные и богослужебные книги. По своему возвышенному характеру, по своей силе и звучности церковнославянский язык является наиболее совершенным средством для выражения религиозных настроений православного русского человека. Высшие стремления духа и глубокие чувствования, отрешенные от земного и направленные к небесному, чистому и вечному, получают наиболее соответствующее выражение в этом языке, далеком от обычного, житейского.
    
    Но для этого необходимо, чтобы все читаемое в храме Божием достигало своей цели, т.е. доходило бы до сознания души христианина, научая ее истинному благочестию и указывая ей путь спасения. А поэтому нужно, чтобы те, которые употребляют язык Церкви~---~чтецы, певцы, церковно- и священнослужители~---~в совершенстве понимали то, что произносят их уста. Да кроме того, без полного изучения языка матери-Церкви нельзя дать ответа вопрошающим о недоуменных словах и выражениях церковного языка. А таких слов и выражений в церковнославянском языке достаточно. Взять хотя бы такие слова: \emph{абие, амо, аще, выну, вкупе, ей, ктому, нань, леть, паче, паки, рамо, сице, тацы, убо, уне, яко}; или такие выражения: \emph{Змий сей, его же создал еси ругахуся ему; Чермнует бо ся небо; И бяху выну в церкви; Обыде нас последняя бездна; Любовию же, Дево, песни ткати спротяженно сложенныя неудобно есть; Противодышущу росоносному Духу, со огнем сущу пояху} и многие другие.
    
    Итак, все что читается и поется в храмах Божиих, должно быть истолковано людьми, стоящими на высоте своего призвания, а такими людьми, прежде всего, должны быть православные пастыри. Только тогда у православного русского человека укрепится сознательное и вместе с тем благоговейное отношение к богослужению нашей Православной Церкви и самое его мировоззрение обогатится достаточным запасом религиозно-церковных понятий.
    
                \subsubsection{Возникновение письменности у славян}
                
    Славянский язык назван именно \emph{славянским} потому, что на нем говорили наши предки славяне.
    
    Славяне, обитавшие первоначально около Карпат, долгое время были непросвещенным языческим народом. Очень часто им случалось соприкасаться с культурными народами Византийской и Римской империй. Мало-помалу славяне стали подчиняться влиянию этих уже просвещенных христианских народов. От греков и римлян к славянам проникло и быстро у них распространилось христианство. Но, слушая богослужение на чуждом им греческом или латинском языке, славяне не могли себе вполне усвоить истины нового для них учения, и потому многие из них были христианами лишь по имени, а по существу сохраняли прежнюю грубость нравов и держались старых языческих суеверий. Так было до половины IX века.
    
    В Половине IX века святые братья Кирилл (в миру Константин) и Мефодий, родом греки, задумали облегчить славянам понимание христианского богослужения~---~и перевели для них богослужебные книги с греческого языка на славянский. Так как у славян в то время не было еще письменности, то святые братья сами составили славянскую азбуку, взяв за образец греческую азбуку и дополнив ее недостающими буквами. Таким образом, богослужебные книги были переведены ими на славянский язык, а сами они стали апостолами-просветителями славян.
    
                \subsubsection{Деятельность святых братьев Кирилла и Мефодия}

    Святые братья Кирилл (827---869 гг.) и Мефодий (\dag 885 г.) были детьми знатного вельможи и родились в Солуни (Фессалониках)~---~главном городе Македонии, страны, населенной по преимуществу славянами. Старший брат, Мефодий, после домашнего воспитания занимал сначала военные, а потом административные должности, но впоследствии принял монашество, поселившись на горе Олимп. Младший, Кирилл, отличался блестящими способностями к учению, обучался словесным, философским и математическим наукам. Его ожидали почести в свете, но он не прельстился этим, а принял сан священника и место библиотекаря в библиотеке при храме св. Софии. Впоследствии св. Кирилл удалился к брату на Олимп. Предполагают, что в это время жизни на Олимпе святые братья и начали переводить богослужебные книги на славянский язык.

    Около 862 года, через посредничество Византийского императора Михаила III, святые братья были приглашены Моравским князем Ростиславом в Моравию для проповеди христианства. Здесь начинается главная деятельность свв. Кирилла и Мефодия. В княжествах Моравском и Панионском (где ныне Венгрия и Чехия) христианство было уже проповедано среди тамошних славян немецким, латинским духовенством. Но латинское духовенство совершало богослужение на чуждом славянам латинском языке. А так как латинский язык был непонятен славянам, то, конечно, проповедь немецкого духовенства оказалась безуспешной.

    Святые братья, начав учить славян Христовой вере, дали им священные книги на славянском языке. На славянский язык были переведены и богослужебные книги. На славянском языке стали они совершать богослужение, а для обучения славян письменности стали заводить и школы. Христианство между славянами начало быстро распространяться.

    Немецкое духовенство, видя успехи святых братьев в деле распространения ими христианства среди славян, из зависти подало на них жалобу в Рим. Римский папа Николай I вызвал к себе святых братьев на суд (в то время не было еще разделения Церкви, все были православные, и только что начинались несогласия между Римом и Византией). Святые братья отправились в Рим, но уже не застали папу Николая I в живых, а преемник его Андриан очень благосклонно отнесся к свв. Кириллу и Мефодию. Кирилл представил папе Евангелие на славянском языке, и папа положил славянский перевод в соборе св. Петра, где святые братья даже отслужили торжественную литургию на славянском языке. Св. Кирилл в Риме заболел и скончался. Умирая, он убеждал св. Мефодия не покидать дела проповеди и распространения веры среди славян. Папа Андриан рукоположил Мефодия в епископа Моравии, отпустил его с честью и дал грамоту, в которой признал богослужение на славянском языке целесообразным.

    По возвращении св. Мефодия к Моравийским славянам враги не оставили его в покое. Они восстановили против него немецкого императора Людовика и добились того, что Мефодий был заточен в темницу, где и пробыл в заточении два с половиной года. Когда ему возвращена была свобода, то его вновь оклеветали перед папой. Пришлось Мефодию опять отправляться в Рим и защищать себя. Оправданный папой, св. Мефодий продолжал свое великое дело.

    Когда св. Мефодий скончался, то на учеников его было воздвигнуто гонение. Они, изгнанные из Моравии, ушли в другие славянские земли, главным образом, в Болгарию. Болгария заботливо хранила труды святых просветителей и потом, когда в конце X века, при св. князе Владимире, Русь приняла христианство, списки этих книг передала нашему отечеству. В те древние времена славянские наречия очень мало отличались друг от друга, и книги одного славянского племени были совершенно пригодны для богослужения и чтения другого.


        \section{Церковнославянское буквоначертание}
                \subsubsection{Церковнославянская азбука}

    Устная наша речь состоит из членораздельных \textbf{звуков}.

    Изображения звуков речи условными знаками в письме или в печати называются \textbf{буквами}.

    Собрание всех букв языка, расположенных в общепринятом порядке, называется \textbf{азбукой} (от названия двух первых славянских букв <<аз>> и <<буки>>). Церковнославянская азбука состоит из 39 букв.


    \begin{center}
        \renewcommand*{\arraystretch}{1.4}    \begin{longtable}{|c|c|c|c|}
            \caption*{Буквы церковнославянской азбуки}\\
            \hline
            
            Прописная буква
            & Малая буква
            & Название
            & Произношение
            \\
            
            \hline
            \endfirsthead
            
            \multicolumn{4}{l}
            {
%                \tablename\ \thetable\ -- 
                \footnotesize\textit{Начало на предыдущей странице}
            }
            \\
            
            \hline
            
            Прописная буква
            & Малая буква
            & Название
            & Произношение
            \\
                
            \hline
            \endhead
            \hline
            
            \multicolumn{4}{r}{\footnotesize\textit{Продолжение на следующей странице}}
            \\
            
            \endfoot
            \hline
            \endlastfoot
            
            {\slv{А}}     & {\slv{а}}       & аз      & а\\\hln
            {\slv{Б}}     & {\slv{б}}       & буки    & б\\\hln
            {\slv{В}}     & {\slv{в}}       & веди    & в\\\hln
            {\slv{Г}}     & {\slv{г}}       & глаголь & г\\\hln
            {\slv{Д}}     & {\slv{д}}       & добро   & д\\\hln
            {\slv{Е}}     & {\slv{є, е}}    & есть    & е\\\hln
            {\slv{Ж}}     & {\slv{ж}}       & живете  & ж\\\hln
            {\slv{Ѕ}}     & {\slv{ѕ}}       & зело    & з\\\hln
            {\slv{З}}     & {\slv{з}}       & земля   & з\\\hln
            {\slv{И}}     & {\slv{и}}       & иже     & и\\\hln
            {\slv{І}}     & {\slv{ї}}       & и       & и\\\hln
            {\slv{К}}     & {\slv{к}}       & како    & к\\\hln
            {\slv{Л}}     & {\slv{л}}       & люди    & л\\\hln
            {\slv{М}}     & {\slv{м}}       & мыслете & м\\\hln
            {\slv{Н}}     & {\slv{н}}       & наш     & н\\\hln
            {\slv{Ѻ, Ѽ}} & {\slv{ѻ, о}}    & он      & о\\\hln
            {\slv{П}}     & {\slv{п}}       & покой   & п\\\hln
            {\slv{Р}}     & {\slv{р}}       & рцы     & р\\\hln
            {\slv{С}}     & {\slv{с}}       & слово   & с\\\hln
            {\slv{Т}}     & {\slv{т}}       & твердо  & т\\\hln
            {\slv{Оу}}    & {\slv{ᲂу, ꙋ, у}} & ук      & у\\\hln
            {\slv{Ф}}     & {\slv{ф}}       & ферт    & ф\\\hln
            {\slv{Х}}     & {\slv{х}}       & хер     & х\\\hln
            {\slv{Ц}}     & {\slv{ц}}       & цы      & ц\\\hln
            {\slv{Ч}}     & {\slv{ч}}       & червь   & ч\\\hln
            {\slv{Ш}}     & {\slv{ш}}       & ша      & ш\\\hln
            {\slv{Щ}}     & {\slv{щ}}       & ща      & щ\\\hln
            {\slv{Ъ}}     & {\slv{ъ}}       & ер      & --\\\hln
            {\slv{Ы}}     & {\slv{ы}}       & ы       & ы\\\hln
            {\slv{Ь}}     & {\slv{ь}}       & ерь     & --\\\hln
            {\slv{Ѣ}}     & {\slv{ѣ}}       & ять     & е\\\hln
            {\slv{Ю}}     & {\slv{ю}}       & ю       & ю\\\hln
            {\slv{Ꙗ, Ѧ}} & {\slv{ꙗ, ѧ}}    & я       & я\\\hln
            {\slv{Ѡ}}     & {\slv{ѡ}}       & омега   & о\\\hln
            {\slv{Ѿ}}     & {\slv{ѿ}}       & от      & от\\\hln
            {\slv{Ѯ}}     & {\slv{ѯ}}       & кси     & кс\\\hln
            {\slv{Ѱ}}     & {\slv{ѱ}}       & пси     & пс\\\hln
            {\slv{Ѳ}}     & {\slv{ѳ}}       & фита    & ф\\\hln
            {\slv{Ѵ}}     & {\slv{ѵ}}       & ижица   & и, в\\\hln

        \end{longtable}
    \end{center}

                    \paragraph{Упражнение}
    
    \textbf{Славянские или церковные буквы, сходные с русскими}
    
    {\slv{а б в г д е ж з и к л м н о п р с т ф х ц ч ш щ ъ ы ь ю}}
    \medskip
    
    \begin{center}
        \begin{slv}
            Бо́гъ.~\textemdash~Спа́съ.~\textemdash~Сы́нъ.~\textemdash~Ма́-ти.~\textemdash~Не́-бо.~\textemdash~Ча̑-да.~\textemdash~Лю́-ди.~\textemdash~По-сты̀.~\textemdash~Ѻ҆-те́цъ.~\textemdash~Це́р-ковь.~\textemdash~Гос-по́дь.~\textemdash~Мо-ли̑т-вы.~\textemdash~Пра́зд-ни-цы.~\textemdash~Спа-си́-тель.~\textemdash~Бла-го-сло-ве́нъ Бо́гъ на́шъ.~\textemdash~Бо́гъ є҆-динъ є҆́сть.~\textemdash~Го́с-по-ди, бла-го-сло-вѝ.~\textemdash~Въ це́рк-вахъ бла-го-сло-ви́-те Бо́-га.~\textemdash~Гла́съ бы́сть съ не-бе-сѐ: Ты̀ є҆-сѝ Сы́нъ Мо́й воз-лю́б-лен-ный.~\textemdash~Бо́гъ на́шъ на не-бе-сѝ и на зе-млѝ.~\textemdash~Воз-лю́-би-ши Го́с-по-да Бо́-га тво-е-го̀.~\textemdash~Ще́дръ и ми́-лос-тивъ Гос-по́дь.~\textemdash~Бра́т-ство воз-лю-би́-те.~\textemdash~Чтѝ ѻ҆т-ца̀ тво-е-го̀ и ма́-терь тво-ю̀.~\textemdash~Сы́нъ не по-кор-ли́-вый в по-ги́-бель.
        \end{slv}
    \end{center}


                \subsubsection{Основные правила церковнославянского чтения и письма}
                
    При церковнославянском чтении нужно соблюдать следующие основные правила:
    
    1. Читать нужно внятно и произносить слова так, как они напечатаны в книге. Например, слово {\slv{є҆го̀}} нужно читать <<его>> (а не <<ево>>), {\slv{моли́тва}} (а не <<малитва>>), {\slv{єди́наго}} (а не <<единова>>).
    
    2. Звука <<ё>> в церковнославянском языке совсем нет, а потому букву {\slv{є}} или {\slv{е}} нужно всегда произносить как <<е>>, а не как <<ё>>. Например, {\slv{твоѐ}} (а не <<твоё>>), {\slv{моѐ}} (а не <<моё>>), {\slv{тве́рдый}} (а не <<твёрдый>>).
    
    3. В церковнославянских словах на одном из слогов ставится \textbf{ударение}, чтобы показать, что этот слог нужно усилить голосом или, как говорят, сделать на нем ударение. Ударение обозначается знаком~~{\slv{́}}, или~~{\slv{̀}}, или~~{\slv{̑}}, например: {\slv{влады́ко, хвала̀, сїѧ̑}}. Если же слово начинается с гласного звука, то над буквой этого звука ставится \textbf{придыхание}, которое изображается знаком~~{\slv{҆}}, например: {\slv{о҆де́жда. є҆сѝ}}.
    
    4. В церковнославянском языке начальное слово предложения (после точки) пишется с прописной (большой) буквы, например: {\slv{Въ нача́лѣ сотворѝ бг҃ъ не́бо и҆ зе́млю. Землѧ̀ же бѣ неви́дима и҆ неꙋстро́ена}} (Быт. 1, 1--2). Прописная буква пишется иногда в начале каждого стиха, даже если стих начинается после запятой или другого какого-нибудь знака помимо точки.
    
    5. Священные имена лиц, собственные имена, т.е. имена, названия городов, стран, гор, рек, морей и пр., если они стоят в середине или в конце предложение (т.е. не после точки), пишутся с малой буквы, например: {\slv{Рече́ же гдⷭ҇ь къ мѡѷсе́ю и҆ а҆арѡ́нꙋ въ землѝ є҆гѵ́петстѣй}} (Исх. 12, 1).


                \subsubsection{Особенности правописания и произношения некоторых букв в церковнославянском языке}

    Буква {\slv{Г}} (<<глаголь>>) перед {\slv{г, к, х}} произносится как звук <<и>>, например: {\slv{сѷгклі́тъ}} (<<синклит>>), {\slv{а҆сѷгкрі́тъ}} (<<асинкрит>>). Исключаются из этого правила слова: {\slv{а҆гге́й}} (<<Аггей>>) и {\slv{а҆́гге́лъ}} (<<аггел>>, в значении злого духа).
    
    Буквы {\slv{є, е}}. Буква {\slv{є}} (так называемое <<удлиненное есть>>) пишется всегда в начале слова; в середине же и на конце слова обычно пишется буква {\slv{е}} (<<простое есть>>), например: {\slv{є҆́зеро, є҆ле́нь, се́рдце}}. Но иногда буква {\slv{є}} пишется и в середине слова и даже в конце слова. Эти случаи будут рассмотрены впоследствии.
    
    Буква {\slv{ѕ}} (<<зело>>) произносится как русское <<з>> и пишется только в семи коренных словах: {\slv{ѕвѣзда̀, ѕвѣ́рь, ѕе́лїе, ѕла́къ, ѕмі́й, ѕло̀, ѕѣлѡ̀}}, а также в производных от этих слов: {\slv{ѕлы́й, ѡ҆ѕло́бити, ѕла́чный, ѕвѣ́рскїй}} и др.
    
    Буква {\slv{ї}} (<<и>>) пишется пред гласными, например: {\slv{воскресе́нїе}}. Но в словах иностранных (еврейских и греческих) эта буква пишется и перед согласными, например: {\slv{і҆́долъ, вїно̀}}. Слово <<мир>> в значении <<вселенная>> в славянском языке пишется через {\slv{ї}}, например: {\slv{ненави́дитъ ва́съ мі́ръ}} (вселенная) (Ин. 15, 19), а это же самое слово, но в значении <<покой>>, <<тишина>> пишется через {\slv{и}} (<<иже>>), например: {\slv{ми́ръ мо́й даю̀ ва́мъ}} (т.е. даю благодатное умиротворение, покой) (Ин. 14, 27).
    
    Буквы {\slv{ѻ, о, ѽ}}. Буква {\slv{ѻ}} (<<он польск\'ое>>) пишется в начале слова, например: {\slv{ѻ҆́гнь, ѻ҆́ко}}. Исключение составляет слово {\slv{і҆ѻрда́нъ}}. Буква {\slv{о}} (<<он простое>>) пишется в середине и в конце слова, например: {\slv{со́нмъ, не́бо}}. Буква {\slv{ѽ}} употребляется только в качестве междометия.
    
    Буквы {\slv{ᲂу, ꙋ, у}} (<<ук>>) изображают звук <<у>>, причем буква {\slv{ᲂу}} пишется в начале слова, а {\slv{ꙋ}}~---~в середине и на конце слова, например: {\slv{ᲂу҆че́нїе, ᲂу҆́бѡ, дꙋша̀, рꙋкꙋ̀}}. Буква {\slv{у}} (без буквы {\slv{о}}) в словах совсем не пишется, а употребляется в значении цифры, о чем будет сказано далее.
    
    Буква {\slv{ѣ}} (<<ять>>) первоначально изображала собою звук как бы средний между нашими звуками <<е>> и <<я>> (почему и названа <<ять>>). Но с утратой этого звука в произношении буква {\slv{ѣ}} в последствии в церковнославянском языке стала означать звук <<е>>.
    
    Буквы {\slv{ъ}} (<<ер>>) и {\slv{ь}} (<<ерь>>). Буква {\slv{ъ}} первоначально обозначала краткий звук <<о>>, а буква {\slv{ь}}~---~краткий звук <<е>>. Эти звуки были потому краткие, что произносились более кратко и слабо, чем другие согласные звуки. В некоторых случаях они были особенно слабы и стали совсем исчезать из произношения. В письме сохранилась постановка буквы {\slv{ъ}} в конце слов, которая теперь показывает твердое произношение конечного согласного звука, например: {\slv{хра́мъ, пра́ведникъ}}. Иногда в этом случае букву {\slv{ъ}} заменяет особый знак <<ерок>>, или <<ерик>>, который обозначается~~{\slv{̾}} и ставится над последней буквой слова. Как правило, ерок ставится в односложных предлогах, оканчивающихся на согласный звук, как-то: {\slv{ѡ҆б̾, и҆з̾, без̾, над̾, под̾}} и др. Над этими предлогами (кроме предлога {\slv{бли́з̾}}) также не ставят ударений. Когда эти предлоги соединяются в качестве приставок с другим словом, начинающимся с гласного звука, в одно слово, то ерок, смотря по произношению, удерживается (в качестве разделительного знака), например: {\slv{ѡ҆б̾ѧтїѧ, и҆з̾ѧсни́ти}}, или опускается, например: {\slv{и҆зы́де}}. Предлоги, состоящие только из согласных звуков: {\slv{въ, къ, съ}}, при раздельном их написании всегда пишутся с буквой {\slv{ъ}}, а не с ероком. Буква {\slv{ь}}, исчезнувшая как гласный звук из произношения, осталась лишь знаком мягкости предыдущего согласного звука, а потому сохранила свою постановку как в середине, так и на конце слов.

                \subsubsection{Правописание и произношение букв, перешедших в церковнославянский язык из греческого языка}

    Буквы, перешедшие с греческого языка в церковнославянский язык, следующие: {\slv{ѡ, ѿ, ѯ, ѱ, ѳ, ѵ}}.

    Буква {\slv{ѡ}} (<<омега>>) пишется в предлогах {\slv{ѡ҆}} и {\slv{ѡ҆б̾}} и в словах, начинающихся этими предлогами (в качестве приставок), например: {\slv{ѡ҆чище́нїе, ѡ҆бличи́ти}}. Если слово, начинающееся буквой {\slv{ѡ}}, соединяется с приставкой в одно слово, то в полученном составном слове буква {\slv{ѡ}} почти всегда удерживается, например: {\slv{преѡбраже́нїе}}. Буква {\slv{ѡ}} пишется также в словах, заимствованных из других языков (греческого и еврейского), например: {\slv{ѡ҆са́нна, і҆кѡ́на, канѡ́нъ, і҆ѡа́ннъ}} и др. Наконец, буква {\slv{ѡ}} пишется в некоторых наречиях, оканчивающихся звуком <<о>>, союзах, а также и в других случаях, о чем будет сказано впоследствии.

    Буква {\slv{ѿ}} (<<от>>), как уже показывает и ее название, представляет соединение двух букв {\slv{ѡ}} и {\slv{т}}, что соответствует и ее начертанию. Эта составная буква употребляется в качестве предлога {\slv{ѿ}}, а также в словах, начинающихся этим предлогом (в качестве приставки), например: {\slv{ѿра́да, ѿмще́нїе}}.

    Буквы {\slv{ѯ}} (<<кси>>), {\slv{ѱ}} (<<пси>>), {\slv{ѳ}} (<<фита>>) употребляются только в словах, заимствованных с греческого языка, например: {\slv{а҆леѯа́ндръ, ѱалти́рь, ѳома̀}}.

    Буква {\slv{ѵ}} (<<ижица>>) пишется в словах, взятых с греческого языка, причем произносится или как звук <<и>>, или как звук <<в>>. Если буква {\slv{ѵ}} стоит в начале слова или после согласной буквы, то она произносится как наш гласный звук <<и>>. В этом случае над нею ставятся две наклонные черточки, например: {\slv{сѷно́дъ}}. Эти черточки заменяются или придыханием, когда с буквы {\slv{ѵ}} начинается слово, или ударением, когда оно падает на слог, содержащий {\slv{ѵ}}, например: {\slv{ѵ҆ссѡ́пъ, тѵ́хѡнъ}}. После гласного звука буква {\slv{ѵ}} произносится как согласный звук <<в>>, например: {\slv{є҆́ѵа, лаѵре́нтїй, леѵі́тъ}}. Исключение составляют некоторые собственные имена, например: {\slv{мѡѷсе́й}}, а также составные слова, например: {\slv{триѷпоста́сный}}. Слово {\slv{мѵ́ро}} пишется через {\slv{ѵ}} для различения его в косвенных падежах от слов {\slv{мі́ръ}} и {\slv{ми́ръ}}.


                \subsubsection{Правописание букв {\large\slv{ꙗ}} и {\large\slv{ѧ}}}

    Буква {\slv{ꙗ}} представляет собою соединение двух звуков: {\slv{ї}} и {\slv{а}}. Прежде буква {\slv{ї}} обозначала тот звук, который изображается в русском языке иногда буквой <<й>> (<<и краткое>>), например в словах <<мой>>, <<лейка>>. Этот звук (а не буква) называется <<йотом>>, а гласные, перед которыми он слышится называются <<йотированными>>. При соединении звука <<йота>> со звуком {\slv{а}} получалось созвучие {\slv{їа}} (<<йа>>), которое и стали изображать буквой {\slv{ꙗ}}, а произносить: <<я>>. Обычно йотированная буква {\slv{ꙗ}} пишется только в начале слова, например: {\slv{ꙗ҆́рость, ꙗ҆́кѡ}}.
    
    Буква {\slv{ѧ}} первоначально изображала носовой звук, произносимый как <<ем>> или <<ен>>. Впоследствии этот носовой звук, называемый в древности <<малым юсом>>, перешел в звук <<я>>, но только не йотированный, т.е. без соединения с ним звука <<йота>>. Но нейотированный звук <<я>> может произноситься только после согласного звука, а потому буква {\slv{ѧ}} пишется в середине и конце слова, например: {\slv{мѧте́жъ, стезѧ̀}}.
    
    Но в слове <<язык>>, которое в славянском языке имеет много значений, как-то: член тела, орган речи, наречие или говор, народ, племя, то условились, что если слово <<язык>> обозначает народ или племя, писать букву {\slv{ꙗ}}, например: {\slv{ꙗ҆зы́къ самарі́йскїй}} (т.е. народ самарийский). Если же слово <<язык>> обозначает член тела, орган речи, наречие (говор), то условились писать букву{\slv{ѧ}}, например: {\slv{Прильпнѝ ѧ҆зы́къ мо́й горта́ни моемꙋ̀}} (Пс. 136, 6); {\slv{церко́вно-славѧ́нскїй ѧ҆зы́къ}}.
    
                    \paragraph{Упражнение}

    \textbf{Славянские или церковные буквы, не сходные с русскими}
    \medskip
    
    \begin{slv}
        Оу҆́хо.~\textemdash~Оу҆́тро.~\textemdash~Оу҆ста̀.~\textemdash~Оу҆́тренѧ.~\textemdash~Оу҆спе́ние.~\textemdash~Оу҆пова́нїе.~\textemdash~Дꙋ́хъ.~\textemdash~Дꙋша̀.~\textemdash~Слꙋ́жба.~\textemdash~І҆исꙋ́съ.~\textemdash~Ꙗ҆́сли.~\textemdash~Ꙗ҆зы́къ.~\textemdash~И҆́мѧ.~\textemdash~Сѣ́ьѧ.~\textemdash~Марі́ѧ.~\textemdash~Клѧ́тва.~\textemdash~Вече́рнѧ.~\textemdash~Лїтꙋргі́ѧ.~\textemdash~І҆ѡ́на.~\textemdash~І҆ѡа́ннъ.~\textemdash~І҆ѡ́сифъ.~\textemdash~І҆ѻрда́нъ.~\textemdash~І҆ѡанні́кїй.
        
        Го́споди, поми́лꙋй.~\textemdash~Приклонѝ ᲂу҆́хо Твоѐ мнѣ̀ и҆ ᲂуслы́ши глаго́лы моѧ̀.~\textemdash~Ѕаꙋ́тра ᲂу҆слы́ши гла́съ мо́й, Царю̀ мо́й и҆ Бо́же мо́й.~\textemdash~Оу҆слы́ши, Го́споди, пра́вдꙋ мою̀, вонмѝ моле́нїю моемꙋ̀.~\textemdash~Да возра́дꙋетсѧ дꙋша̀ моѧ̀ ѡ҆ Го́сподѣ.~\textemdash~Положѝ, Го́споди, хране́нїе ᲂу҆стѡ́мъ мои̑мъ.~\textemdash~Сла́ва Тебѣ̀ Бо́гꙋ Благода́телю на́шемꙋ во вѣ́ки вѣко́въ.~\textemdash~Сла́ва Ѻ҆тцꙋ̀ и҆ Сы́нꙋ и҆ Свѧто́мꙋ Дꙋ́хꙋ.~\textemdash~Бо́же, ми́лостивъ бꙋ́ди мнѣ̀ грѣ́шникꙋ.
        
        Ꙗ҆вѝ нам, Го́споди, ми́лость Твою̀ и҆ спасе́нїе Твоѐ да́ждь на́мъ.~\textemdash~И҆сповѣ́дайтесѧ Го́споди, ꙗ҆́кѡ благ: ꙗ҆́кѡ въ вѣ́къ ми́лость є҆гѡ́.~\textemdash~Да бꙋ́детъ во́лѧ твоѧ̀, ꙗ҆́кѡ на небесѝ и҆ на землѝ.~\textemdash~А҆́нгели, ᲂу҆спѣ́нїе Пречи́стыѧ ви́дѣвше, ᲂу҆диви́шасѧ.
        
        Благословѝ, дꙋшѐ моѧ̀, Го́спода.~\textemdash~Го́споди, воззва́хъ къ Тебѣ̀, ᲂу҆слы́ши мѧ̀.~\textemdash~Поми́лꙋй мѧ̀, Бо́же, поми́лꙋй мѧ̀.~\textemdash~Пресвѧта́ѧ Тро́ице, поми́лꙋй на́съ.~\textemdash~Свѧ́тъ, Свѧ́тъ, Свѧ́тъ є҆сѝ Бо́же, Богоро́дицею поми́лꙋй на́съ.
        
        Го́споди, ѡ҆чи́сти грѣхѝ на́шѧ.~\textemdash~Моли́ Бо́га ѡ҆ на́съ, свѧти́телю ѻ҆́тче Нїко́лае, ꙗ҆́кѡ мы̀ ᲂу҆се́рднѡ къ Тебѣ̀ прибега́емъ.~\textemdash~Ѽ, Всепѣ́таѧ Ма́ти, ро́ждшаѧ всѣ́хъ свѧты́хъ Свѧтѣ́йшее Сло́во!~\textemdash~Ѽ, пресла́дкїй и҆ всеще́дрый І҆исꙋ́се, прїимѝ ны́нѣ ма́лое моле́нїе сїѐ на́ше, ꙗ҆́коже прїѧ́лъ є҆сѝ вдови́цы два̀ ле́пта.
    \end{slv}

                \subsubsection{Надстрочные знаки}

    В церковнославянском языке употребляются следующие надстрочные знаки: \textbf{ударения}, \textbf{придыхание} и \textbf{титла}.

    \medskip
    \underline{1. Ударения}    
    \medskip

    Ударения ставятся над гласными буквами для усиления слога. Первый слог слова, если он начинается с гласного звука, и последний слог слова, если он оканчивается гласным звуком, называются \emph{открытыми} слогами; все же остальные слоги слова называются \emph{закрытыми} слогами. Ударение ставится также над односложными словами.
    
    Ударения в церковнославянском языке бывают: \textbf{острое} ({\slv{~́}}), \textbf{тупое} ({\slv{~̀}}) и \textbf{облегченное} ({\slv{~̑}}).
    
    Острое ударение может стоять на любом слоге слова, кроме последнего открытого слога, например: {\slv{си́ла, мо́ре, любо́вь, горта́нь, є҆диномꙋ́дрие}}.
    
    Тупое ударение ставится всегда только на последнем открытом слоге, например: {\slv{рꙋка̀, млеко̀, сотворѝ, рцы̀}}.
    
    Облегченное ударение может стоять на всех слогах слова, кроме первого открытого слога. Это ударение вообще употребляется для различения созвучных грамматических форм, о чем будет сказано в надлежащем месте.
    
    Если за словом с ударением на последнем открытом слоге следует одно из односложных слов, оканчивающихся на гласный звук, т.е. состоящих из одного открытого слога, а именно: {\slv{бо̀, мѧ̀, мѝ, тѧ̀, тѝ, сѧ̀, сѝ, жѐ}} (но не предлоги {\slv{во, на, по, при}} и проч.), то ударение над этими односложными словами опускается, а предшествующее им слово меняет тупое ударение на острое. Например: {\slv{Сꙋди́ ми, бж҃е}} (Пс. 42, 1); {\slv{Бѣ́ же во дни̑ во це́ркви ᲂу҆чѧ̀}} (Лк. 21, 37).
    
    Односложные предлоги и частицы {\slv{не, ни, но, на, по, да, же, при, пре, бо, во, ко, со, за}} и др. не имеют над собою никакого ударения. Исключение из этих правил составляет слово {\slv{сѐ}} (всю), которое сохраняет над собою ударение, и иногда частица {\slv{ни}}, когда она употребляется в значении русского слова <<нет>>, например: {\slv{Бꙋ́ди же сло́во ва́ше: є҆́й, є҆́й: нѝ, нѝ}} (Мф. 5, 37).
    
    \medskip
    \underline{2. Придыхание}
    \medskip
    
    Когда слово начинается с открытого слога, то над гласной буквой этого слога ставится придыхание ({\slv{~҆}}), например: {\slv{є҆ле́й, и҆мѣ́ти, ᲂу҆чени́къ, ꙗ҆зы́къ}}.
    
    В словах, где ударение падает на начальный открытый слог, то над гласной буквой ставится придыхание вместе с острым ударением, например: {\slv{ѻ҆́ко, а҆́ще, и҆́мѧ, є҆́зеро}}. Такое соединение придыхания с острым ударением называется словом {\slv{и҆́со}}.
    
    Союзы {\slv{и҆}} и {\slv{а҆}}, а также предлоги {\slv{ᲂу҆}} и {\slv{ѡ҆}} пишутся только с одним придыханием (без ударения).
    
    Иногда бывает соединение придыхания с тупым ударением. Такое соединение называется словом {\slv{а҆́построфь}}.
    
    Апострофь ставится на начальном открытом ударном слоге для различения смысла некоторых созвучных слов. Например: {\slv{и҆}} (союз <<и>>) и {\slv{и҆̀}} (он, его), {\slv{ᲂу҆}} (предлог <<у>>) и {\slv{ᲂу҆̀}} (еще), {\slv{и҆́же}} (который) и {\slv{и҆̀же}} (которые) и т.п.

    Эти апострофорные слова сохраняют свое ударение и после слов, оканчивающихся ударным открытым слогом. Например: {\slv{И̑ а҆́бїе воззва̀ ѧ҆̀}} (Мк. 1, 20).
    
    \medskip
    \underline{3. Титла}
    \medskip
    
    В церковнославянском языке некоторые слова пишутся сокращенно: с пропуском одной или нескольких букв, а иногда и целых слогов в середине слова. Сокращение это обозначается особыми знаками \textbf{титлами}.
    
    Различают простое титло и буквенные титла.
    
    \emph{Простое} титло изображается знаком {\slv{~҃}}, который ставится вверху слова над той буквой, перед которой или после которой имеется пропуск одной или нескольких букв. Например: {\slv{бг҃ъ}} (Бог), {\slv{ѻ҆ц҃ъ}} (Отец), {\slv{бл҃года́ть}} (благодать).
    
    \emph{Буквенное} титло представляет собою дугообразный титловой знак {\slv{~҇}} с какой-либо подписной буквой, именно с той, какая буква пропущена или находится среди пропущенных (если их несколько). Буквенные титла следующие:

    \bigskip\autorows{c}{1}{l}{
        {{\slv{~ⷢ҇}}~---~глаголь-титло: {\slv{є҆ѵⷢ҇лїе}} (Евангелие),},
        {{\slv{~ⷪ҇}}~---~он-титло: {\slv{прⷪ҇ро́къ}} (пророк),},
        {{\slv{~ⷬ҇}}~---~рцы-титло: {\slv{и҆мⷬ҇къ}} (имярек),},
        {{\slv{~ⷭ҇}}~---~слово-титло: {\slv{гдⷭ҇ь}} (Господь).}
    }

    К буквенным титлам еще относится добро-титло, которое изображается без титлового знака {\slv{~҇}}, а просто ставится буква <<добро>> вверху слова. Например: {\slv{бцⷣа}} (Богородица), {\slv{влⷣко}} (Владыко).
    
    Общее правило постановки титл таково. Под титло подводится слово тогда, когда оно относится к лицам и предметам священным, почитаемым; наоборот, те же самые слова, но относящиеся к обыкновенным именам, предметам и понятиям, в особенности же к языческим, пишутся без титла. Например:

    \bigskip\autorows{c}{1}{l}{
        {{\slv{бг҃ъ}} (истинный Бог) и {\slv{бо́гъ}} (бог языческий)},
        {{\slv{і҆и҃съ}} (имя Господа) и {\slv{і҆исꙋ́съ}} (имя человека)},
        {{\slv{мр҃і́а}} (Пресвятая Дева Мария) и {\slv{марі́а}} (имя женщины)},
        {{\slv{мт҃и}} (Матерь Господа) и {\slv{ма́ти}} (земная мать)},
        {{\slv{ѻ҆ц҃ъ}} (Отец Небесный) и {\slv{ѻ҆те́цъ}} (земной отец)},
        {{\slv{дв҃а}} (Дева Пресвятая) и {\slv{дѣ́ва}} (девица)},
        {{\slv{дх҃ъ}} (Дух Святый) и {\slv{дꙋ́хъ}} (дыхание)},
        {{\slv{а҆́гг҃лъ}} (Ангел) и {\slv{а҆́ггелъ}} (аггел, или злой дух)}
    }

    Изредка под титлами пишутся и обыкновенные слова, имеющие в отдельных случаях священное значение или священную память:
    
    {\slv{мцⷭ҇ъ}} (месяц), {\slv{чл҃вѣ́къ}} (человек), {\slv{нн҃ѣ}} (ныне), {\slv{срⷣце}} (сердце), {\slv{сл҃нце}} (солнце), {\slv{дн҃ь}} (день).
    
    Есть сокращения на особые случаи:
    
    1. В венце образа Христа Спасителя встречаются надписи: {\slv{о҆ ѡ҆́н}}. Это греческое слово ({$o~\omega\nu$}), написанное славянскими буквами и означающее {\slv{сы́й}}, т.е. сущий.
    
    2. В венце образа распятого на кресте Христа Спасителя имеются такие буквы: {\slv{І҆.Н҃.Ц҃.І҆.}}, или {\slv{І҆Н҃Ц҃І҆}}. Это означает: <<Иисус Назарянин Царь Иудейский>>.
    
    3. На образе Божией Матери встречается надпись {\slv{мр҃ ѳꙋ҃}}, т.е. по-гречески ({$\mu\eta\tau\eta\rho~\tau o \upsilon~\theta\varepsilon o \upsilon$})~---~Матерь Бога.
        
                \subsubsection{Буквы как числовые знаки (цифры)}

    Буквы церковнославянской азбуки служат одновременно и цифрами. Чтобы показать, что данная буква является цифрой, над буквой ставится простое титло {\slv{~҃}}.

    \begin{center}
        \renewcommand*{\arraystretch}{1.4}    \begin{longtable}{|c|c|c|c|}
%            \caption*{Числа}\\

            \hline
            
            \emph{Единицы}
            & \emph{Десятки}
            & \emph{Сотни}
            & \emph{Тысячи}
            \\
            
            \scriptsize\makecell{изображаются\\следующими\\буквами}
            & \scriptsize\makecell{изображаются\\следующими\\буквами}
            & \scriptsize\makecell{изображаются\\следующими\\буквами}
            & \scriptsize\makecell{изображаются теми же буквами,\\какими единицы,\\десятки и сотни,\\но с прибавлением впереди\\буквы значка}
            \\
        
            &
            &
            & \huge{\slv{҂}}
            \\
            
            \hline
            \endfirsthead
            
            \multicolumn{4}{l}%
            {
                %                \tablename\ \thetable\ -- 
                \footnotesize\textit{Начало на предыдущей странице}
            }
            \\
            
%            \hline
%            Единицы & Десятки & Сотни & Тысячи\\ % This is the "next" header.
            \hline
            \endhead
            \hline
            
            \multicolumn{4}{r}{\footnotesize\textit{Продолжение на следующей странице}}
            \\
            
            \endfoot
            \hline
            \endlastfoot
            
            {\slv{а҃}} (1) & {\slv{і҃}} (10) & {\slv{р҃}} (100) & {\slv{҂а҃}} (1000) \\\hln
            {\slv{в҃}} (2) & {\slv{к҃}} (20) & {\slv{с҃}} (200) & {\slv{҂в҃}} (2000) \\\hln
            {\slv{г҃}} (3) & {\slv{л҃}} (30) & {\slv{т҃}} (300) & {\slv{҂г҃}} (3000) \\\hln
            {\slv{д҃}} (4) & {\slv{м҃}} (40) & {\slv{у҃}} (400) & {\slv{҂і҃}} (10000) \\\hln
            {\slv{є҃}} (5) & {\slv{н҃}} (50) & {\slv{ф҃}} (500) & {\slv{҂к҃}} (20000) \\\hln
            {\slv{ѕ҃}} (6) & {\slv{ѯ҃}} (60) & {\slv{х҃}} (600) & {\slv{҂л҃}} (30000) \\\hln
            {\slv{з҃}} (7) & {\slv{ѻ҃}} (70) & {\slv{ѱ҃}} (700) & {\slv{҂р҃}} (100000) \\\hln
            {\slv{и҃}} (8) & {\slv{п҃}} (80) & {\slv{ѿ}} (800) & {\slv{҂с҃}} (200000) \\\hln
            {\slv{ѳ҃}} (9) & {\slv{ч҃}} (90) & {\slv{ц҃}} (900) & {\slv{҂҂а҃}} (1000000) \\\hln
            
        \end{longtable}
    \end{center}

    Десять тысяч ({\slv{҂і҃}}) произносится по-славянски {\slv{тьма̀}}, двадцать тысяч ({\slv{҂к҃}})~---~{\slv{двѣ̀ тьмы̀}}, тридцать тысяч ({\slv{҂л҃}})~---~{\slv{трѝ тьмы̀}} и т.д. Сто тысяч ({\slv{҂р҃}}) называется по-славянски {\slv{легеѡ́нъ}}, двести тысяч ({\slv{҂с҃}})~---~{\slv{два̀ легеѡ́на}} и т.д.
    
    Числа от 10 до 20 составляются следующим образом: сначала ставят единицы и справа приписывают десятки, причем титло ставится над единицами. Вот нумерация чисел второго десятка: {\slv{а҃і}} (11), {\slv{в҃і}} (12), {\slv{г҃і}} (13), {\slv{д҃і}} (14), {\slv{є҃і}} (15), {\slv{ѕ҃і}} (16), {\slv{з҃і}} (17), {\slv{и҃і}} (18), {\slv{ѳ҃і}} (19).
    
    Числа от 20 до 100 составляются порядком, т.е. сперва пишут десятки, а справа приписывают единицы; титло ставится над десятками: {\slv{к҃а}} (21), {\slv{л҃в}} (32), {\slv{м҃д}} (44), {\slv{н҃ѕ}} (56), {\slv{ѯ҃г}} (63), {\slv{ѻ҃ѳ}} (79).
    
    Таким же обычным порядком составляются числа и от 100 до 1000 и далее; титло во всех этих числах ставится всегда на предпоследней цифре (второй цифре справа): {\slv{сл҃є}} (235), {\slv{тн҃з}} (357), {\slv{фп҃ѕ}} (586), {\slv{хч҃д}} (694), {\slv{ѱм҃ф}} (749), {\slv{ѿо҃ⷢⷢг}} (873), {\slv{цк҃в}} (922), {\slv{҂ацн҃з}} (1957), {\slv{҂аѿп҃ѕ}} (1886), {\slv{҂аѿч҃д}} (1894), {\slv{҂зтч҃а}} (7391), {\slv{҂кцг҃і}} (20913), {\slv{҂ѻєу҃ѕ}} (75406).
    \bigskip

    \textbf{Упражнения в чтении употребительнейших слов под титлами}

                    \paragraph{Упражнение}
    
    \begin{slv}
        Бг҃ъ (Бо́гъ), Бг҃а (Бо́га), Бг҃ꙋ (Бо́гꙋ), Бг҃ови (Бо́гови), Бг҃омъ (Бо́гомъ), Бж҃е (Бо́же), ѡ҆ Бз҃ѣ (Бо́зѣ).
        
        Бг҃ъ любы̀ є҆́сть.~\textemdash~Бг҃ъ застꙋ́прикъ мо́й.~\textemdash~По́йте Бг҃ꙋ на́шемꙋ, по́йте!~\textemdash~А҆́ще кто̀ рече́тъ, ꙗ҆́кѡ люблю̀ Бг҃а, а҆ бра́та своего̀ ненави́дитъ, ло́жъ є҆́сть.~\textemdash~Бж҃е, ѡ҆чи́сти мѧ̀ грѣ́шнаго.~\textemdash~Ѡ҆ Бз҃ѣ спасе́нїе моѐ и҆ сла́ва моѧ̀.~\textemdash~Бж҃їѧ никто́же вѣ́сть.
        
        Гдⷭ҇ь (Госпо́дь), Гдⷭ҇а (Го́спода), Гдⷭ҇ꙋ (Го́сподꙋ), Гдⷭ҇ви (Го́сподеви), Гдⷭ҇и (Го́споди), ѡ҆ Гдⷭ҇ѣ (Го́споде).
        
        Гдⷭ҇ь мнѣ̀ помо́щникъ.~\textemdash~Не во́змеши и҆́мене Гдⷭ҇а Бг҃а твоегѡ̀ всꙋ́е.~\textemdash~Гдⷭ҇ꙋ помо́лимсѧ! Гдⷭ҇и, поми́лꙋй.~\textemdash~Гдⷭ҇и, не введѝ менѐ въ напа́сть.~\textemdash~Да возра́дꙋетсѧ дꙋша̀ моѧ̀ ѡ҆ Гдⷭ҇ѣ.
        
        І҆и҃съ (І҆исꙋ̀съ) Хрⷭ҇то́съ (Хрїсто́съ), І҆и҃са Хрⷭ҇та̀, І҆и҃сомъ Хрⷭ҇то́мъ, ѡ҆ І҆и҃сѣ Хрⷭ҇тѣ̀, Сп҃съ (Спа́съ), Сп҃си́тель (Спаси́тель), Хрⷭ҇то́въ (Христо́въ).
        
        Воскр҃съ І҆и҃съ ѿ гро́ба, ꙗ҆́кѡже проречѐ.~\textemdash~Вѣ́рꙋю во Є҆ди́наго Гдⷭ҇а І҆и҃са Хрⷭ҇та̀.~\textemdash~І҆и҃се, Сы́не Бж҃їй, поми́лꙋй мѧ̀.~\textemdash~І҆и҃се, Сп҃се мо́й, спасѝ мѧ҃!~\textemdash~Возра́довасѧ дх҃ъ мо́й ѡ҆ Бз҃ѣ Сп҃сѣ мое́мъ.~\textemdash~Дрꙋ́къ дрꙋ́га тѧгѡты̀ носи́те, и҆ та́кѡ и҆спо́лните зако́нъ Хрⷭ҇то́въ.
    \end{slv}

                    \paragraph{Упражнение}
    
    \begin{slv}
        Ст҃ъ (Свѧ́тъ), Ст҃ы́й (Свѧты́й), Ст҃а́гѡ, Ст҃о́мꙋ, Ст҃и, Ст҃а́ѧ, Ст҃ы̑мъ.
        
        Ст҃ъ Гдⷭ҇ь Бг҃ъ на́шъ.~\textemdash~Ст҃ъ, Ст҃ъ, Ст҃ъ Гдⷭ҇ь Вседержи́тель.~\textemdash~Поклони́мсѧ Ст҃о́мꙋ Гдⷭ҇ꙋ І҆и҃сꙋ, Є҆ди́номꙋ безгрѣ́шномꙋ.~\textemdash~Ст҃ы́й Бж҃е, Ст҃ы́й Крѣ́пкїй, Ст҃ы́й Безсме́ртный поми́лꙋй на́съ!~\textemdash~Свѧ́ти бꙋ́дите, ꙗ҆́кѡ А҆́зъ ст҃ъ (є҆́смь) Гдⷭ҇ь Бг҃ъ ва́шъ.~\textemdash~Ст҃а̑ѧ ст҃ы̑мъ.
        
        Ѻ҆цъ (Ѻ҆те́цъ), Сн҃ъ (Сы́нъ), Дх҃ъ (Дꙋ́хъ), Дш҃а̀ (Дꙋша̀), Трⷪ҇ца (Тро́ица), Нн҃ѣ (Ны́нѣ), Прⷭ҇нѡ (При́снѡ).
        
        Ѻ҆ц҃ъ, Сн҃ъ и҆ Ст҃ы́й Дх҃ъ Трⷪ҇ца Ст҃а́ѧ.~\textemdash~Оу҆пова́нїе моѐ Ѻ҆ц҃ъ, прибѣ́жище моѐ Сн҃ъ, покро́въ мо́й Дх҃ъ Ст҃ы́й, Трⷪ҇ца Ст҃а́ѧ, сла́ва Тебѣ̀.~\textemdash~Во и҆́мѧ Ѻ҆ц҃а̀ и҆ Сн҃а и҆ Ст҃а́гѡ Дх҃а. А҆ми́нь.~\textemdash~Сла́ва Ѻ҆ц҃ꙋ̀ и҆ Сн҃ꙋ и҆ Ст҃о́мꙋ Дх҃ꙋ, и҆ нн҃ѣ и҆ прⷭ҇нѡ, и҆ во вѣ́ки вѣкѡ́въ. А҆ми́нь.~\textemdash~Вѣ́рꙋю и҆ въ Дх҃а Ст҃а́го, Гдⷭ҇а животворѧ́щаго, и҆́же со Ѻ҆ц҃емъ и҆ Сн҃омъ спокланѧ́ема и҆ ссла́вима.~\textemdash~Прест҃а́ѧ Трⷪ҇це, Бж҃е на́шъ, сла́ва Тебѣ̀.~\textemdash~Ѿ сна̀ воста́въ, бл҃годарю̀ Тѧ, Ст҃а́ѧ Трⷪ҇це.~\textemdash~Да возра́дꙋетсѧ дш҃а҃ моѧ̀ ѡ҆ Гдⷭ҇ѣ, ѡ҆блече́ бо мѧ̀ въ ри́зꙋ спасе́нїѧ.
        
        Бцⷣа (Богоро́дица), Дв҃а (Дѣ́ва), Мр҃і́ѧ (Марі́ѧ), Мт҃и (Ма́ти), Прⷭ҇нодв҃а (Приснодѣ́ва).
        
        Мл҃твами Бцⷣы, млⷭ҇тиве, ѡ҆чи́сти мно́жество согрѣше́нїй на́шихъ.~\textemdash~Пребл҃гослове́нна є҆сѝ Бцⷣе Дв҃о.~\textemdash~Ра́дꙋйсѧ, бл҃года́тнаѧ Мр҃і́е.~\textemdash~Спасѝ ѿ бѣ́дъ рабы̑ твоѧ̑, Бцⷣе.~\textemdash~Пресла́внаѧ Прⷭ҇нодв҃о, Мт҃и Хрⷭ҇та̀ Бг҃а, принесѝ на́шꙋ мл҃твꙋ Сн҃ꙋ Твоемꙋ̀ и҆ Бг҃ꙋ на́шемꙋ, да спасе́тъ Тобо́ю дш҃ы на́ша.
    \end{slv}
    \medskip

                    \paragraph{Упражнение}

    \begin{slv}
        А҆́гг҃лъ (А҆́нгелъ), Влⷣко (Влады́ко), Влⷣчца (Влады́чица), Бл҃гъ (Бла́гъ), Цр҃ь (Ца́рь), Цр҃ица (Цари́ца).
        
        Ѡ҆полчи́тсѧ А҆́гг҃лъ Гдⷭ҇ень ѡ҆́крестъ боѧ́щихсѧ є҆гѡ̀, и҆ и҆зба́витъ и҆̀хъ.~\textemdash~Ст҃і́и а҆рха́г҃гли, моли́те Бг҃а ѡ҆ на́съ.~\textemdash~А҆́гг҃лѡмъ свои̑мъ заповѣ́сть ѡ҆ тебѣ̀, сохрани́ти тѧ̀ во всѣ́хъ пꙋте́хъ твои́хъ.~\textemdash~По́йте Бг҃ꙋ на́шемꙋ, по́йте Цр҃е́ви на́шемꙋ, по́йте: ꙗ҆́кѡ цр҃ь всеѧ̀ землѝ Бг҃ъ.~\textemdash~А҆́гг҃ле, храни́телю мо́й ст҃ы́й.~\textemdash~Бл҃гословѝ, Влⷣка, ст҃ы́й вхо́ди.~\textemdash~Прест҃а́ѧ Влⷣчце Бцⷣе, молѝ ѡ҆ на́съ грѣ́шныхъ.~\textemdash~Гдⷭ҇и и҆ Влⷣко живота́ моегѡ̀, дꙋ́хъ пра́здности, ᲂу҆ны́нїѧ, любонача́лїѧ, и҆ праздносло́вїѧ не да́ждь мѝ.~\textemdash~Бл҃гъ мнѣ̀ зако́нъ ᲂу҆́стъ Твои́хъ, па́че ты́сѧщъ зла́та и сребра̀.~\textemdash~Црⷭ҇тво Твоѐ црⷭ҇тво всѣ́хъ вѣкѡ́въ.
        
        Є҆ѵⷢ҇лі́стъ (Є҆ѵангелі́стъ), Є҆пⷭ҇кпъ (Є҆пі́скопъ), І҆и҃ль (І҆зра́иль), Крⷭ҇тъ (Кре́стъ), І҆ерⷭ҇ли́мъ (І҆ерꙋсали́мъ).
        
        Велича́емъ тѧ̀, а҆пⷭ҇ле Хрⷭ҇то́въ и҆ є҆ѵⷢ҇лі́сте І҆ѡа́нне Бг҃осло́ве.~\textemdash~Крⷭ҇тꙋ̀ Твоемꙋ̀ поклонѧ́емсѧ, Влⷣко!~\textemdash~Сла́ва, Гдⷭ҇и, крⷭ҇тꙋ̀ Твоемꙋ̀ чтⷭ҇но́мꙋ.~\textemdash~И̑сповѣ́дꙋю є҆ди́но кр҃ще́нїе во ѡ҆ставле́нїе грѣхѡ́въ.~\textemdash~Ѡ̑гради́ мѧ, Гдⷭ҇и, си́лою чтⷭ҇на́гѡ и҆ животворѧ́щагѡ Твоегѡ̀ крⷭ҇та̀, и҆ сохрани́ мѧ ѿ всѧ́кагѡ ѕла̀.
    \end{slv}
    \medskip

                    \paragraph{Упражнение}

    \begin{slv}
        Мрⷣость (Мꙋ̀дрость), Премⷣръ (Премꙋ́дръ), Млⷭ҇ть (Ми́лость), Мл҃тва (Моли́тва), Прⷭ҇то́лъ (Престо́лъ).
        
        И҆́стинꙋ возлюби́лъ є҆сѝ, безвѣ́стнаѧ и҆ та́йнаѧ премⷣрости Твоеѧ̀ ꙗ҆ви́лъ мѝ є҆сѝ.~\textemdash~Оу҆тверди́лъ є҆́сть Гдⷭ҇ь млⷭ҇ть Свою̀ на боѧ́щихсѧ Є҆гѡ́.~\textemdash~Прⷭ҇то́лъ Тво́й, Бж҃е, въ вѣ́къ вѣ́ка.
        
        Всѧ̑ нбⷭ҇ныѧ си̑лы ст҃ы́хъ а҆́гг҃лъ и҆ а҆рха̑гг҃лъ, моли́те ѡ҆ на́съ грѣ́шныхъ.~\textemdash~Пра́ведницы просвѣтѧ́тсѧ ꙗ҆́кѡ со́лнце, въ Ца́рствїи Ѻ҆ц҃а̀ и҆́хъ.~\textemdash~Сщ҃е́нницы Твоѝ ѡ҆блекꙋ̀тсѧ пра́вдою, и҆ пре́пбнїи Твоѝ возра́дꙋютсѧ.~\textemdash~Ржⷭ҇тво̀ Твоѐ, Хрⷭ҇те Бж҃е нашъ, возсїѧ̀ мі́рови свѣ́тъ ра́зꙋма.~\textemdash~Нб҃о прⷭ҇то́лъ Мо́й, землѧ̀ же подно́жїе но́гъ Мои́хъ.~\textemdash~Па́мѧть прⷣвныхъ съ похвала́ми: и҆́мѧ же нечести́выхъ ᲂу҆гаса́етъ.~\textemdash~И̑з̾ ᲂу҆́стъ младє́нецъ и҆ ссꙋ́щихъ соверши́лъ є҆сѝ хвалꙋ̀.~\textemdash~Совѣ́тъ Гдⷭ҇ень во вѣ́къ пребыва́етъ, помышлє҆нїѧ срⷣца Є҆гѡ́ въ ро́дъ и҆ ро́дъ.~\textemdash~Въ нача́лѣ сотворѝ Бг҃ъ не́бо и҆ зе́млю.~\textemdash~Ѻ҆́чи Гдⷭ҇ни на прⷣвныѧ, и҆ ᲂу҆́ши Є҆гѡ́ въ моли́твꙋ и҆̀хъ.~\textemdash~Млⷭ҇рдїѧ двє́ри ѿве́рзи на́мъ, Бл҃гослове́ннаѧ Бцⷣе!~\textemdash~Не клени́тесѧ ни нб҃ом, ни земле́ч, ни и҆но́ю ко́ею клѧ́твою.~\textemdash~Во Црⷭ҇твїи Твое́мъ помѧнѝ на́съ, Гдⷭ҇и, є҆гда̀ прїи́деши во Црⷭ҇твїи Твое́мъ.
    \end{slv}


        \section[Общие понятия о грамматических формах]{Общие понятия о грамматических формах церковнославянского языка}
                \subsubsection{Разделение звуков}

    \medskip
    Звуки церковнославянского языка, как и в русском языке, разделяются на \textbf{гласные} и \textbf{согласные}.
    \medskip
    
    Гласные звуки бывают:
    \medskip
    
    \emph{твердые}: {\slv{а, о, ꙋ, ы}};

    \emph{мягкие}: {\slv{ꙗ}} ({\slv{ѧ}}){\slv{, е, ю, и}} ({\slv{ї, ѵ}}){\slv{, ѣ}}.
    \medskip
    
    Согласные звуки делятся на:
    \medskip
    
    \emph{губные}: {\slv{б, в, м, п, ф}} ({\slv{ѳ}}){\slv{, ѵ}};
    
    \emph{гортанные}: {\slv{г, к, х}};
    
    \emph{зубные}: {\slv{д, з}} ({\slv{ѕ}}){\slv{, с, т, ц}}~---~эти звуки, кроме {\slv{д}} и {\slv{т}}, 
    
    называются еще свистящими;
    
    \emph{шипящие}: {\slv{ж, ч, ш, щ}};
    
    \emph{плавные}: {\slv{л, м, н, р}}; плавные звуки {\slv{м}} и {\slv{н}} 
    
    называются еще носовыми.
    \medskip

                \subsubsection{Чередование звуков}

    При изменении слов гласные и согласные звуки могут чередоваться ({\slv{берꙋ̀~\textemdash~соб{\Large и}ра́ю, сꙋ́хъ~\textemdash~сꙋ{\Large ш}и́ти}}). Особенно часто чередуются твердые согласные звуки с более мягкими. Это чередование происходит или посредством замены твердых согласных мягкими, или посредством вставок мягкого согласного перед твердым или после него. Такое чередование согласных носит название \emph{смягчение согласных звуков}.
    
    Вот основные случаи смягчения:

    \bigskip
    \underline{1. Смягчение гортанных звуков}
    \medskip
    
    Гортанные звуки {\slv{г, к, х}} перед гласными звуками {\slv{е}} и {\slv{и}} смягчаются посредством замены их шипящими, а именно:

    \begin{flushleft}
        \renewcommand*{\arraystretch}{1.2}
        \begin{tabular}[l]{clcl}

            ~~~~~
            & {\slv{г~\textemdash~ж:}}
            & ~~~~~
            & {\slv{дрꙋ́гъ~\textemdash~дрꙋ́же~\textemdash~дрꙋжи́ти}}
            \\
            
            ~~~~~
            & {\slv{к~\textemdash~ч:}}
            & ~~~~~
            & {\slv{ре́къ~\textemdash~речѐ~\textemdash~рѣ́чи}}
            \\
            
            ~~~~~
            & {\slv{х~\textemdash~ш:}}
            & ~~~~~
            & {\slv{дꙋ́хъ~\textemdash~дꙋ́ше}}
            \\
            
        \end{tabular}
    \end{flushleft}

    Но эти же самые гортанные звуки, находясь перед падежными окончаниями {\slv{ѣ}} и {\slv{и}} ({\slv{ы}}), а также перед {\slv{и}} ({\slv{ы}}) в некоторых глагольных формах,~---~смягчаются несколько иначе, посредством замены их свистящими:

    \begin{flushleft}
        \renewcommand*{\arraystretch}{1.2}
        \begin{tabular}[l]{clcl}
            
            ~~~~~
            & {\slv{г~\textemdash~з:}}
            & ~~~~~
            & {\slv{дрꙋ́гъ~\textemdash~дрꙋ́зи~\textemdash~ѡ҆ дрꙋзѣ́хъ}}
            \\
            
            ~~~~~
            & {\slv{к~\textemdash~ц:}}
            & ~~~~~
            & {\slv{ре́къ~\textemdash~рцы̀}}
            \\
            
            ~~~~~
            & {\slv{х~\textemdash~с:}}
            & ~~~~~
            & {\slv{лꙋ́хъ~\textemdash~лꙋ́си~\textemdash~ѡ҆ лꙋ́сѣх}}
            \\
            
        \end{tabular}
    \end{flushleft}

    Итак, гортанные звуки могут смягчаться или шипящими (1-й закон смягчения), или свистящими (2-й закон смягчения.)
    
    \bigskip
    \underline{2. Смягчение зубных звуков}
    \medskip
    
    Зубные звуки смягчаются шипящими, а именно:

    \begin{flushleft}
        \renewcommand*{\arraystretch}{1.2}
        \begin{tabular}[l]{clcl}
            
            ~~~~~
            & {\slv{з~\textemdash~ж:}}
            & ~~~~~
            & {\slv{вѧза́ти~\textemdash~вѧжꙋ̀}}
            \\
            
            ~~~~~
            & {\slv{с~\textemdash~ш:}}
            & ~~~~~
            & {\slv{писа́ти~\textemdash~пишꙋ̀}}
            \\
            
            ~~~~~
            & {\slv{т~\textemdash~щ:}}
            & ~~~~~
            & {\slv{свѣ́тъ~\textemdash~свѣщꙋ̀~\textemdash~свѣща̀}}
            \\
            
            ~~~~~
            & {\slv{ц~\textemdash~ч:}}
            & ~~~~~
            & {\slv{ѻ҆те́цъ~\textemdash~ѻ҆те́чествїе}}
            \\
            
        \end{tabular}
    \end{flushleft}

    Зубной звук {\slv{д}} смягчается вставкой перед ним звука {\slv{ж}}:
    
    \begin{flushleft}
        \renewcommand*{\arraystretch}{1.2}
        \begin{tabular}[l]{cl}
            
            ~~~~~
            & {\slv{ходи́ти~\textemdash~хождꙋ̀~\textemdash~хожде́нїе}}
            \\
            
        \end{tabular}
    \end{flushleft}
    
    \medskip
    \underline{3. Смягчение губных звуков}
    \medskip

    Губные звуки {\slv{б, в, м, п}} перед некоторыми гласными смягчаются вставкой после губных плавного звука {\slv{л}}:
    
    \begin{flushleft}
        \renewcommand*{\arraystretch}{1.2}
        \begin{tabular}[l]{cl}
            
            ~~~~~ & {\slv{люби́ти~\textemdash~люблю̀}}\\
            ~~~~~ & {\slv{ꙗзви́ти~\textemdash~ꙗ҆звлю̀}}\\
            ~~~~~ & {\slv{ломи́ти~\textemdash~ломлю̀}}\\
            ~~~~~ & {\slv{спа́ти~\textemdash~сплю̀}}\\
            
        \end{tabular}
    \end{flushleft}

                \subsubsection{Состав слова}

    Слова образуются через сочетания отдельных звуков.
    
    Соединение гласного звука с одним или несколькими согласными составляет \textbf{слог}. Очевидно, что в слове столько слогов, сколько в нем гласных звуков, а поэтому слова могут быть \emph{односложные, двухсложные} и вообще \emph{многосложные}.
    
    Одни из звуков, из которых образуются слова, являются неизменными, другие же могут изменяться. Так, например, в словах: {\slv{да́ти, дарова́нїе, воздаѧ́нїе}} и т.д. Звуки {\slv{да}} во всех этих словах не изменяются, остальные же звуки: {\slv{-ти, -рованїе, воз-, -ѧнїе}}~---~представляют собою различные сочетания.
    
    Неизменяемые звуки в слове составляют корень слова. Существенным признаком корня слова служит его \emph{односложность}.
    
    Часть слова, стоящая перед корнем, называется \textbf{приставкой}, например: в слове {\slv{воздаѧ́нїе}} слог {\slv{воз-}}, стоящий перед корнем {\slv{-да-}}, есть приставка.
    
    Часть слова, стоящая после корня и указывающая на определенный падеж в именах или на лицо в глаголах, называется \textbf{окончанием}, например в словах {\slv{сы́нъ, сы́на, сы́нꙋ}} или {\slv{нестѝ, несꙋ́тъ}} звуки {\slv{-ъ, -а, -ꙋ, -ти, -ꙋтъ}} являются окончаниями.
    
    Иногда между корнем слова и окончанием вставляется один или несколько звуков, которые образуют вставочные слоги в слове. Такие вставочные слоги в слове называются \textbf{суффиксами}; так, например, в слове {\slv{дарова́нїе}} слоги {\slv{-ро-, -ва-, -нї-}} являются суффиксами.
    
    Окончание слова меняется в зависимости от падежей и чисел в именах или в зависимости от времени, лица, числа в глаголах. Остальные же части слова (корень, приставка, суффиксы) в этих случаях не меняются и в своей совокупности называются \textbf{основой} слова. Например, в слове {\slv{воздаѧ́нїе}} основой слова будет {\slv{воздаѧ́нї}}, а окончанием~---~{\slv{е}}.

                \subsubsection{Слова простые и сложные}

    Слова бывают простые и сложные.
    
    Если в состав слова входит только один корень, то такое слово называется \textbf{простым}. Например: {\slv{вѣ́ра, ѡ҆бходи́ти}}.
    
    Если же слово произошло от соединения двух или более простых слов, то такое слово называется \textbf{сложным}. Возьмем, например, два таких простых слова: {\slv{смире́нїе}} и {\slv{мꙋ̀дрость}}. Из этих двух слов мы можем составить одно сложное слово: {\slv{смиренномꙋ́дрїе}}. Ясно, что в сложном слове столько корней, сколько в состав этого слова входит простых слов. В приведенном сложном слове {\slv{смиренномꙋ́дрїе}} два корня: {\slv{-мир-}} и {\slv{-мꙋдр-}}.
    
    Простые слова соединяются в сложные посредством так называемой \textbf{соединительной гласной}. Такими соединительными гласными служат звуки {\slv{е}} или {\slv{о}}, например: {\slv{пꙋт{\Large е}ше́ствие, благ{\Large о}лѣ́пїе}}.

                \subsubsection{Славянское неполногласие}

    \textbf{Славянским неполногласием} называется наличие в словах некоторых звуковых сочетаний между согласными. Эти звуковые сочетания следующие: {\slv{ра, ла, ре, ле}}.
    
    Неполногласием эти звуковые сочетания названы в противоположность соответствующим звуковым сочетаниям: {\slv{оро, оло, ере, еле}}, которые называются \textbf{русским полногласием}.
    
    Вот несколько примеров славянского неполногласия:

    \bigskip\autorows{c}{3}{l}{
        {{\slv{гра́дъ}}~---~город},{{\slv{сребро̀}}~---~серебро},{{\slv{брада̀}}~---~борода},
        {{\slv{зла́то}}~---~золото},{{\slv{млеко̀}}~---~молоко},{{\slv{глава̀}}~---~голова},
        {{\slv{дре́во}}~---~дерево},{{\slv{кла́съ}}~---~колос},{{\slv{врата̀}}~---~ворота}
    }

                \subsubsection{Части речи и их грамматические формы}

    В церковнославянском языке, как и в русском, десять частей речи. Все части речи разделяются на изменяемые и неизменяемые.

    К \textbf{изменяемым частям речи} относятся имена существительные, прилагательные, числительные, местоимения и глаголы. Это слова самостоятельные, обозначающие предметы, признаки, количество, действия и т.д.

    \underline{Имя существительное} бывает \emph{трех родов}:

    \bigskip\autorows{c}{4}{l}{
        {мужского:}, {\slv{человѣ́къ}}, {\slv{ѻ҆лта́рь}}, {\slv{хра́мъ}},
        {женского:}, {\slv{жена̀}}, {\slv{це́рковь}}, {\slv{си́ла}},
        {среднего:}, {\slv{вре́мѧ}}, {\slv{село̀}}, {\slv{благоволе́нїе}}
    }

    \underline{Имя прилагательное} и некоторые имена числительные, а также некоторые местоимения в свою очередь сами могут изменяться по родам. Например:

    \bigskip\autorows{c}{4}{l}{
        {мужской род:}, {\slv{благі́й}}, {\slv{пѧ́тый}}, {\slv{ѻ҆́нъ}},
        {женский род:}, {\slv{блага́ѧ}}, {\slv{пѧ́таѧ}}, {\slv{ѻна̀}},
        {средний род:}, {\slv{благо́е}}, {\slv{пѧ́тое}}, {\slv{ѻ҆но̀}}
    }

    Эти четыре части речи изменяются также по числам и падежам.

    \emph{Чисел} в церковнославянском языке три: единственное, двойственное, множественное.
    
    \emph{Падежей} в церковнославянском языке семь: именительный, родительный, дательный, винительный, звательный, творительный, предложный.
    
    Звательный падеж служит обращением к лицу или предмету и во множественном числе почти всегда сходен с именительным (имена числительные и местоимения звательного падежа не имеют).
    
    В двойственном числе во всех именах (кроме некоторых числительных и местоимений) сходны между собою следующие падежи:
    
    \begin{flushleft}
        \renewcommand*{\arraystretch}{1.2}
        \begin{tabular}[l]{clcl}
            
            ~~~~~ & 1) & ~ & именительный, винительный, звательный\\
            ~~~~~ & 2) & ~ & родительный, предложный\\
            ~~~~~ & 3) & ~ & дательный, творительный\\
            
        \end{tabular}
    \end{flushleft}

    Изменение слов по падежам называется \emph{склонением}.
    
    Все четыре перечисленные части речи называются \emph{склоняемыми}.
    
    \underline{Глагол} изменяется по временам, лицам и числам.
    
    Действия или состояния предметов, выражаемые глаголами, совершаются \emph{во времени}. Они могут происходить в настоящем времени, прошедшем времени и будущем времени.
    
    Глагол имеет \emph{три лица}: 1-е, 2-е и 3-е.
    
    \emph{Чисел} церковнославянский глагол имеет три: единственное, двойственное, множественное.
    
    Двойственное число глагола имеет особенность, заключающуюся в том, что это число изменяется по родам: у женского и среднего рода окончания сходны, но различаются от окончаний мужского рода; кроме того, 2-е и 3-е лица в каждом роде одинаковы по окончаниям.
    
    Изменения глагола по временам, лицам и числам, в двойственном числе и по родам называется \emph{спряжением}.
    
    Глагол имеет еще особые формы, о которых будет сказано впоследствии.
    
    К \textbf{неизменным частям речи} относятся наречия, предлоги, союзы, частицы и междометия. Из них только наречия являются самостоятельными словами, а предлоги, союзы и частицы представляют собою так называемые \emph{служебные слова}, которые придают самостоятельным словам надлежащие взаимные соотношения или различные оттенки смысла. Что же касается междометий, то они занимают совершенно отдельное место, не являются сами по себе ни самостоятельными, ни служебными словами, а выражают только различные чувства и переживания.

                \subsubsection{Понятие о предложении}

    Мысль, выраженная словами, называется \textbf{предложением}.
    
    Слова, входящие в состав предложения, бывают связаны между собою известными сочетаниями. Возьмем, например, такое предложение:
    
    \medskip
    {\slv{Небеса̀ повѣ́даютъ сла́вꙋ бж҃їю}} (Пс. 18, 2)
    \medskip
    
    Здесь слово {\slv{небеса̀}} связано (или, как говорят, согласовано) со словом {\slv{повѣ́даютъ}}, а слово {\slv{сла́вꙋ}}~---~со словом {\slv{бж҃їю}}.
    
    Каждое слово в предложении (если это слово не является служебным словом или междометием) отвечает на какой-то вопрос; так, в приведенном предложении
    
    \begin{flushleft}
        \renewcommand*{\arraystretch}{1.2}
        \begin{tabular}[l]{cllcl}
            
            ~~~~~
            & слово
            & {\slv{небеса̀}}
            & отвечает на вопрос
            & \emph{что?} (имен. падеж)
            \\
            
            ~~~~~
            &
            & {\slv{повѣ́даютъ}}
            & --
            & \emph{что делают?}
            \\
            
            ~~~~~
            &
            & {\slv{сла́вꙋ}}
            & --
            & \emph{что?} (винит. падеж)
            \\
            
            ~~~~~
            &
            & {\slv{бж҃їю}}
            & --
            & \emph{чью?}
            \\
            
        \end{tabular}
    \end{flushleft}

    Самостоятельные слова, входящие в состав предложения и отвечающие на какой-нибудь вопрос, называются \textbf{членами предложения}.
    
    В каждом предложении говорится о ком- или о чем-либо.
    
    То, о чем говорится в предложении, называется \emph{подлежащим}. Подлежащее всегда отвечает на вопрос именительного падежа: {\large кто}? или {\large что}? В данном предложении слово {\slv{небеса́}} является подлежащим.
    
    То, что говорится о подлежащем, называется \emph{сказуемым}. Сказуемое отвечает на вопрос: {\large что делает подлежащее}? или {\large что с ним делается}? или {\large что оно такое}? В приведенном предложении сказуемым является слово {\slv{повѣ́даютъ}}.
    
    Подлежащее и сказуемое могут иметь при себе объяснительные слова. В предложении слова {\slv{сла́вꙋ бж҃їю}} являются объяснительными словами.
    
    Подлежащее и сказуемое называются \emph{главными членами}, а объяснительные слова~---~\emph{второстепенными членами}.
    
    Из второстепенных членов различают дополнения, определения и обстоятельства.
    
    \emph{Дополнением} в предложении называется слово, относящееся к сказуемому и отвечающее на вопросы косвенных падежей, т.е. всех падежей, кроме именительного и звательного. В том же предложении дополнением служит слово {\slv{сла́вꙋ}} ({\large что}?~---~винит. падеж).
    
    \emph{Определение} может отвечать на вопросы: {\large какой}? {\large чей}? {\large который}? {\large сколько}? В данном предложении определением служит слово {\slv{бж҃їю}} ({\large чью}? {\large какую}?).
    
    \emph{Обстоятельства} выражают различные условия действия или состояния подлежащего. Они могут обозначать место действия, время, образ, цель и причину действия, отвечая на соответствующие вопросы.

                \subsubsection{Пунктуация}

    \emph{Пунктуацией} называется правило расстановки знаков препинания между словами в предложении. В церковнославянской письменности знаки препинания следующие: точка, запятая, двоеточие, знак вопроса, из которых точка и запятая ставятся в тех же случаях, как и в русском языке.
    
    Двоеточие ({\slv{:}}) употребляется перед собственной речью, или когда последующее предложение объясняет предыдущее. Но в церковнославянском тексте двоеточие употребляется и в том случае, когда по-русски должна стоять точка с запятой, например:
    
    \bigskip\autorows{c}{1}{c}{
        {\slv{Воскли́кнемъ бг҃ꙋ сп҃си́телю на́шемꙋ: предвари́мъ лицѐ є҆гѡ̀}},
        {{\slv{во и҆сповѣ́данїи}} (Пс. 94, 1--2)}
    }

    Кроме того, двоеточие употребляется вместо русского многоточия,  когда указывается только начало церковного чтения или песнопения, например:
    
    \medskip
    {\slv{Сла́ва, и҆ ны́нѣ: Вѣ́рꙋю во є҆ди́наго бг҃а ѻ҆ц҃а̀: ({\footnotesize листъ} г҃). трис҃тое. Прес҃та́ѧ трⷪ҇це: По ѻ҆́ч҃е на́шъ: {\footnotesize свѧще҆́нникъ:} Ꙗ҆́кѡ твоѐ є҆́сть црⷭ҇тво:}}
    \medskip
    
    Для изображения знака вопрос в церковнославянском тексте употребляется знак <<точка с запятой>> ({\slv{;}}). Например:

    \bigskip\autorows{c}{1}{c}{
        {\slv{А҆́зъ тре́бꙋю тобо́ю крⷭ҇ти́тисѧ, и҆ ты́ ли}},
        {{\slv{грѧде́ши ко мнѣ̀;}} (Мф. 3, 14)}
    }

    Иногда знак вопроса в церковнославянском тексте обозначает знак восклицания ({\slv{!}}). Например:

    \bigskip\autorows{c}{1}{c}{
        {{\slv{Гдⷭ҇и, что́ сѧ ᲂу҆мно́жиша стꙋжа́ющїи мѝ;}} (Пс. 3, 2)},
        {(Господи, как умножились враждующие против меня!)}
    }

        \section{Глагол}
                \subsubsection{Понятие о глаголе}

    Глагол есть самая основная часть речи во всех языках. В предложении глагол фигурирует почти всегда в качестве главного члена предложения~---~сказуемого. Поэтому-то, прежде изучения других самостоятельных частей речи церковнославянского языка, рекомендуется предварительно изучить простейшие церковнославянские глагольные формы.
    
    \textbf{Глаголом} называется часть речи, выражающая действие или состояние предмета или явления. Например:

    \bigskip\autorows{c}{1}{c}{
        {{\slv{И̑щи́те и҆ ѡбрѧ́щете}} (Мф. 7, 7)},
        {{\slv{А҆́зъ ᲂуснꙋ́хъ и҆ спа́хъ, воста́хъ}} (Пс. 3, 6)}
    }


                \subsubsection{Глаголы архаические}

    Глаголы подчиняются известным законам образования и спряжения. Но в церковнославянском языке существует несколько глаголов, которые не подчиняются общим законам образования и спряжения. У этих глаголов сохранились еще древние первоначальные формы славянского языка, а потому такие глаголы получили название \emph{архаических}, т.е. древних, старинных. К таким глаголам относятся следующие пять: {\slv{бы́ти, вѣ́дѣти, ꙗ҆́сти, да́ти, и҆мѣ́ти}}.
    
    Из этих архаических глаголов глагол {\slv{бы́ти}} имеет особо важное значение. В силу такого исключительного значения архаического глагола {\slv{бы́ти}} его принято называть \emph{вспомогательным глаголом}.

                \subsubsection{Неопределенная форма глаголов}

    У всех глаголов есть такая форма, которая не спрягается и служит лишь указанием действия или состояния. Например: {\slv{писа́ти, хвали́ти, нестѝ}}.
    
    Такая глагольная форма называется \emph{неопределенной формой} глагола и считается \emph{началом глагола}.
    
    Неопределенная форма глагола всегда имеет окончание {\slv{-ти}}. Если иногда неопределенная форма глагола и имеет окончание {\slv{-щи}} ({\slv{мощѝ, рещѝ}}), то такое окончание не представляет собою никакого исключения: это то же окончание {\slv{-ти}}, но только с предшествующим гортанным {\slv{-г-}} или {\slv{-к-}} ({\slv{мог-тѝ, рек-тѝ}}), а эти гортанные ({\slv{-г-}} и {\slv{-к-}}), соединяясь с {\slv{-т-}}, переходят, по закону смягчения, вместе с ним в {\slv{-щ-}}.
    
    Таким образом, вместо {\slv{могтѝ}} получилось {\slv{мощѝ}} ({\slv{г}} + {\slv{т}} = {\slv{щ}}), а вместо {\slv{ректѝ}}~---~{\slv{рещѝ}} ({\slv{к}} + {\slv{т}} = {\slv{щ}}).
    
%    \pagebreak

                \subsubsection{Спряжение в настоящем времени вспомогательного глагола {\slv{бы҆́ти}}}

    Глагол в настоящем времени служит для выражения действия или состояния в данный момент. Например:

    \bigskip\autorows{c}{1}{c}{
        {{\slv{Гдⷭ҇ь пасе́тъ мѧ̀}} (Пс. 22, 1)}
    }

    Важное значение глагола {\slv{бы́ти}} среди других глаголов, как глагола вспомогательного, заставляет изучать его спряжение прежде спряжения других глаголов.

    \begin{center}
        Спряжение глагола {\slv{бы́ти}} в настоящем времени
        \renewcommand*{\arraystretch}{1.2}
        \begin{tabular}[c]{|c|c|c|c|c|}
            \hline

            \multirow{2}{*}{\spheading[2em]{\scriptsize{Лицо}}}
            & \multirow{2}{*}{\scriptsize\makecell{Единственное\\число}}
            & \multicolumn{2}{c|}{\scriptsize\makecell{Двойственное число}}
            & \multirow{2}{*}{\scriptsize\makecell{Множественное\\число}}
            \\
            
            \cline{3-4}
            
            &
            & \scriptsize\makecell{Мужской род}
            & \scriptsize\makecell{Жен. и сред. род}
            &
            \\
            
            \hline
            
            1
            & \makecell{{\footnotesize\slv{а҆́зъ}} {\slv{є҆́смь}}}
            & \makecell{{\footnotesize\slv{мы̀}} {\slv{є҆сва́}}\\({\scriptsize или} {\slv{є҆сма̀}})}
            & \makecell{{\footnotesize\slv{мы̀}} {\slv{є҆свѣ̀}}}
            & \makecell{{\footnotesize\slv{мы̀}} {\slv{є҆смы̀}}}
            \\\hline
            
            2
            & \makecell{{\footnotesize\slv{ты̀}} {\slv{є҆сѝ}}}
            & \makecell{{\footnotesize\slv{вы̀}} {\slv{є҆ста̀}}}
            & \makecell{{\footnotesize\slv{вы̀}} {\slv{є҆стѣ̀}}}
            & \makecell{{\footnotesize\slv{вы̀}} {\slv{є҆стѐ}}}
            \\\hline
            
            3
            & \makecell{{\footnotesize\slv{ѻ҆́нъ, ѻ҆на̀, ѻ҆но̀}}\\{\slv{є҆́сть}}}
            & \makecell{{\footnotesize\slv{ѻ҆́на}} {\slv{є҆ста̀}}}
            & \makecell{{\footnotesize\slv{ѻ҆́нѣ}} {\slv{є҆стѣ̀}}}
            & \makecell{{\footnotesize\slv{ѻ҆нѝ, ѻ҆нѣ̀, ѻ҆нѝ}}\\{\slv{сꙋ́ть}}}
            \\\hline
        
        \end{tabular}
    \end{center}

    Если при глаголе {\slv{бы́ти}} находится отрицание {\slv{не}},то в настоящем времени {\slv{не}} и {\slv{є҆́смь}} сливаются. Происходит так называемое \emph{стяжание} двух звуков {\slv{е}}. При стяжании двойной звук {\slv{ее}} переходит в {\slv{ѣ}}, и образуется слово {\slv{нѣсмь}}. В таком положении глагол {\slv{бы́ти}} с отрицанием {\slv{не}} и спрягается в настоящем времени во всех лицах и числах: {\slv{нѣ́смь, нѣ́си, нѣ́сть}}\ldots~{\slv{нѣ́смы, нѣ́сте}}\ldots~Исключение из этого правила составляет только 3-е лицо множественного числа, где отрицание {\slv{не}} пишется раздельно: {\slv{не сꙋ́ть}}.

                    \paragraph{Упражнение}

    Поставьте приведенные в скобках слова в соответствующую форму.
    
    \begin{flushleft}
        \renewcommand*{\arraystretch}{1.2}
        \begin{tabular}[l]{cll}
            
            ~~~~~
            & \emph{Образец}:
            & \makecell[l]{В начале ты, Господи, основал землю,\\и небеса _____ ({\slv{бы́ти}}) дело рук Твоих (Пс. 101, 26).}
            \\
            
            ~~~~~
            &
            &
            \\
            
            ~~~~~
            & \emph{Ответ}:
            & \makecell[l]{В начале ты, Господи, основал землю,\\и небеса {\slv{сꙋ́ть}} дело рук Твоих.}
            \\
            
        \end{tabular}
    \end{flushleft}

    Правильность ответа проверьте в церковнославянском тексте Нового Завета.
    
    1. Петр же поднял его, говоря: встань; я тоже _____ ({\slv{бы́ти}}) человек (Деян. 10, 26).
    
    2. Грех не должен над вами господствовать, ибо вы _____ ({\slv{бы́ти}}) не под законом, но под благодатью (Рим. 6, 14).
    
    3. Но злой дух сказал в ответ: Иисуса знаю, и павел мне известен, а вы кто _____ ({\slv{бы́ти}})? (Деян. 19, 15).
    
    4. Итак, смотрите, поступайте осторожно, не как неразумные, потому что дни лукавы _____ ({\slv{бы́ти}}) (Еф. 5, 15).
    
    5. Ибо теперь мы живы _____ ({\slv{бы́ти}}), когда вы стоите в Господе (1 Фес. 3, 8).
    
    6. Итак, неизвинителен ты _____ ({\slv{бы́ти}}), всякий человек (Рим. 2, 1).
    
    7. Потому что немудрое Божие премудрее человеков _____ ({\slv{бы́ти}}), и немощное Божие сильнее человеков _____ ({\slv{бы́ти}}) (1 Кор. 1, 25).
    
    8. Что же? станем ли грешить, потому что мы _____ ({\slv{бы́ти}}) не под законом, а под благодатью? (Рим. 6, 15).
    
    9. Ибо нет _____ ({\slv{бы́ти}}) лицеприятия у Бога (Рим. 2, 11).
    
    10. Ибо все вы _____ ({\slv{бы́ти}}) сыны света и сыны дня: мы не _____ ({\slv{бы́ти}}) сыны ночи, ни тьмы (1 Фес. 5, 5).

                \subsubsection{Спряжение в настоящем времени прочих глаголов}

    Для образца спряжения прочих глаголов в настоящем времени возьмем глаголы {\slv{нестѝ}} и {\slv{хвали́ти}}.
    
    \begin{center}
        \renewcommand*{\arraystretch}{1.2}
        \begin{tabular}[c]{|c|c|c|}
            \hline
            
            \footnotesize\makecell{Лицо}
            & \footnotesize\makecell{Единственное число}
            & \footnotesize\makecell{Множественное число}
            \\\hline
            
            \makecell{1}
            & \makecell{{\slv{несꙋ̀, хвалю̀}}}
            & \makecell{{\slv{несе́мъ, хва́лимъ}}}
            \\\hln
            
            \makecell{2}
            & \makecell{{\slv{несе́ши, хва́лиши}}}
            & \makecell{{\slv{несе́те, хва́лите}}}
            \\\hln
            
                        
            \makecell{3}
            & \makecell{{\slv{несе́тъ, хва́литъ}}}
            & \makecell{{\slv{несꙋ́тъ, хва́лѧтъ}}}
            \\\hline

            
            \multicolumn{3}{|c|}{\footnotesize\makecell{Двойственное число}}
            \\\hline

            & \footnotesize\makecell{Мужской род}
            & \footnotesize\makecell{Женский и средний род}
            \\\hline
            
            \makecell{1}
            & \makecell{{\slv{несе́ва, хва́лива}}}
            & \makecell{{\slv{несе́вѣ, хва́ливѣ}}}
            \\\hln
            
            \makecell{2}
            & \makecell{{\slv{несе́та, хва́лита}}}
            & \makecell{{\slv{несе́тѣ, хва́литѣ}}}
            \\\hln
            
            \makecell{3}
            & \makecell{{\slv{несе́та, хва́лита}}}
            & \makecell{{\slv{несе́тѣ, хва́литѣ}}}
            \\\hline

        \end{tabular}
    \end{center}

    Необходимо заметить:
    
    1. Во 2-м и 3-м лице двойственного числа женского и среднего рода глаголы оканчиваются на {\slv{ѣ}}, а во 2-м лице множественного числа~---~на {\slv{е}}; нужно помнить, что первоначально в славянском языке звуки {\slv{ѣ}} и {\slv{е}} звучали не одинаково, когда же эти звуки впоследствии стали однозвучны, то буквы (но не звуки) {\slv{ѣ}} (в двойственном числе) и {\slv{е}} (во множественном числе) стали служить лишь для различия числовых форм друг от друга; такое правописание для этих чисел распространяется на спряжение глаголов и в остальных временах.
    
    Второе и третье лицо двойственного числа в настоящем времени (а также и в других временах) имеют одинаковую форму.
    
    2. Если в 1-м лице настоящего времени основа глагола оканчивается гортанным звуком {\slv{г}} или {\slv{к}}, то эти звуки перед другими личными окончаниями (кроме 3-го лица множественного числа) соответственно смягчаются в {\slv{ж}} или {\slv{ч}}. Например:
    
    \autorows{c}{1}{c}{
        {\slv{могꙋ̀~\textemdash~мо́жеши, рекꙋ̀~\textemdash~рече́ши}}
    }
    
    3. Если основа неопределенной формы оканчивается на юсовское {\slv{ѧ}}, то это {\slv{ѧ}} перед окончаниями настоящего времени разлагается на свои древние звуки. Например:
    
    \autorows{c}{1}{c}{
        {{\slv{клѧ́-ти~\textemdash~клен-ꙋ̀, клен-е́-ши}} и т.д.},
        {{\slv{ꙗ҆́-ти~\textemdash~є҆́мл-ю, є҆́мл-е-ши}} и т.д.}
    }

    В этом последнем случае губной звук {\slv{м}} смягчается плавным звуком {\slv{л}}.

                    \paragraph{Упражнение}
    
    Восстановите в нижеследующих примерах правильные окончания церковнославянских глаголов. В скобках приведена неопределенная форма соответствующих глаголов. Правильность выполнения проверьте по церковнославянскому тексту, ссылки на который приводятся в конце каждого предложения.
    
    \begin{flushleft}
        \renewcommand*{\arraystretch}{1.2}
        \begin{tabular}[l]{cll}
            
            ~~~~~
            & \emph{Образец}:
            & \makecell[l]{Когда Иисус шел оттуда, за ним следовали двое слепых\ldots\\И говорит им Иисус: {\slv{Верꙋ}}\ldots~({\slv{ве́ровати}}) ли, яко могу\\сие сотворите? (Мф. 9, 27--28).}
            \\
            
            ~~~~~
            &
            &
            \\
            
            ~~~~~
            & \emph{Ответ}:
            & \makecell[l]{Когда Иисус шел оттуда, за ним следовали \textbf{двое} слепых\ldots\\И говорит им Иисус: {\slv{Ве́рꙋета}} ли, яко могу\\сие сотворите? (Мф. 9, 27--28).}
            \\
            
        \end{tabular}
    \end{flushleft}

    (Выделенное в примере слово \textbf{двое} указывает на то, что глагол стоит в двойственном числе. Подробнее о двойственном числе см. в книге: В.И. Супрун. <<Учебник старославянского языка>>, с. 63--66).
    
    1. Тогда подошли к Нему сыновья Зеведеевы \textbf{Иаков} и \textbf{Иоанн} и сказали: Учитель! {\slv{Хоще}}\ldots~({\slv{хотѣ́ти}}), чтобы Ты сделал нам, о чем {\slv{проси}}\ldots~({\slv{проси́ти}}). Он сказал им: что {\slv{хоще}}\ldots~({\slv{хотѣ́ти}}), чтобы Я сделал вам? Они сказали Ему: дай нам {\slv{да сѧде}}\ldots~({\slv{сѣдѣ́ти}}) у Тебя, одному по правую сторону, а другому по левую в славе Твоей. Но Иисус казал им: не знаете, чего {\slv{проси}}\ldots~({\slv{про́сити}}). {\slv{Мож}}\ldots~({\slv{мощѝ}}) ли пить чашу, которую Я пью, и креститься крещением, которым Я крещусь? Они отвечали: {\slv{мож}}\ldots~({\slv{мощѝ}}). Иисус же сказал им: чашу, которую Я пью, {\slv{испї}}\ldots~({\slv{испи́ти}}) и крещением, которым Я крещусь, {\slv{крест}}\ldots~({\slv{крести́тисѧ}}) (Мк. 10, 35--39).
    
    Истинно также говорю вам, что если \textbf{двое} из вас {\slv{совѣща}}\ldots~({\slv{совѣща́ти}}) на земле просить о всяком деле, то, чего бы ни {\slv{прос}}\ldots~({\slv{проси́ти}}), будет им от Отца Моего Небесного (Мф. 18, 19).
    
    3. Господь~---~престол Его на небесах, \textbf{очи} Его {\slv{призира}}\ldots~({\slv{призира́ти}}) на нищего; \textbf{вежды} Его {\slv{испыта}}\ldots~({\slv{испыта́ти}}) сынов человеческих (Пс. 10, 4).
    
    4. \textbf{Фавор} и \textbf{Ермон} о имени Твоем {\slv{возрадꙋ}}\ldots~({\slv{возра́доватисѧ}}) (Пс. 88, 13).

                \subsubsection{Глагольные основы}

    В глаголах нужно различать две основы: основу неопределенной формы и основу настоящего времени. Эти две основы имеют очень важное значение, потому что от них образуются все глагольные формы.
    
    Основа неопределенной формы глагола~---~это та его часть, которая останется, если отбросить окончание {\slv{-ти}}. Так, например, в глаголах:
    
    \begin{center}
        \begin{tabular}[c]{l|c|l}
            
            {\slv{бра́-ти}} & основа & {\slv{бра-}} \\
            {\slv{зва́-ти}} &        & {\slv{зва-}} \\
            
        \end{tabular}
    \end{center}

    Основа настоящего времени глагола~---~это та его часть, которая останется, если отбросить личное окончание 1-го лица единственного числа. Например, в этих же глаголах:
    
    \begin{center}
        \begin{tabular}[c]{l|c|l}
            
            {\slv{бер-ꙋ̀}} & основа & {\slv{бер-}} \\
            {\slv{зов-ꙋ̀}} &        & {\slv{зов-}} \\
            
        \end{tabular}
    \end{center}

    Но не всегда основа неопределенной формы глагола отличается от основы настоящего времени; нередко обе основы совпадают, а иногда они совпадают даже с корнем глагола. Например, в глаголах:
    
    \bigskip\autorows{c}{2}{l}{
        {\slv{дѣ́ла-ти}}, {обе основы},
        {\slv{дѣ́ла-ю}}, {\slv{дѣла-}},
        {\slv{нес-ти́}}, {обе основы совпадают},
        {\slv{нес-ꙋ̀}}, {с корнем {\slv{нес-}}}
    }
                \subsubsection{Глаголы тематические и разделение их на два спряжения}

    По основе настоящего времени различают глаголы: тематические и нетематические (архаические).
    
    \textbf{Тематические} глаголы~---~это те, в которых между основой настоящего времении личным окончанием (кроме 1-го лица единственного числа и 3-го лица множественного числа) находится гласный звук (\emph{тема}) {\slv{-е-}} (например, {\slv{зов-е́-ши}}) или {\slv{-и-}} (например, {\slv{хва́л-и-ши}}).
    
    \textbf{Нетематические} же (архаические) глаголы присоединяют личные окончания непосредственно к корню.
    
    Тематические глаголы разделяются на два \textbf{спряжения}.
    
    К \textbf{первому спряжению} относятся те глаголы, которые в настоящем времени принимают личные окончания с помощью тематической гласной {\slv{е}}.
    
    Ко \textbf{второму спряжению} относятся те глаголы, которые в настоящем времени принимают личные окончания с помощью тематической гласной {\slv{и}}.
    
    Первое спряжение тематических глаголов характеризуется (за очень малыми исключениями) еще тем, что в 3-м лице множественного числа настоящего времени они имеют окончание {\slv{-ꙋтъ}} или {\slv{-ютъ}} ({\slv{нес-ꙋ́тъ, повелѣва́-ютъ}}), а глаголы второго спряжения в том же числе того же времени имеют окончание {\slv{-атъ}} или {\slv{-ѧтъ}} ({\slv{де́рж-атъ, хва́л-ѧтъ}}).

                \subsubsection{Преходящее время вспомогательного глагола {\slv{бы́ти}}}

    \textbf{Преходящим временем} называется такое прошедшее время глагола, которое выражает действие, долго продолжавшееся или несколько раз повторявшееся. Например:
    
    {\slv{Во́ды мно́ги бѧ́хꙋ тꙋ̀}} (Ин. 3, 23)
    
    Рассмотрим преходящее время вспомогательного глагола {\slv{бы́ти}}.


    \begin{center}
%        Преходящее время вспомогательного глагола {\slv{бы́ти}}
        \renewcommand*{\arraystretch}{1.2}
        \begin{tabular}[c]{|c|c|c|c|c|}
            \hline

            \multirow{2}{*}{\spheading[2em]{\scriptsize{Лицо}}}
            & \multirow{2}{*}{\scriptsize\makecell{Единственное\\число}}
            & \multicolumn{2}{c|}{\scriptsize\makecell{Двойственное число}}
            & \multirow{2}{*}{\scriptsize\makecell{Множественное\\число}}
            \\
            
            \cline{3-4}
            
            &
            & \scriptsize\makecell{Мужской род}
            & \scriptsize\makecell{Жен. и сред. род}
            &
            \\
            
            \hline
            
            1
            & \makecell{\slv{бѧ́хъ, бѣ́хъ}}
            & \makecell{\slv{бѧ́хова, бѣ́хова}}
            & \makecell{\slv{бѧ́ховѣ, бѣ́ховѣ}}
            & \makecell{\slv{бѧ́хомъ, бѣ́хомъ}}
            \\\hline
            
            2
            & \makecell{\slv{бѣ̀}}
            & \makecell{\slv{бѧ́ста, бѣ́ста}}
            & \makecell{\slv{бѧ́стѣ, бѣ́стѣ}}
            & \makecell{\slv{бѧ́сте, бѣ́сте}}
            \\\hline
            
            3
            & \makecell{\slv{бѧ́ше, бѣ̀}}
            & \makecell{\slv{бѧ́ста, бѣ́ста}}
            & \makecell{\slv{бѧ́стѣ, бѣ́стѣ}}
            & \makecell{\slv{бѧ́хꙋ, бѣ́хꙋ, бѣ́ша}}
            \\\hline
            
        \end{tabular}
    \end{center}

    \textbf{Примечание}. Преходящее время имеет в учебной литературе по церковнославянскому языку несколько различных наименований. Его называют также прошедшим продолжительным, прошедшим несовершенным, прошедшим многократным временем или \emph{имперфектом}.

    Оно используется преимущественно для всякого рода описаний. В этом времени написаны, например, описание земли в начале сотворения мира (Быт. 1, 2) и описание Овчей купели в Евангелии от Иоанна (Ин. 5, 2--4). Более понятной его функция станет ниже, когда мы познакомимся с \emph{аористом}.
    
                    \paragraph{Упражнение}

    Поставьте приведенные в скобках церковнославянские слова в соответствующую форму преходящего времени. Правильность решения можно проверить, обратившись к церковнославянскому тексту. Ответ считается правильным, если приведена одна из форм, перечисленных в таблице выше. Например, если в тексте написано {\slv{бѧ́хомъ}}, а обучающийся написал {\slv{бѣ́хомъ}}, ответ считается правильным, так как в таблице это две равноправные формы 1-го лица множественного числа.
    
\begin{flushleft}
    \renewcommand*{\arraystretch}{1.2}
    \begin{tabular}[l]{cll}
        
        ~~~~~
        & \emph{Образец}:
        & \makecell[l]{В Антиохии, в тамошней церкви _____ ({\slv{бы́ти}}) некоторые\\пророки и учители (Деян. 13, 1).}
        \\
        
        ~~~~~
        &
        &
        \\
        
        ~~~~~
        & \emph{Ответ}:
        & \makecell[l]{В Антиохии, в тамошней церкви {\slv{бѧ́хꙋ}} некоторые пророки\\и учители.}
        \\
        
    \end{tabular}
\end{flushleft}

    1. Ибо вы _____ ({\slv{бы́ти}}), как овцы блуждающие (не имея пастыря), но возвратились ныне к Пастырю и Блюстителю душ ваших (1 Пет. 2, 25).
    
    2. И вот, в тот самый час три человека стали перед домом, в котором я _____ ({\slv{бы́ти}}), посланные из Кесарии ко мне (Деян. 11, 11).
    
    3. Скиния свидетельства _____ ({\slv{бы́ти}}) у отцов наших в пустыне, как повелел Говоривший Моисею сделать ее по образцу, им виденному (Деян. 4, 44).
    
    4. А до пришествия веры мы заключены _____ ({\slv{бы́ти}}) под стражею закона, до того времени, как надлежало открыться вере (Гал. 3, 23).
    
    5. \ldots и видит отверстое небо и сходящий к нему некоторый сосуд, как бы большое полотно, привязанное за четыре угла и опускаемое на землю; в нем _____ ({\slv{бы́ти}}) всякие четвероногие земные, звери, пресмыкающиеся и птицы небесные (Деян. 10, 11--12).
    
    6. Ибо, когда вы были рабами греха, тогда _____ ({\slv{бы́ти}}) свободны от праведности (Рим. 6, 20).
    
    7. А за что убил его? За то, что дела его _____ ({\slv{бы́ти}}) злы, а дела брата его праведны (1 Ин. 3, 12).
    
    8. Церквам Христовым в Иудее лично я не _____ ({\slv{бы́ти}}) известен (Гал. 1, 22).
    
    9. У множества же уверовавших _____ ({\slv{бы́ти}}) одно сердце и одна душа; и никто ничего из имения своего не называл своим, но все у них было общее (Деян. 4, 32).

                \subsubsection{Аорист вспомогательного глагола {\slv{бы́ти}}}

    \textbf{Аористом} называется такое прошедшее время глагола, которое характеризует действие вполне законченное. Например:
    
    
    \bigskip\autorows{c}{1}{c}{
        {{\slv{То́й речѐ и҆}} {\slv{\large бы́ша}} (Пс. 32, 9)}
    }

    \textbf{Примечание}: \emph{Аорист} не имеет прямого аналога в русском языке. Его называют также прошедшим совершенным, прошедшим однократным временем. Однако эти названия отражают его сущность лишь отчасти. Наиболее точно будет сказать, что аорист используется для \textbf{ведения рассказа}, а \emph{имперфект} (преходящее) употребляется для обозначения сопутствующих и производных от основных фактов (см. Иеромонах Алипий. Грамматика церковно-славянского языка М., 1991, с. 200--204):
    
    Царь Ирод, \textbf{услышав} об Иисусе,~---~ибо имя Его стало гласно,~---~\emph{говорил}: это Иоанн Креститель воскрес из мертвых, и потому чудеса делаются им. Другие \emph{говорили}: это Илия, а иные \emph{говорили}: это пророк, или как один из пророков. Ирод же, услышав, \textbf{сказал}: это Иоанн, которого я обезглавил; он воскрес из мертвых. Ибо сей Ирод, послав, \textbf{взял} Иоанна и \textbf{заключил} его в темницу за Иродиаду, жену Филиппа, брата своего, потому что \textbf{женился} на ней. Ибо Иоанн \emph{говорил} Ироду: не должно тебе иметь жену брата твоего. Иродиада же \emph{злобясь} на него, \emph{желала} убить его; но не \emph{могла}. Ибо Ирод \emph{боялся} Иоанна, зная, что он муж праведный и святой, и \emph{берег} его; многое \emph{делал}, слушаясь его, и с удовольствием \emph{слушал} его (Мк. 6, 14--20).
    
    Выделенные в тексте жирным шрифтом формы переданы в церковнославянском тексте аористом и образуют как бы \textbf{костяк} действия:

    \begin{center}
        услышал~---~сказал~---~взял~---~заключил~---~женился.
    \end{center}

    Последний факт является как бы напоминанием о том, что было раньше.
    
    Формы, выделенные курсивом, образуют некий \textbf{фон} указанного действия.
    
    Рассмотрим аорист вспомогательного глагола {\slv{бы́ти}}.


    \begin{center}
        %        Аорист вспомогательного глагола {\slv{бы́ти}}
        \renewcommand*{\arraystretch}{1.2}
        \begin{tabular}[c]{|c|c|c|c|c|}
            \hline
            
            \multirow{2}{*}{\spheading[2em]{\scriptsize{Лицо}}}
            & \multirow{2}{*}{\scriptsize\makecell{Единственное\\число}}
            & \multicolumn{2}{c|}{\scriptsize\makecell{Двойственное число}}
            & \multirow{2}{*}{\scriptsize\makecell{Множественное\\число}}
            \\
            
            \cline{3-4}
            
            &
            & \scriptsize\makecell{Мужской род}
            & \scriptsize\makecell{Жен. и сред. род}
            &
            \\
            
            \hline
            
            1
            & \makecell{\slv{бы́хъ}}
            & \makecell{\slv{бы́хова, бы́сва}}
            & \makecell{\slv{бы́ховѣ, бы́свѣ}}
            & \makecell{\slv{бы́хомъ}}
            \\\hline
            
            2
            & \makecell{\slv{бы̀}}
            & \makecell{\slv{бы́ста}}
            & \makecell{\slv{бы́стѣ}}
            & \makecell{\slv{бы́сте}}
            \\\hline
            
            3
            & \makecell{\slv{бы́сть, бы̀}}
            & \makecell{\slv{бы́ста}}
            & \makecell{\slv{бы́стѣ}}
            & \makecell{\slv{бы́ша}}
            \\\hline
            
        \end{tabular}
    \end{center}

    При переводе некоторых славянских предложений на русский язык, где фигурирует глагол {\slv{бы́ти}} в аористе 3-го лица единственного числа в своей форме {\slv{бы́сть}}, эта форма иногда переводится русским безличным глаголом \emph{стало} или \emph{случилось}. Например:

    \bigskip\autorows{c}{1}{c}{
        {{\slv{Бы́сть же во дни̑ ты̑ѧ}} (Лк. 2, 1) (Случилось же в те дни)}
    }

                    \paragraph{Упражнение}

    Поставьте приведенные в скобках церковнославянские слова в соответствующую форму аориста.
    
    \begin{flushleft}
        \renewcommand*{\arraystretch}{1.2}
        \begin{tabular}[l]{cll}
            
            ~~~~~
            & \emph{Образец}:
            & \makecell[l]{И угодно _____ ({\slv{бы́ти}}) это предложение всему собранию\\(Деян. 6, 5).}
            \\
            
            ~~~~~
            &
            &
            \\
            
            ~~~~~
            & \emph{Ответ}:
            & \makecell[l]{И угодно {\slv{бы́сть}} это предложение всему собранию.}
            \\
            
        \end{tabular}
    \end{flushleft}

    1. И _____ ({\slv{бы́ти}}) я у вас в немощи и в страхе и в великом трепете (1 Кор. 2, 3).

    2. Иосиф открылся братьям своим, и известен _____ ({\slv{бы́ти}}) фараону род Иосифов (Деян. 7, 13).

    3. Как и вы некогда были непослушны Богу, а ныне помилованы _____ ({\slv{бы́ти}}) (Рим. 11, 30).

    4. Иаков перешел в Египет, и скончался сам и отцы наши; и перенесены _____ ({\slv{бы́ти}}) в Сихем и положены во гробе (Деян. 7, 15--16).

    5. Злословят нас, мы благословляем; гонят нас, мы терпим; хулят нас, мы молим; мы _____ ({\slv{бы́ти}}) как сор для мира, как прах, всеми попираемый доныне (1 Кор. 4, 12--13).

    6. Для иудеев я был как иудей, чтобы приобрести иудеев; для подзаконных _____ ({\slv{бы́ти}}) как подзаконный, чтобы приобрести подзаконных (1Кор. 9, 20).

    7. Ибо если, избегнув скверн мира чрез познание Господа и Спасителя нашего Иисуса Христа, опять запутываются в них и побеждаются ими, то последнее _____ ({\slv{бы́ти}}) для таковых хуже первого (2 Пет. 2, 20).

    8. Они убили предвозвестивших пришествие Праведника, Которого предателями и убийцами _____ ({\slv{бы́ти}}) ныне вы (Деян. 7, 52).

    9. Ныне радуюсь в страданиях моих за вас и восполняю недостаток в плоти моей скорбей Христовых за Тело Его, которое есть Церковь, которой _____ ({\slv{бы́ти}}) я служителем (Кол. 1, 24--25).

    \emph{Пояснение к упражнению 10}. Два предыдущих упражнения наглядно показывают различие между преходящим временем и аористом. Если в упражнении 9 значения глагола {\slv{бы́ти}} практически совпадают с русскими, то в упражнении 10 каждый пример приобретает дополнительное значение, непонятное на основе только русского текста. Приведем списком дополнительные значения:
    
    \begin{flushleft}
        \renewcommand*{\arraystretch}{1.2}
        \begin{tabular}[l]{cll}
            
            ~~~~~
            & \emph{Образец}:
            & \makecell[l]{({\slv{бы́ти}}) угодно, т.е. \textbf{понравиться};}
            \\
            
            ~~~~~
            &
            & 1: я у вас \textbf{оказался} в состоянии немощи\ldots
            \\
            
            ~~~~~
            &
            & 2: \textbf{стал} известен;
            \\

            
            ~~~~~
            &
            & 3: вам было \textbf{предоставлено} помилование;
            \\
            
                        
            ~~~~~
            &
            & 4: наших отцов \textbf{перенесли} в Сихем\ldots
            \\
            
                        
            ~~~~~
            &
            & 5: мы \textbf{превратились} в сор для мира;
            \\
            
                        
            ~~~~~
            &
            & 6: я \textbf{становился} иудеем и подзаконным;
            \\
            
                        
            ~~~~~
            &
            & 7: \textbf{стало} хуже;
            \\
            
                        
            ~~~~~
            &
            & 8: \textbf{стали} вы;
            \\
            
                        
            ~~~~~
            &
            & 9: я \textbf{стал} служителем.
            \\

        \end{tabular}
    \end{flushleft}

    Суть явления здесь состоит в том, что глагол {\slv{бы́ти}}, сам по себе обозначающий \textbf{состояние}, через форму аориста приобратает значение \textbf{события}, а конкретный русский перевод (стать, превратиться и т.д.) зависит уже от контекста. После этого необходимого пояснения выполним упражнение 11, которое поможет уяснить рассмотренное различие.

                    \paragraph{Упражнение}

    Поставьте приведенные в скобках церковнославянские слова в соответствующую форму преходящего времени или аориста.
    
    1. При наступлении дня Пятидесятницы все они были единодушно вместе. И внезапно _____ ({\slv{бы́ти}}) шум с неба, как бы от несущегося сильного ветра, и наполнил весь дом (Деян. 2, 1--2).
    
    2. Вместе с ним лицемерили и прочие Иудеи, так что даже Варнава _____ ({\slv{бы́ти}}) увлечен их лицемением (Гал. 2, 13).
    
    3. Когда я _____ ({\slv{бы́ти}}) младенцем, то по-младенчески говорил, по-младенчески мыслил, по-младенчески рассуждал; а как _____ ({\slv{бы́ти}}) мужем, то оставил младенческое (1 Кор. 13, 11).
    
    4. \ldots вы _____ ({\slv{бы́ти}}) в то время без Христа, отчуждены от общества Израильского, чужды заветов обетования, не имели надежды и были безбожники в мире. А теперь во Христе Иисусе вы, бывшие некогда далеко, _____ ({\slv{бы́ти}}) близки Кровию Христовою (Еф. 2, 12--13).
    
    5. В Иоппии находилась одна ученица, именем Тавифа; она _____ ({\slv{бы́ти}}) исполнена добрых дел и творила много милостынь (Деян. 9, 36).
    
    6. Они же, пригрозив, отпустили их, не находя возможности наказать их, по причине народа; потому что все прославляли Бога за происшедшее. Ибо лет более сорока _____ ({\slv{бы́ти}}) тому человеку, над которым сделалось сие чудо исцеления (Деян. 4, 21--22).
    
    7. Руками же апостолов совершались в народе многие знамения и чудеса; и все единодушно _____ ({\slv{бы́ти}}) в притворе Соломоновом (Деян. 5, 12).

                \subsubsection{Преходящее время тематических глаголов}

    Рассмотрим спряжение в преходящем времени тематических глаголов {\slv{нестѝ}} и {\slv{хвали́ти}}.

    \begin{center}
        %        Спряжение в настоящем времени тематических глаголов {\slv{нестѝ}} и {\slv{хвали́ти}}
        \renewcommand*{\arraystretch}{1.2}
        \begin{tabular}[c]{|c|c|c|c|c|}
            \hline
            
            \multirow{2}{*}{\spheading[2em]{\scriptsize{Лицо}}}
            & \multirow{2}{*}{\scriptsize\makecell{Единственное\\число}}
            & \multicolumn{2}{c|}{\scriptsize\makecell{Двойственное число}}
            & \multirow{2}{*}{\scriptsize\makecell{Множественное\\число}}
            \\
            
            \cline{3-4}
            
            &
            & \scriptsize\makecell{Мужской род}
            & \scriptsize\makecell{Жен. и сред. род}
            &
            \\
            
            \hline
            
            1
            & \makecell{{\slv{несѧ́хъ}}\\{\slv{хвалѧ́хъ}}}
            & \makecell{{\slv{несѧ́хова}}\\{\slv{хвалѧ́хова}}}
            & \makecell{{\slv{несѧ́ховѣ}}\\{\slv{хвалѧ́ховѣ}}}
            & \makecell{{\slv{несѧ́хомъ}}\\{\slv{хвалѧ́хомъ}}}
            \\\hline
            
            2
            & \makecell{{\slv{несѧ́ше}}\\{\slv{хвалѧ́ше}}}
            & \makecell{{\slv{несѧ́ста}}\\{\slv{хвалѧ́ста}}}
            & \makecell{{\slv{несѧ́стѣ}}\\{\slv{хвалѧ́стѣ}}}
            & \makecell{{\slv{несѧ́сте}}\\{\slv{хвалѧ́сте}}}
            \\\hline
            
            3
            & \makecell{{\slv{несѧ́ше}}\\{\slv{хвалѧ́ше}}}
            & \makecell{{\slv{несѧ́ста}}\\{\slv{хвалѧ́ста}}}
            & \makecell{{\slv{несѧ́стѣ}}\\{\slv{хвалѧ́стѣ}}}
            & \makecell{{\slv{несѧ́хꙋ}}\\{\slv{хвалѧ́хꙋ}}}
            \\\hline
            
        \end{tabular}
    \end{center}

    Преходящее время тематических глаголов образуется от неопределенной формы глаголов следующим образом:
    
    а) если в неопределенной форме глагола перед окончанием {\slv{-ти}} находится гласный звук {\slv{а}}, то {\slv{-ти}} меняется на окончание {\slv{-хъ}}, например:

    \bigskip\autorows{c}{1}{c}{
        {\slv{велича́-ти~\textemdash~велича́-хъ}}
    }

    б) во всех же остальных тематических глаголах окончание неопределенной формы {\slv{-ти}} меняется (вместе с простым суффиксом, если он имеется) на окончание {\slv{-ѧхъ}}, например:

    \bigskip\autorows{c}{1}{c}{
        {\slv{нес-тѝ~\textemdash~нес-ѧ́хъ}},
        {\slv{хвал-и́-ти~\textemdash~хвал-ѧ́хъ}}
    }

    Гортанные {\slv{г}} и {\slv{к}}, смягчаясь перед {\slv{-ѧхъ}} соответственно в {\slv{ж}} и {\slv{ч}}, принимают после себя {\slv{а}}, а не {\slv{ѧ}}, например:

    \bigskip\autorows{c}{1}{c}{
        {\slv{мощѝ}}({\slv{мо{\Large г}тѝ}}){\slv{~\textemdash~мо{\Large ж}а́хъ}},
        {\slv{тещѝ}}({\slv{те{\Large к}тѝ}}){\slv{~\textemdash~те{\Large ч}а́хъ}}
    }

    Губные согласные смягчаются вставкой после них плавного звука {\slv{л}}, например:

    \bigskip\autorows{c}{1}{c}{
        {\slv{люби́ти~\textemdash~люблѧ́хъ}},
        {\slv{лови́ти~\textemdash~ловлѧ́хъ}}
    }

    2-е и 3-е лицо единственного числа тематических глаголов в преходящем времени имеют одинаковую форму
    
    От глаголов {\slv{ꙗ҆́ти}} и {\slv{клѧ́ти}} преходящее время будет: {\slv{є҆́млѧхъ, кленѧ́хъ}}.

                    \paragraph{Упражнение}
                    
    Поставьте приведенные в скобках церковнославянские слова в соответствующую форму преходящего времени. Русский глагол пропускается. (Русский текст приводится по переводу П. Юнгерова).
    
    \begin{flushleft}
        \renewcommand*{\arraystretch}{1.2}
        \begin{tabular}[l]{cll}
            
            ~~~~~
            & \emph{Образец}:
            & \makecell[l]{Когда сокрушались кости мои, \textbf{поносили} ({\slv{поноша́ти}})\\меня враги мои (Пс. 41, 11).}
            \\
            
            ~~~~~
            &
            &
            \\
            
            ~~~~~
            & \emph{Ответ}:
            & \makecell[l]{Когда сокрушались кости мои, {\slv{поноша́хꙋ}} меня враги мои.}
            \\
            
        \end{tabular}
    \end{flushleft}

    1. Поношения и страдания \textbf{ожидала} ({\slv{ча́ѧти}}) душа моя (Пс. 68, 21).
    
    2. Это я вспоминал и \textbf{изливал} ({\slv{излиѧ́ти}}) душу мою в себя (Пс. 41, 5).
    
    3. Против меня \textbf{шептали} ({\slv{шепта́ти}}) враги мои, на меня \textbf{замышляли} ({\slv{помышлѧ́ти}}) зло (Пс. 40, 8).
    
    4. Ты\ldots~с которым мы \textbf{ходили} ({\slv{ходи́ти}}) в дом Божий единомысленно (Пс. 54, 15).
    
    5. Освободил Он от тяжестей хребет его; руки его \textbf{работали} ({\slv{порабо́тати}}) корзиною (Пс. 80, 7).
    
    6. Когда же я \textbf{говорил} ({\slv{глаго́лати}}) с ними, они без вины враждовали (борити) со мною (Пс. 119, 7).
    
    7. Тайно клевещущего на ближнего своего~---~сего я \textbf{изгонял} ({\slv{и҆згонѧ́ти}}); со смотрящим гордо и ненасытным~---~с ним я не \textbf{ел} ({\slv{ꙗ҆́сти}})\ldots~Не \textbf{жил} ({\slv{жи́ти}}, в форме {\slv{живѧ}})\ldots~внутри моего дома поступающий гордо, говорящий неправду \textbf{не был прав} ({\slv{и҆справлѧ́ти}}) пред глазами моими. Поутру \textbf{избивал} ({\slv{и҆збива́ти}}) всех грешников земли\ldots~(Пс. 100, 5--8).
    
    8. Устами своими \textbf{благословляли} ({\slv{благословлѧ́ти}}), а сердцем своим \textbf{проклинали} ({\slv{клѧ́ти}}) (Пс. 61, 5).

                \subsubsection{Аорист тематических глаголов}

    Для образца спряжения в аористе тематических глаголов рассмотрим уже знакомые нам глаголы {\slv{нестѝ}} и {\slv{хвали́ти}}.

    \begin{center}
        %        Аорист тематических глаголов {\slv{нестѝ}} и {\slv{хвали́ти}}
        \renewcommand*{\arraystretch}{1.2}
        \begin{tabular}[c]{|c|c|c|c|c|}
            \hline
            
            \multirow{2}{*}{\spheading[2em]{\scriptsize{Лицо}}}
            & \multirow{2}{*}{\scriptsize\makecell{Единственное\\число}}
            & \multicolumn{2}{c|}{\scriptsize\makecell{Двойственное число}}
            & \multirow{2}{*}{\scriptsize\makecell{Множественное\\число}}
            \\
            
            \cline{3-4}
            
            &
            & \scriptsize\makecell{Мужской род}
            & \scriptsize\makecell{Жен. и сред. род}
            &
            \\
            
            \hline
            
            1
            & \makecell{{\slv{несо́хъ}}\\{\slv{хвали́хъ}}}
            & \makecell{{\slv{несо́хова}}\\{\slv{хвали́хова}}}
            & \makecell{{\slv{несо́ховѣ}}\\{\slv{хвали́ховѣ}}}
            & \makecell{{\slv{несо́хомъ}}\\{\slv{хвали́хомъ}}}
            \\\hline
            
            2
            & \makecell{{\slv{несѐ}}\\{\slv{хвалѝ}}}
            & \makecell{{\slv{несо́ста}}\\{\slv{хвали́ста}}}
            & \makecell{{\slv{несо́стѣ}}\\{\slv{хвали́стѣ}}}
            & \makecell{{\slv{несо́сте}}\\{\slv{хвали́сте}}}
            \\\hline
            
            3
            & \makecell{{\slv{несѐ}}\\{\slv{хвалѝ}}}
            & \makecell{{\slv{несо́ста}}\\{\slv{хвали́ста}}}
            & \makecell{{\slv{несо́стѣ}}\\{\slv{хвали́стѣ}}}
            & \makecell{{\slv{несо́ша}}\\{\slv{хвали́ша}}}
            \\\hline
            
        \end{tabular}
    \end{center}

    Аорист тематических глаголов образуется от неопределенной формы следующим образом:
    
    а) если основа неопределенной формы оканчивается на гласный звук, то окончание {\slv{-ти}} непосредственно меняется на окончание {\slv{-хъ}}:

    \bigskip\autorows{c}{1}{c}{
        {\slv{хвали́-ти~\textemdash~хвали́-хъ}}
    }

    б) если же основа неопределенной формы оканчивается на согласный звук, то окончание аориста {\slv{-хъ}} присоединяется к этой основе посредством соединительной гласной {\slv{-о-}}:

    \bigskip\autorows{c}{1}{c}{
        {\slv{нес-тѝ~\textemdash~нес-о́-хъ}}
    }

    При окончании аориста (вместе с соединительной гласной {\slv{-о-}}) на {\slv{-охъ}} 2-е и 3-е лицо единственного числа оканчивается на {\slv{-е}}, причем при основе на гортанные {\slv{г}} и {\slv{к}} последние соответственно смягчаются в {\slv{ж}} и {\slv{ч}}. Например:

    \bigskip\autorows{c}{1}{c}{
        {\slv{нес-о́хъ~\textemdash~нес-ѐ}},
        {\slv{поверг-о́хъ~\textemdash~пове́рж-е}},
        {\slv{тек-о́хъ~\textemdash~теч-ѐ}},
        {\slv{мог-о́хъ~\textemdash~мо́ж-е}}
    }
    
    В остальных же глаголах 2-е и 3-е лицо единственного числа оканчивается на конечный гласный звук основы. Например:

    \bigskip\autorows{c}{1}{c}{
        {\slv{глаго́ла-хъ~\textemdash~глаго́ла}},
        {\slv{хвали́-хъ~\textemdash~хвалѝ}}
    }

    Глагол {\slv{ꙗ҆́ти}} в 1-м лице единственного числа аориста имеет форму {\slv{ꙗхъ}}, а во 2-м и 3-м лице к корню {\slv{ꙗ-}} присоединяется окончание {\slv{-тъ}}: {\slv{ꙗ҆́тъ}}.
    
    Если этот глагол употребляется с приставками {\slv{при-}}, {\slv{под-}} и т.п., то 2-е лицо единственного числа оканчивается на конечный гласный звук основы (т.е. на корень {\slv{-ꙗ-}}), а в 3-м лице присоединяется к основе окончание {\slv{-тъ}}. Например:

    \bigskip\autorows{c}{4}{l}{
        {1-е лицо:}, {\slv{прїѧ́-хъ}}, {\slv{под̾ѧ́-хъ}}, {\slv{взѧ́-хъ}},
        {2-е лицо:}, {\slv{прїѧ̀}}, {\slv{под̾ѧ}}, {\slv{взѧ̀}},
        {3-е лицо:}, {\slv{прїѧ́-тъ}}, {\slv{под̾ѧ́-тъ}}, {\slv{взѧ́-тъ}}
    }
    
    Глагол {\slv{рещѝ}} в аористе во всех числах имеет две формы:

    \begin{center}
        %        Аорист тематического глагола {\slv{рещѝ}}
        \renewcommand*{\arraystretch}{1.2}
        \begin{tabular}[c]{|c|c|c|c|c|}
            \hline
            
            \multirow{2}{*}{\spheading[2em]{\scriptsize{Лицо}}}
            & \multirow{2}{*}{\scriptsize\makecell{Единственное\\число}}
            & \multicolumn{2}{c|}{\scriptsize\makecell{Двойственное число}}
            & \multirow{2}{*}{\scriptsize\makecell{Множественное\\число}}
            \\
            
            \cline{3-4}
            
            &
            & \scriptsize\makecell{Мужской род}
            & \scriptsize\makecell{Жен. и сред. род}
            &
            \\
            
            \hline
            
            1
            & \makecell{{\slv{реко́хъ}}\\{\slv{рѣ́хъ}}}
            & \makecell{{\slv{реко́хова}}\\{\slv{рѣ́хова}}}
            & \makecell{{\slv{реко́ховѣ}}\\{\slv{рѣ́ховѣ}}}
            & \makecell{{\slv{реко́хомъ}}\\{\slv{рѣ́хомъ}}}
            \\\hline
            
            2
            & \makecell{{\slv{ре́че}}\\~}
            & \makecell{{\slv{реко́ста}}\\{\slv{рѣ́ста}}}
            & \makecell{{\slv{реко́стѣ}}\\{\slv{рѣ́стѣ}}}
            & \makecell{{\slv{реко́сте}}\\{\slv{рѣ́сте}}}
            \\\hline
            
            3
            & \makecell{{\slv{ре́че}}\\~}
            & \makecell{{\slv{реко́ста}}\\{\slv{рѣ́ста}}}
            & \makecell{{\slv{реко́стѣ}}\\{\slv{рѣ́стѣ}}}
            & \makecell{{\slv{реко́ша}}\\{\slv{рѣ́ша}}}
            \\\hline
            
        \end{tabular}
    \end{center}

                    \bigskip\paragraph{Упражнение}

Поставьте приведенные в скобках церковнославянские слова в соответствующую форму аориста. Русский глагол пропускается.
    
    \begin{flushleft}
        \renewcommand*{\arraystretch}{1.2}
        \begin{tabular}[l]{cll}
            
            ~~~~~
            & \emph{Образец}:
            & \makecell[l]{И ныне, как только Он \textbf{возвысил} ({\slv{вознестѝ}}) главу мою\\над врагами, я \textbf{обошел} ({\slv{ѡ҆бытѝ}}) и принес в скинии\\\textbf{жертву} ({\slv{пожре́ти}}) хвалы и восклицания (Пс. 26, 6).}
            \\
            
            ~~~~~
            &
            &
            \\
            
            ~~~~~
            & \emph{Ответ}:
            & \makecell[l]{И ныне, как только Он {\slv{вознесѐ}} главу мою\\над врагами, я {\slv{ѡ҆быдо́хъ}} и {\slv{пожро́хъ}} в скинии\\жертву хвалы и восклицания.}
            \\
            
        \end{tabular}
    \end{flushleft}

    1. \textbf{Услышал} ({\slv{слы́шати}}) Господь и \textbf{помиловал} ({\slv{поми́ловати}}) меня, Господь \textbf{стал} ({\slv{бы́ти}}) помощником мне (Пс. 29, 11).
    
    2. Беззаконие мое я \textbf{создал} ({\slv{позна́ти}}) и греха моего не \textbf{скрыл} ({\slv{покры́ти}}), \textbf{сказал} ({\slv{рещѝ}}): <<Исповедуюсь Господу в беззаконии моем\ldots>> (Пс. 31, 5).
    
    3. Но я на Тебя, Господи, \textbf{уповал} ({\slv{ᲂу҆пова́ти}}), сказал: Ты~---~Бог мой! (Пс. 30, 15).
    
    4. Ибо они без вины \textbf{скрыли} ({\slv{скры́ти}}) для меня пагубную сеть своего, напрасно \textbf{поносили} ({\slv{поноси́ти}}) душу мою (Пс. 34, 7).
    
    5. Ибо \textbf{окружили} ({\slv{ѡ҆бытѝ}}) меня беды, коим нет числа, \textbf{постигли} ({\slv{пости́гнꙋти}}) меня беззакония мои, так что я не \textbf{мог} ({\slv{возмощѝ}}) смотреть, \textbf{умножились} ({\slv{ᲂу҆мно́житисѧ}}) более волос на голове моей, и сердце мое \textbf{оставило} ({\slv{ѡ҆ста́вити}}) меня (Пс. 39, 13).
    
    6. Все сие \textbf{постигло} (букв.: <<постигли>>~---~глагол {\slv{прїитѝ}}) нас, но мы не забыли ({\slv{забы́ти}}) Тебя и не \textbf{нарушили} ({\slv{непра́вдовати}}) завета Твоего (Пс. 43, 18).
    
    7. \textbf{Восшел} ({\slv{взы́ти}}) Бог при восклицании, Господь при звуке трубном (Пс. 46, 6).
    
    8. И \textbf{убоялся} ({\slv{ᲂу҆боѧ́тисѧ}}) всякий человек и \textbf{возвестили} ({\slv{возвѣсти́ти}}) о делах Бога и \textbf{уразумели} ({\slv{разꙋмѣ́ти}}) действия Его (Пс. 63, 10).
    
    9. Войду в дом Твой со всесожжением, исполню пред Тобою обеты мои, которые \textbf{изрекли} ({\slv{и҆зрещѝ}}) уста моя и \textbf{произнес} (букв.: <<произнесли>>~---~{\slv{глаго́лати}}) язык мой в скорби моей (Пс. 65, 13--14).
    
    10. К нему устами моими я \textbf{воззвал} ({\slv{воззва́ти}}) и \textbf{превознес} ({\slv{вознестѝ}}) Его языком моим (Пс. 65, 17).
    
    11. И \textbf{отверг} ({\slv{ѿри́нꙋти}}) селение Иосифа и колена Ефремова не \textbf{избрал} ({\slv{и҆збра́ти}}) (Пс. 77, 67).
    
                    \paragraph{Упражнение}
                    
    Поставьте приведенные в скобках церковнославянские слова в соответствующую форму преходящего времени или аориста.
    
    1. Гордые до крайности \textbf{преступали закон} ({\slv{законопрестꙋпова́ти}}), а я не \textbf{уклонялся} ({\slv{ᲂу҆клони́тисѧ}}) от закона Твоего (Пс. 118, 51).
    
    2. Воздающим мне злом за добро \textbf{клеветали} ({\slv{ѡ҆болга́ти}}) на меня, так как я \textbf{следовал} ({\slv{гонѧ́ти}}) добру (Пс. 37, 21).
    
    3. Хотя они честь мою \textbf{замыслили} ({\slv{совѣща́ти}}) низринути, жадно \textbf{спешили} ({\slv{тещѝ}}), устами своими \textbf{благословляли} ({\slv{благословлѧ́ти}}), а сердцем своим \textbf{проклинали} ({\slv{клѧ́ти}}) (Пс. 61, 5).
    
    4. Но они, вскричав громким голосом, \textbf{затыкали} ({\slv{затыка́ти}}) уши свои и единодушно \textbf{устремились} ({\slv{ᲂу҆стреми́тисѧ}}) на него (Деян. 7, 57).
    
    5. И когда Петр \textbf{пришел} ({\slv{взы́ти}}) в Иерусалим, обрезанные \textbf{упрекали} ({\slv{препира́тисѧ}}) его (Деян. 11, 2).
    
    6. Язычники, слыша это, \textbf{радовались} ({\slv{ра́доватисѧ}}) и \textbf{прославляли} ({\slv{сла́вити}}) (Деян. 13, 48).
    
    7. Дойдя до Мисии, \textbf{предпринимали} ({\slv{покꙋша́тисѧ}}) идти в Вифинию; но Дух \textbf{не допустил} ({\slv{ѡ҆ста́вити}}) их (Деян. 16, 7).
    
    8. И, придя, \textbf{извинились} ({\slv{ᲂу҆моли́ти}}) перед ними и, выведя, \textbf{просили} ({\slv{моли́ти}}) удалиться из города (Деян. 16, 39).
    
    9. Услышав о воскресении мертвых, одни \textbf{насмехались} ({\slv{рꙋга́тисѧ}}) а другие \textbf{говорили} ({\slv{рещѝ}}): об этом послушаем тебя в другое время (Деян. 17, 32).
    
    10. Весь город \textbf{пришел в движение} ({\slv{подви́гнꙋтисѧ}}), и \textbf{сделалось} ({\slv{бы́ти}}) стечение народа; и, схватив Павла, \textbf{повлекли} ({\slv{влещѝ}}) его вон из храма, и тотчас \textbf{заперты были} ({\slv{затвори́тисѧ}}) двери (Деян. 21, 30).
    
    11. В приведенном ниже церковнославянском тексте замените формы выделенных глаголов, стоящих во множественном числе, на соответствующие формы единственного числа (подобные случаи нередко встречаются в богослужебной практике):
    
    И аще что яко человецы плоть носяще, и в мире живуще, от диавола \textbf{прельстишаяся}. Аще же в слове или в деле или в ведении или в неведении, или слово священническое \textbf{попраша}, или под клятвою священническом \textbf{быша}, или под свою анафему \textbf{падоша}, или под клятву ведошася, Сам яко благ и незлобивый Владыко, сия рабы Твоя словом разрешися благоволи (Требник. М., 1991, с. 76).

                    \paragraph{Упражнение}
                    
    В приведенном ниже церковнославянском тексте замените формы выделенных глаголов, стоящих во единственном числе, на соответствующие формы множественного числа (подобные случаи нередко встречаются в богослужебной практике):
    
    \ldots и аще что \textbf{согреши} словом, или делом, или помышлением, или в нощи, или во дни; или под клятву священническую или своему проклятию \textbf{подпаде}; или клятвою \textbf{огоричся}, и \textbf{проклят} себе; Тебе просим, и Тебе молимся, ослаби, остави, прости ему, Боже\ldots~и аще что от заповедей Твоих \textbf{преступи} или \textbf{согреши}\ldots~Сам яко Благ и Человеколюбец Бог прости\ldots~(Требник. М., 1991, с. 351--352).

                \subsubsection{Глаголы вида совершенного и несовершенного}

    Глаголы в церковнославянском языке бывают вида совершенного и несовершенного.
    
    \textbf{Совершенный вид} глаголов выражает действие или состояние с определенной продолжительностью, под условием оконченности совершения его. Например: {\slv{взѧ́ти, повелѣ́ти}}.
    
    \textbf{Несовершенный вид} глаголов выражает действие или состояние с оттенком неопределенной продолжительности. Например: {\slv{плы́ти, повелѣва́ти}}.
    
    Совершенный вид глаголов, кроме вспомогательного глагола {\slv{бы́ти}}, не может иметь настоящего времени и преходящего.

                \subsubsection{Будущее время глаголов}

    Глагол в будущем времени служит для выражения действия или состояния, которое должно совершиться впоследствии. Например:

    \bigskip\autorows{c}{1}{c}{
        {{\slv{Ѿ со́нмищъ {\large и҆жденꙋ́тъ} вы̀}} (Ин. 16, 2)}
    }

    Будущее время в церковнославянском языке может быть \textbf{простое} и \textbf{составное} (сложное).
    
    \underline{1. Будущее время простое}
    
    Будущее время простое имеет глаголы совершенного вида. Образование и спряжение этого времени вполне сходно с образованием и спряжением настоящего времени глаголов несовершенного вида, например: {\slv{несꙋ̀~\textemdash~понесꙋ̀, творю̀~\textemdash~сотворю̀}}.
    
    Вспомогательный глагол {\slv{бы́ти}} в этом времени также имеет окончания, сходные с окончаниями тематических глаголов.
    
    Вот образец этого спряжения:

    \begin{center}
        %        Будущее время вспомогательного глагола {\slv{бы́ти}}
        \renewcommand*{\arraystretch}{1.2}
        \begin{tabular}[c]{|c|c|c|c|c|}
            \hline
            
            \multirow{2}{*}{\spheading[2em]{\scriptsize{Лицо}}}
            & \multirow{2}{*}{\scriptsize\makecell{Единственное\\число}}
            & \multicolumn{2}{c|}{\scriptsize\makecell{Двойственное число}}
            & \multirow{2}{*}{\scriptsize\makecell{Множественное\\число}}
            \\
            
            \cline{3-4}
            
            &
            & \scriptsize\makecell{Мужской род}
            & \scriptsize\makecell{Жен. и сред. род}
            &
            \\
            
            \hline
            
            1
            & \makecell{{\slv{бꙋ́дꙋ}}}
            & \makecell{{\slv{бꙋ́дева}}}
            & \makecell{{\slv{бꙋ́девѣ}}}
            & \makecell{{\slv{бꙋ́демъ}}}
            \\\hline
            
            2
            & \makecell{{\slv{бꙋ́деши}}}
            & \makecell{{\slv{бꙋ́дета}}}
            & \makecell{{\slv{бꙋ́детѣ}}}
            & \makecell{{\slv{бꙋ́дете}}}
            \\\hline
            
            3
            & \makecell{{\slv{бꙋ́детъ}}}
            & \makecell{{\slv{бꙋ́дета}}}
            & \makecell{{\slv{бꙋ́детѣ}}}
            & \makecell{{\slv{бꙋ́дꙋтъ}}}
            \\\hline
            
        \end{tabular}
    \end{center}

    \underline{2. Будущее время составное}
    
    Будущее время составное имеют глаголы несовершенного вида. Образуется оно из будущего времени глагола {\slv{бы́ти}} и неопределенной формы глагола. Например:
    
    \autorows{c}{1}{l}{
        {{\slv{бꙋ́дꙋ нестѝ, бꙋ́деши нестѝ}} и т.д.},
        {{\slv{бꙋ́дꙋ слы́шати, бꙋ́деши слы́шати}} и т.д.}
    }

    Вспомогательный глагол {\slv{бы́ти}}, как глагол совершенного вида, составного будущего времени не имеет.

                \subsubsection{Понятие о наклонениях глагола}

    \textbf{Наклонениями глагола} называются такие его формы, которые служат для выражения различных способов действий или состояний.
    
    До сего времени рассматривались формы изъявительного наклонения.
    
    \textbf{Изъявительное наклонение} выражает определенное действие или состояние предмета или явления в настоящем, или прошедшем, или будущем времени. Например: {\slv{спаса́ю, спасо́хъ, спа́сꙋ}}.

                \subsubsection{Желательное наклонение глаголов}

    Желательное наклонение глаголов выражает действие, которое можно пожелать в будущем. Поэтому оно имеет только одно будущее время, которое образуется из частицы {\slv{да}} и настоящего или будущего времени спрягаемого глагола. Например: {\slv{да просвѣща́ю, да просвещꙋ̀}}.
    
%    Вот образцы спряжения в этом наклонении глаголов {\slv{бы́ти}} и {\slv{хвали́ти}}:

\begin{center}
    Спряжение в желательном наклонении глаголов {\slv{бы́ти}} и {\slv{хвали́ти}}
    \renewcommand*{\arraystretch}{1.2}
    \begin{tabular}[c]{|c|c|c|c|c|c|}
        \hline
        
        \multirow{2}{*}{\spheading[2.5em]{\scriptsize{Глагол}}}
        &\multirow{2}{*}{\spheading[2em]{\scriptsize{Лицо}}}
        & \multirow{2}{*}{\scriptsize\makecell{Единственное\\число}}
        & \multicolumn{2}{c|}{\scriptsize\makecell{Двойственное число}}
        & \multirow{2}{*}{\scriptsize\makecell{Множественное\\число}}
        \\
        
        \cline{4-5}
        
        &
        &
        & \scriptsize\makecell{Мужской род}
        & \scriptsize\makecell{Жен. и сред. род}
        &
        \\
        
        \hline
        
        \multirow{3}{*}{\spheading[3em]{\slv{бы́ти}}}
        &1
        & \makecell{{\slv{да бꙋ́дꙋ}}}
        & \makecell{{\slv{да бꙋ́дева}}}
        & \makecell{{\slv{да бꙋ́девѣ}}}
        & \makecell{{\slv{да бꙋ́демъ}}}
        \\\cline{2-6}
        
        &2
        & \makecell{{\slv{да бꙋ́деши}}}
        & \makecell{{\slv{да бꙋ́дета}}}
        & \makecell{{\slv{да бꙋ́детѣ}}}
        & \makecell{{\slv{да бꙋ́дете}}}
        \\\cline{2-6}
        
        &3
        & \makecell{{\slv{да бꙋ́детъ}}}
        & \makecell{{\slv{да бꙋ́дета}}}
        & \makecell{{\slv{да бꙋ́детѣ}}}
        & \makecell{{\slv{да бꙋ́дꙋтъ}}}
        \\\hline

        \multirow{3}{*}{\spheading[5.5em]{\slv{хвали́ти}}}
        &1
        & \makecell{{\slv{да хвалю̀}}\\{\slv{да похвалю̀}}}
        & \makecell{{\slv{да хва́лива}}\\{\slv{да похва́лива}}}
        & \makecell{{\slv{да хва́ливѣ}}\\{\slv{да похва́ливѣ}}}
        & \makecell{{\slv{да хва́лимъ}}\\{\slv{да похва́лимъ}}}
        \\\cline{2-6}
        
        &2
        & \makecell{{\slv{да хва́лиши}}\\{\slv{да похва́лиши}}}
        & \makecell{{\slv{да хва́лита}}\\{\slv{да похва́лита}}}
        & \makecell{{\slv{да хва́литѣ}}\\{\slv{да похва́литѣ}}}
        & \makecell{{\slv{да хва́лите}}\\{\slv{да похва́лите}}}
        \\\cline{2-6}
        
        &3
        & \makecell{{\slv{да хва́литъ}}\\{\slv{да похва́литъ}}}
        & \makecell{{\slv{да хва́лита}}\\{\slv{да похва́лита}}}
        & \makecell{{\slv{да хва́литѣ}}\\{\slv{да похва́литѣ}}}
        & \makecell{{\slv{да хва́лѧтъ}}\\{\slv{да похва́лѧтъ}}}
        \\\hline
        
    \end{tabular}
\end{center}

                \subsubsection{Повелительное наклонение глаголов}

    Повелительное наклонение глаголов выражает просьбу или требование совершить действие. Например: {\slv{спасѝ! повели́те!}}
    
    Особенность этого наклонения та, что оно не имеет времен; кроме того, в единственном числе имеет только второе лицо, а в двойственном или множественном~---~только 1-е и 2-е.
    
    Рассмотрим спряжение в этом наклонении тех же глаголов {\slv{бы́ти, хвали́ти}}.

    \begin{center}
        %        Спряжение в повелительном наклонении глаголов {\slv{бы́ти}} и {\slv{хвали́ти}}
        \renewcommand*{\arraystretch}{1.2}
        \begin{tabular}[c]{|c|c|c|c|c|c|}
            \hline
            
            \multirow{2}{*}{\spheading[2.5em]{\scriptsize{Глагол}}}
            &\multirow{2}{*}{\spheading[2em]{\scriptsize{Лицо}}}
            & \multirow{2}{*}{\scriptsize\makecell{Единственное\\число}}
            & \multicolumn{2}{c|}{\scriptsize\makecell{Двойственное число}}
            & \multirow{2}{*}{\scriptsize\makecell{Множественное\\число}}
            \\
            
            \cline{4-5}
            
            &
            &
            & \scriptsize\makecell{Мужской род}
            & \scriptsize\makecell{Жен. и сред. род}
            &
            \\
            
            \hline
            
            \multirow{2}{*}{\spheading[2.3em]{\scriptsize\slv{бы́ти}}}
            &1
            & \makecell{--}
            & \makecell{{\slv{бꙋ́дива}}}
            & \makecell{{\slv{бꙋ́дивѣ}}}
            & \makecell{{\slv{бꙋ́димъ}}}
            \\\cline{2-6}
            
            &2
            & \makecell{{\slv{бꙋ́дꙋ}}}
            & \makecell{{\slv{бꙋ́дита}}}
            & \makecell{{\slv{бꙋ́дитѣ}}}
            & \makecell{{\slv{бꙋ́дите}}}
            \\\hline
            
            \multirow{2}{*}{\spheading[2.9em]{\scriptsize\slv{хвали́ти}}}
            &1
            & \makecell{--}
            & \makecell{{\slv{хвали́ва}}}
            & \makecell{{\slv{хвали́вѣ}}}
            & \makecell{{\slv{хвали́мъ}}}
            \\\cline{2-6}
            
            &2
            & \makecell{{\slv{хвалѝ}}}
            & \makecell{{\slv{хвали́та}}}
            & \makecell{{\slv{хвали́тѣ}}}
            & \makecell{{\slv{хвали́те}}}
            \\\hline
            
        \end{tabular}
    \end{center}

    Повелительное наклонение тематических глаголов образуется от неопределенной формы глаголов следующим образом: окончание {\slv{-ти}} (и с простым суффиксом, если он имеется) отбрасывается и к основе присоединяется окончание {\slv{-и}}. Например: {\slv{нес-тѝ~\textemdash~нес-ѝ, повел-ѣ́-ти~\textemdash~повел-ѝ, хвал-и́-ти~\textemdash~хвал-ѝ, ꙗ҆́-ти~\textemdash~є҆́мл-и}}.
    
    Если же основа глагола оканчивается на гласный {\slv{а}}, то вместо окончания неопределенной формы {\slv{-ти}} непосредственно присоединяется окончание {\slv{й}}. Например: {\slv{повелѣва́-ти~\textemdash~повелѣва́-й, пита́-ти~\textemdash~пита́-й}}.
    
    Окончание 2-го лица единственного числа {\slv{-и}} иногда меняется на {\slv{-ь}}. Например: {\slv{да́ждь}} (вместо {\slv{да́жди}}), {\slv{ви́ждь}} (вместо {\slv{ви́жди}}).
    
    При наличии в основе гортанных {\slv{г}} и {\slv{к}} последние перед гласным {\slv{и}} смягчаются соответственно в {\slv{з}} и {\slv{ц}} по второму закону смягчения, причем после {\slv{ц}} пишется не {\slv{-и}}, но {\slv{-ы}}. Например: {\slv{помощѝ}} ({\slv{помог-тѝ}})~---~{\slv{помозѝ}}, {\slv{тещѝ}} ({\slv{тек-тѝ}})~---~{\slv{тецы̀}}, {\slv{рещѝ}} ({\slv{рек-тѝ}})~---~{\slv{рцы̀}}.
    
    Третье лицо желательного наклонения употребляется иногда в значении третьего лица повелительного наклонения. Например:
    
    \bigskip\autorows{c}{1}{l}{
        {{\slv{Да бꙋ́детъ свѣ́тъ}} (Быт. 1, 3)}
    }

                \subsubsection{Спряжение архаических глаголов}

    Рассмотрим особенности спряжения архаических глаголов: {\slv{вѣ́дѣти, ꙗ҆́сти, да́ти}} и {\slv{и҆мѣ́ти}}.
    
    Эти особенности касаются настоящего времени изъявительного наклонения глаголов {\slv{вѣ́дети, ꙗ҆́сти}} и {\slv{и҆мѣ́ти}}; в отношении глагола {\slv{да́ти}} особенности касаются аориста и будущего времени. В отношении же повелительного наклонения все четыре глагола имеют свои особенности.

    \begin{center}
        %        Спряжение архаических глаголов {\slv{вѣ́дѣти}}, {\slv{ꙗ҆́сти}} и {\slv{и҆мѣ́ти}}
        \renewcommand*{\arraystretch}{1.2}
        \begin{tabular}[c]{|c|c|c|c|c|c|}
            \hline
            
            \multirow{2}{*}{\spheading[2.5em]{\scriptsize{~}}}
            &\multirow{2}{*}{\spheading[2em]{\scriptsize{Лицо}}}
            & \multirow{2}{*}{\scriptsize\makecell{Единственное\\число}}
            & \multicolumn{2}{c|}{\scriptsize\makecell{Двойственное число}}
            & \multirow{2}{*}{\scriptsize\makecell{Множественное\\число}}
            \\\cline{4-5}
            
            &
            &
            & \scriptsize\makecell{Мужской род}
            & \scriptsize\makecell{Жен. и сред. род}
            &
            \\\hline
            
            \multicolumn{6}{|c|}{{\slv{вѣ́дети}}}
            \\\hline
            
            \multirow{2}{*}{\spheading[4em]{\scriptsize Настоящее время}}
            &{\small 1}
            & \makecell{{\slv{вѣ́мъ}}}
            & \makecell{{\slv{вѣ́ва}}}
            & \makecell{{\slv{вѣ́вѣ}}}
            & \makecell{{\slv{вѣ́мы}}}
            \\\cline{2-6}
            
            &{\small 2}
            & \makecell{{\slv{вѣ́си}}}
            & \makecell{{\slv{вѣ́ста}}}
            & \makecell{{\slv{вѣ́стѣ}}}
            & \makecell{{\slv{вѣ́сте}}}
            \\\cline{2-6}
            
            &{\small 3}
            & \makecell{{\slv{вѣ́сть}}}
            & \makecell{{\slv{вѣ́ста}}}
            & \makecell{{\slv{вѣ́стѣ}}}
            & \makecell{{\slv{вѣ́дѧтъ}}}
            \\\hline
            
            \multirow{2}{*}{\spheading[2.9em]{\scriptsize Повелит. наклон.}}
            &{\small 1}
            & \makecell{--}
            & \makecell{{\slv{вѣ́дива}}}
            & \makecell{{\slv{вѣ́дивѣ}}}
            & \makecell{{\slv{вѣ́димъ}}}
            \\\cline{2-6}
            
            &{\small 2}
            & \makecell{{\slv{вѣ́ждь}}}
            & \makecell{{\slv{вѣ́дита}}}
            & \makecell{{\slv{вѣ́дитѣ}}}
            & \makecell{{\slv{вѣ́дите}}}
            \\\hline
    
            \multicolumn{6}{|c|}{{\slv{ꙗ҆́сти}}}
            \\\hline
            
            \multirow{2}{*}{\spheading[4em]{\scriptsize Настоящее время}}
            &{\small 1}
            & \makecell{{\slv{ꙗ҆́мъ}}}
            & \makecell{{\slv{ꙗ҆́ва}}}
            & \makecell{{\slv{ꙗ҆́вѣ}}}
            & \makecell{{\slv{ꙗ҆́мы}}}
            \\\cline{2-6}
            
            &{\small 2}
            & \makecell{{\slv{ꙗ҆́си}}}
            & \makecell{{\slv{ꙗ҆́ста}}}
            & \makecell{{\slv{ꙗ҆́стѣ}}}
            & \makecell{{\slv{ꙗ҆́сте}}}
            \\\cline{2-6}
            
            &{\small 3}
            & \makecell{{\slv{ꙗ҆́стъ}}}
            & \makecell{{\slv{ꙗ҆́ста}}}
            & \makecell{{\slv{ꙗ҆́стѣ}}}
            & \makecell{{\slv{ꙗ҆́дѧтъ}}}
            \\\hline
            
            \multirow{2}{*}{\spheading[2.9em]{\scriptsize Повелит. наклон.}}
            &{\small 1}
            & \makecell{--}
            & \makecell{{\slv{ꙗ҆ди́ва}}}
            & \makecell{{\slv{ꙗ҆ди́вѣ}}}
            & \makecell{{\slv{ꙗ҆ди́мъ}}}
            \\\cline{2-6}
            
            &{\small 2}
            & \makecell{{\slv{ꙗ҆́ждь}}}
            & \makecell{{\slv{ꙗ҆ди́та}}}
            & \makecell{{\slv{ꙗ҆ди́тѣ}}}
            & \makecell{{\slv{ꙗ҆ди́те}}}
            \\\hline
    
            \multicolumn{6}{|c|}{{\slv{и҆мѣ́ти}}}
            \\\hline
            
            \multirow{2}{*}{\spheading[5em]{\scriptsize Настоящее время}}
            &{\small 1}
            & \makecell{{\slv{и҆́мамъ}}}
            & \makecell{{\slv{и҆́мава}}}
            & \makecell{{\slv{и҆́мавѣ}}}
            & \makecell{{\slv{и҆́мамы}}}
            \\\cline{2-6}
            
            &{\small 2}
            & \makecell{{\slv{и҆́маши}}}
            & \makecell{{\slv{и҆́мата}}}
            & \makecell{{\slv{и҆́матѣ}}}
            & \makecell{{\slv{и҆́мате}}}
            \\\cline{2-6}
            
            &{\small 3}
            & \makecell{{\slv{и҆́мать}}}
            & \makecell{{\slv{и҆́мата}}}
            & \makecell{{\slv{и҆́матѣ}}}
            & \makecell{{\slv{и҆́мꙋтъ}}\\{\slv{и҆мѣ́ютъ}}}
            \\\hline
            
            \multirow{2}{*}{\spheading[3.5em]{\scriptsize Повелит. наклон.}}
            &{\small 1}
            & \makecell{--}
            & \makecell{{\slv{и҆мѣ́йва}}}
            & \makecell{{\slv{и҆мѣ́йвѣ}}}
            & \makecell{{\slv{и҆мѣ́имъ}}}
            \\\cline{2-6}
            
            &{\small 2}
            & \makecell{{\slv{и҆мѣ́й}}\\{\slv{и҆мѝ}}}
            & \makecell{{\slv{и҆мѣ́йта}}\\{\slv{и҆ми́та}}}
            & \makecell{{\slv{и҆мѣ́йтѣ}}\\{\slv{и҆ми́тѣ}}}
            & \makecell{{\slv{и҆мѣ́йте}}\\{\slv{и҆ми́те}}}
            \\\hline
            
        \end{tabular}
    \end{center}

    При рассмотрении спряжения архаических глаголов {\slv{вѣ́дѣти, ꙗ҆́сти}}, и {\slv{и҆мѣ́ти}} необходимо заметить:
    
    1. Глагол {\slv{вѣ́дѣти}}, как глагол несовершенного вида, имеет только будущее составное время ({\slv{бꙋ́дꙋ вѣ́дѣти}}). В будущем простом он употребляется с какой-либо приставкой (например, {\slv{ᲂу҆вѣ́дети}}), т.е. приобретает совершенный вид и спрягается по образу настоящего времени.
    
    2. Глагол {\slv{ꙗ҆́сти}}, как глагол также несовершенного вида, употребляется в будущем простом времени с приставкой {\slv{с}}; причем звук {\slv{ꙗ}} после {\slv{с}} переходит в {\slv{ѣ}} и между ними для благозвучия вставляется согласный звук {\slv{н}}; получается глагол совершенного вида {\slv{снѣ́сти}}.
    
    3. Глагол {\slv{имѣ́ти}}, как глагол несовершенного вида, будущего простого времени не имеет.
    
    Глагол {\slv{да́ти}}, как глагол совершенного вида, в изъявительном наклонении не имеет настоящего времени, преходящего и будущего составного. Для этих глагольных форм он заменяется производным глаголом несовершенного вида {\slv{даѧ́ти}}, который представляет собою уже тематический глагол 1-го спряжения.
    
    Во 2-м и 3-м лице единственного числа аориста глагол {\slv{да́ти}} имеет форму {\slv{дадѐ}} (вместо {\slv{да}}), а в 3-м лице множественного числа того же времени имеет две формы: {\slv{да́ша}} и {\slv{дадо́ша}}.

    \begin{center}
        %        Спряжение глагола совершенного вида {\slv{да́ти}}
        \renewcommand*{\arraystretch}{1.2}
        \begin{tabular}[c]{|c|c|c|c|c|c|}
            \hline
            
            \multirow{2}{*}{\spheading[2.5em]{\scriptsize{~}}}
            &\multirow{2}{*}{\spheading[2em]{\scriptsize{Лицо}}}
            & \multirow{2}{*}{\scriptsize\makecell{Единственное\\число}}
            & \multicolumn{2}{c|}{\scriptsize\makecell{Двойственное число}}
            & \multirow{2}{*}{\scriptsize\makecell{Множественное\\число}}
            \\\cline{4-5}
            
            &
            &
            & \scriptsize\makecell{Мужской род}
            & \scriptsize\makecell{Жен. и сред. род}
            &
            \\\hline
            
            \multirow{3}{*}{\spheading[4em]{\scriptsize Аорист}}
            &{\small 1}
            & \makecell{{\slv{да́хъ}}}
            & \makecell{{\slv{да́хова}}}
            & \makecell{{\slv{да́ховѣ}}}
            & \makecell{{\slv{да́хомъ}}}
            \\\cline{2-6}
            
            &{\small 2}
            & \makecell{{\slv{дадѐ}}}
            & \makecell{{\slv{да́ста}}}
            & \makecell{{\slv{да́стѣ}}}
            & \makecell{{\slv{да́сте}}}
            \\\cline{2-6}
            
            &{\small 3}
            & \makecell{{\slv{дадѐ}}}
            & \makecell{{\slv{да́ста}}}
            & \makecell{{\slv{да́стѣ}}}
            & \makecell{{\slv{да́ша}}\\{\slv{дадо́ша}}}
            \\\hline
            
            \multirow{3}{*}{\spheading[4em]{\scriptsize Будущее время}}
            &{\small 1}
            & \makecell{{\slv{да́мъ}}}
            & \makecell{{\slv{да́дива}}}
            & \makecell{{\slv{да́дивѣ}}}
            & \makecell{{\slv{да́мы}}}
            \\\cline{2-6}
            
            &{\small 2}
            & \makecell{{\slv{да́си}}}
            & \makecell{{\slv{дади́та}}}
            & \makecell{{\slv{дади́тѣ}}}
            & \makecell{{\slv{дади́те}}}
            \\\cline{2-6}
            
            &{\small 3}
            & \makecell{{\slv{да́стъ}}}
            & \makecell{{\slv{дади́та}}}
            & \makecell{{\slv{дади́тѣ}}}
            & \makecell{{\slv{дадꙋ́тъ}}\\{\slv{дадѧ́ть}}}
            \\\hline
            
            \multirow{2}{*}{\spheading[2.9em]{\scriptsize Повелит. наклон.}}
            &{\small 1}
            & \makecell{--}
            & \makecell{{\slv{дади́ва}}}
            & \makecell{{\slv{дади́вѣ}}}
            & \makecell{{\slv{дади́мъ}}}
            \\\cline{2-6}
            
            &{\small 2}
            & \makecell{{\slv{да́ждь}}}
            & \makecell{{\slv{дади́та}}}
            & \makecell{{\slv{дади́тѣ}}}
            & \makecell{{\slv{да́жд̾те}}}
            \\\hline
            
        \end{tabular}
    \end{center}

    \addtocounter{paragraph}{1} % Following the original typo.
    
                    \bigskip\paragraph{Упражнение}
    
    В приведенных предложениях замените выделенный глагол на соответствующую форму церковнославянского глагола {\slv{вѣ́дѣти}}.
    
    \begin{flushleft}
        \renewcommand*{\arraystretch}{1.2}
        \begin{tabular}[l]{cll}
            
            ~~~~~
            & \emph{Образец}:
            & \makecell[l]{Вы сами \textbf{знаете}, что я всеми силами служил\\отцу вашему (Быт. 31, 6).}
            \\
            
            ~~~~~
            &
            &
            \\
            
            ~~~~~
            & \emph{Ответ}:
            & \makecell[l]{Вы сами {\slv{вѣ́сте}}, что я всеми силами служил\\отцу вашему.}
            \\
            
        \end{tabular}
    \end{flushleft}

    1.Пойти пойду с тобою; только \textbf{знай}, что не тебе уже будет слава на сем пути (Суд. 4, 9).
    
    2. И если \textbf{знаешь}, что между ними есть способные люди, поставь их смотрителями над моим скотом (Быт. 47, 6).
    
    3. Путь же беззаконных~---~как тьма; они не \textbf{знают}, обо что споткнутся (Притч. 4, 19).
    
    4. Пришельца не обижай и не притесняй его: вы \textbf{знаете} душу пришельца (Исх. 23, 9).
    
    5. Но Я \textbf{знаю}, что фараон, царь Египетский, не позволит вам идти (Исх. 3, 19).
    
    6. Но доколе не придем туда, мы не \textbf{знаем}, что принести в жертву Господу Богу нашему (Исх. 10, 26).
    
    7. Так, и когда вы увидите то сбывающимся, \textbf{знайте}, что близко Царствие Божие (Лк. 21, 31).
    
    8. Слышала ли ты, что Адония, сын Аггифин, сделался царем, а господин наш Давид не \textbf{знает} о том? (3 Цар. 1, 11).
    
                \paragraph{Упражнение}

    В приведенных предложениях замените выделенный глагол на соответствующую форму церковнославянского глагола {\slv{ꙗ҆́сти}}.
    
    \begin{flushleft}
        \renewcommand*{\arraystretch}{1.2}
        \begin{tabular}[l]{cll}
            
            ~~~~~
            & \emph{Образец}:
            & \makecell[l]{Итак берегись, не пей вина и сикера, и не \textbf{ешь} ничего\\нечистого (Суд. 13, 4).}
            \\
            
            ~~~~~
            &
            &
            \\
            
            ~~~~~
            & \emph{Ответ}:
            & \makecell[l]{Итак берегись, не пей вина и сикера, и не {\slv{ꙗ҆́ждь}} ничего\\нечистого.}
            \\
            
        \end{tabular}
    \end{flushleft}

    1. Горе тебе, земля, когда царь твой отрок, и когда князья твои \textbf{едят} рано (Еккл. 10, 16).
    
    2. И когда вы \textbf{едите} и когда пьете, не для себя ли вы \textbf{едите}, не для себя ли вы пьете? (Зах. 7, 6).
    
    3. И если какой человек \textbf{ест} и пьет, и видит доброе во всяком труде своем, то это~---~дар Божий (Еккл. 3, 13).
    
    4. \textbf{Ем} ли Я мясо волов и пью ли кровь козлов? (Пс. 49, 13).
    
    5. Вот это оставлено, положи пред собою и \textbf{ешь} (1 Цар. 9, 24).
    
    6. Пища не приближает нас к Богу: ибо, \textbf{едим} ли мы, ничего не приобретаем; не \textbf{едим}, ничего не теряем (1 Кор. 8, 8).
    
    7. И приноси жертвы мирные, и \textbf{ешь} и насыщайся там, и веселись пред Господом Богом твоим (Втор. 27, 7).
    
    8. И сказал Моисей: \textbf{ешьте} его сегодня, ибо сегодня суббота Господня; сегодня не найдете его на поле (Исх. 16, 25).
    
                        \paragraph{Упражнение}

    В приведенных предложениях замените выделенный глагол на соответствующую форму церковнославянского глагола {\slv{и҆мѣ́ти}}.
    
    \begin{flushleft}
        \renewcommand*{\arraystretch}{1.2}
        \begin{tabular}[l]{cll}
            
            ~~~~~
            & \emph{Образец}:
            & \makecell[l]{\ldots каждый оставит войну в своей собственной стране,\\которую они \textbf{имеют} между собою (3 Езд. 13, 33).}
            \\
            
            ~~~~~
            &
            &
            \\
            
            ~~~~~
            & \emph{Ответ}:
            & \makecell[l]{\ldots каждый оставит войну в своей собственной стране,\\которую они {\slv{и҆́мꙋтъ}} между собою.}
            \\
            
        \end{tabular}
    \end{flushleft}

    1. Иудеи отвечали ему: мы \textbf{имеем} закон (Ин. 19, 7).
    
    2. Хотя бы он и оскудел разумом, \textbf{имей} снисхождение (Сир. 3, 13).
    
    3. Пилат сказал им: \textbf{имеете} стражу; пойдите, охраняйте, как знаете (Мф. 27, 65).
    
    4. \textbf{Имейте} в себе соль, и мир \textbf{имейте} между собою (Мк. 9, 50).
    
    5. Ибо я не \textbf{имею} никого равно усердного, кто бы столь искренно заботился о вас (Флп. 2, 20).
    
    6. Пойди, все, что \textbf{имеешь}, продай и раздай нищим, и будешь \textbf{иметь} сокровище на небесах (Мк. 10, 21).
    
    7. И не могут они, заметив это, оставить их, потому что не \textbf{имеют} смысла (Посл. Иер. 1, 41).
    
    8. Велик он и не \textbf{имеет} конца, высок и неизмерим (Вар. 3, 25).
    
                    \medskip\paragraph{Упражнение}
                    
    В приведенных предложениях замените выделенный глагол на соответствующую форму церковнославянского глагола {\slv{да́ти}}.
    
    \begin{flushleft}
        \renewcommand*{\arraystretch}{1.2}
        \begin{tabular}[l]{cll}
            
            ~~~~~
            & \emph{Образец}:
            & \makecell[l]{Лаван расседлал верблюдов и \textbf{дал} соломы и корму\\верблюдам (Быт. 24, 32).}
            \\
            
            ~~~~~
            &
            &
            \\
            
            ~~~~~
            & \emph{Ответ}:
            & \makecell[l]{Лаван расседлал верблюдов и {\slv{да́де}} соломы и корму\\верблюдам.}
            \\
            
        \end{tabular}
    \end{flushleft}

    1. Если ты праведен, что \textbf{даешь} Ему? (Иов. 35, 7).
    
    2. Я истреблю Израиля с лица земли Моей, которую Я \textbf{дал} им (2 Пар. 7, 20).
    
    3. Она сказала: \textbf{дай} мне благословение (Нав. 15, 19).
    
    4. Господь \textbf{даст} силу народу Своему (Пс. 28, 11).
    
    5. Ибо вы знаете, какие мы \textbf{дали} вам заповеди от Господа Иисуса (1 Фес. 4, 2).
    
    6. Я \textbf{дам} тебе две тысячи коней (4 Цар. 18, 23).
    
    7. И \textbf{дала} кушанье и хлеб, которые она приготовила, в руки Иакову, сыну своему (Быт. 27, 17).
    
    8. И \textbf{дали} на устроение дома Божия пять тысяч талантов (1 Пар. 29, 7).

                \subsubsection{Будущее время описательное}

    Будущее время описательное является разновидностью будущего составного времени и отличается от него только тем, что вместо вспомогательного глагола {\slv{бы́ти}} берется в формах настоящего времени архаический глагол {\slv{имѣ́ти}}, спрягаемый же глагол ставится в неопределенной форме. Глагол {\slv{бы́ти}} также имеет будущее описательное время, так как оно приложимо для глаголов как несовершенного, так и совершенного вида. Например:
    
    \bigskip\autorows{c}{1}{l}{
        {{\slv{и҆́мамъ бы́ти, и҆́маши бы́ти}} и т.д.},
        {{\slv{и҆́мамъ хвали́ти, и҆́маши хвали́ти}} и т.д.}
    }

    У глагола {\slv{и҆мѣ́ти}} в будущем описательном времени в качестве вспомогательного глагола присоединяется тематический глагол {\slv{хотѣ́ти}}. Например: {\slv{хощꙋ̀ и҆мѣ́ти, хо́щеши и҆мѣ́ти}} и т.д.
    
    Будущее описательное время большей частью употребляется тогда, когда нужно не только ограничиться передачей мысли, но сделать ее и более наглядной, образной. Например:
    
    {\slv{Глаго́лю же ва́мъ, ꙗ҆́кѡ не и҆́мамъ пи́ти ѿны́нѣ ѿ плода̀ ло́знагѡ}} (Мф. 26, 29).

        \section{Склоняемые части речи}
            \subsection{Имя существительное}
                \subsubsection{Родовые окончания имен существительных}

    \textbf{Именем существительным} называется часть речи, обозначающая предмет, явление или понятие. Например: {\slv{землѧ̀, па́стырь, до́ждь, вѣ́тр, помышле́нїе}}.
    
    Каждому роду имен существительных соответствуют особые окончания именительного падежа единственного числа, которые поэтому называются \emph{родовыми} окончаниями. Так, родовые окончания в именах
    
    \bigskip\autorows{c}{2}{l}{
        {мужского рода:}, {{\slv{-ъ, -ь, -й}} ({\slv{до́мъ, па́стырь, мра́вїй}});},
        {женского рода:}, {{\slv{-а, -ѧ, -ь}} ({\slv{вода̀, ладїѧ̀, ма́терь}});},
        {среднего рода:}, {{\slv{-о, -е, -ѧ}} ({\slv{чело̀, мо́ре, и҆́мѧ}}).}
    }

    Имена существительные в церковнославянском языке имеют \textbf{три склонения}. Так как склоняемые имена могут иметь или твердые, или мягкие окончания, то и некоторые склонения подразделяются на \emph{твердые} и \emph{мягкие}.

                \subsubsection{Первое склонение имен существительных}

    К первому склонению имен существительных относятся имена \textbf{женского} рода с окончанием на {\slv{-а}} и {\slv{-ѧ}}. Но в некоторых случаях эти окончания имеют и имена существительные по значению мужского рода, например: {\slv{воево́да, сꙋдїѧ̀}}, которые также склоняются по первому склонению.
    
    Имена существительные с окончанием на {\slv{-а}} (при основе на согласный) принадлежат к твердому склонению, а с окончанием на {\slv{ѧ}} (а также и на {\slv{-а}} с основой на гласный)~---~к мягкому склонению.
    
    Возьмем для образца первого склонения существительные: {\slv{жена̀, пꙋсты́нѧ}} и {\slv{ски́нїѧ}} ({\slv{ски́нїа}}).
    
    \begin{center}
        \renewcommand*{\arraystretch}{1.4}
        \footnotesize\begin{tabular}[c]{|c|c|c|c|c|c|c|}
            \hline
            
            \makecell{Па-\\деж}
            & \multicolumn{3}{c|}{Единственное число}
            & \multicolumn{3}{c|}{Множественное число}
            \\\hline
            
            И.
            & {\slv{жена̀}}
            & {\slv{пꙋсты́нѧ}}
            & \makecell{{\slv{ски́нїѧ}}\\{\slv{ски́нїа}}}
            & {\slv{жєны̀}}
            & {\slv{пꙋсты̑ни}}
            & {\slv{ски̑нїи}}
            \\\hline
            
            Р.
            & {\slv{жены̀}}
            & {\slv{пꙋсты́ни}}
            & {\slv{ски́нїи}}
            & {\slv{же́нъ}}
            & {\slv{пꙋсты́нь}}
            & {\slv{ски́ний}}
            \\\hline
            
            Д.
            & {\slv{женѣ̀}}
            & {\slv{пꙋсты́ни}}
            & {\slv{ски́нїи}}
            & {\slv{жена́мъ}}
            & {\slv{пꙋсты́нѧмъ}}
            & {\slv{ски́нїѧмъ}}
            \\\hline
    
            В.
            & {\slv{женꙋ̀}}
            & {\slv{пꙋсты́ню}}
            & {\slv{ски́нїю}}
            & {\slv{жєны}}
            & {\slv{пꙋсты̑ни}}
            & {\slv{ски̑нїи}}
            \\\hline
    
            З.
            & {\slv{же́но}}
            & \makecell{{\slv{пꙋсты́не}}\\{\slv{пꙋсты́нѧ}}}
            & \makecell{{\slv{ски́нїе}}\\{\slv{ски́нїѧ}}}
            & {\slv{же́ны}}
            & {\slv{пꙋсты̑ни}}
            & {\slv{ски̑нїи}}
            \\\hline
            
            Т.
            & {\slv{жено́ю}}
            & {\slv{пꙋсты́нею}}
            & {\slv{ски́нїею}}
            & {\slv{жена́ми}}
            & {\slv{пꙋсты́нѧми}}
            & {\slv{ски́нїѧми}}
            \\\hline
            
            П.
            & {\slv{ѡ҆ женѣ̀}}
            & {\slv{ѡ҆ пꙋсты́ни}}
            & {\slv{ѡ҆ ски́нїи}}
            & \makecell{{\slv{ѡ҆ жена́хъ}}\\{\slv{ѡ҆ женѣ́хъ}}}
            & {\slv{ѡ҆ пꙋсты́нѧхъ}}
            & {\slv{ѡ҆ ски́нїѧхъ}}
            \\\hline
            
            \makecell{~\\~}
            & \multicolumn{3}{c|}{Двойственное число}
            \\\cline{1-4}
    
            \makecell{И.\\В. З.}
            & {\slv{жєны̀}}
            & {\slv{пꙋсты̑ни}}
            & {\slv{ски̑нїи}}
            \\\cline{1-4}
            
            \makecell{Р. П.}
            & \makecell{\slv{жєнꙋ̀}}
            & {\slv{пꙋсты́ню}}
            & {\slv{ски̑нїю}}
            \\\cline{1-4}
    
            \makecell{Д. Т.}
            & \makecell{{\slv{жена́ма}}\\{\slv{жено́ма}}}
            & \makecell{{\slv{пꙋсты́нѧма}}\\{\slv{пꙋсты́ньми}}}
            & {\slv{ски́нїѧма}}
            \\\cline{1-4}
            
        \end{tabular}
    \end{center}

                \subsubsection{Общие замечания к первому склонению имен существительных}

    1. Звательный падеж единственного числа в мягком склонении большей частью оканчивается на {\slv{-е}}: {\slv{ѕемлѐ, пꙋсты́не}}, но иногда он бывает сходен и с именительным падежом единственного числа: {\slv{землѧ̀, пꙋсты́нѧ}}.
    
    2. Дательный и творительный падежи двойственного числа иногда, в виде исключения, в твердом склонении оканчиваются на {\slv{-ома}} ({\slv{жено́ма}}), а предложный падеж множественного числа~---~на {\slv{-ѣхъ}} ({\slv{ѡ҆ женѣ́хъ}}).
    
    3. Формы родительного падежа множественного числа в твердом склонении весьма разнообразны: {\slv{мольба̀~\textemdash~моле́бъ, татьба̀~\textemdash~та́тьбъ, тьма̀~\textemdash~те́мъ, та́йна~\textemdash~та́инъ, смо́ква~\textemdash~смо́квїй}} (и {\slv{смо́квей}}) и т.п.
    
    4. Некоторые существительные первого склонения не имеют или двойственного и множественного чисел, например: {\slv{мгла̀, мзда̀, прѧ̀, лжа̀, тлѧ̀}} и др., или единственного и двойственного чисел: {\slv{вери́ги, ѡ҆ко́вы}} и др.
    
    5. Некоторые падежи двойственного и множественного чисел, созвучные с некоторыми падежами единственного числа, различаются от них или иным начертанием букв ({\slv{е}} или {\slv{о}}~---~в единственном числе и {\slv{є}} или {\slv{ѡ}}~---~в двойственном и множественном числах), или ударением (острое или тупое~---~в единственном числе и облегченное~---~в двойственном и множественном числах). Эта особенность в начертании букв или постановке ударений соблюдается и в прочих склонениях, а также в склонениях других частей речи.
    
    6. Имена существительные с основой на гортанные {\slv{г, к, х}} смягчаются перед {\slv{ѣ}} и {\slv{и}} ({\slv{ы}}) соответственно в {\slv{з, ц, с}}, например:
    
    \bigskip\autorows{c}{1}{l}{
        \slv{но{\large г}а̀~\textemdash~{но́\large з}ѣ~\textemdash~но́{\large з}и},
        \slv{рꙋ{\large к}а̀~\textemdash~{рꙋ́\large ц}ѣ~\textemdash~рꙋ́{\large ц}ы},
        \slv{сно{\large х}а̀~\textemdash~сно{\large с}ѣ̀~\textemdash~сно́{\large с}и}
    }

    Исключение из этого правила составляет слово {\slv{па́сха}} ({\slv{па́сцѣ}}).


                \subsubsection{Имена существительные первого склонения с основой на шипящие и на {\slv{ц}}}

    Имена существительные с окончанием на {\slv{-а}}, но с основой на шипящие {\slv{ж, ч, ш, щ}}, а также на {\slv{ц}} имеют в некоторых падежах окончания твердого склонения, а в некоторых~---~мягкого.
    
    Возьмем для образца склонения таких существительных: {\slv{дꙋша̀}} и {\slv{дѣви́ца}}.
    
    \begin{center}
        Склонения существительных {\slv{дꙋша̀}} и {\slv{дѣви́ца}}
        \renewcommand*{\arraystretch}{1.4}
        \footnotesize\begin{tabular}[c]{|c|c|c|c|c|}
            \hline
            
            \makecell{Па-\\деж}
            & \multicolumn{2}{c|}{Единственное число}
            & \multicolumn{2}{c|}{Множественное число}
            \\\hline
            
            И.
            & {\slv{дꙋша̀}}
            & {\slv{дѣви́ца}}
            & {\slv{дꙋ́ши}}
            & {\slv{дѣви̑цы}}
            \\\hline
            
            Р.
            & {\slv{дꙋшѝ}}
            & {\slv{дѣви́цы}}
            & {\slv{дꙋ́шъ}}
            & {\slv{дѣви́цъ}}
            \\\hline
            
            Д.
            & {\slv{дꙋшѝ}}
            & {\slv{деви́цѣ}}
            & {\slv{дꙋша́мъ}}
            & {\slv{дѣви́цамъ}}
            \\\hline
            
            В.
            & {\slv{дꙋ́шꙋ}}
            & {\slv{дѣви́цꙋ}}
            & {\slv{дꙋ́ши}}
            & {\slv{дѣви́цъ}}
            \\\hline
            
            З.
            & {\slv{дꙋшѐ}}
            & \makecell{{\slv{дѣви́це}}}
            & {\slv{дꙋ́шы}}
            & {\slv{дѣви̑цы}}
            \\\hline
            
            Т.
            & {\slv{дꙋше́ю}}
            & {\slv{дѣви́цею}}
            & {\slv{дꙋша́ми}}
            & {\slv{дѣви́цами}}
            \\\hline
            
            П.
            & {\slv{ѡ҆ дꙋшѝ}}
            & {\slv{ѡ҆ дѣви́цѣ}}
            & \makecell{{\slv{ѡ҆ дꙋша́хъ}}}
            & {\slv{ѡ҆ дѣви́цахъ}}
            \\\hline
            
            \makecell{~\\~}
            & \multicolumn{2}{c|}{Двойственное число}
            \\\cline{1-3}
            
            \makecell{И.\\В. З.}
            & {\slv{дꙋши̑}}
            & {\slv{дѣви̑цы}}
            \\\cline{1-3}
            
            \makecell{Р. П.}
            & \makecell{\slv{дꙋ̀шꙋ}}
            & {\slv{дѣви̑цꙋ}}
            \\\cline{1-3}
            
            \makecell{Д. Т.}
            & \makecell{{\slv{дꙋша́ма}}}
            & \makecell{{\slv{дѣви́цами}}}
            \\\cline{1-3}
            
        \end{tabular}
    \end{center}

    1. Имена существительные с основой на шипящие в дательном и предложном падежах единственного числа имеют окончание {\slv{-и}}, а не {\slv{-ѣ}}, а некоторые из таких существительных имеют также в родительном падеже множественного числа две формы, например: {\slv{кꙋ́ща~\textemdash~кꙋ́щей}} и {\slv{кꙋ́щъ}}.
    
    В винительном падеже множественного числа эти существительные имеют окончание {\slv{-ы}} ({\slv{дꙋ́шы}}) для различения этого падежа от именительного того же числа, которые имеют окончание {\slv{-и}} ({\slv{дꙋ́ши}}).
    
    2. Имена существительные с основой на {\slv{ц}} вместо окончания {\slv{-и}} везде имеют окончание {\slv{-ы}} или {\slv{-ѣ}} ({\slv{дѣви́цы, ѡ҆ дѣви́цѣ}}).

                \subsubsection{Второе склонение имен существительных}

    Ко второму склонению имен существительных относятся имена \textbf{мужского} рода с окончанием на {\slv{-ъ}} (твердое склонение) и с окончанием на {\slv{-ь}} и {\slv{-й}} (мягкое склонение), имена \textbf{среднего} рода с окончанием на {\slv{-о}} (твердое склонение) и с окончанием на {\slv{-е}} (мягкое склонение).
    
    \medskip\underline{1. Твердое склонение}
    \medskip
    
    Приведем образцы твердого склонения имен существительных мужского рода: {\slv{сы́нъ, ра́бъ, дꙋ́хъ, проро́къ}}~---~и среднего рода: {\slv{лѣ́то}}.
    \begin{center}
        Склонения существительных {\slv{сы́нъ, ра́бъ, дꙋ́хъ, проро́къ, лѣ́то}}
        \renewcommand*{\arraystretch}{1.4}
        \footnotesize\begin{tabular}[c]{|c|c|c|c|c|c|c|}
            \hline
            
            \multirow{7}{*}{\spheading[10em]{Единственное число}}
            & И.
            & {\slv{сы́нъ}}
            & {\slv{ра́бъ}}
            & {\slv{дꙋ́хъ}}
            & {\slv{проро́къ}}
            & {\slv{лѣ́то}}
            \\\cline{2-7}
            
            & Р.
            & {\slv{сы́на}}
            & {\slv{раба̀}}
            & {\slv{дꙋ́ха}}
            & {\slv{проро́ка}}
            & {\slv{лѣ́та}}
            \\\cline{2-7}
            
            & Д.
            & \makecell{{\slv{сы́нꙋ}}\\{\slv{сы́нови}}}
            & {\slv{рабꙋ̀}}
            & \makecell{{\slv{дꙋ́хꙋ}}\\{\slv{дꙋ́хови}}}
            & {\slv{проро́кꙋ}}
            & {\slv{лѣ́тꙋ}}
            \\\cline{2-7}
            
            & В.
            & {\slv{сы́на}}
            & {\slv{раба̀}}
            & \makecell{{\slv{дꙋ́хъ}}\\{\slv{дꙋ́ха}}}
            & {\slv{проро́ка}}
            & {\slv{лѣ́то}}
            \\\cline{2-7}
            
            & З.
            & {\slv{сы́не}}
            & \makecell{{\slv{ра́бе}}}
            & {\slv{дꙋ́ше}}
            & {\slv{проро́че}}
            & {\slv{лѣ́то}}
            \\\cline{2-7}
            
            & Т.
            & {\slv{сы́номъ}}
            & {\slv{рабо́мъ}}
            & {\slv{дꙋ́хомъ}}
            & {\slv{проро́комъ}}
            & {\slv{лѣ́томъ}}
            \\\cline{2-7}
            
            & П.
            & {\slv{ѡ҆ сы́нѣ}}
            & {\slv{ѡ҆ рабѣ̀}}
            & \makecell{{\slv{ѡ҆ дꙋ́се}}}
            & {\slv{ѡ҆ проро́цѣ}}
            & {\slv{ѡ҆ лѣ́тѣ}}
            \\\hline
            
            \multirow{3}{*}{\spheading[4.5em]{Дв. число}}
            & \makecell{И.\\В. З.}
            & {\slv{сы̑на}}
            & {\slv{раба̑}}
            & {\slv{дꙋ̑ха}}
            & {\slv{прорѡ́ка}}
            & {\slv{лѣ̑та}}
            \\\cline{2-7}
            
            & Р. П.
            & {\slv{сы̑нꙋ}}
            & {\slv{рабꙋ̑}}
            & {\slv{дꙋ̑хꙋ}}
            & {\slv{прорѡ́кꙋ}}
            & {\slv{лѣ̑тꙋ}}
            \\\cline{2-7}
            
            & Д. Т.
            & {\slv{сыно́ма}}
            & {\slv{рабо́ма}}
            & {\slv{дꙋ́хома}}
            & {\slv{проро́кома}}
            & {\slv{лѣ́тома}}
            \\\hline
            
            \multirow{7}{*}{\spheading[10em]{Множественное число}}
            & И.
            & \makecell{{\slv{сы́ны}}\\{\slv{сы́нове}}}
            & {\slv{рабѝ}}
            & {\slv{дꙋ́си}}
            & {\slv{проро́цы}}
            & {\slv{лѣ̑та}}
            \\\cline{2-7}
            
            & Р.
            & {\slv{сынѡ́въ}}
            & \makecell{{\slv{ра̑бъ}}\\{\slv{рабѡ́въ}}}
            & {\slv{дꙋхѡ́въ}}
            & \makecell{{\slv{проро́кѡвъ}}\\{\slv{прорѡ́къ}}}
            & {\slv{лѣ́тъ}}
            \\\cline{2-7}
            
            & Д.
            & \makecell{{\slv{сынѡ́мъ}}\\{\slv{сыновѡ́мъ}}}
            & {\slv{рабѡ́мъ}}
            & \makecell{{\slv{дꙋхѡ́мъ}}\\{\slv{дꙋховѡ́мъ}}}
            & {\slv{проро́кѡмъ}}
            & {\slv{лѣ́тѡмъ}}
            \\\cline{2-7}
            
            & В.
            & \makecell{{\slv{сыны̀}}\\{\slv{сынѡ́въ}}}
            & \makecell{{\slv{рабы̀}}\\{\slv{рабѡ́въ}}}
            & {\slv{дꙋ́хи}}
            & {\slv{проро́ки}}
            & {\slv{лѣ̑та}}
            \\\cline{2-7}
            
            & З.
            & {\slv{сы́ны}}
            & \makecell{{\slv{рабѝ}}}
            & {\slv{дꙋ́си}}
            & {\slv{проро́цы}}
            & {\slv{лѣ̑та}}
            \\\cline{2-7}
            
            & Т.
            & \makecell{{\slv{сы̑ны}}\\{\slv{сы́нми}}}
            & \makecell{{\slv{рабы̑}}\\{\slv{раба́ми}}}
            & {\slv{дꙋ̑хи}}
            & \makecell{{\slv{прорѡ́ки}}\\{\slv{проро́ками}}}
            & {\slv{лѣ́ты}}
            \\\cline{2-7}
            
            & П.
            & {\slv{ѡ҆ сынѣ́хъ}}
            & {\slv{ѡ҆ рабѣ́хъ}}
            & \makecell{{\slv{ѡ҆ дꙋ́сѣхъ}}}
            & {\slv{ѡ҆ проро́цѣхъ}}
            & {\slv{ѡ҆ лѣ́техъ}}
            \\\hline
            
        \end{tabular}
    \end{center}

    Существительные с основой на {\slv{ц}} в Р., Д., П. падежах ед. ч. имеют окончание твердого варианта склонения ({\slv{о҆трокови́ца~\textemdash~о҆трокови́цѣ}}).
    
    \medskip\underline{2. Мягкое склонение}
    \medskip
    
    Приведем образцы мягкого склонения имен существительных мужского рода: {\slv{па́стырь, жре́бїй}}; среднего рода: {\slv{мо́ре, писа́нїе}}.
    
    \begin{center}
        Склонения существительных {\slv{па́стырь, жре́бий, мо́ре, писа́нїе}}
        \renewcommand*{\arraystretch}{1.4}
        \footnotesize\begin{tabular}[c]{|c|c|c|c|c|c|}
            \hline
            
            \multirow{7}{*}{\spheading[10em]{Единственное число}}
            & И.
            & {\slv{па́стырь}}
            & {\slv{жре́бїй}}
            & {\slv{мо́ре}}
            & {\slv{писа́нїе}}
            \\\cline{2-6}
            
            & Р.
            & {\slv{па́стырѧ}}
            & {\slv{жре́бїѧ}}
            & {\slv{мо́рѧ}}
            & {\slv{писа́нїѧ}}
            \\\cline{2-6}
            
            & Д.
            & \makecell{{\slv{па́стырю}}\\{\slv{па́стыреви}}}
            & {\slv{жре́бїю}}
            & {\slv{мо́рю}}
            & {\slv{писа́нїю}}
            \\\cline{2-6}
            
            & В.
            & {\slv{па́стырѧ}}
            & {\slv{жре́бїй}}
            & {\slv{мо́ре}}
            & {\slv{писа́нїе}}
            \\\cline{2-6}
            
            & З.
            & {\slv{па́стырю}}
            & \makecell{{\slv{жре́бїе}}}
            & {\slv{мо́ре}}
            & {\slv{писа́ние}}
            \\\cline{2-6}
            
            & Т.
            & {\slv{па́стыремъ}}
            & {\slv{жре́бїемъ}}
            & {\slv{мо́ремъ}}
            & {\slv{писа́нїем}}
            \\\cline{2-6}
            
            & П.
            & \makecell{{\slv{ѡ҆ па́стырѣ}}\\{\slv{ѡ҆ па́стыри}}}
            & {\slv{ѡ҆ жре́бїи}}
            & {\slv{ѡ҆ мо́ри}}
            & {\slv{ѡ҆ писа́нїи}}
            \\\hline
            
            \multirow{3}{*}{\spheading[4.5em]{Дв. число}}
            & \makecell{И.\\В. З.}
            & {\slv{па̑стырѧ}}
            & {\slv{жрє́бїѧ}}
            & {\slv{мѡ́рѧ}}
            & {\slv{писа̑нїѧ}}
            \\\cline{2-6}
            
            & Р. П.
            & {\slv{па̑стырю}}
            & {\slv{жрє́бию}}
            & {\slv{мѡ́рю}}
            & {\slv{писа̑нию}}
            \\\cline{2-6}
            
            & Д. Т.
            & \makecell{{\slv{па́стырема}}\\{\slv{па́стырьма}}}
            & {\slv{жре́бїима}}
            & {\slv{мо́рема}}
            & {\slv{писа́нїима}}
            \\\hline
            
            \multirow{7}{*}{\spheading[10em]{Множественное число}}
            & И.
            & {\slv{па́стырїе}}
            & {\slv{жрє́бїи}}
            & {\slv{морѧ̀}}
            & {\slv{писа̑нїѧ}}
            \\\cline{2-6}
            
            & Р.
            & {\slv{па́стырей}}
            & \makecell{{\slv{жрє́бїй}}\\{\slv{жре́бїевъ}}}
            & {\slv{море́й}}
            & {\slv{писа́ний}}
            \\\cline{2-6}
            
            & Д.
            & {\slv{па́стырємъ}}
            & {\slv{жре́бїємъ}}
            & {\slv{мѡ́рем}}
            & {\slv{писа́нїємъ}}
            \\\cline{2-6}
            
            & В.
            & \makecell{{\slv{па́стыри}}\\{\slv{па́стырей}}}
            & {\slv{жрє́бїи}}
            & {\slv{морѧ̀}}
            & {\slv{писа̑нїѧ}}
            \\\cline{2-6}
            
            & З.
            & {\slv{па́стырїе}}
            & \makecell{{\slv{жрє́бїи}}}
            & {\slv{морѧ̀}}
            & {\slv{писа̑нїѧ}}
            \\\cline{2-6}
            
            & Т.
            & \makecell{{\slv{па̑стыри}}\\{\slv{па́стырьми}}}
            & {\slv{жре́бїими}}
            & \makecell{{\slv{мѡ́ри}}\\{\slv{морѧ́ми}}}
            & \makecell{{\slv{писа̑нїи}}\\{\slv{писа́нми}}}
            \\\cline{2-6}
            
            & П.
            & {\slv{ѡ҆ па́стырехъ}}
            & \makecell{{\slv{ѡ҆ жре́бїихъ}}\\{\slv{ѡ҆ жре́бїѧхъ}}}
            & \makecell{{\slv{ѡ҆ мо́рехъ}}\\{\slv{ѡ҆ морѧ́хъ}}}
            & {\slv{ѡ҆ писа́нїихъ}}
            \\\hline
            
        \end{tabular}
    \end{center}

                \subsubsection{Общие замечания ко второму склонению имен существительных}

    Во втором склонении имен существительных необходимо обратить внимание на следующие особенности:
    
    1. Имена существительные {\slv{ве́рхъ, до́мъ}} в родительном падеже единственного числа оканчиваются на {\slv{-ꙋ}} вместо {\slv{-а}}. Например:
    
    \bigskip\autorows{c}{1}{c}{
        {{\slv{И̑ ведо́ша є҆го̀}} ({\slv{і҆и҃са}}) {\slv{до верхꙋ̀ горы̀}} (Лк. 4, 29)},
        {{\slv{Во дво́рѣхъ до́мꙋ гдⷭ҇нѧ}} (Пс. 115, 10)}
    }

    2. Имена существительные собственные, а также обозначающие звание или профессию, в дательном падеже единственного числа принимают наращение {\slv{-ови}} в твердом склонении и {\slv{-еви}} в мягком склонении. Например:
    
    \bigskip\autorows{c}{1}{c}{
        {\slv{И̑ дꙋ́хови твоемꙋ̀}},
        {{\slv{Бж҃е, сꙋ́дъ тво́й царе́ви да́ждь}} (Пс. 71, 1)}
    }

    Наращенное окончание {\slv{-ови}} в дательном падеже употребляется также в словах иностранных, вошедших в церковнославянский язык, хотя бы это слово и было мягкого склонения. Например:
    
    \bigskip\autorows{c}{1}{c}{
        {{\slv{Почерпни́те ны́нѣ и҆ принеси́те а҆рхїтрїклі́нови}} (Ин. 2, 9)},
        {{\slv{А҆́ще бо бы́сте вѣ́ровали мѡѷсе́ови, вѣ́ровали}}},
        {{\slv{бы́сте ᲂу҆́бо и҆ мнѣ}} (Ин. 5, 46)},
        {{\slv{Во́ини же є҆́мше і҆и҃са, ведо́ша къ каїа́фѣ а҆рхїере́ови}} (Мф. 26, 57)}
    }

    3. Винительный падеж единственного числа при именах одушевленных сходен с родительным, а при именах неодушевленных~---~с именительным. Но иногда и при именах одушевленных предметов винительный падеж сходен с именительным. Например:
    
    \bigskip\autorows{c}{1}{c}{
        {{\slv{Приве́дше теле́цъ ᲂу҆пита́нный заколи́те}} (Лк. 15, 23)}
    }

    4. В именах существительных мужского рода при окончании именительного падежа единственного числа на {\slv{-ъ}} и {\slv{й}} звательный падеж того же числа имеет окончание {\slv{-е}}: {\slv{о҆́тче, фарїсе́е, андре́е}}, а при окончании именительного падежа на {\slv{-ь}}~---~окончание {\slv{-ю}}: {\slv{царю̀, па́стырю, ᲂу҆чи́телю}}; но если основа оканчивается на свистящий {\slv{-з}}, то окончание в звательном падеже бывает {\slv{-е}}: {\slv{кнѧ́зь~\textemdash~кнѧ́же}}.
    
    5. Предложный падеж единственного числа может оканчиваться на {\slv{-ѣ}}, {\slv{-и}}, и {\slv{-ꙋ}}. Например: {\slv{ѡ҆ свѣ́тѣ, въ раѝ, въ домꙋ̀, ѡ҆ челѣ̀, ѡ҆ мо́ри}}.
    
    6. Именительный, винительный и звательный падежи двойственного числа имеют окончание {\slv{-а}} в твердом склонении и окончание {\slv{-ѧ}} в мягком склонении. Например:
    
    \bigskip\autorows{c}{1}{c}{
        {{\slv{Человѣ́къ нѣ́кий и҆мѣ два̀ сы̑на}} (Лк. 15, 11)},
        {{\slv{И̑ сѐ, два̀ жрє́бїѧ мета́ша}}}
    }

    7. Родительный и предложный падежи этого же числа оканчиваются на {\slv{}} в твердом склонении и на {\slv{}} в мягком склонении. Например:
    
    \bigskip\autorows{c}{1}{c}{
        {{\slv{Сн҃ъ є҆диноро́дный... во двою̀ є҆стествꙋ̑ несли́тнѡ познава́емый}} (Догм. гл. 6)}
    }

    8. Дательный и творительный падежи двойственного числа оканчиваются на {\slv{-ома}} в твердом склонении и на {\slv{-ема}} или {\slv{-има}} в мягком склонении. Например: {\slv{сыно́ма, царе́ма, писа́нїима}}.
    
    9. Именительный и звательный падежи множественного числа иногда оканчиваются на {\slv{}}, или на {\slv{}}, или на {\slv{}}. Например:
    
    \bigskip\autorows{c}{1}{c}{
        {{\slv{Сы́нове же ца́рствїѧ и҆згна́нии бꙋ́дꙋтъ}} (Мф. 8, 12)},
        {\slv{Па́стырїе и҆ ᲂу҆чи́телїе вселе́́нныѧ, моли́те}},
        {{\slv{ѡ҆ насъ грѣ́шныхъ}} (Вел. повеч.)},
        {{\slv{Фарїсе́є ше́дше, совѣ́тъ сотвори́ша на него̀}} (Мф. 8, 12)}
    }

    10. В родительном падеже множественного числа почти все имена существительные имеют:
    
    а) или окончание {\slv{-ѡвъ, -євъ, -ей, -ъ, -й}}, например: {\slv{рогѡ́въ, крає́въ, царе́й, лѣ́тъ, писа́нїй}};
    
    б) или окончание, созвучное с именительным падежом единственного числа, например:
    
    \bigskip\autorows{c}{1}{c}{
        {{\slv{Ка́мени запеча́танꙋ ѿ і҆ꙋдє́й}} (вместо: {\slv{ѿ і҆ꙋде́євъ}})},
        {{\slv{Нѣ́цыи ѿ кни̑жникъ}} (вместо: {\slv{ѿ кни́жникѡвъ}})}
    }

    11. Дательный падеж множественного числа оканчивается на {\slv{-ѡмъ}} (или на {\slv{-овѡмъ}} при наращении), на {\slv{-ємъ}}. Например:
    
    \bigskip\autorows{c}{1}{c}{
        {{\slv{Речѐ гдⷭ҇ь свои̑мъ ᲂу҆ченикѡмъ}}},
        {{\slv{Тогда̀ і҆и҃съ нача́тъ поноша́ти градовѡмъ}} (Лк. 11, 20)},
        {{\slv{Заповѣ́да о҆тцє́мъ сказа́ти сыновѡмъ свои̑мъ}} (Пс. 77, 5)}
    }

    12. Творительный падеж множественного числа часто принимает сокращенную форму, а именно: вместо окончания {\slv{-ами}} или {\slv{ѧми}} принимает окончание {\slv{-ы}} или {\slv{-и}}. Например:
    
    \bigskip\autorows{c}{1}{c}{
        {{\slv{Досто́инъ є҆сѝ во всѧ̑ времена̀ пѣ́ть бы́ти гла̑сы}}},
        {{\slv{прпⷣбными}} (Послед. вечерни)},
        {{\slv{Ꙗ҆́вльшїѧсѧ же ны́нѣ, писа̑нїи прⷪ҇ро́ческими, по повелѣ́нїю}}},
        {{\slv{вѣ́чнагѡ бг҃а}} (Рим. 14, 25)}
    }

    Конечно, в этом случае эта форма творительного падежа различается от созвучной иногда ему формы винительного падежа или начертанием букв, или облегченным ударением.
    
    13. В предложном падеже множественного числа с основой на согласный (но не шипящий) звук может быть окончание {\slv{-ѣхъ}} и {\slv{-ехъ}}. Окончание {\slv{-ѣхъ}} пишется тогда, когда в русском языке в соответствующем слове пишется окончание \textbf{-ах}, а {\slv{-ехъ}} пишется тогда, когда по-русски пишется \textbf{-ях}. Например: {\slv{ѡ҆ сынѣ́хъ}} (о сынах), {\slv{ѡ҆ па́стырехъ}} (о пастырях).
    
    14. Имена существительные с основой на гортанный звук смягчают его перед гласными {\slv{}} и {\slv{}} по известным законам смягчения. Но в винительном и творительном падежах множественного числа, в отличие их от именительного, гортанные не смягчаются. Например:
    
    \medskip\autorows{c}{1}{c}{
        {{\slv{Вни́де са́мъ}} ({\slv{і҆и҃съ}}) {\slv{и҆ ᲂу҆чн҃цы}} (им. п. мн. ч.) {\slv{є҆гѡ̀}} (Ин. 18, 1)},
        {{\slv{Пое́мъ}} ({\slv{і҆и҃съ}}) {\slv{ѻ҆бана́десѧте ᲂу҆чн҃ки}} (вин. п. мн. ч.) {\slv{своѧ̑}} (Лк. 18, 31)},
        {{\slv{Сѣдѧ́ше}} ({\slv{і҆и҃съ}}) {\slv{со ᲂу҆чн҃кѝ}} (тв. п. мн. ч.) {\slv{свои́ми}} (Ин. 6, 3)},
        {{\slv{Цр҃ь нбный за чл҃вѣколю́иїе на землѝ ꙗви́сѧ и҆ съ человѣ̑ки}}},
        {(тв. п. мн. ч.) {\slv{поживѐ}} (Догм. гл. 8)}
    }

    15. Имена существительные среднего рода, имеющие перед гласным окончанием две и более согласных, принимают в некоторых косвенных падежах беглый гласный. Например: {\slv{дно̀~\textemdash~дѡ́нъ, ѕло̀~\textemdash~ѕѡ́лъ}}.
    
    Употребляемые только во множественном числе существительные среднего рода: {\slv{врата̀, чре́сла}} и др.~---~склоняются по образцу существительного {\slv{лѣ́то}} во множественном числе.
    
    16. Имена существительные, взятые с еврейского языка, как например: {\slv{а҆донаі̀, равві̀}} и подобные им, не склоняются. Иногда не склоняется и имя {\slv{і҆и҃съ}}, когда это имя стоит не одно, а в сочетании с именем {\slv{хрⷭ҇то́съ}}. Например:
    
    \bigskip\autorows{c}{1}{c}{
        {{\slv{Блⷣгть же и҆ и҆́стина і҆и҃съ хрⷭ҇то́мъ бы́сть}} (Ин. 1, 17)}
    }

    
                \subsubsection{Имена существительные второго склонения с основой на шипящие и на {\slv{ц}}}

    Приведем образцы склонений таких существительных: {\slv{мꙋ́жъ, лицѐ}}.
    
    \begin{center}
%        Склонения существительных {\slv{мꙋ́жъ}} и {\slv{лицѐ}}
        \renewcommand*{\arraystretch}{1.4}
        \footnotesize\begin{tabular}[c]{|c|c|c|c|c|}
            \hline
            
            \makecell{Па-\\деж}
            & \multicolumn{2}{c|}{Единственное число}
            & \multicolumn{2}{c|}{Множественное число}
            \\\hline
            
            И.
            & {\slv{мꙋ́жъ}}
            & {\slv{лицѐ}}
            & {\slv{мꙋ́жїе}}
            & {\slv{ли́ца}}
            \\\hline
            
            Р.
            & {\slv{мꙋ́жа}}
            & {\slv{лица̀}}
            & \makecell{{\slv{мꙋ̑жъ}}\\{\slv{мꙋже́й}}}
            & {\slv{ли́цъ}}
            \\\hline
            
            Д.
            & \makecell{{\slv{мꙋ́жꙋ}}\\{\slv{мꙋ́жеви}}}
            & {\slv{лицꙋ̀}}
            & {\slv{мꙋжє́мъ}}
            & {\slv{ли́цам}}
            \\\hline
            
            В.
            & {\slv{мꙋ́жа}}
            & {\slv{лицѐ}}
            & {\slv{мꙋ́жы}}
            & \makecell{{\slv{ли́ца}}\\{\slv{лицы̀}}}
            \\\hline
            
            З.
            & {\slv{мꙋ́жꙋ}}
            & \makecell{{\slv{лицѐ}}}
            & {\slv{мꙋ́жїе}}
            & \makecell{{\slv{ли́ца}}}
            \\\hline
            
            Т.
            & {\slv{мꙋ́жемъ}}
            & {\slv{лице́мъ}}
            & \makecell{{\slv{мꙋ̑жи}}\\{\slv{мꙋжа́ми}}}
            & \makecell{{\slv{ли̑цы}}\\{\slv{ли́цами}}}
            \\\hline
            
            П.
            & {\slv{ѡ҆ мꙋ́жи}}
            & {\slv{ѡ҆ лицѣ̀}}
            & \makecell{{\slv{ѡ҆ мꙋ́жахъ}}\\{\slv{ѡ҆ мꙋже́хъ}}}
            & {\slv{ѡ҆ ли́цахъ}}
            \\\hline
            
            \makecell{~\\~}
            & \multicolumn{2}{c|}{Двойственное число}
            \\\cline{1-3}
            
            \makecell{И.\\В. З.}
            & {\slv{мꙋ̑жа}}
            & {\slv{лица̑}}
            \\\cline{1-3}
            
            Р. П.
            & {\slv{мꙋ̑жꙋ}}
            & {\slv{лицꙋ̑}}
            \\\cline{1-3}
            
            Д. Т.
            & {\slv{мꙋже́ма}}
            & {\slv{лице́ма}}
            \\\cline{1-3}
            
        \end{tabular}
    \end{center}

    Склонение имен существительных с основой на шипящие и на {\slv{ц}} имеет свои особенности.
    
    1. Имена существительные мужского рода с основой на шипящий {\slv{ч}} имеют в именительном падеже единственного числа окончание {\slv{-ь}}. Например: {\slv{ме́чь, вра́чь, клю́чь}}.
    
    2. Звательный падеж единственного числа с основой на шипящие {\slv{ж}} и {\slv{ч}} оканчивается на {\slv{-ꙋ}}: {\slv{мꙋ́жꙋ, врачꙋ̀}}.
    
    3. Предложный падеж единственного числа с основой на {\slv{ж}} и {\slv{щ}} оканчивается на {\slv{-и}}, а с основой на {\slv{ц}} оканчивается на {\slv{ѣ}}. Например: {\slv{ѡ҆ мꙋ́жи, на ло́жи, въ со́нмищи, ѡ҆ лицѣ̀}}.
    
    Но существительное {\slv{се́рдце}} имеет иногда в предложном падеже окончание {\slv{-ы}}: {\slv{Лю́дїе под̾ тобо́ю падꙋ́тъ въ се́рдцы вра̑гъ цр҃е́выхъ}} (Лк. 44, 6).
    
    4. Дательный и творительный падежи двойственного числа существительных {\slv{мꙋ́жъ}} и {\slv{се́рдце}} имеют формы: {\slv{мꙋже́ма, сердца́ма}}.
    
    5. Именительный и звательный падежи множественного числа с основой на {\slv{ч}} имеют окончание {\slv{-еве}}: {\slv{Вра́чеве воскресѧ́тъ и҆ и҆сповѣ́дѧтсѧ тебѣ̀}} (Пс. 87, 11).
    
    6. Родительный падеж множественного числа существительных среднего рода с основой на {\slv{ж, щ, ц}} оканчивается на {\slv{-ъ}}: {\slv{ло́жъ, сокро́вищъ, ли́цъ}}.
    
    7. Имена существительные мужского рода с основой на {\slv{ж}} имеют в винительном падеже множественного числа окончание {\slv{-ы}} для различения от предложного падежа единственного числа, имеющего окончание {\slv{-и}}, а также от творительного падежа множественного числа, тоже имеющего в одной из своих форм окончание {\slv{-и}}. Например:
    
    \medskip\autorows{c}{1}{c}{
        {{\slv{Оу҆смотри́те ᲂу҆̀бо бра́тїе, мꙋ́жы}} (вин. п. мн. ч.) {\slv{ѿтъ ва́съ}} (Деян. 6, 3)},
        {{\slv{Цари́ца ю҆́жскаѧ воста́нетъ на сꙋ́дъ съ мꙋ̑жи}} (тв. п. мн. ч.)},
        {{\slv{ро́да сегѡ̀}} (Лк. 11, 31)},
        {{\slv{Гдⷭ҇и, слы́шахъ ѿ мно́гихъ ѡ҆ мꙋ́жи}} (предл. п. ед. ч.)},
        {{\slv{се́мъ}} (Деян. 9, 13)}
    }

    8. В предложном падеже множественного числа существитльные с основой на {\slv{ж}} и {\slv{ч}} имеют двоякое окончание: {\slv{-ахъ}} и {\slv{-ехъ}} (но не {\slv{-ѣхъ}}), например {\slv{ѡ҆ мꙋ́жахъ, ѡ҆ мꙋже́хъ}}, а с основой на {\slv{щ}} могут оканчиваться на {\slv{-ахъ, -ихъ, -ехъ}}, например: {\slv{въ со́нмищахъ}}, или {\slv{въ со́нмищихъ}}, или {\slv{въ со́нмищехъ}}. И только в существительных с основой на {\slv{ц}} окончание бывает {\slv{-ахъ}}: {\slv{ѡ҆ ли́цахъ, ѡ҆ сердца́хъ}}.

                \subsubsection{Третье склонение имен существительных}

    К третьему склонению имен существительных относятся имена \textbf{женского} рода с окончанием на {\slv{-ь}}. Так как окончание {\slv{-ь}} является мягким, то это склонение только мягкое.
    
    К третьему склонению относятся лишь немногие имена существительные мужского рода на {\slv{-ь}}: {\slv{горта́нь, пꙋ́ть}}.
    
    {\slv{Госпо́дїе}}~---~во множественном числе удерживает окончания по третьему склонению.
    
    Возьмем для образца третьего склонения существительные {\slv{ча́сть}} и {\slv{це́рковь}}.
    
    \begin{center}
        Склонения существительных {\slv{ча́сть}} и {\slv{це́рковь}}
        \renewcommand*{\arraystretch}{1.4}
        \footnotesize\begin{tabular}[c]{|c|c|c|c|c|}
            \hline
            
            \makecell{Па-\\деж}
            & \multicolumn{2}{c|}{Единственное число}
            & \multicolumn{2}{c|}{Множественное число}
            \\\hline
            
            \makecell{И.\\В. З.}
            & {\slv{ча́сть}}
            & {\slv{це́рковь}}
            & {\slv{ча̑сти}}
            & {\slv{цє́ркви}}
            \\\hline
            
            Р.
            & {\slv{ча́сти}}
            & {\slv{це́ркве}}
            & {\slv{часте́й}}
            & {\slv{церкве́й}}
            \\\hline
            
            Д.
            & {\slv{ча́сти}}
            & {\slv{це́ркви}}
            & {\slv{часте́мъ}}
            & {\slv{це́рквамъ}}
            \\\hline
            
            Т.
            & {\slv{ча́стїю}}
            & {\slv{це́рковїю}}
            & {\slv{частмѝ}}
            & {\slv{це́рквами}}
            \\\hline
            
            П.
            & {\slv{ѡ҆ ча́сти}}
            & {\slv{ѡ҆ це́ркви}}
            & \makecell{{\slv{ѡ҆ часте́хъ}}}
            & {\slv{ѡ҆ це́рквахъ}}
            \\\hline
            
            \makecell{~\\~}
            & \multicolumn{2}{c|}{Двойственное число}
            \\\cline{1-3}
            
            \makecell{И.\\В. З.}
            & {\slv{ча̑сти}}
            & {\slv{цє́ркви}}
            \\\cline{1-3}
            
            \makecell{Р. П.}
            & \makecell{\slv{ча̑стїю}}
            & {\slv{цє́рковїю}}
            \\\cline{1-3}
            
            \makecell{Д. Т.}
            & \makecell{{\slv{часте́ма}}}
            & \makecell{{\slv{церква́ма}}}
            \\\cline{1-3}
            
        \end{tabular}
    \end{center}

    1. По образцу склонения существительного {\slv{ча́сть}} склоняется существительное {\slv{дла́нь}}, а также некоторые имена существительные, употребляемые только во множественном числе, как-то: {\slv{мо́щи, гꙋ́сли, ꙗ҆́сли, пе́рси}} и др.
    
    2. Некоторые имена существительные третьего склонения, склоняющиеся по образцу существительного {\slv{це́рковь}}, имеют иногда в именительном падеже единственного числа двоякое окончание, например: {\slv{любо́вь}} и {\slv{любы̀}}, {\slv{свекро́вь}} и {\slv{свекры̀}}, {\slv{неплодо́вь}} и {\slv{непло́ды}}.
    
    3. Имя существительное {\slv{мꙋ́дрость}}, относимое к Иисусу Христу, имеет в звательном падеже единственного числа окончание {\slv{-е}}. Например: {\slv{Ѽ мꙋ́дросте, и҆ сло́ве бж҃їй, и҆ си́ло!}} (Тропарь 9-й песни канона Пасхи).

            \subsection{Имя прилагательное}
                \subsubsection{Разделение имен прилагательных}

    \textbf{Именем прилагательным} называется часть речи, обозначающая признак предмета, явления или понятия. Например: {\slv{пра́ведный мꙋ́жъ, ве́лїѧ сла́ва, го́рнее мѣ́сто}}.
    
    Имена прилагательные, как обозначающие признаки предметов, явлений или понятий, могут указывать или на их качество, или на отношение предмета к материалу, месту, времени и т.д. Поэтому имена прилагательные разделяются на \emph{качественные} ({\slv{лю́тый звѣ́рь, блага́ѧ вѣ́сть, до́брое дѣ́ло}}) и \emph{относительные} ({\slv{зла́тъ вѣне́цъ, морска́ѧ волна̀, нощно́е вре́мѧ}}).
    
    Особую разновидность относительных имен прилагательных составляют так называемые имена прилагательные {\slv{притяжательные}}, которые имеют значение принадлежности. Например: {\slv{Но́евъ ковче́гъ, хра́мъ і҆ерⷭ҇ли́мль}}.
    
    По характеру окончаний имена прилагательные разделяются на:
    
    \medskip\autorows{c}{1}{l}{
        {\emph{краткие} ({\slv{до́бръ, зла́тъ}}) и \emph{полные} ({\slv{до́брый, златы́й}})},
        {\emph{твердые} ({\slv{мꙋ́дръ, мꙋ́дрый}}) и \emph{мягкие} ({\slv{си́нїй, си́нь}})}
    }

                \subsubsection{Краткие имена прилагательные}

    Краткие имена прилагательные по своим окончаниям сходны с родовыми окончаниями имен существительных, а именно:
    
    \medskip\autorows{c}{3}{l}{
        {мужской род}, {оканчивается на}, {{\slv{-ъ, -ь}} ({\slv{мꙋ́дръ, си́нь}})},
        {женский род}, {оканчивается на}, {{\slv{-а, -ѧ}} ({\slv{мꙋ́дра, си́нѧ}})},
        {средний род}, {оканчивается на}, {{\slv{-о, -е}} ({\slv{мꙋ́дро, си́не}})}
    }

    а потому и склоняются по соответствующим склонениям имен существительных.

                \subsubsection{Склонение кратких имен прилагательных с твердым окончанием}

    Для образца возьмем склонение краткого имени прилагательного (совместно с существительным) с твердым окончанием: {\slv{до́бръ}} ({\slv{пло́дъ}}), {\slv{до́бра}} ({\slv{ри́за}}), {\slv{до́бро}} ({\slv{дѣ́ло}}).
    
    \begin{center}
%        Склонения прилагательных {\slv{}}
        \renewcommand*{\arraystretch}{1.4}
        \footnotesize\begin{tabular}[c]{|c|c|c|c|c|}
            \hline

            ~
            & \makecell{Па-\\деж}
            & Мужской род
            & Женский род
            & Средний род
            \\\hline
            
            \multirow{7}{*}{\spheading[10em]{Единственное число}}
            & И.
            & {\slv{до́бръ {\scriptsize пло́дъ}}}
            & {\slv{до́бра {\scriptsize ри́за}}}
            & {\slv{до́бро {\scriptsize дѣ́ло}}}
            \\\cline{2-5}
            
            & Р.
            & {\slv{добра̀ {\scriptsize плода̀}}}
            & {\slv{добры̀ {\scriptsize ри́зы}}}
            & {\slv{до́бра {\scriptsize дѣ́ла}}}
            \\\cline{2-5}
            
            & Д.
            & {\slv{до́брꙋ {\scriptsize плодꙋ̀}}}
            & {\slv{до́брѣ {\scriptsize ри́зѣ}}}
            & {\slv{до́брꙋ {\scriptsize дѣ́лꙋ}}}
            \\\cline{2-5}
            
            & В.
            & {\slv{до́бръ}}({\slv{а̀}}) {\slv{\scriptsize пло́дъ}}
            & {\slv{до́брꙋ {\scriptsize ри́зꙋ}}}
            & {\slv{до́бро {\scriptsize дѣ́ло}}}
            \\\cline{2-5}
            
            & З.
            & {\slv{до́бръ {\scriptsize пло́де}}}
            & {\slv{до́бра {\scriptsize ри́зо}}}
            & {\slv{до́бро {\scriptsize дѣ́ло}}}
            \\\cline{2-5}
            
            & Т.
            & {\slv{до́брымъ {\scriptsize пло́домъ}}}
            & {\slv{до́брою {\scriptsize ри́зою}}}
            & {\slv{до́брым {\scriptsize дѣ́ломъ}}}
            \\\cline{2-5}
            
            & П.
            & {\slv{ѡ҆ до́брѣ {\scriptsize плодѣ̀}}}
            & {\slv{ѡ҆ до́брѣ {\scriptsize ри́зѣ}}}
            & {\slv{ѡ҆ до́брѣ {\scriptsize дѣ́лѣ}}}
            \\\hline
            
            \multirow{3}{*}{\spheading[4.5em]{Дв. число}}
            & \makecell{И.\\В. З.}
            & {\slv{дѡ́бра {\scriptsize плѡда̀}}}
            & {\slv{дѡ́брѣ {\scriptsize ри̑зѣ}}}
            & {\slv{дѡ́бра {\scriptsize дѣ̑ла}}}
            \\\cline{2-5}
            
            & Р. П.
            & {\slv{дѡ́брꙋ {\scriptsize плѡдꙋ̀}}}
            & {\slv{дѡ́брꙋ {\scriptsize ри̑зꙋ}}}
            & {\slv{дѡ́брꙋ {\scriptsize дѣ̑лꙋ}}}
            \\\cline{2-5}
            
            & Д. Т.
            & {\slv{до́брыма {\scriptsize пло́дома}}}
            & {\slv{до́брыма {\scriptsize ри́зами}}}
            & {\slv{до́брыма {\scriptsize дѣло́ма}}}
            \\\hline
            
            \multirow{7}{*}{\spheading[10em]{Множественное число}}
            & И.
            & {\slv{до́бри {\scriptsize пло́ди}}}
            & {\slv{дѡбры̀ {\scriptsize ри̑зы}}}
            & {\slv{дѡ́бра {\scriptsize дѣла̀}}}
            \\\cline{2-5}
            
            & Р.
            & {\slv{дѡ́бръ {\scriptsize плѡ́дъ}}}
            & {\slv{дѡ́бръ {\scriptsize ри́зъ}}}
            & {\slv{дѡ́бръ {\scriptsize дѣ́лъ}}}
            \\\cline{2-5}
            
            & Д.
            & {\slv{дѡ́брымъ {\scriptsize пло́дѡмъ}}}
            & {\slv{дѡ́брымъ {\scriptsize ри́замъ}}}
            & {\slv{дѡ́брымъ {\scriptsize дѣлѡ́мъ}}}
            \\\cline{2-5}
            
            & В.
            & {\slv{добры̀ {\scriptsize плоды̀}}}
            & {\slv{дѡбры̀ {\scriptsize ри̑зы}}}
            & {\slv{дѡбра̀ {\scriptsize дѣла̑}}}
            \\\cline{2-5}
            
            & З.
            & {\slv{до́бри {\scriptsize пло́ди}}}
            & {\slv{дѡбры̀ {\scriptsize ри̑зы}}}
            & {\slv{до́бри {\scriptsize дѣла̀}}}
            \\\cline{2-5}
            
            & Т.
            & {\slv{дѡ́бры {\scriptsize плѡ́ды}}}
            & {\slv{до́брыми {\scriptsize ри́зами}}}
            & {\slv{дѡ́бры {\scriptsize дѣ́лы}}}
            \\\cline{2-5}
            
            & П.
            & {\slv{ѡ҆ до́брыхъ {\scriptsize плодѣ́хъ}}}
            & {\slv{ѡ҆ до́брыхъ {\scriptsize ри́захъ}}}
            & {\slv{ѡ҆ до́брыхъ {\scriptsize дѣ́лѣхъ}}}
            \\\hline
            
        \end{tabular}
    \end{center}

    Имена прилагательные в значении существительных в звательном падеже единственного числа принимают их окончания, например: {\slv{млⷭ҇тиве}}. В остальных случаях звательный падеж сходен с именительным.
    
    По образцу склонения кратких прилагательных с твердым окончанием склоняются имена прилагательные притяжательные, например: {\slv{сі́мѡновъ, царе́въ}} и др.
    
    Краткие прилагательные: {\slv{є҆диноро́дъ, сꙋгꙋ҆́бъ}} и т.п.~---~не склоняются.
    
    Имена прилагательные с основой на гортанные смягчают их в соответствующих падежах:
    
    \bigskip\autorows{c}{1}{c}{
        {{\slv{мно́гъ~\textemdash~мно́зи, бла́гъ~\textemdash~бла́зи, крѣ́покъ~\textemdash~крѣ́пцы}}}
    }

    Краткие прилагательные в единственном числе творительном падеже мужского и среднего родов принимают окончания полных прилагательных {\slv{-ымъ, -имъ}}: {\slv{мꙋ́дрымъ, госпо́днимъ}}.

                \subsubsection{Склонение кратких имен прилагательных с мягким окончанием}

    Для образца возьмем склонение краткого имени прилагательного (совместно с существительным) с мягким окончанием: {\slv{си́нь}} ({\slv{пла́тъ}}), {\slv{си́нѧ}} ({\slv{пелена̀}}), {\slv{си́не}} ({\slv{мо́ре}}).

    \begin{center}
        %        Склонения прилагательных {\slv{}}
        \renewcommand*{\arraystretch}{1.4}
        \footnotesize\begin{tabular}[c]{|c|c|c|c|c|}
            \hline
            
            ~
            & \makecell{Па-\\деж}
            & Мужской род
            & Женский род
            & Средний род
            \\\hline
            
            \multirow{7}{*}{\spheading[10em]{Единственное число}}
            & И.
            & {\slv{си́нь {\scriptsize пла́тъ}}}
            & {\slv{си́нѧ {\scriptsize пелена̀}}}
            & {\slv{си́не {\scriptsize мо́ре}}}
            \\\cline{2-5}
            
            & Р.
            & {\slv{си́нѧ {\scriptsize пла́та}}}
            & {\slv{си́ни {\scriptsize пелены̀}}}
            & {\slv{си́нѧ {\scriptsize мо́рѧ}}}
            \\\cline{2-5}
            
            & Д.
            & {\slv{си́ню {\scriptsize пла́тꙋ}}}
            & {\slv{си́ни {\scriptsize пеленѣ̀}}}
            & {\slv{си́ню {\scriptsize мо́рю}}}
            \\\cline{2-5}
            
            & В.
            & {\slv{си́нь(ѧ) {\scriptsize пла́тъ}}}
            & {\slv{си́ню {\scriptsize пеленꙋ̀}}}
            & {\slv{си́не {\scriptsize мо́ре}}}
            \\\cline{2-5}
            
            & З.
            & {\slv{си́нь {\scriptsize пла́те}}}
            & {\slv{си́нѧ {\scriptsize пелено̀}}}
            & {\slv{си́не {\scriptsize мо́ре}}}
            \\\cline{2-5}
            
            & Т.
            & {\slv{си́нимъ {\scriptsize пла́том}}}
            & {\slv{си́нею {\scriptsize пелено́ю}}}
            & {\slv{си́нимъ {\scriptsize мо́ремъ}}}
            \\\cline{2-5}
            
            & П.
            & {\slv{ѡ҆ си́ни {\scriptsize пла́тѣ}}}
            & {\slv{ѡ҆ си́ни {\scriptsize пеленѣ̀}}}
            & {\slv{ѡ҆ си́ни {\scriptsize мо́ри}}}
            \\\hline
            
            \multirow{3}{*}{\spheading[4.5em]{Дв. число}}
            & \makecell{И.\\В. З.}
            & {\slv{си̑ни {\scriptsize пла̑та}}}
            & {\slv{си̑ни {\scriptsize пелєнѣ̀}}}
            & {\slv{си̑ни {\scriptsize мѡ́рѧ}}}
            \\\cline{2-5}
            
            & Р. П.
            & {\slv{си̑ню {\scriptsize пла̑тꙋ}}}
            & {\slv{си̑ню {\scriptsize пелєнꙋ̀}}}
            & {\slv{си̑ню {\scriptsize мѡ́рю}}}
            \\\cline{2-5}
            
            & Д. Т.
            & {\slv{си́нима {\scriptsize пла́тома}}}
            & {\slv{си́нима {\scriptsize пелена́ма}}}
            & {\slv{си́нима {\scriptsize мо́рема}}}
            \\\hline
            
            \multirow{6}{*}{\spheading[10em]{Множественное число}}
            & И. З.
            & {\slv{си̑ни {\scriptsize пла́ти}}}
            & {\slv{си̑ни {\scriptsize пелєны̀}}}
            & {\slv{си̑нѧ {\scriptsize мо́рѧ}}}
            \\\cline{2-5}
            
            & Р.
            & {\slv{си̑нь {\scriptsize пла̑тъ}}}
            & {\slv{си̑нь {\scriptsize пеле́нъ}}}
            & {\slv{си̑нь {\scriptsize море́й}}}
            \\\cline{2-5}
            
            & Д.
            & {\slv{си̑нимъ {\scriptsize пла́тѡмъ}}}
            & {\slv{си̑нимъ {\scriptsize пелена́мъ}}}
            & {\slv{си̑нимъ {\scriptsize мо́рємъ}}}
            \\\cline{2-5}
            
            & В.
            & {\slv{си̑ни {\scriptsize пла̑ты}}}
            & {\slv{си̑ни {\scriptsize пелєны̀}}}
            & {\slv{си̑нѧ {\scriptsize морѧ̀}}}
            \\\cline{2-5}
            
            & Т.
            & \makecell{{\slv{си̑ни {\scriptsize пла̑ты}}}\\{\slv{си́ними}}}
            & {\slv{си́нѧми {\scriptsize пелена́ми}}}
            & {\slv{си̑ни {\scriptsize мѡ́ри (морѧ́ми)}}}
            \\\cline{2-5}
            
            & П.
            & {\slv{ѡ҆ си́нихъ {\scriptsize пла́тѣхъ}}}
            & {\slv{ѡ҆ си́нѧхъ {\scriptsize пелена́хъ}}}
            & \makecell{{\slv{ѡ҆ си́нихъ {\scriptsize мо́рехъ}}}\\{\slv{\scriptsize (морѧ́хъ)}}}
            \\\hline
            
        \end{tabular}
    \end{center}

    По образцу склонения кратких прилагательных с мягким окончанием склоняются имена прилагательные притяжательные, например: {\slv{саꙋ́ль, і҆а́ковль}} и т.п.
    
    Краткие прилагательные, как например {\slv{свобо́дь}} и т.п., не склоняются.

                \subsubsection{Склонение кратких имен прилагательных с основой на шипящие}

    Приведем образец склонения краткого имени прилагательного (также совместно с существительными) с основой на шипящий звук {\slv{щ}}~---~{\slv{то́щь}} (напрасен, тщетен): {\slv{то́щь}} ({\slv{да́ръ}}), {\slv{то́ща}} ({\slv{бра́нь}}), {\slv{то́ще}} ({\slv{помышле́нїе}}).
    
    \begin{center}
        %        Склонения прилагательных {\slv{}}
        \renewcommand*{\arraystretch}{1.4}
        \footnotesize\begin{tabular}[c]{|c|c|c|c|c|}
            \hline
            
            ~
            & \makecell{Па-\\деж}
            & Мужской род
            & Женский род
            & Средний род
            \\\hline
            
            \multirow{7}{*}{\spheading[10em]{Единственное число}}
            & И.
            & {\slv{то́щь {\scriptsize да́ръ}}}
            & {\slv{то́ща {\scriptsize бра́нь}}}
            & {\slv{то́ще {\scriptsize помышле́нїе}}}
            \\\cline{2-5}
            
            & Р.
            & {\slv{то́ща {\scriptsize да́ра}}}
            & {\slv{то́щи {\scriptsize бра́ни}}}
            & {\slv{то́ща {\scriptsize помышле́нїѧ}}}
            \\\cline{2-5}
            
            & Д.
            & {\slv{то́щꙋ {\scriptsize да́рꙋ}}}
            & {\slv{то́щи {\scriptsize бра́ни}}}
            & {\slv{то́щꙋ {\scriptsize помышле́нїю}}}
            \\\cline{2-5}
            
            & В.
            & {\slv{то́щь(а) {\scriptsize да́ръ}}}
            & {\slv{то́щꙋ {\scriptsize бра́нь}}}
            & {\slv{то́ще {\scriptsize помышле́нїе}}}
            \\\cline{2-5}
            
            & З.
            & {\slv{то́щь {\scriptsize да́ре}}}
            & {\slv{то́ща {\scriptsize бра́нь}}}
            & {\slv{то́ще {\scriptsize помышле́нїе}}}
            \\\cline{2-5}
            
            & Т.
            & {\slv{то́щимъ {\scriptsize да́ромъ}}}
            & {\slv{то́щею {\scriptsize бра́нїю}}}
            & {\slv{то́щимъ {\scriptsize помышле́ниемъ}}}
            \\\cline{2-5}
            
            & П.
            & {\slv{ѡ҆ то́щи {\scriptsize да́рѣ}}}
            & {\slv{ѡ҆ то́щи {\scriptsize бра́ни}}}
            & {\slv{ѡ҆ то́щи {\scriptsize помышле́нїи}}}
            \\\hline
            
            \multirow{3}{*}{\spheading[4.5em]{Дв. число}}
            & \makecell{И.\\В. З.}
            & {\slv{тѡ́ща {\scriptsize да̑ра}}}
            & {\slv{тѡ́щи {\scriptsize бра̑ни}}}
            & {\slv{тѡ́щи {\scriptsize помышлє́нїѧ}}}
            \\\cline{2-5}
            
            & Р. П.
            & {\slv{тѡ́щꙋ {\scriptsize да̑рꙋ}}}
            & {\slv{тѡ́щꙋ {\scriptsize бра̑нїю}}}
            & {\slv{тѡ́щꙋ {\scriptsize помышлє́нїю}}}
            \\\cline{2-5}
            
            & Д. Т.
            & {\slv{то́щима {\scriptsize да́рома}}}
            & {\slv{то́щима {\scriptsize бра́нема}}}
            & {\slv{то́щима {\scriptsize помышле́нїима}}}
            \\\hline
            
            \multirow{6}{*}{\spheading[10em]{Множественное число}}
            & И. З.
            & {\slv{тѡ́щи {\scriptsize да́ри}}}
            & {\slv{тѡ́щи {\scriptsize бра̑ни}}}
            & {\slv{тѡ́ща {\scriptsize помышлє́нїѧ}}}
            \\\cline{2-5}
            
            & Р.
            & {\slv{тѡ́щь {\scriptsize да̑ръ}}}
            & {\slv{тѡ́щь {\scriptsize бра́ней}}}
            & {\slv{тѡ́щь {\scriptsize помышле́нїй}}}
            \\\cline{2-5}
            
            & Д.
            & {\slv{то́щымъ {\scriptsize дарѡ́мъ}}}
            & {\slv{то́щымъ {\scriptsize бра́немъ}}}
            & {\slv{то́щымъ {\scriptsize помышле́нїємъ}}}
            \\\cline{2-5}
            
            & В.
            & {\slv{то́щы {\scriptsize да́ры}}}
            & {\slv{то́щы {\scriptsize бра̑ни}}}
            & {\slv{тѡ́ща {\scriptsize помышлє́нїѧ}}}
            \\\cline{2-5}
            
            & Т.
            & {\slv{тѡ́щы {\scriptsize да̑ры}}}
            & {\slv{тѡ́щы {\scriptsize бра́нми}}}
            & {\slv{тѡ́щы {\scriptsize помышлє́нїи}}}
            \\\cline{2-5}
            
            & П.
            & {\slv{ѡ҆ то́щихъ {\scriptsize дарѣ́хъ}}}
            & {\slv{ѡ҆ то́щихъ {\scriptsize бране́хъ}}}
            & {\slv{ѡ҆ то́щихъ {\scriptsize помышле́нїихъ}}}
            \\\hline
            
        \end{tabular}
    \end{center}

    Имена прилагательные краткие с основой на шипящие хотя и имеют твердое окончание, но склоняются по образцу кратких прилагательных с мягким окончанием, заменяя после шипящих {\slv{ж, ч, ш, щ}} гласные {\slv{ѧ}} на {\slv{а}}, {\slv{ю}} на {\slv{ꙋ}}, {\slv{ѣ}} на {\slv{и}}. Отступления от этого правила допускаются только для различения созвучных форм в различных числах и падежах. Так, например, буква {\slv{ы}} пишется только для различия падежей множественного числа от созвучных падежей единственного числа.
    
    Имена прилагательные с основой на шипящую, как и имена существительные, имеют смешанное склонение.


            \subsection{Местоимение}
                \subsubsection{Понятие о местоимении}

    \textbf{Местоимением} называется часть речи, употребляемая вместо имени существительного или его заменяющих имен. Например:
    
    \medskip\autorows{c}{1}{c}{
        {{\slv{Кто̀ ѿ ва́съ ѡ҆блича́етъ мѧ̀ ѡ҆ грѣсѣ̀;}} (Мф. 8, 46)}
    }

    Все местоимения разделяются на неродовые и родовые.
    
    \textbf{Неродовые} местоимения~---~это те, которые не изменяются по родам, а следовательно, могут относиться ко всем родам. К таким местоимениям относятся {\slv{а҆́зъ}} и {\slv{ты̀}}. Эти местоимения называются \emph{личными}, так как они относятся к тому или другому лицу; {\slv{а҆́зъ}}~---~это местоимение 1-го лица, а {\slv{ты̀}}~---~местоимение 2-го лица. Кроме того, к неродовым местоимениям относится \emph{возвратное} местоимение {\slv{себє̀}} и \emph{вопросительные} {\slv{кто̀}} и {\slv{что̀}}.
    
    \textbf{Родовыми} местоимениями называются такие, которые изменяются по родам в зависимости от определенного слова.
    
    Все местоимения звательного падежа не имеют.
    
    Рассмотрим склонение некоторых местоимений.

                \subsubsection{Склонение личных местоимений {\slv{а҆́зъ}} и {\slv{ты̀}} и возвратного {\slv{себє̀}}}
    
    \begin{center}
%        Склонение личных местоимений
        \renewcommand*{\arraystretch}{1.4}
        \footnotesize\begin{tabular}[c]{|c|c|c|c|c|}
            \hline
            
            \makecell{Па-\\деж}
            & \multicolumn{2}{c|}{Единственное число}
            & \multicolumn{2}{c|}{Множественное число}
            \\\hline
            
            \makecell{И.}
            & {\slv{а҆́зъ}}
            & {\slv{ты̀}}
            & {\slv{мы̀}}
            & {\slv{вы̀}}
            \\\hline
            
            Р.
            & {\slv{менє̀}}
            & {\slv{тебє̀}}
            & {\slv{на́съ}}
            & {\slv{ва́съ}}
            \\\hline
            
            Д.
            & {\slv{мнѣ̀, мѝ}}
            & {\slv{тебѣ̀, тѝ}}
            & {\slv{на́мъ}}
            & {\slv{ва́мъ}}
            \\\hline
            
            В.
            & {\slv{менѐ, мѧ̀}}
            & {\slv{тебѐ, тѧ̀}}
            & {\slv{на́съ, ны̀}}
            & {\slv{ва́съ, вы̀}}
            \\\hline

            Т.
            & {\slv{мно́ю}}
            & {\slv{тобо́ю}}
            & {\slv{на́ми}}
            & {\slv{ва́ми}}
            \\\hline
            
            П.
            & {\slv{ѡ҆ мнѣ̀}}
            & {\slv{ѡ҆ тебѣ̀}}
            & \makecell{{\slv{ѡ҆ на́съ}}}
            & {\slv{ѡ҆ ва́съ}}
            \\\hline
            
            \makecell{~\\~}
            & \multicolumn{2}{c|}{Двойственное число}
            \\\cline{1-3}
            
            \makecell{И.}
            & {\slv{мы̀}}
            & {\slv{вы̀}}
            \\\cline{1-3}
            
            \makecell{Р. П.}
            & \makecell{\slv{на́ю}}
            & {\slv{ва́ю}}
            \\\cline{1-3}
            
            \makecell{Д. Т.}
            & \makecell{{\slv{на́ма}}}
            & \makecell{{\slv{ва́ма}}}
            \\\cline{1-3}
            
            \makecell{В.}
            & {\slv{ны̀}}
            & {\slv{вы̀}}
            \\\cline{1-3}

        \end{tabular}
    \end{center}

    Возвратное местоимение {\slv{себє̀}} именительного падежа не имеет, а также не имеет двойственного и множественного чисел.
    
    Созвучные падежи (родительный и винительный единственного числа) различаются между собою по начертанию: родительный падеж имеет в окончании {\slv{є}} (удлиненное), а винительный падеж~---~{\slv{е}} (обыкновенное).
    
    Дательный и предложный падежи единственного числа имеют окончание {\slv{-ѣ}}.
    
    Дательный и винительный падежи единственного числа, а также винительный падеж множественного числа имеют две формы: \emph{полную} и \emph{энклитическую} ({\slv{мѝ, тѝ, сѝ; мѧ̀, тѧ̀, сѧ̀; ны̀, вы̀}}). Эти формы в сочетании с предыдущим словом как бы составляют с ним одно целое и даже иногда, когда предыдущее слово имеет конечный ударный открытый слог, утрачивают собственное ударение (см. \S9, п. 1). Например: {\slv{Оу҆слы́ши мѧ̀, гдⷭ҇и! Спасѝ мѧ̀, бж҃е мо́й!}}
    
    Такие формы и названы \emph{энклитическими} от греческого {$\varepsilon\gamma\kappa\lambda\iota\nu\omega$}~---~склоняюсь, т.е. применительно к данному случаю~---~утрачиваю ударение

                \subsubsection{Особенности глагольного сказуемого в предложении}

    Глагол в предложении почти всегда является \textbf{сказуемым}. В этом случае глагол-сказуемое может быть простым и составным.
    
    \emph{Простым} глагольным сказуемым называется такое сказуемое, которое выражено только одним глаголом. Например: {\slv{А҆́зъ къ бг҃ꙋ воззва́хъ}} (Пс. 54, 17).
    
    \emph{Составным} глагольным сказуемым называется такое сказуемое, которое выражено или двумя глаголами, или глаголом совместно с именем существительным, кратким прилагательным или местоимением. Например: {\slv{А҆́зъ снидо́хъ и҆з̾ѧти и҆̀хъ}} (Деян. 7, 34). {\slv{Вѣ́ра без̾ дѣ́лъ мертва̀ є҆́сть}} (Иак. 2, 20).
    
    Во втором примере, где составное сказуемое выражено кратким прилагательным совместно с глаголом {\slv{бы́ти}} в настоящем времени, заслуживает особого внимания. Глагол {\slv{бы́ти}} в составе подобных сказуемых называется \emph{связкой}. Глагол-связка выполняет в составном сказуемом лишь вспомогательную роль, устанавливая связь между подлежащим и сказуемым. Основное же значение сказуемого выражается входящим в его состав именным словом.
    
    Если глагол-связка стоит в настоящем времени (как во втором примере), то при переводе такого предложения на русский язык, эта связка опускается. Так, например, второе предложение переводится на русский язык так: <<Вера без дел мертва>>.

                \subsubsection{Согласование слов в предложении}

    \textbf{Согласованием} называется такое сочетание слов в предложении, когда эти слова поставлены или в одинаковом падеже, или в одном роде, числе и падеже, или, наконец, в одном числе и лице.
    
    Подлежащее, выраженное именем существительным, согласуется с глагольным сказуемым в одинаковом числе (и роде, если число двойственное):
    
    \bigskip\autorows{c}{1}{c}{
        {{\slv{Два̀ ᲂу҆чн҃ка и҆до́ста въ ве́сь}}}
    }

    При подлежащем, выраженном именем существительным собирательным, согласование его с глагольным сказуемым бывает по смыслу, т.е. при таком подлежащем в единственном числе сказуемое ставится во множественном, так как собирательное имя заключает в себе многие предметы. Например:
    
    \bigskip\autorows{c}{1}{c}{
        {{\slv{Наро́дъ же стоѧ́й слы́шавъ глаго́лахꙋ}} (Ин. 12, 29)}
    }

    При двух подлежащих глагольное сказуемое ставится в предложении большей частью в двойственном числе, причем в роде преимущественном. Например:
    
    \medskip\autorows{c}{1}{c}{
        {{\slv{Пра́вда}} (жен. р.) {\slv{и҆ ми́ръ}} (муж. р.) {\slv{ѡ҆блобыза́стасѧ}} (муж. р.) (Пс. 84, 11)}
    }

    При подлежащем во множественном числе или при многих подлежащих глагольное сказуемое ставится во множественном числе. Например:
        
    \bigskip\autorows{c}{1}{c}{
        {{\slv{Всѝ дні́е на́ши ѡ҆скꙋдѣ́ша}} (Пс. 89, 9)}
    }

    Имена прилагательные употребляются в речи совместно с именами существительными, а потому согласуются с ними в роде, числе и падеже, например: {\slv{но́ваѧ за́повѣдь, мꙋ́дри мꙋ́жїе}}, причем полные имена прилагательные бывают определениями, а краткие~---~сказуемыми, например:
        
    \bigskip\autorows{c}{1}{c}{
        {{\slv{Ско́рбное се́рдце. Се́й чꙋ́ждъ є҆́сть кро́ве}}}
    }

                \subsubsection{Управление слов в предложении}

    \textbf{Управлением} называется сочетание слов в предложении, когда одно слово зависит от другого и выражает свою зависимость определенным падежом. Например: {\slv{Добро̀ сотвори́ти человѣ́кꙋ. Надѣ́ѧтисѧ на бг҃а}}.
    
    Здесь слова {\slv{}}, {\slv{}} поставлены в определенном падеже, а иногда соединены с предлогом по требованию других слов: {\slv{}}, {\slv{}}. Эти слова, зависящие от других слов, называются \emph{управляемыми}.

                \subsubsection{Обращение}

    \textbf{Обращением} называется слово, называющее того, к кому или к чему обращаются с речью.
    
    Пунктуация при обращении следует тем же правилам, что и в русском языке:
    
    \bigskip\autorows{c}{1}{c}{
        {{\slv{Гдⷭ҇и, кто́ ѡ҆бита́етъ въ жили́щи твое́мъ;}} (Пс. 14, 1)},
        {{\slv{Оу҆слы́ши, гдⷭ҇и, пра́вдꙋ мою̀}} (Пс. 16, 1)},
        {{\slv{Ко́ль возлю́блєнна селє́нїѧ твоѧ̑, гдⷭ҇и си́лъ!}}}
    }

    Обращение не является членом предложения, т.к. не отвечает ни на какой вопрос и грамматически не связано с ним. Ставится обращение всегда в звательном падеже.

    \chapter*{2-й класс}
    \label{ch:secondgrade}
    \addcontentsline{toc}{chapter}{\nameref{ch:secondgrade}}
    \setcounter{section}{0}
    \setcounter{subsubsection}{0}
        \section{Склоняемые части речи}
            \subsection{Имя существительное}
                \subsubsection{Имена существительные неравносложные}

    Некоторые имена существительные принимают в косвенных падежах между основой и окончанием \emph{суффикс} (или, как говорят, \emph{наращение}), отсутствующий в именительном падеже единственного числа. Этот суффикс (наращение) образует в слове лишний слог, а потому такие имена существительные называются \textbf{неравносложными}.
    
    Имена существительные могут принимать такие наращения: {\slv{-ер-}}, {\slv{-ес-}}, {\slv{-ат-}} ({\slv{-ѧт-}}), {\slv{-ен-}}.
    
    \bigskip\underline{1. Склонение существительных с наращением {\slv{-ер-}}}
    \bigskip
    
    К таким именам существительным относятся только два: {\slv{ма́ти}} и {\slv{дщѝ}}.
    
    \begin{center}
        %        Склонение существительных с наращением {\slv{-ер-}}
        \renewcommand*{\arraystretch}{1.4}
        \footnotesize\begin{tabular}[c]{|c|c|c|c|c|}
            \hline
            
            \makecell{Па-\\деж}
            & \multicolumn{2}{c|}{Единственное число}
            & \multicolumn{2}{c|}{Множественное число}
            \\\hline
            
            \makecell{И. З.}
            & {\slv{ма́ти}}
            & {\slv{дщѝ}} ({\slv{дще́рь}})
            & {\slv{ма́тєри}}
            & {\slv{дщє́ри}}
            \\\hline
            
            Р.
            & {\slv{ма́тере}}
            & {\slv{дще́ре}}
            & {\slv{ма́терей, ма́терїй}}
            & {\slv{дще́рей}}
            \\\hline
            
            Д.
            & {\slv{ма́тери}}
            & {\slv{дще́ри}}
            & {\slv{ма́теремъ}}
            & {\slv{дще́ремъ}}
            \\\hline
            
            В.
            & {\slv{ма́терь}}
            & {\slv{дще́рь}}
            & {\slv{матере́й}}
            & {\slv{дщє́ри}}
            \\\hline
            
            Т.
            & {\slv{ма́терїю}}
            & {\slv{дще́рїю}}
            & {\slv{ма́терьми}}
            & {\slv{дще́рьми}}
            \\\hline
            
            П.
            & {\slv{ѡ҆ ма́тери}}
            & {\slv{ѡ҆ дще́ри}}
            & \makecell{{\slv{ѡ҆ ма́терехъ}}}
            & {\slv{ѡ҆ дще́рехъ}}
            \\\hline
            
            \makecell{~\\~}
            & \multicolumn{2}{c|}{Двойственное число}
            \\\cline{1-3}
            
            \makecell{И.\\В. З.}
            & {\slv{ма́тєри}}
            & {\slv{дщє́ри}}
            \\\cline{1-3}
            
            \makecell{Р. П.}
            & \makecell{\slv{ма́тєрїю}}
            & {\slv{дщє́рїю}}
            \\\cline{1-3}
            
            \makecell{Д. Т.}
            & \makecell{{\slv{ма́терема}}}
            & \makecell{{\slv{дще́рема}}}
            \\\cline{1-3}
            
        \end{tabular}
    \end{center}

    Как видно из рассмотрения этого склонения, оно является мягким и принимает окончания третьего склонения имен существительных, причем винительный падеж единственного и множественного чисел сходен с именительным.
        
    \bigskip\underline{2. Склонение существительных с наращением {\slv{-ес-}}}
    \bigskip
    
    К такому склонению относятся некоторые имена существительные \textbf{среднего} рода, оканчивающиеся на {\slv{-о}}, например: {\slv{не́бо, сло́во, чꙋ́до, тѣ́ло}} и др.
    
    Для образца этого склонения возьмем имя существительное {\slv{не́бо}}.
    
    \begin{center}
        %        Склонение существительных с наращением {\slv{-ес-}}
        \renewcommand*{\arraystretch}{1.4}
        \footnotesize\begin{tabular}[c]{|c|c|c|c|c|c|}
            \hline
            
            \multicolumn{2}{|c|}{Единственное число}
            & \multicolumn{2}{c|}{Двойственное число}
            & \multicolumn{2}{c|}{Множественное число}
            \\\hline
            
            \makecell{И. В. З.}
            & {\slv{не́бо}}
            & \makecell{И. В. З.}
            & {\slv{небєсѝ}}
            & \makecell{И. В. З.}
            & {\slv{небеса̀}}
            \\\hline

            \makecell{Р.}
            & {\slv{небесѐ}}
            & \makecell{Р. П.}
            & {\slv{небесꙋ̀}}
            & \makecell{Р.}
            & {\slv{небе́съ}}
            \\\hline

            \makecell{Д.}
            & {\slv{небесѝ}}
            & \multirow{3}{*}{Д. Т.}
            & \multirow{3}{*}{\slv{небесе́ма}}
            & \makecell{Д.}
            & {\slv{небесє́мъ}}
            \\\cline{1-2}\cline{5-6}

            \makecell{Т.}
            & {\slv{небесе́мъ}}
            &
            &
            & \makecell{Т.}
            & {\slv{небесы̀}}
            \\\cline{1-2}\cline{5-6}

            \makecell{П.}
            & {\slv{ѡ҆ небесѝ}}
            &
            &
            & \makecell{П.}
            & {\slv{ѡ҆ небесѣ́хъ}}
            \\\hline

        \end{tabular}
    \end{center}

    Существительное {\slv{сло́во}}, если оно означает Сына Божия, наращения {\slv{-ес-}} не принимает, склоняется только в единственном числе по второму склонению и в звательном падеже имеет окончание {\slv{-е}}: {\slv{сло́ве бж҃їй}}.
    
    Склонение имен существительных с наращением {\slv{-ес-}} является смешанным: частью твердым, а частью мягким, как это видно из прилагаемого образца.
        
    \bigskip\underline{3. Склонение существительных с наращением {\slv{-ат-}} ({\slv{-ѧт-}})}
    \bigskip
    
    К такому склонению относятся некоторые имена существительные \textbf{среднего} рода с окончанием на {\slv{-а}} и {\slv{-ѧ}}, например: {\slv{ѻ҆троча̀, ѻ҆вча̀, ꙗ҆гнѧ̀, ѻ҆слѧ̀, жребѧ̀}} и др.
    
    Приведем склонение имени существительного {\slv{ѻ҆троча̀}}.
    
    \begin{center}
        %        Склонение существительных с наращением {\slv{-ат-}} ({\slv{-ѧт-}})
        \renewcommand*{\arraystretch}{1.4}
        \footnotesize\begin{tabular}[c]{|c|c|c|c|c|c|}
            \hline
            
            \multicolumn{2}{|c|}{Единственное число}
            & \multicolumn{2}{c|}{Двойственное число}
            & \multicolumn{2}{c|}{Множественное число}
            \\\hline
            
            \makecell{И. В. З.}
            & {\slv{ѻ҆троча̀}}
            & \makecell{И. В. З.}
            & {\slv{ѻ҆трѡча́ти}}
            & \makecell{И. В. З.}
            & {\slv{ѻ҆троча́та}}
            \\\hline
            
            \makecell{Р.}
            & {\slv{ѻ҆троча́те}}
            & \makecell{Р. П.}
            & {\slv{ѻ҆троча́тꙋ}}
            & \makecell{Р.}
            & {\slv{ѻ҆троча́тъ}}
            \\\hline
            
            \makecell{Д.}
            & {\slv{ѻ҆троча́ти}}
            & \multirow{3}{*}{Д. Т.}
            & \multirow{3}{*}{\slv{ѻ҆троча́тема}}
            & \makecell{Д.}
            & {\slv{ѻ҆троча́тємъ}} ({\slv{-ѡмъ}})
            \\\cline{1-2}\cline{5-6}
            
            \makecell{Т.}
            & {\slv{ѻ҆троча́темъ}}
            &
            &
            & \makecell{Т.}
            & {\slv{ѻ҆троча́ты}}
            \\\cline{1-2}\cline{5-6}
            
            \makecell{П.}
            & {\slv{ѡ҆ ѻ҆троча́ти}}
            &
            &
            & \makecell{П.}
            & {\slv{ѡ҆ ѻ҆троча́техъ}}
            \\\hline
            
        \end{tabular}
    \end{center}

    Как видно из прилагаемого образца, это склонение в большей своей части~---~твердое, причем предложный падеж множественного числа имеет окончание {\slv{-ехъ}} (а не {\slv{-ѣхъ}}).

    \bigskip\underline{4. Склонение существительных с наращением {\slv{-ен-}}}
    \bigskip
    
    Имена существительные с наращением {\slv{-ен-}} принадлежат к мягкому склонению с окончанием на {\slv{-ѧ}}, если основа оканчивается на согласный {\slv{м}}, например: {\slv{и҆́мѧ, вре́мѧ, сѣ́мѧ, бре́мѧ}} и др. Все такие имена \textbf{среднего} рода. В предложном падеже множественного числа они имеют окончание {\slv{-ехъ}}. 
    
    Возьмем для образца склонение существительного {\slv{и҆́мѧ}}.

    \begin{center}
        %        Склонение существительных с наращением {\slv{-ен-}}}
        \renewcommand*{\arraystretch}{1.4}
        \footnotesize\begin{tabular}[c]{|c|c|c|c|c|c|}
            \hline
            
            \multicolumn{2}{|c|}{Единственное число}
            & \multicolumn{2}{c|}{Двойственное число}
            & \multicolumn{2}{c|}{Множественное число}
            \\\hline
            
            \makecell{И. В. З.}
            & {\slv{и҆́мѧ}}
            & \makecell{И. В. З.}
            & {\slv{и҆́мєни}}
            & \makecell{И. В. З.}
            & {\slv{и҆мена̀}}
            \\\hline
            
            \makecell{Р.}
            & {\slv{и҆́мене}}
            & \makecell{Р. П.}
            & {\slv{и҆менꙋ̀}}
            & \makecell{Р.}
            & {\slv{и҆ме́нъ}}
            \\\hline
            
            \makecell{Д.}
            & {\slv{и҆́мени}}
            & \multirow{3}{*}{Д. Т.}
            & \multirow{3}{*}{\slv{и҆мене́ма}}
            & \makecell{Д.}
            & {\slv{и҆менє́мъ}} ({\slv{-ѡ́мъ}})
            \\\cline{1-2}\cline{5-6}
            
            \makecell{Т.}
            & {\slv{и҆́менемъ}}
            &
            &
            & \makecell{Т.}
            & {\slv{и҆мены̀}}
            \\\cline{1-2}\cline{5-6}
            
            \makecell{П.}
            & {\slv{ѡ҆ и҆́мени}}
            &
            &
            & \makecell{П.}
            & {\slv{ѡ҆ и҆́менехъ}}
            \\\hline
            
        \end{tabular}
    \end{center}

                \subsubsection{Имена существительные разносклоняемые}

    В церковнославянском языке есть немного имен существительных, из которых каждое склоняется по окончаниям различных склонений. Такие имена существительные обыкновенно называются \textbf{разносклоняемыми}.
    
    Приведем образцы склонений этих разносклоняемых существительных.

    \bigskip\underline{1. Склонение имен существительных {\slv{гдⷭ҇ь}} и {\slv{господи́нъ}}}
    \bigskip
    
    Существительное {\slv{гдⷭ҇ь}} склоняется только в единственном числе по второму склонению. Но, несмотря на то, что это слово оканчивается на {\slv{-ь}} и потому как бы принадлежит к мягкому склонению, в действительности имеет окончания твердого склонения, причем звательный падеж оканчивается на {\slv{-и}}, например {\slv{гдⷭ҇и, поми́лꙋй}}.

    Существительное {\slv{господи́нъ}} (в смысле~---~земной господин) в единственном и двойственном числах склоняется правильно по второму склонению, но во множественном числе имеет значительные отступления, а также многообразные падежные формы.

    \begin{center}
        %        Склонение существительных {\slv{гдⷭ҇ь}} и {\slv{господи́нъ}}
        \renewcommand*{\arraystretch}{1.4}
        \footnotesize\begin{tabular}[c]{|c|c|}
            \hline
            
            \multicolumn{2}{|c|}{Множественное число}
            \\\hline
            
            \makecell{И. З.}
            & {\slv{госпо́дїе}}
            \\\hline
            
            \makecell{Р.}
            & {\slv{господе́й, госпо́дїй, госпѡ́дъ}}
            \\\hline
            
            \makecell{Д.}
            & {\slv{господє́мъ, госпо́дїѧмъ, господа́мъ}}
            \\\hline
            
            \makecell{В.}
            & {\slv{господы̀, госпо́дїй}}
            \\\hline

            \makecell{Т.}
            & {\slv{господмѝ, госпо́дїѧми, господа́ми, госпѡды̀}}
            \\\hline
            
            \makecell{П.}
            & {\slv{ѡ҆ господѣ́хъ, ѡ҆ госпо́дїѧхъ}}
            \\\hline
            
        \end{tabular}
    \end{center}

    \bigskip\underline{2. Склонение имени существительного {\slv{бра́тъ}}}
    \bigskip
    
    Существительное {\slv{бра́тъ}} в единственном и двойственном числах склоняется правильно по второму склонению. Но во множественном числе это существительное принимает два различных смысла: {\slv{бра́тїе}}, т.е. несколько братьев, и {\slv{бра́тїѧ}}, т.е. существительное собирательное (в смысле хотя бы монастырской братии), склоняющееся только во множественном числе с некоторыми особенностями по первому склонению имен существительных.
    
    \begin{center}
        %        Склонение существительного {\slv{бра́тъ}}
        \renewcommand*{\arraystretch}{1.4}
        \footnotesize\begin{tabular}[c]{|c|c|c|}
            \hline
            
            \multicolumn{3}{|c|}{Множественное число}
            \\\hline
            
            \makecell{И. З.}
            & {\slv{бра́тїе}}
            & {\slv{бра́тїѧ}}
            \\\hline
            
            \makecell{Р.}
            & {\slv{бра́тїй}}
            & {\slv{бра́тїи}}
            \\\hline
            
            \makecell{Д.}
            & {\slv{бра́тїємъ, бра́тїѧмъ}}
            & {\slv{бра́тїи}}
            \\\hline
            
            \makecell{В.}
            & {\slv{бра́тїй}}
            & {\slv{бра́тїю}}
            \\\hline
            
            \makecell{Т.}
            & {\slv{бра́тїѧми}}
            & {\slv{бра́тїею, бра́тїю}}
            \\\hline
            
            \makecell{П.}
            & {\slv{ѡ҆ бра́тїѧхъ}}
            & {\slv{ѡ҆ бра́тїи}}
            \\\hline
            
        \end{tabular}
    \end{center}


    \bigskip\underline{3. Склонение имени существительного {\slv{де́нь}}}
    \bigskip
    
    Существительное {\slv{де́нь}} склоняется с некоторыми отступлениями.

    \begin{center}
        %        Склонение существительного {\slv{де́нь}}}
        \renewcommand*{\arraystretch}{1.4}
        \footnotesize\begin{tabular}[c]{|c|c|c|c|c|c|}
            \hline
            
            \multicolumn{2}{|c|}{Единственное число}
            & \multicolumn{2}{c|}{Двойственное число}
            & \multicolumn{2}{c|}{Множественное число}
            \\\hline
            
            \makecell{И. З.}
            & {\slv{де́нь}}
            & \multirow{2}{*}{И. В. З.}
            & \multirow{2}{*}{\slv{дни̑}}
            & \makecell{И. З.}
            & {\slv{дні́е}}
            \\\cline{1-2}\cline{5-6}
            
            \makecell{Р.}
            & {\slv{днѐ}}
            &
            &
            & \makecell{Р.}
            & {\slv{дні́й}}
            \\\hline
            
            \makecell{Д.}
            & {\slv{днѝ, дне́ви}}
            & \multirow{2}{*}{Р. П.}
            & \multirow{2}{*}{\slv{дню̀, дні́ю}}
            & \makecell{Д.}
            & {\slv{днє́мъ}}
            \\\cline{1-2}\cline{5-6}
            
            \makecell{В.}
            & {\slv{де́нь}}
            &
            &
            & \makecell{В.}
            & {\slv{дни̑}}
            \\\hline
            
            \makecell{Т.}
            & {\slv{дне́мъ}}
            & \multirow{2}{*}{Д. Т.}
            & \multirow{2}{*}{\slv{де́нма}}
            & \makecell{Т.}
            & {\slv{де́нми}}
            \\\cline{1-2}\cline{5-6}

            \makecell{П.}
            & {\slv{ѡ҆ днѝ}}
            &
            &
            & \makecell{П.}
            & {\slv{ѡ҆ дне́хъ}}
            \\\hline
            
        \end{tabular}
    \end{center}

    По образу склонения существительного {\slv{де́нь}} склоняются существительные: {\slv{ка́мень, ко́рень, пла́мень, сте́пень}}.
    
    \bigskip\underline{4. Склонение имени существительного {\slv{пꙋ́ть}}}
    \bigskip
    
    Существительное {\slv{пꙋ́ть}} хотя и мужского рода, но по падежным окончаниям относится к третьему склонению и при этом имеет в окончаниях некоторые отступления.

    \begin{center}
        %        Склонение существительного {\slv{пꙋ́ть}}}
        \renewcommand*{\arraystretch}{1.4}
        \footnotesize\begin{tabular}[c]{|c|c|c|c|c|c|}
            \hline
            
            \multicolumn{2}{|c|}{Единственное число}
            & \multicolumn{2}{c|}{Двойственное число}
            & \multicolumn{2}{c|}{Множественное число}
            \\\hline
            
            \makecell{И. В.}
            & {\slv{пꙋ́ть}}
            & \multirow{2}{*}{И. В. З.}
            & \multirow{2}{*}{\slv{пꙋти̑}}
            & \makecell{И. З.}
            & {\slv{пꙋтїѐ}}
            \\\cline{1-2}\cline{5-6}
            
            \makecell{Р.}
            & {\slv{пꙋтѝ}}
            &
            &
            & \makecell{Р.}
            & {\slv{пꙋті́й}}
            \\\hline
            
            \makecell{Д.}
            & {\slv{пꙋтѝ}}
            & \multirow{2}{*}{Р. П.}
            & \multirow{2}{*}{\slv{пꙋтїю̀}}
            & \makecell{Д.}
            & {\slv{пꙋтє́мъ}}
            \\\cline{1-2}\cline{5-6}
            
            \makecell{З.}
            & {\slv{пꙋтѝ}}
            &
            &
            & \makecell{В.}
            & {\slv{пꙋти̑}}
            \\\hline
            
            \makecell{Т.}
            & {\slv{пꙋте́мъ}}
            & \multirow{2}{*}{Д. Т.}
            & \multirow{2}{*}{\slv{пꙋтьма̀}}
            & \makecell{Т.}
            & {\slv{пꙋтьмѝ}}
            \\\cline{1-2}\cline{5-6}
            
            \makecell{П.}
            & {\slv{ѡ҆ пꙋтѝ}}
            &
            &
            & \makecell{П.}
            & {\slv{ѡ҆ пꙋте́хъ}}
            \\\hline
            
        \end{tabular}
    \end{center}

    \bigskip\underline{5. Склонение имен существительных {\slv{ѻ҆́ко}} и {\slv{ᲂу҆́хо}}}
    \bigskip
    
    Существительные {\slv{ѻ҆́ко}} и {\slv{ᲂу҆́хо}} в единственном числе склоняются по образцу существительного {\slv{не́бо}} двояким образом:
    
    а) или без наращения, например: {\slv{ѻ҆́ка}}, {\slv{ᲂу҆́ха}}; {\slv{ѻ҆́кꙋ}}, {\slv{ᲂу҆́хꙋ}} и т.д.,
    
    б) или с наращением {\slv{-ес-}} в косвенных падежах, например: {\slv{ѻ҆чесѐ}}, {\slv{ᲂу҆шесѐ}}; {\slv{ѻ҆чесѝ}}, {\slv{ᲂу҆шесѝ}} и т.д., причем в соответствующих падежах в том и другом случае основы {\slv{к}} и {\slv{х}} по законам смягчения принимают {\slv{ц}}, {\slv{ч}} и {\slv{ш}}, например: {\slv{во ѻ҆́цѣ}}.
    
    В двойственном числе эти существительные имеют совершенно особые формы.
    
    \begin{center}
        %        Склонение имен существительных {\slv{ѻ҆́ко}} и {\slv{ᲂу҆́хо}}
        \renewcommand*{\arraystretch}{1.4}
        \footnotesize\begin{tabular}[c]{|c|c|c|}
            \hline
            
            \multicolumn{3}{|c|}{Двойственное число}
            \\\hline
            
            \makecell{И. З.}
            & {\slv{ѻ҆́чи}}
            & {\slv{ᲂу҆́ши}}
            \\\hline
            
            \makecell{Р. П.}
            & {\slv{ѻ҆́чїю}}
            & {\slv{ᲂу҆́шїю}}
            \\\hline
            
            \makecell{Д. Т.}
            & {\slv{ѻ҆чи́ма}}
            & {\slv{ᲂу҆ши́ма}}
            \\\hline
            
            \makecell{В.}
            & {\slv{ѻ҆́цѣ}}
            & {\slv{ᲂу҆́ши}}
            \\\hline
            
        \end{tabular}
    \end{center}

    Во множественном числе они склоняются по образу существительного {\slv{не́бо}} с наращением {\slv{-ес-}}.
    
    \bigskip\underline{6. Склонение имени существительного {\slv{ᲂу҆ста̀}} ({\slv{ᲂу҆стна̀}})}
    \bigskip
    
    Существительное {\slv{ᲂу҆ста̀}} ({\slv{ᲂу҆стна̀}}) единственного числа не имеет. В остальных числах имеет особые окончания, причем во множественном числе иеет во всех падежах двоякую форму, как то видно из прилагаемого образца.

    \begin{center}
        %        Склонение существительного {\slv{ᲂу҆ста̀}} ({\slv{ᲂу҆стна̀}})}
        \renewcommand*{\arraystretch}{1.4}
        \footnotesize\begin{tabular}[c]{|c|c|c|c|c|}
            \hline
            
            \multicolumn{2}{|c|}{Двойственное число}
            & \multicolumn{3}{c|}{Множественное число}
            \\\hline
            
            \multirow{2}{*}{И. В. З.}
            & \multirow{2}{*}{\slv{ᲂу҆стнѣ̀}}
            & \makecell{И. В. З.}
            & {\slv{ᲂу҆стна̀}}
            & {\slv{ᲂу҆ста̀}}
            \\\cline{3-5}
            
            &
            & \makecell{Р.}
            & {\slv{ᲂу҆сте́нъ}}
            & {\slv{ᲂу҆́стъ}}
            \\\hline
            
            \multirow{2}{*}{Р. П.}
            & \multirow{2}{*}{\slv{ᲂу҆стнꙋ̀}}
            & \makecell{Д.}
            & {\slv{ᲂу҆стна́мъ}}
            & {\slv{ᲂу҆стѡ́мъ}}
            \\\cline{3-5}
            
            &
            & \makecell{Т.}
            & {\slv{ᲂу҆стна́ми}}
            & {\slv{ᲂу҆сты̀}}
            \\\hline
            
            Д. Т.
            & {\slv{ᲂу҆стна́ма}}
            & П.
            & \makecell{{\slv{ѡ҆ ᲂу҆стнѣ́хъ,}}\\{\slv{ѡ҆ ᲂу҆стна́хъ}}}
            & \makecell{{\slv{ѡ҆ ᲂу҆стѣ́хъ,}}\\{\slv{ѡ҆ ᲂу҆ста́хъ}}}
            \\\hline

        \end{tabular}
    \end{center}

            \subsection{Местоимение}
                \subsubsection{Склонение личного местоимения 3-го лица {\slv{ѻ҆́нъ}}}

    Как уже известно, местоимения могут быть \emph{неродовые} и \emph{родовые}. Из неродовых местоимений нами уже рассмотрены ранее личные местоимения 1-го лица {\slv{а҆́зъ}} и 2-го лица {\slv{ты̀}}. Но личное местоимение 3-го лица уже родовое. К рассмотрению его склонения мы и переходим.
    
    \begin{center}
        %        Склонение местоимения {\slv{ѻ҆́нъ}}
        \renewcommand*{\arraystretch}{1.4}
        \footnotesize\begin{tabular}[c]{|c|c|c|c|c|c|c|c|}
            \hline
            
            \multicolumn{4}{|c|}{Единственное число}
            & \multicolumn{4}{c|}{Множественное число}
            \\\hline
            
            \makecell{Па-\\деж}
            & м. р.
            & ж. р.
            & ср. р.
            & \makecell{Па-\\деж}
            & м. р.
            & ж. р.
            & ср. р.
            \\\hline
            
            И.
            & {\slv{ѻ҆́нъ}}
            & {\slv{ѻ҆на̀}}
            & {\slv{ѻ҆но̀}}
            & И.        
            & {\slv{ѻ҆нѝ}}
            & {\slv{ѻ҆нѣ̀}}
            & {\slv{ѻ҆нѝ}}
            \\\hline
    
            Р.
            & {\slv{є҆гѡ̀}}
            & {\slv{є҆ѧ̀}}
            & {\slv{є҆гѡ̀}}
            & Р.        
            & {\slv{и҆́хъ}}
            & {\slv{и҆́хъ}}
            & {\slv{и҆́хъ}}
            \\\hline
            
            Д.
            & {\slv{є҆мꙋ̀}}
            & {\slv{є҆́й}}
            & {\slv{є҆мꙋ̀}}
            & Д.        
            & {\slv{и҆̀мъ}}
            & {\slv{и҆̀мъ}}
            & {\slv{и҆̀мъ}}
            \\\hline
            
            В.
            & {\slv{є҆го̀, и҆̀}}
            & {\slv{ю҆̀}}
            & {\slv{є҆̀}}
            & В.        
            & {\slv{и҆̀хъ, ѧ҆̀}}
            & {\slv{ѧ҆̀}}
            & {\slv{ѧ҆̀}}
            \\\hline
            
            Т.
            & {\slv{и҆́мъ}}
            & {\slv{є҆́ю}}
            & {\slv{и҆́мъ}}
            & Т.        
            & {\slv{и҆́ми}}
            & {\slv{и҆́ми}}
            & {\slv{и҆́ми}}
            \\\hline
            
            П.
            & {\slv{ѡ҆ не́мъ}}
            & {\slv{ѡ҆ не́й}}
            & {\slv{ѡ҆ не́мъ}}
            & П.        
            & {\slv{ѡ҆ ни́хъ}}
            & {\slv{ѡ҆ ни́хъ}}
            & {\slv{ѡ҆ ни́хъ}}
            \\\hline
    
            \multicolumn{4}{|c|}{Двойственное число}
            \\\cline{1-4}
            
            \makecell{И.}
            & {\slv{ѻ҆́на}}
            & {\slv{ѻ҆́нѣ}}
            & {\slv{ѻ҆́на}}
            \\\cline{1-4}
            
            \makecell{Р. П.}
            & {\slv{є҆ю̀}}
            & {\slv{є҆ю̀}}
            & {\slv{є҆ю̀}}
            \\\cline{1-4}
    
            \makecell{Д. Т.}
            & {\slv{и҆́ма}}
            & {\slv{и҆́ма}}
            & {\slv{и҆́ма}}
            \\\cline{1-4}
    
            \makecell{В.}
            & {\slv{ѧ҆̀}}
            & {\slv{ѧ҆̀}}
            & {\slv{ѧ҆̀}}
            \\\cline{1-4}
    
        \end{tabular}
    \end{center}

    Личное местоимение 3-го лица первоначально произносилось: {\slv{и҆̀}}~---~в мужском роде, {\slv{ꙗ҆̀}}~---~в женском роде и {\slv{є҆̀}}~---~в среднем роде. Эта первоначальная форма именительного падежа осталась в церковнославянском языке в мужском и среднем роде только в винительном падеже единственного числа, в женском же роде в том же падеже и числе осталась первоначальная форма {\slv{ю҆̀}}.
    
    После предлогов {\slv{ѡ҆, въ, за, на}} и др. местоимение {\slv{ѻ҆́нъ}} в видах благозвучия принимает согласный звук {\slv{н}}, например: {\slv{на ню̀, на нѐ, на нѧ̀, ѡ҆ не́мъ, въ не́мъ}} и т.п., а в винительном падеже единственного числа мужского рода первоначальная форма {\slv{и҆̀}} в этом случае переходит в {\slv{ь}}, например: {\slv{на́нь}}, где предлог {\slv{на}} пишется слитно с местоимением.
    
    По образцу склонения личного местоимения 3-го лица {\slv{ѻ҆́нъ}} склоняется относительное местоимение {\slv{и҆́же}} (который). Это местоимение состоит из первоначальной формы {\slv{и҆̀}} (он) и частицы {\slv{же}}. При склонении частица {\slv{же}} присоединяется к падежным окончаниям без изменения. Например:
    
    \medskip
    Единственное число:
    
    И. {\slv{и҆́же, ꙗ҆́же, є҆́же}}
    
    Р. {\slv{є҆гѡ́же, є҆ѧ́же, є҆гѡ́же}}
    
    Т. {\slv{и҆́мже, є҆́юже, и҆́мже}}
    
    \medskip
    Двойственное число:
    
    И. В. {\slv{ꙗ҆̀же}}
    
    Р. П. {\slv{є҆ю́же}} во всех родах
    
    Д. П. {\slv{и҆́маже}}
    
    \medskip
    Множественное число:
    
    Д. {\slv{и҆̀мже}} во всех родах и т.д.

                    \paragraph{Упражнение}

    В приведенных ниже предложениях замените формы местоимений 3-го лица мужского рода на соответствующие формы женского рода:
    
    \begin{flushleft}
        \renewcommand*{\arraystretch}{1.2}
        \begin{tabular}[l]{cll}
            
            ~~~~~
            & \emph{Образец}:
            & \makecell[l]{Прости \textbf{ему} согрешения \textbf{его}.}
            \\
            
            ~~~~~
            &
            &
            \\
            
            ~~~~~
            & \emph{Ответ}:
            & \makecell[l]{Прости {\slv{є҆й}} согрешения {\slv{є҆ѧ̀}}.}
            \\
            
        \end{tabular}
    \end{flushleft}

    1. Исцели и раба Твоего (и рабу Твою) от обдержащия \textbf{его} телесныя и душевныя немощи, и оживотвори \textbf{его} благодатию Христа Твоего (Требник. М., 1991, с. 318).
    
    2. Аще что согрешил есть словом, делом или помышлением, прости, очисти \textbf{его} и чиста сотвори от всякого греха и, присно спребывая \textbf{ему}, сохрани прочее лето живота \textbf{его} ходяща во оправданиях Твоих, во еже не ктому быть \textbf{ему} посмеяние диаволу; яко и в \textbf{нем} прославится пресвятое имя Твое (Там же, с. 324).
    
    3. Воздвигни \textbf{его} от одра болезненнаго и от ложа озлобления цела и всесовершенна, даруй \textbf{его} Церкви Твоей благоухающа и творяща волю Твою (Там же, с. 329).
    
    4. Посети \textbf{его} милостию и щедротами Твоими, отжени от \textbf{него} всякую болезнь и немощь (Там же, с. 333).
    
    5. Подая \textbf{ему} оставление грехов, и прощение согрешений, вольных же и невольных, уврачуй \textbf{его} язвы неисцельныя, всякий же недуг и всякую язю, даруй ему душевное исцеление (Там же, с. 339).
    
    6. Презри яко непамятозлобивый Бог согрешения \textbf{его} вся, свободи \textbf{его} от вечныя муки, уста \textbf{его} Твоего хваления исполни, устне \textbf{его} отверзи к славословию имене Твоего, руце \textbf{его} простри к деланию заповедей Твоих, нозе \textbf{его} к течению благовествования Твоего исправи, вся \textbf{его} уды и мысль Твоею укрепляя благодатию (Там же, с. 345--346).

                \subsubsection[Склонение притяжательных местоимений]{Склонение притяжательных местоимений {\slv{мо́й}} ({\slv{тво́й}}, {\slv{сво́й}}) и {\slv{на́шъ}} ({\slv{ва́шъ}})}
    
    \begin{center}
        %        Склонения притяжательных местоимений
        \renewcommand*{\arraystretch}{1.4}
        \footnotesize\begin{tabular}[c]{|c|c|c|c|c|c|c|c|}
            \hline
            
            ~
            & \makecell{Па-\\деж}
            & Муж. р.
            & Жен. р.
            & Ср. р.
            & Муж. р.
            & Жен. р.
            & Ср. р.
            \\\hline
            
            \multirow{6}{*}{\spheading[10em]{Единственное число}}
            & И.
            & {\slv{мо́й}}
            & {\slv{моѧ̀}}
            & {\slv{моѐ}}
            & {\slv{на́шъ}}
            & {\slv{на́ша}}
            & {\slv{на́ше}}
            \\\cline{2-8}
            
            & Р.
            & {\slv{моегѡ̀}}
            & {\slv{моеѧ̀}}
            & {\slv{моегѡ̀}}
            & {\slv{на́шегѡ}}
            & {\slv{на́шеѧ}}
            & {\slv{на́шегѡ}}
            \\\cline{2-8}
            
            & Д.
            & {\slv{моемꙋ̀}}
            & {\slv{мое́й}}
            & {\slv{моемꙋ̀}}
            & {\slv{на́шемꙋ}}
            & {\slv{на́шей}}
            & {\slv{на́шемꙋ}}
            \\\cline{2-8}
            
            & В.
            & \makecell{{\slv{моего̀,}}\\{\slv{мо́й}}}
            & {\slv{мою̀}}
            & {\slv{моѐ}}
            & \makecell{{\slv{на́шего,}}\\{\slv{на́шъ}}}
            & {\slv{на́шꙋ}}
            & {\slv{на́ше}}
            \\\cline{2-8}
            
            & Т.
            & {\slv{мои́мъ}}
            & {\slv{мое́ю}}
            & {\slv{мои́мъ}}
            & {\slv{на́шим}}
            & {\slv{на́шею}}
            & {\slv{на́шим}}
            \\\cline{2-8}
            
            & П.
            & {\slv{ѡ҆ мое́мъ}}
            & {\slv{ѡ҆ мое́й}}
            & {\slv{ѡ҆ мое́мъ}}
            & {\slv{ѡ҆ на́шемъ}}
            & {\slv{ѡ҆ на́шей}}
            & {\slv{ѡ҆ на́шемъ}}
            \\\hline
            
            \multirow{4}{*}{\spheading[4.5em]{Дв. число}}
            & \makecell{И.}
            & {\slv{моѧ̑}}
            & {\slv{мои̑}}
            & {\slv{мои̑}}
            & {\slv{на̑ши}}
            & {\slv{на̑ша}}
            & {\slv{на̑ша}}
            \\\cline{2-8}
            
            & Р. П.
            & {\slv{моє́ю}}
            & {\slv{моє́ю}}
            & {\slv{моє́ю}}
            & {\slv{на́шєю}}
            & {\slv{на́шєю}}
            & {\slv{на́шєю}}
            \\\cline{2-8}
            
            & Д. Т.
            & {\slv{мои́ма}}
            & {\slv{мои́ма}}
            & {\slv{мои́ма}}
            & {\slv{на́шима}}
            & {\slv{на́шима}}
            & {\slv{на́шима}}
            \\\cline{2-8}
            
            & В.
            & {\slv{моѧ̑}}
            & {\slv{моѧ̑}}
            & {\slv{моѧ̑}}
            & {\slv{на̑ши}}
            & {\slv{на̑ша}}
            & {\slv{на̑ша}}
            \\\hline
            
            \multirow{6}{*}{\spheading[10em]{Множественное число}}
            & И.
            & {\slv{моѝ}}
            & {\slv{моѧ̑}}
            & {\slv{моѧ̑}}
            & {\slv{на́ши}}
            & {\slv{на́шѧ}}
            & {\slv{на̑ша}}
            \\\cline{2-8}
            
            & Р.
            & {\slv{мои́хъ}}
            & {\slv{мои́хъ}}
            & {\slv{мои́хъ}}
            & {\slv{на́шихъ}}
            & {\slv{на́шихъ}}
            & {\slv{на́шихъ}}
            \\\cline{2-8}
            
            & Д.
            & {\slv{мои̑мъ}}
            & {\slv{мои̑мъ}}
            & {\slv{мои̑мъ}}
            & {\slv{на́шымъ}}
            & {\slv{на́шымъ}}
            & {\slv{на́шымъ}}
            \\\cline{2-8}
            
            & В.
            & \makecell{{\slv{мои́хъ,}}\\{\slv{моѧ̑}}}
            & {\slv{моѧ̑}}
            & {\slv{моѧ̑}}
            & \makecell{{\slv{на́шихъ,}}\\{\slv{на́шѧ}}}
            & {\slv{на́шѧ}}
            & {\slv{на̑ша}}
            \\\cline{2-8}
            
            & Т.
            & {\slv{мои́ми}}
            & {\slv{мои́ми}}
            & {\slv{мои́ми}}
            & {\slv{на́шими}}
            & {\slv{на́шими}}
            & {\slv{на́шими}}
            \\\cline{2-8}
            
            & П.
            & {\slv{ѡ҆ мои́хъ}}
            & {\slv{ѡ҆ мои́хъ}}
            & {\slv{ѡ҆ мои́хъ}}
            & {\slv{ѡ҆ на́шихъ}}
            & {\slv{ѡ҆ на́шихъ}}
            & {\slv{ѡ҆ на́шихъ}}
            \\\hline
            
        \end{tabular}
    \end{center}

    Притяжательные местоимения {\slv{мо́й, тво́й, сво́й}} к своим основам {\slv{мо-, тво-, сво-}} присоединяют в косвенных падежах полностью падежные формы личного местоимения 3-го лица {\slv{ѻ҆́нъ}}.
    
    Притяжательные местоимения {\slv{}} и {\slv{}} к своим основам {\slv{}} и {\slv{}} также присоединяют в косвенных падежах формы личного местоимения 3-го лица {\slv{}}. Кроме того, для различения созвучных падежей употребляются буквы {\slv{}} и {\slv{}}, {\slv{}} и {\slv{}}, а также облегченное ударение.

                \subsubsection{Склонение определительного местоимения {\slv{ве́сь}}}
    
    \begin{center}
        %        Склонение определительного местоимения {\slv{ве́сь}}
        \renewcommand*{\arraystretch}{1.4}
        \footnotesize\begin{tabular}[c]{|c|c|c|c|c|c|c|c|}
            \hline
            
            \multicolumn{4}{|c|}{Единственное число}
            & \multicolumn{4}{c|}{Множественное число}
            \\\hline
            
            \makecell{Па-\\деж}
            & м. р.
            & ж. р.
            & ср. р.
            & \makecell{Па-\\деж}
            & м. р.
            & ж. р.
            & ср. р.
            \\\hline
            
            И.
            & {\slv{ве́сь}}
            & {\slv{всѧ̀}}
            & {\slv{всѐ}}
            & И.        
            & {\slv{всѝ}}
            & {\slv{всѧ̑}}
            & {\slv{всѧ̑}}
            \\\hline
            
            Р.
            & {\slv{всегѡ̀}}
            & {\slv{всеѧ̀}}
            & {\slv{всегѡ̀}}
            & Р.        
            & {\slv{всѣ́хъ}}
            & {\slv{всѣ́хъ}}
            & {\slv{всѣ́хъ}}
            \\\hline
            
            Д.
            & {\slv{всемꙋ̀}}
            & {\slv{все́й}}
            & {\slv{всемꙋ̀}}
            & Д.        
            & {\slv{всѣ̑мъ}}
            & {\slv{всѣ̑мъ}}
            & {\slv{всѣ̑мъ}}
            \\\hline
            
            В.
            & {\slv{всего̀, ве́сь}}
            & {\slv{всю̀}}
            & {\slv{всѐ}}
            & В.        
            & {\slv{всѣ́хъ, всѧ̑}}
            & {\slv{всѧ̑}}
            & {\slv{всѧ̑}}
            \\\hline
            
            Т.
            & {\slv{всѣ́мъ}}
            & {\slv{все́ю}}
            & {\slv{всѣ́мъ}}
            & Т.        
            & {\slv{всѣ́ми}}
            & {\slv{всѣ́ми}}
            & {\slv{всѣ́ми}}
            \\\hline
            
            П.
            & {\slv{ѡ҆ все́мъ}}
            & {\slv{ѡ҆ все́й}}
            & {\slv{ѡ҆ все́мъ}}
            & П.        
            & {\slv{ѡ҆ всѣ́хъ}}
            & {\slv{ѡ҆ всѣ́хъ}}
            & {\slv{ѡ҆ всѣ́хъ}}
            \\\hline
            
        \end{tabular}
    \end{center}

    Местоимение {\slv{ве́сь}} двойственного числа не может иметь по своему значению.

                \subsubsection{Склонение указательного местоимения {\slv{се́й}}}
    
\begin{center}
    %        Склонение указательного местоимения {\slv{се́й}}
    \renewcommand*{\arraystretch}{1.4}
    \footnotesize\begin{tabular}[c]{|c|c|c|c|c|c|c|c|}
        \hline
        
        \multicolumn{4}{|c|}{Единственное число}
        & \multicolumn{4}{c|}{Множественное число}
        \\\hline
        
        \makecell{Па-\\деж}
        & м. р.
        & ж. р.
        & ср. р.
        & \makecell{Па-\\деж}
        & м. р.
        & ж. р.
        & ср. р.
        \\\hline
        
        И.
        & {\slv{се́й, сі́й}}
        & {\slv{сїѧ̀}}
        & {\slv{сїѐ, сѐ}}
        & И.        
        & {\slv{сі́и}}
        & {\slv{сїѧ̑}}
        & {\slv{сїѧ̑}}
        \\\hline
        
        Р.
        & {\slv{сегѡ̀}}
        & {\slv{сеѧ̀}}
        & {\slv{сегѡ̀}}
        & Р.        
        & {\slv{си́хъ}}
        & {\slv{си́хъ}}
        & {\slv{си́хъ}}
        \\\hline
        
        Д.
        & {\slv{семꙋ̀}}
        & {\slv{се́й}}
        & {\slv{семꙋ̀}}
        & Д.        
        & {\slv{си̑мъ}}
        & {\slv{си̑мъ}}
        & {\slv{си̑мъ}}
        \\\hline
        
        В.
        & {\slv{сего̑, се́й}}
        & {\slv{сїю̀}}
        & {\slv{сїѐ, сѐ}}
        & В.        
        & {\slv{си́хъ, сїѧ̑}}
        & {\slv{сїѧ̑}}
        & {\slv{сїѧ̑}}
        \\\hline
        
        Т.
        & {\slv{си́мъ}}
        & {\slv{се́ю}}
        & {\slv{си́мъ}}
        & Т.        
        & {\slv{си́ми}}
        & {\slv{си́ми}}
        & {\slv{си́ми}}
        \\\hline
        
        П.
        & {\slv{ѡ҆ се́мъ}}
        & {\slv{ѡ҆ се́й}}
        & {\slv{ѡ҆ се́мъ}}
        & П.        
        & {\slv{ѡ҆ си́хъ}}
        & {\slv{ѡ҆ си́хъ}}
        & {\slv{ѡ҆ си́хъ}}
        \\\hline
        
        \multicolumn{4}{|c|}{Двойственное число}
        \\\cline{1-4}
        
        \makecell{И. В.}
        & {\slv{сїѧ̑}}
        & {\slv{сі̑и}}
        & {\slv{сїи̑}}
        \\\cline{1-4}
        
        \makecell{Р. П.}
        & {\slv{сею̀}}
        & {\slv{сїю̑}}
        & {\slv{сею̀}}
        \\\cline{1-4}
        
        \makecell{Д. Т.}
        & {\slv{си́ма}}
        & {\slv{си́ма}}
        & {\slv{си́ма}}
        \\\cline{1-4}
        
    \end{tabular}
\end{center}




                \subsubsection{Склонение вспомогательных местоимений {\slv{кто̀}} и {\slv{что̀}}}

    Все до сего времени рассмотренные образцы склонений имели свои падежные формы личного местоимения 3-го лица {\slv{ѻ҆́нъ}}. Такие местоимения относятся к мягкому склонению.
    
    Неродовые вспомогательные местоимения {\slv{}} и {\slv{}} относятся уже к твердому склонению, отличительной чертой которых является в родительном падеже окончание {\slv{}}, в дательном~---~{\slv{}} и в винительном {\slv{}}. Рассмотрим это склонение, заметив при этом, что местоимение {\slv{}} имеет в некоторых падежах двойные и даже тройные формы, несколько уклоняющиеся от твердого склонения и более примыкащие к мягкому.
    
    Местоимения {\slv{}} и {\slv{}} двойственного и множественного чисел не имеют.
    
    \bigskip\autorows{c}{3}{l}{
        {И.}, {{\slv{кто̀}}}, {{\slv{что̀}}},
        {Р.}, {{\slv{когѡ̀}}}, {{\slv{чегѡ̀, чесѡ̀, чесогѡ̀}}},
        {Д.}, {{\slv{комꙋ̀}}}, {{\slv{чемꙋ̀, чесомꙋ̀}}},
        {В.}, {{\slv{кого̀}}}, {{\slv{что̀, чесо̀}}},
        {Т.}, {{\slv{ки́мъ}}}, {{\slv{чи́мъ}}},
        {П.}, {{\slv{ѡ҆ ко́мъ}}}, {{\slv{ѡ҆ че́мъ, ѡ҆ чесо́мъ}}}
    }

    По образцу склонения этих местоимений склоняются неродовые местоимения: {\slv{никто̀, ничто̀, никто́же, ничто́же, кто̀-ли́бо}} и пр.
    
    При склонении местоимений {\slv{никто́же}} и {\slv{ничто́же}} неизменяемая частица {\slv{же}} присоединяется к падежному окончанию, например, р.п.~---~{\slv{никогѡ́же, ничесѡ́же}}; д.п.~---~{\slv{никомꙋ́же, ничемꙋ́же}} и т.д.
    
    При склонении местоимений {\slv{никто̀}} и {\slv{ничто̀}}, а также {\slv{никто́же}} и {\slv{ничто́же}}, предлог, относящийся к этим местоимениям, ставится в падежах между слитной частицей {\slv{ни}} и местоимением. Например: {\slv{ни во что̀, ни ᲂу҆ когѡ̀, ни на чесо́мъ, ни ѡ҆ ко́мже}} и т.п.

                \subsubsection{Склонение указательного местоимения {\slv{то́й}} (тот)}

    Указательное местоимение {\slv{то́й}} и некоторые другие также относятся к твердому склонению, причем в некоторых падежах множественного числа местоимение {\slv{то́й}} имеет многообразные формы, а в винительном падеже во всех родах и числах и энклитические формы.

    \begin{center}
        %        Склонение указательного местоимения {\slv{то́й}}
        \renewcommand*{\arraystretch}{1.4}
        \footnotesize\begin{tabular}[c]{|c|c|c|c|c|c|c|c|}
            \hline
            
            \multicolumn{4}{|c|}{Единственное число}
            & \multicolumn{4}{c|}{Множественное число}
            \\\hline
            
            \makecell{Па-\\деж}
            & м. р.
            & ж. р.
            & ср. р.
            & \makecell{Па-\\деж}
            & м. р.
            & ж. р.
            & ср. р.
            \\\hline
            
            И.
            & \makecell{{\slv{то́й}}}
            & \makecell{{\slv{та́ѧ, та̀}}}
            & \makecell{{\slv{то́е, то̀}}}
            & И.        
            & \makecell{{\slv{ті́и}}}
            & \makecell{{\slv{ты́ѧ}}}
            & \makecell{{\slv{та̑ѧ, та̑}}}
            \\\hline
            
            Р.
            & \makecell{{\slv{тогѡ̀}}}
            & \makecell{{\slv{тоѧ̀}}}
            & \makecell{{\slv{тогѡ̀}}}
            & Р.        
            & \makecell{{\slv{тѣ́хъ}}}
            & \makecell{{\slv{тѣ́хъ}}}
            & \makecell{{\slv{тѣ́хъ}}}
            \\\hline
            
            Д.
            & \makecell{{\slv{томꙋ̀}}}
            & \makecell{{\slv{то́й}}}
            & \makecell{{\slv{томꙋ̀}}}
            & Д.        
            & \makecell{{\slv{тѣ̑мъ,}}\\{\slv{ты́мъ}}}
            & \makecell{{\slv{тѣ̑мъ,}}\\{\slv{ты́мъ}}}
            & \makecell{{\slv{тѣ̑мъ,}}\\{\slv{ты́мъ}}}
            \\\hline
            
            В.
            & \makecell{{\slv{того̀,}}\\{\slv{то́й}}}
            & \makecell{{\slv{тꙋ́ю, тꙋ̀}}}
            & \makecell{{\slv{то́е, то̀}}}
            & В.        
            & \makecell{{\slv{тѣ́хъ,}}\\{\slv{ты́ѧ, ты̑}}}
            & \makecell{{\slv{ты҆́ѧ}}}
            & \makecell{{\slv{та̑ѧ, та̑}}}
            \\\hline
            
            Т.
            & \makecell{{\slv{тѣ́мъ}}}
            & \makecell{{\slv{то́ю}}}
            & \makecell{{\slv{тѣ́мъ}}}
            & Т.        
            & \makecell{{\slv{тѣ́ми}}}
            & \makecell{{\slv{тѣ́ми}}}
            & \makecell{{\slv{тѣ́ми}}}
            \\\hline
            
            П.
            & \makecell{{\slv{ѡ҆ то́мъ}}}
            & \makecell{{\slv{ѡ҆ то́й}}}
            & \makecell{{\slv{ѡ҆ то́мъ}}}
            & П.        
            & \makecell{{\slv{ѡ҆ тѣ́хъ,}}\\{\slv{ѡ҆ ты́хъ}}}
            & \makecell{{\slv{ѡ҆ тѣ́хъ,}}\\{\slv{ѡ҆ ты́хъ}}}
            & \makecell{{\slv{ѡ҆ тѣ́хъ,}}\\{\slv{ѡ҆ ты́хъ}}}
            \\\hline
            
            \multicolumn{4}{|c|}{Двойственное число}
            \\\cline{1-4}
            
            \makecell{И. В.}
            & \makecell{{\slv{та̑}}}
            & \makecell{{\slv{тѣ̀}}}
            & \makecell{{\slv{та̑, тѣ̀}}}
            \\\cline{1-4}
            
            \makecell{Р. П.}
            & \makecell{{\slv{тѡ́ю}}}
            & \makecell{{\slv{тѡ́ю}}}
            & \makecell{{\slv{тѡ́ю}}}
            \\\cline{1-4}
            
            \makecell{Д. Т.}
            & \makecell{{\slv{тѣ́ма}}}
            & \makecell{{\slv{тѣ́ма}}}
            & \makecell{{\slv{тѣ́ма}}}
            \\\cline{1-4}
            
        \end{tabular}
    \end{center}

    По образцу склонения местоимения {\slv{то́й}} склоняется и местоимение {\slv{то́йжде}}, причем частица {\slv{-жде}} присоединяется без изменения непосредственно к падежным окончаниям.
    
    По этому же образцу склоняется местоимение {\slv{и҆ны́й}}, причем это местоимение в родительном падеже единственного числа женского рода имеет окончание {\slv{-ыѧ}}, например:
    
    {\slv{Не и҆́мамы и҆ны́ѧ по́мощи}} (Кондак Пресвятой Богородице).

                \subsubsection{Склонение указательного местоимения {\slv{ѻ҆́нъ, ѻ҆́ный}}}

    Указательное местоимение {\slv{ѻ҆́ный}} в своей краткой форме {\slv{ѻ҆́нъ}} в именительном падеже всех родов заменило впоследствии первоначальную форму  местоимения 3-го лица в этом же падеже {\slv{и҆̀, ꙗ҆̀, є҆̀}}.
    
    По своим падежным окончаниям это местоимение принадлежит также к твердому склонению, но имеет и некоторые особенности. Вот образец этого склонения.
    
    \begin{center}
        %        Склонение указательного местоимения {\slv{ѻ҆́нъ, ѻ҆́ный}}
        \renewcommand*{\arraystretch}{1.4}
        \footnotesize\begin{tabular}[c]{|c|c|c|c|c|c|c|c|}
            \hline
            \multirow{2}{*}{\makecell{Па-\\деж}}
            & \multicolumn{3}{c|}{Единственное число}
            & \multicolumn{3}{c|}{Множественное число}
            \\\cline{2-7}
            
            & м. р.
            & ж. р.
            & ср. р.
            & м. р.
            & ж. р.
            & ср. р.
            \\\hline
            
            И.
            & \makecell{{\slv{ѻ҆́нъ,}}\\{\slv{ѻ҆́ный}}}
            & \makecell{{\slv{ѻ҆́на,}}\\{\slv{ѻ҆́наѧ}}}
            & \makecell{{\slv{ѻ҆́но}}\\{\slv{ѻ҆́ное}}}
            & \makecell{{\slv{ѻ҆́ни, ѻ҆́ны,}}\\{\slv{ѻ҆́нїи}}}
            & \makecell{{\slv{ѻ҆́ны,}}\\{\slv{ѻ҆̀ныѧ}}}
            & \makecell{{\slv{ѻ҆̀на,}}\\{\slv{ѻ҆̀наѧ}}}
            \\\hline
            
            Р.
            & \makecell{{\slv{ѻ҆́нагѡ}}}
            & \makecell{{\slv{ѻ҆́ноѧ,}}\\{\slv{ѻ҆́ныѧ}}}
            & \makecell{{\slv{ѻ҆́нагѡ}}}
            & \makecell{{\slv{ѻ҆́нѣхъ}}}
            & \makecell{{\slv{ѻ҆́нѣхъ}}}
            & \makecell{{\slv{ѻ҆́нѣхъ}}}
            \\\hline
            
            Д.
            & \makecell{{\slv{ѻ҆́номꙋ}}}
            & \makecell{{\slv{ѻ҆́ной,}}\\{\slv{ѻ҆́нѣй}}}
            & \makecell{{\slv{ѻ҆́номꙋ}}}
            & \makecell{{\slv{ѻ҆̀нымъ}}\\{\slv{ѻ҆́нѣмъ}}}
            & \makecell{{\slv{ѻ҆̀нымъ}}\\{\slv{ѻ҆́нѣмъ}}}
            & \makecell{{\slv{ѻ҆̀нымъ}}\\{\slv{ѻ҆́нѣмъ}}}
            \\\hline
            
            В.
            & \makecell{{\slv{ѻ҆́нъ, ѻ҆́на,}}\\{\slv{ѻ҆́наго}}}
            & \makecell{{\slv{ѻ҆́нꙋ,}}\\{\slv{ѻ҆́нꙋю}}}
            & \makecell{{\slv{ѻ҆́но,}}\\{\slv{ѻ҆́ное}}}
            & \makecell{{\slv{ѻ҆́ны, ѻ҆̀ныѧ,}}\\{\slv{ѻ҆́нѣхъ, ѻ҆́ныхъ}}}
            & \makecell{{\slv{ѻ҆́ны,}}\\{\slv{ѻ҆̀ныѧ}}}
            & \makecell{{\slv{ѻ҆̀на,}}\\{\slv{ѻ҆̀наѧ}}}
            \\\hline
            
            Т.
            & \makecell{{\slv{ѻ҆́нымъ,}}\\{\slv{ѻ҆́нѣмъ}}}
            & \makecell{{\slv{ѻ҆́ною}}}
            & \makecell{{\slv{ѻ҆́нымъ}}\\{\slv{ѻ҆́нѣмъ}}}
            & \makecell{{\slv{ѻ҆́нѣми}}}
            & \makecell{{\slv{ѻ҆́нѣми}}}
            & \makecell{{\slv{ѻ҆́нѣми}}}
            \\\hline
            
            П.
            & \makecell{{\slv{ѡ҆ ѻ҆́номъ}}}
            & \makecell{{\slv{ѡ҆ ѻ҆́ной,}}\\{\slv{ѡ҆ ѻ҆́нѣй}}}
            & \makecell{{\slv{ѡ҆ ѻ҆́номъ}}}
            & \makecell{{\slv{ѡ҆ ѻ҆́нѣхъ}}}
            & \makecell{{\slv{ѡ҆ ѻ҆́нѣхъ}}}
            & \makecell{{\slv{ѡ҆ ѻ҆́нѣхъ}}}
            \\\hline
            
            \multicolumn{4}{|c|}{Двойственное число}
            \\\cline{1-4}
            
            \makecell{И. В.}
            & \makecell{{\slv{ѻ҆̀на}}}
            & \makecell{{\slv{ѻ҆́нѣ}}}
            & \makecell{{\slv{ѻ҆̀на}}}
            \\\cline{1-4}
            
            \makecell{Р. П.}
            & \makecell{{\slv{ѻ҆̀нꙋ}}}
            & \makecell{{\slv{ѻ҆̀нꙋ}}}
            & \makecell{{\slv{ѻ҆̀нꙋ}}}
            \\\cline{1-4}
            
            \makecell{Д. Т.}
            & \makecell{{\slv{ѻ҆́нѣма}}}
            & \makecell{{\slv{ѻ҆́нѣма}}}
            & \makecell{{\slv{ѻ҆́нѣма}}}
            \\\cline{1-4}
            
        \end{tabular}
    \end{center}
    
    По этому образцу склоняется местоимение {\slv{є҆ли́кїй}} (в краткой форме {\slv{є҆ли́къ}}). В краткой форме гортанный {\slv{к}} смягчается перед соответствующими гласными в {\slv{ц}} ({\slv{є҆ли́цы}}), а в родительном падеже единственного числа женского рода в своей полной форме имеет {\slv{є҆ли́кїѧ}}.
    
    В значении 3-го лица в церковнославянском языке часто употребляются указательные местоимения {\slv{се́й, то́й}}, например: {\slv{Се́й прїи́де во свидѣ́тельство}} (Ин. 1, 7); {\slv{Не бѣ̀ то́й свѣ́тъ}} (Ин. 1, 8). В русском переводе в обоих текстах стоит <<он>>.

            \subsection{Имя прилагательное}
                \subsubsection{Склонение полных имен прилагательных}

    Полные имена прилагательные образовались из кратких через присоединение к их окончаниям личного местоимения 3-го лица в своей первоначальной форме {\slv{и҆̀, ꙗ҆̀, є҆̀}}. Вот примеры таких образований: 1) с твердым окончанием ({\slv{до́брый}}) и 2) с мягким окончанием ({\slv{си́нїй}}).
    
    \bigskip\autorows{c}{1}{l}{
        {{\slv{до́бръ}} + {\slv{и҆̀}} = {\slv{до́бро}} + {\slv{и҆̀}} = {\slv{до́бр}} + {\slv{ый}} = {\slv{до́брый}}},
        
        {{\slv{до́бра}} + {\slv{є҆гѡ̀}} = {\slv{до́бра}} + {\slv{а҆гѡ̀}} = {\slv{до́бр}} + {\slv{а}} + {\slv{гѡ}} = {\slv{до́брагѡ}} и т.д.},
        
        {{\slv{си́нь}} + {\slv{и҆̀}} = {\slv{си́ни}} + {\slv{и҆̀}} = {\slv{си́н}} + {\slv{ий}} = {\slv{си́нїй}}},
        
        {{\slv{си́нѧ}} + {\slv{є҆гѡ̀}} = {\slv{си́нѧ}} + {\slv{ѧ҆гѡ̀}} = {\slv{си́н}} + {\slv{ѧ}} + {\slv{гѡ}} = {\slv{си́нѧгѡ}} и т.д.}
    }

    Рассмотрим склонение этих прилагательных в их полной форме.
    
    \begin{center}
        %        Склонение полных имен прилагательных
        \renewcommand*{\arraystretch}{1.4}
        \footnotesize\begin{tabular}[c]{|c|c|c|c|c|c|c|c|}
            \hline
            
            ~
            & \makecell{Па-\\деж}
            & Муж. р.
            & Жен. р.
            & Ср. р.
            & Муж. р.
            & Жен. р.
            & Ср. р.
            \\\hline
            
            \multirow{6}{*}{\spheading[10em]{Единственное число}}
            & И. З.
            & {\slv{до́брый}}
            & {\slv{до́браѧ}}
            & {\slv{до́брое}}
            & {\slv{си́нїй}}
            & {\slv{си́нѧѧ}}
            & {\slv{си́нее}}
            \\\cline{2-8}
            
            & Р.
            & {\slv{до́брагѡ}}
            & {\slv{до́брыѧ}}
            & {\slv{до́брагѡ}}
            & {\slv{си́нѧгѡ}}
            & {\slv{си́нїѧ}}
            & {\slv{си́нѧгѡ}}
            \\\cline{2-8}
            
            & Д.
            & {\slv{до́бромꙋ}}
            & {\slv{до́брѣй}}
            & {\slv{до́бромꙋ}}
            & {\slv{си́немꙋ}}
            & {\slv{си́ней}}
            & {\slv{си́немꙋ}}
            \\\cline{2-8}
            
            & В.
            & \makecell{{\slv{до́браго,}}\\{\slv{до́брый}}}
            & {\slv{до́брꙋю}}
            & {\slv{до́брое}}
            & \makecell{{\slv{си́нѧго,}}\\{\slv{си́нїй}}}
            & {\slv{си́нюю}}
            & {\slv{си́нее}}
            \\\cline{2-8}
            
            & Т.
            & {\slv{до́брым}}
            & {\slv{до́брою}}
            & {\slv{до́брымъ}}
            & {\slv{си́нимъ}}
            & {\slv{си́нею}}
            & {\slv{си́нимъ}}
            \\\cline{2-8}
            
            & П.
            & {\slv{ѡ҆ до́брѣмъ}}
            & {\slv{ѡ҆ до́брѣй}}
            & {\slv{ѡ҆ до́брѣмъ}}
            & {\slv{ѡ҆ си́нѣмъ}}
            & {\slv{ѡ҆ си́нѣй}}
            & {\slv{ѡ҆ си́нѣмъ}}
            \\\hline
            
            \multirow{4}{*}{\spheading[4.5em]{Дв. число}}
            & \makecell{И. В.\\З.}
            & {\slv{дѡ́браѧ}}
            & {\slv{до́брѣи}}
            & {\slv{до́брѣи}}
            & {\slv{си̑нѧѧ}}
            & {\slv{си̑нїи}}
            & {\slv{си̑нїи}}
            \\\cline{2-8}
            
            & Р. П.
            & {\slv{дѡ́брꙋю}}
            & {\slv{дѡ́брꙋю}}
            & {\slv{дѡ́брꙋю}}
            & {\slv{си̑нюю}}
            & {\slv{си̑нюю}}
            & {\slv{си̑нюю}}
            \\\cline{2-8}
            
            & Д. Т.
            & {\slv{до́брыма}}
            & {\slv{до́брыма}}
            & {\slv{до́брыма}}
            & {\slv{си́нима}}
            & {\slv{си́нима}}
            & {\slv{си́нима}}
            \\\hline
            
            \multirow{6}{*}{\spheading[10em]{Множественное число}}
            & И. З.
            & {\slv{до́брїи}}
            & {\slv{дѡ́брыѧ}}
            & {\slv{дѡ́браѧ}}
            & {\slv{си́нїи}}
            & {\slv{си̑нїѧ}}
            & {\slv{си̑нѧѧ}}
            \\\cline{2-8}
            
            & Р.
            & {\slv{до́брыхъ}}
            & {\slv{до́брыхъ}}
            & {\slv{до́брыхъ}}
            & {\slv{си́нихъ}}
            & {\slv{си́нихъ}}
            & {\slv{си́нихъ}}
            \\\cline{2-8}
            
            & Д.
            & {\slv{дѡ́брымъ}}
            & {\slv{дѡ́брымъ}}
            & {\slv{дѡ́брымъ}}
            & {\slv{си̑нимъ}}
            & {\slv{си̑нимъ}}
            & {\slv{си̑нимъ}}
            \\\cline{2-8}
            
            & В.
            & \makecell{{\slv{до́брыхъ,}}\\{\slv{дѡ́брыѧ}}}
            & {\slv{дѡ́брыѧ}}
            & {\slv{дѡ́браѧ}}
            & \makecell{{\slv{си́нихъ,}}\\{\slv{си̑нїѧ}}}
            & {\slv{си̑нїѧ}}
            & {\slv{си̑нѧѧ}}
            \\\cline{2-8}
            
            & Т.
            & {\slv{до́брыми}}
            & {\slv{до́брыми}}
            & {\slv{до́брыми}}
            & {\slv{си́ними}}
            & {\slv{си́ними}}
            & {\slv{си́ними}}
            \\\cline{2-8}
            
            & П.
            & {\slv{ѡ҆ до́брыхъ}}
            & {\slv{ѡ҆ до́брыхъ}}
            & {\slv{ѡ҆ до́брыхъ}}
            & {\slv{ѡ҆ си́нихъ}}
            & {\slv{ѡ҆ си́нихъ}}
            & {\slv{ѡ҆ си́нихъ}}
            \\\hline
            
        \end{tabular}
    \end{center}


                \subsubsection{Общие замечания к склонению полных имен прилагательных}

    1. При склонении полных имен прилагательных гортанные звуки в основе смягчаются перед {\slv{}} и {\slv{}} на общем основании, например:
    
    \bigskip\autorows{c}{1}{c}{
        {{\slv{бла{\large г}і́й}}~---~{\slv{бла́{\large з}ѣй}}~---~{\slv{бла́{\large з}ѣмъ}}}
    }

    2. Имя прилагательное {\slv{}} при склонении имеет следущие особенности:
    
    \begin{center}
        \begin{tabular}[c]{ll}
            
            творительный падеж един. числа муж. и ср. рода
            & {\slv{мно́земъ}}
            \\
            
            родительный падеж множ. числа во всех родах
            & {\slv{мно́зехъ}}
            \\
            
            дательный падеж множ числа во всех родах
            & {\slv{мнѡ́земъ}}
            \\
            
            предложный падеж множ. числа во всех родах
            & {\slv{ѡ҆ мно́зѣхъ}}
            \\
            
        \end{tabular}
    \end{center}

    3. Полные имена прилагательные с основой на шипящие склоняются по следующему образцу ({\slv{то́щїй}}~---~напрасный, тщетный).
    
    \begin{center}
        %        Полное имя прилагательное с основой на шипящий звук {\slv{то́щїй}}
        \renewcommand*{\arraystretch}{1.4}
        \footnotesize\begin{tabular}[c]{|c|c|c|c|c|c|c|c|}
            \hline
            \multirow{2}{*}{\makecell{Па-\\деж}}
            & \multicolumn{3}{c|}{Единственное число}
            & \multicolumn{3}{c|}{Множественное число}
            \\\cline{2-7}
            
            & м. р.
            & ж. р.
            & ср. р.
            & м. р.
            & ж. р.
            & ср. р.
            \\\hline
            
            И. З.
            & \makecell{{\slv{то́щїй}}}
            & \makecell{{\slv{то́щаѧ}}}
            & \makecell{{\slv{то́щее}}}
            & \makecell{{\slv{то́щїи}}}
            & \makecell{{\slv{то́щыѧ}}}
            & \makecell{{\slv{тѡ́щаѧ}}}
            \\\hline
            
            Р.
            & \makecell{{\slv{то́щагѡ}}}
            & \makecell{{\slv{то́щїѧ}}}
            & \makecell{{\slv{то́щагѡ}}}
            & \makecell{{\slv{то́щихъ}}}
            & \makecell{{\slv{то́щихъ}}}
            & \makecell{{\slv{то́щихъ}}}
            \\\hline
            
            Д.
            & \makecell{{\slv{то́щемꙋ}}}
            & \makecell{{\slv{то́щей}}}
            & \makecell{{\slv{то́щемꙋ}}}
            & \makecell{{\slv{то́щымъ}}}
            & \makecell{{\slv{то́щымъ}}}
            & \makecell{{\slv{то́щымъ}}}
            \\\hline
            
            В.
            & \makecell{{\slv{то́щаго,}}\\{\slv{то́щїй}}}
            & \makecell{{\slv{то́щꙋю}}}
            & \makecell{{\slv{то́щее}}}
            & \makecell{{\slv{то́щихъ,}}\\{\slv{то́щыѧ}}}
            & \makecell{{\slv{то́щыѧ}}}
            & \makecell{{\slv{тѡ́щаѧ}}}
            \\\hline
            
            Т.
            & \makecell{{\slv{то́щїимъ}}}
            & \makecell{{\slv{то́щею}}}
            & \makecell{{\slv{то́щїимъ}}}
            & \makecell{{\slv{то́щими}}}
            & \makecell{{\slv{то́щими}}}
            & \makecell{{\slv{то́щими}}}
            \\\hline
            
            П.
            & \makecell{{\slv{ѡ҆ то́щемъ}}}
            & \makecell{{\slv{ѡ҆ то́щей}}}
            & \makecell{{\slv{ѡ҆ то́щемъ}}}
            & \makecell{{\slv{ѡ҆ то́щихъ}}}
            & \makecell{{\slv{ѡ҆ то́щихъ}}}
            & \makecell{{\slv{ѡ҆ то́щихъ}}}
            \\\hline
            
            \multicolumn{4}{|c|}{Двойственное число}
            \\\cline{1-4}
            
            \makecell{И. В.\\З.}
            & \makecell{{\slv{тѡ́щаѧ}}}
            & \makecell{{\slv{тѡ́щїи}}}
            & \makecell{{\slv{тѡ́щїи}}}
            \\\cline{1-4}
            
            \makecell{Р. П.}
            & \makecell{{\slv{тѡ́щꙋю}}}
            & \makecell{{\slv{тѡ́щꙋю}}}
            & \makecell{{\slv{тѡ́щꙋю}}}
            \\\cline{1-4}
            
            \makecell{Д. Т.}
            & \makecell{{\slv{то́щима}}}
            & \makecell{{\slv{то́щима}}}
            & \makecell{{\slv{то́щима}}}
            \\\cline{1-4}
            
        \end{tabular}
    \end{center}

    4. Имена прилагательные {\slv{бж҃їй, ве́лїй}} и подобные им склоняются с небольшими отступлениями. Приведем для образца склонение имени прилагательного {\slv{ве́лїй}}.
    
    \begin{center}
        %        Полное имя прилагательное {\slv{ве́лїй}}
        \renewcommand*{\arraystretch}{1.4}
        \footnotesize\begin{tabular}[c]{|c|c|c|c|c|c|c|}
            \hline
            \multirow{2}{*}{\makecell{Па-\\деж}}
            & \multicolumn{3}{c|}{Единственное число}
            & \multicolumn{2}{c|}{Множественное число}
            \\\cline{2-6}
            
            & м. р.
            & ж. р.
            & ср. р.
            & м. р.
            & ж. и ср. р.
            \\\hline
            
            И. З.
            & \makecell{{\slv{ве́лїй}}}
            & \makecell{{\slv{ве́лїѧ}}}
            & \makecell{{\slv{ве́лїе}}}
            & \makecell{{\slv{ве́лїи}}}
            & \makecell{{\slv{вє́лїѧ}}}
            \\\hline
            
            Р.
            & \makecell{{\slv{ве́лїѧгѡ}}}
            & \makecell{{\slv{ве́лїѧ}}}
            & \makecell{{\slv{ве́ліѧгѡ}}}
            & \makecell{{\slv{ве́лїихъ}}}
            & \makecell{{\slv{ве́лїихъ}}}
            \\\hline
            
            Д.
            & \makecell{{\slv{ве́лїемꙋ}}}
            & \makecell{{\slv{ве́лїей}}}
            & \makecell{{\slv{ве́лїемꙋ}}}
            & \makecell{{\slv{вє́лїимъ}}}
            & \makecell{{\slv{вє́лїимъ}}}
            \\\hline
            
            В.
            & \makecell{{\slv{ве́лїѧго,}}\\{\slv{ве́лїй}}}
            & \makecell{{\slv{ве́лїю}}}
            & \makecell{{\slv{ве́лїе}}}
            & \makecell{{\slv{ве́лїихъ,}}\\{\slv{вє́лїѧ}}}
            & \makecell{{\slv{вє́лїѧ}}}
            \\\hline
            
            Т.
            & \makecell{{\slv{ве́лїимъ}}}
            & \makecell{{\slv{ве́лїею}}}
            & \makecell{{\slv{ве́лїимъ}}}
            & \makecell{{\slv{ве́лїими}}}
            & \makecell{{\slv{ве́лїими}}}
            \\\hline
            
            П.
            & \makecell{{\slv{ѡ҆ ве́лїемъ}}}
            & \makecell{{\slv{ѡ҆ ве́лїей}}}
            & \makecell{{\slv{ѡ҆ ве́лїемъ}}}
            & \makecell{{\slv{ѡ҆ ве́лїихъ,}}\\{\slv{ѡ҆ ве́лїѧхъ}}}
            & \makecell{{\slv{ѡ҆ ве́лїихъ,}}\\{\slv{ѡ҆ ве́лїѧхъ}}}
            \\\hline
            
            \multicolumn{4}{|c|}{Двойственное число}
            \\\cline{1-4}
            
            \makecell{И. В.\\З.}
            & \makecell{{\slv{вє́лїѧ}}}
            & \makecell{{\slv{вє́лїи}}}
            & \makecell{{\slv{вє́лїи}}}
            \\\cline{1-4}
            
            \makecell{Р. П.}
            & \makecell{{\slv{вє́лїю}}}
            & \makecell{{\slv{вє́лїю}}}
            & \makecell{{\slv{вє́лїю}}}
            \\\cline{1-4}
            
            \makecell{Д. Т.}
            & \makecell{{\slv{ве́лїима}}}
            & \makecell{{\slv{ве́лїима}}}
            & \makecell{{\slv{ве́лїима}}}
            \\\cline{1-4}
            
        \end{tabular}
    \end{center}

    5. Если в полном имени прилагательном имеется суффикс {\slv{-ск-}}, то он изменяется в некоторых падежах в {\slv{-ст-}}, например: {\slv{лю́дскїй}}~---~{\slv{людсті́и}}.

                \subsubsection{Понятие о степенях сравнения имен прилагательных}

    Качественные имена прилагательные имеют \textbf{степени сравнения}.
    
    Если качественное прилагательное обозначает такое количество, которое у предмета или явления может быть в большой и в меньшей мере, то такое прилагательное может иметь сравнительную степень и превосходную степень.
    
    \emph{Сравнительная} степень показывает, что в одном предмете или явлении какого-нибудь качества больше, чем в другом.
    
    \emph{Превосходная} степень показывает, что в одном предмете или явлении какого-нибудь качества больше, чем во всех других предметах или явлениях, то есть, иначе говоря, показывает наивысшую меру качества сравнительно с другими однородными предметами или явлениями.
    
    В соответствии этим двум степеням, первоначальная форма имени прилагательного, от которой образуются степени сравнения и превосходная, называется \emph{положительной} степенью. Например:
    
    \begin{center}
        \begin{tabular}[c]{lcl}
            
            {\slv{свѧты́й}}
            & ~--~
            & степень положительная
            \\
            
            {\slv{свѧтѣ́е}}
            & ~--~
            & степень сравнительная
            \\
            
            {\slv{свѧтѣ́йшїй}}
            & ~--~
            & степень превосходная
            \\
            
        \end{tabular}
    \end{center}

                \subsubsection{Сравнительная степень имен прилагательных}

    Сравнительная степень имен прилагательных образуется из положительной через замену полного окончания прилагательных на {\slv{-шїй}} (м. р.), {\slv{-шаѧ}} (ж. р.), {\slv{-шее}} (ср. р.). Например:

%                \subsubsection{Превосходная степень имен прилагательных}
%                \subsubsection{Неправильные степени сравнения}
%            \subsection{Имя числительное}
%                \subsubsection{Понятие об имени числительном}
%                \subsubsection{Имена числительные количественные}
%                \subsubsection{Имена числительные порядковые}
%                \subsubsection{Склонение количественных числительных}
%        \section{Глагол}
%                \subsubsection{Понятие о причастии}
%                \subsubsection{Несклоняемое (спрягаемое) причастие}
%                \subsubsection{Прошедшее совершенное время глаголов}
%                \subsubsection{Давнопрошедшее время глаголов}
%                \subsubsection{Условное наклонение глаголов}
%                \subsubsection{Безличные глаголы}
%                \subsubsection{Возвратная форма глагола}
%                \subsubsection{Страдательная форма глагола}
%                \subsubsection{Склоняемые причастия}
%                \subsubsection{Краткие действительные причастия настоящего времени}
%                \subsubsection{Полные действительные причастия настоящего времени}
%                \subsubsection{Краткие действительные причастия прошедшего времени}
%                \subsubsection{Полные действительные причастия прошедшего времени}
%                \subsubsection{Краткие страдательные причастия настоящего времени}
%                \subsubsection{Полные страдательные причастия настоящего времени}
%                \subsubsection{Краткие страдательные причастия прошедшего времени}
%                \subsubsection{Полные страдательные причастия прошедшего времени}
%                \subsubsection{Сложная страдательная форма глаголов}
%        \section{Неизменяемые части речи}
%                \subsubsection{Наречие}
%                \subsubsection{Предлог}
%                \subsubsection{Союз}
%                \subsubsection{Частицы}
%                \subsubsection{Междометия}
%        \section{Некоторые особенности церковнославянского синтаксиса}
%                \subsubsection{Двойной винительный падеж}
%                \subsubsection{Падеж родительный-разделительный}
%                \subsubsection{Замена притяжательных местоимений личными и возвратными в дательном падеже}
%                \subsubsection{Употребление местоимения и имени прилагательного в смысле имени существительного}
%                \subsubsection{Отсутствие отрицания при глаголе при наличии местоимений{\slv{никто̀, ничто̀}}}
%                \subsubsection{Славянский член и его употребление}
%                \subsubsection{Особенности славянского придаточного предложения}
%                \subsubsection{Особенности славянских придаточных предложений с союзными словами {\slv{є҆́же, во є҆́же, ѡ є҆́же}}}
%                \subsubsection{Особенности славянского придаточного предложения}
%                \subsubsection{Оборот <<Дательный самостоятельный>>}
%                \subsubsection{Оборот <<Дательный с неопределенным>>}
%                \subsubsection{Оборот <<Винительный с неопределенным>>}

\end{document}
