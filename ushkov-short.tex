\documentclass[11pt,a4paper]{memoir}

\usepackage{hyperref}

\usepackage[usenames,dvipsnames,svgnames,table]{xcolor}
\usepackage{fontspec}
\usepackage{xltxtra}
% code borrowed from Polyglossia documentation — Thanks!
\definecolor{myblue}{rgb}{0.02,0.04,0.48}
\definecolor{lightblue}{rgb}{0.61,.8,.8}
\definecolor{myred}{rgb}{0.65,0.04,0.07}

\usepackage{polyglossia}
\setmainlanguage{russian}
\setotherlanguages{churchslavonic,english}
\usepackage{churchslavonic}
\usepackage{lettrine}

\setmainfont[Mapping=tex-text]{Liberation Serif}
\setsansfont[Mapping=tex-text]{Liberation Sans}
\setmonofont[Mapping=tex-text]{Liberation Mono}

\newfontfamily\churchslavonicfont[Script=Cyrillic,Ligatures=TeX,Scale=1.33333333,HyphenChar="005F]{PonomarUnicode.otf} 
\newfontfamily{\slv}[Scale=MatchLowercase]{Ponomar Unicode TT}
\newfontfamily{\ust}[Scale=MatchLowercase]{Menaion Unicode TT}
\newfontfamily{\ind}{Indiction Unicode TT}

\usepackage{indentfirst}
\frenchspacing
\clubpenalty=10000
\widowpenalty=10000

\sloppy

%% Ensure sequential numbering of subsubsections.
\setsecnumdepth{subsubsection}
\counterwithout{subsubsection}{subsection}
\renewcommand{\thechapter}{\arabic{chapter}}
\renewcommand{\thesection}{\arabic{section}}
\renewcommand{\thesubsection}{\Roman{subsection}}
\renewcommand{\thesubsubsection}{\S\arabic{subsubsection}}

\setcounter{tocdepth}{3} % Must not precede the above

%% Insert Church Slavonic snippet into (Russian) text
\newcommand{\chs}[2][slv]{
    \begin{#1}\ignorespaces
        #2
    \ignorespacesafterend\end{#1}
}

\begin{document}
    
    \tableofcontents
    
    \chapter*{}
        \section*{Благолепие церковнославянского языка}
        \label{sec:blago}
        \addcontentsline{toc}{section}{\nameref{sec:blago}}

    Прошло уже более тысячи лет со времени Крещения Руси. Воспринятое ею Православие, благодаря трудам святых Мефодия и Кирилла, совершает свои благоговейные богослужения на церковнославянском языке. Он по своей структуре наиболее близок к греческому, и это позволило сохранить существующий в Греческой церкви строй богослужения, что явилось для многочисленных славянских народов великим благом и живительным источником благочестия и совершенствования нравственного в духе истины правой веры, а также руководством в земных делах: образования, иконографии, храмо- и градостроительства. Наша Русская Православная Церковь восприняла в неповрежденности и чистоте от Греческой Матери Церкви все догматическое, литургическое, духовное, святоотеческое наследие и предание, связав все это, с помощью славянского языка, с духом народа христианского на Руси, с жизнью и бытом и просвещением, ибо на славянский язык были переведены все богослужебные книги и писания греческих отцов, прием сохранялась точность, ясность и истинность, хотя перевод и являлся творческим делом.
    
    Славянский язык помогал памятовать о Христе, хранить Его живой образ и Его учение в сердце и душе и приносить плоды живого творческого восприятия слова Божия. В Церкви шло становление и образование самого славянского языка, его внутренняя христианизация и воцерковление, преображение самой стихии славянской мысли и слова, славянского голоса, души народа, На долю русского народа выпало редкое счастье принять христианство тогда, когда прогремели громом своим Вселенские Соборы и утверждено в чистоте апостольское Православие, и в Церкви Греческой утвердились и литургические, и догматические, и нравственные истины и самый язык славянский, ничем не запятнанный и чистый, воспринял для своего народа сосуд благодати, и в благодати Христовой народ верный обновляется, и просвещается, и спасается, и становится причастным вечной Истине, и славянский язык богатство веры и культуры передал многим полудиким племенам и народам, и в Церкви они нашли единение и братство Христово как дети Божии. На Руси потрудились многие ревнители духовного просвещения: и князь Владимир~---~просветитель, и св. Андрей Боголюбский, и свв. Александр Невский, Даниил Московский, святитель Петр, митрополит Алексий, преподобный Сергий Радонежский и многие другие подвижники благочестия. Созидались монастыри, храмы, школы. Создавались библиотеки, делались переводы на славянский язык, составлялись азбуки для малых и полудиких народностей, писались летописи.
    
    Славянский язык хранил церковное единство и православные традиции. 2 августа 1581 г. впервые была напечатана полная Библия на славянском языке в г. Остроге, что способствовало глубокому изучению Священного Писания. Русская Церковь в XVI в., когда протестанты превращали христианство в систему философских построений, как плод похи Возрождения язычества, начала вести длительную и тяжелую борьбу за сохранение чистоты веры, церковнославянского языка, церковно-национальной культуры.
    
    Славянская Библия занимает самое первое место по своей полноте и точности воспроизведения библейских древних писаний, о чем свидетельствуют и списки Кумранских рукописей, написанных за 200---300 лет до Р.Х., когда не было нужды искажать тексты Священного Писания. Древние рукописи своим текстом совпадают с острожской славянской Библией. В славянской Библии точно сохранено летоисчисление~---~5508 лет от сотворения мира до Рождества Христова, и все мессианские места. Известно, что текст 70 толковников, взятый в основу славянской Библии, долгое время считался в александрийских синагогах у евреев священным переводом, и его читали на греческом языке 200---300 лет. Масоретский же текст Библии идет от Акибы (около II века после Р.Х.).
    
    Кирилло-Мефодианский перевод Евангелия на славянский язык отличается точностью, верностью подлиннику и ясностью, чистотою выражения христианских понятий.
    
    Славянский текст Священного Писания и православного богослужения~---~свидетель первого тысячелетия, то есть неразделенного христианства, и в то же время он является драгоценным наследием равноапостольских трудов свв. Кирилла и Мефодия, на котором воспитываются чада Русской Православной Церкви.
    
    Церковнославянский богослужебный текст является важным и святым. Христиане разных конфессий стремятся осмыслить свои традиции в сравнении с традициями других исповеданий, ведь священные книги Нового Завета переведены на 1050 языков, и среди этих переводов самое важное место занимает славянский. Поскольку перевод Писания был сделан с греческого языка, а греческая текстология устанавливала единый печатный греческий текст XVII в., потом в XX веке его вытеснили из употребления, то в вопросе изысканий и оценки конечных результатов многих реконструкций текстов важную роль играет славянский текст как свидетель текстовой традиции первого тысячелетия, то есть времени неразделенного христианства.
    
    Острожская Библия 1581 года на славянском языке ни по форме, ни по содержанию не изменилась и является эталоном, по которому можно проследить реконструкцию греческого текста по времени.
    
    Анализ и сопоставление текстов показывает, что славянский текст стоит на твердой, нерушимой основе древней богослужебной практики Православия, а западное христианство отошло от древних традиций.
    
    Святые равноапостольные братья Кирилл и Мефодий использовали для перевода церковный византийский текст, который не искажался, был под контролем и оберегался всегда и всюду в Церкви, как нечто постоянное, неизменное и непреложное, как сама Истина, как церковно-текстуальное предание, уходящее в апостольскую древность. Это предание Церкви святое, соборное, апостольское, как поток, исходит из свежего родника живой воды евангельской, течет по чистому руслу под ясным небом и при тихой погоде, и его прозрачные воды менее доступны для загрязнения человеческими мудрованиями во имя кичащегося ума и в ущерб созидающей любви.
    
    В текстах встречаются несколько десятков слов труднопонятных, как тектон~---~дровосек, усмен~---~кожан, измена~---~выкуп, выну~---~всегда и др., но здравый человек за неделю их запомнит, ведь даже иностранные языки учат, да не один, а два-три языка, заучивая 40-45 слов в день, а язык, когда-то бывший родным, разговорным языком наших предков, язык, освященный веками, надо любить и хранить.
    
    Славянский язык в Церкви на Руси освящен непрерывным богослужение, и внешняя гармония, чистота, сердечность, молитвенность, мир Христов и утешение благодатное водворяются в душе православного христианина, как плод веры и любви.
    
    Большое значение славянскому языку для православного богослужения придает и Грузинская Церковь. Архимандрит Рафаил (Карелин) из Тбилиси выразил то так: <<В настоящее время в литургической жизни Грузинской Автокефальной Церкви славянский язык является по своему значению вторым, после древнегрузинского, богослужебным языком. Церковная славистика имеет для нас практическую важность и теоретический интерес>>.
    
    В связи с этим необходимо остановиться на нескольких вопросах, касающихся значения славянского языка для современного православного богослужения, рассматривая при этом славянский язык обобщенно и условно, как единую структурную и знаковую систему.
    
    В данном докладе славянский язык рассматривается обобщенно и условно, как единая структурная и знаковая система.
    
    Славянский язык насчитывает более чем тысячелетнюю литературную историю. Он не стал мертвым языком, не сделался достоянием одних археологов и лингвистов; славянский язык звучит на всех континентах, на нем совершается богослужение в православных церквах, на нем миллионы людей воссылают молитвы к Богу.
    
    В последнее время неоднократно поднимался вопрос: не является ли славянский язык литургическим архаизмом, преградой к пониманию текстов, и не следует ли перевести все богослужебные книги на современные языки?
    
    Такие попытки предпринимались в России и других странах.
    
    В России языковые литургические реформы не встретили поддержки основной массы верующих, и на сегодняшний день богослужебным языком остается церковнославянский. Можно ли объяснить это явление только одной приверженностью к традиции, или же оно имеет еще другие основания?
    
    Полагаем, что эти основания следующие.
    
    I. Славянскому языку, как и другим древним языкам (имеется в виду древнегреческий и древнегрузинский), присуща особая динамическая структура; он лучше передает пульс религиозной жизни, глубже выражает молитвенные чувства, чем современные языки.
    
    Древние языки более подходят к цельному синтетическому восприятию, новые~---~к аналитическому, дробному; древние~---~к созерцанию, новые~---~к логизированию; древние языки полны энергии и эмоций, новые, в сравнении с ними, носят рационалистический, описательный характер. Древние языки дают большую возможность соприкоснуться с глубиной явлений, с духовными субстанциями, сделать человека участником событий, современные~---~изложить явления в определенной системе и дать их анализ.
    
    Сравнивая лексическую и грамматическую ткань древних и новых языков, мы видим, что древние языки располагают меньшим по объему словарным фондом, но их грамматический строй отличается большим многообразием, пластичностью и совершенством.
    
    Мобильность и экспрессивность глагольных форм, богатство аффиксов и флексий, не имеющих аналогии в современных языках, лаконизм и динамика синтаксических структур, этимологическая глубина лексики создают неповторимую выразительность и, по словам патриарха Пимена, <<особую красоту>> славянского языка. Эти возможности славянского языка позволяют выразить в богослужебных текстах многоплановость событий, синергизм и параллелизм действий, объединить различные хронологические периоды в единые смысловые циклы (как бы преолодев линейную протяженность времени), проявить и усилить подтекст каждого предложения.
    
    Сложность и гибкость глагольных форм славянского языка делают одну и ту же по словарному составу фразу то ажурно-легкой и полетной, то~---~до физической ощутимости тяжелой и твердой, как бы изваянной из мрамора. Лаконизм, внутренняя чеканность и в то же время как бы внешняя незавершенность и обрывистость предложений, часто без последовательных переходов от одного предмета к другому, наделяют смысловые паузы эмоциональным содержанием, подчеркивая, что богослужебный текст~---~это не монолог, а диалог, таинственная беседа души с Божеством, что в священных событиях присутствует, как их постоянный субъект, личность каждого молящегося.
    
    Говоря образно, динамика древних языков созвучна динамике света. Богослужение~---~это симфония из лучей Божественных энергий. Священная история изображается в этой Божественной симфонии на светящемся фоне вечности, земные реалии~---~в их логоистическом преображении.
    
    Следует отметить, что в богослужебных текстах мастерски использован точно и строго очертанный круг изобразительных средств. Это делалось древними гимнографами для того, чтобы не подменить духовных переживаний эстетическими, т.е. душевными, не обременить ума молящегося художественной информацией и, тем самым, не оземлить молитву.
    
    Древние гимнографы мыслили Церковь как идею единства в Ее реальном и мистическом воплощении. Унификация изобразительных средств~---~одно из условий общности молитвы, духовного познания, эмоций и сопереживаний верующих~---~тела Церкви.
    
    Динамичные структуры, внутренние смысловые емкости и скульптурная пластичность славянского языка помогают воспринимать традиционные символы и метафористические образы в разных контекстах как вечно новые и неповторимые.
    
    Одним словом, древние языки более приспособлены для выражения явлений и динамики духовной жизни. Это первая и главная причина их сохранения в православном богослужении.
    
    II. Второе основание~---~трудность самого перевода.
    
    Богослужебные тексты~---~это шедевры священной поэзии особого типа и порядка. Переводчик должен хорошо знать святоотеческое богословие, именно святоотеческое, а не схоластическое, чтобы понять глубокие по догматическому содержанию литургические тексты. Далее, переводчик должен обладать одновременно глубокой филологической подготовкой и незаурядным поэтическим дарованием. Без этого перевод богослужебных книг будет казаться обесцвеченным и прозаическим пересказом великих поэм.
    
    Православные церковные службы называют опоэтизированным, иконографическим, поющим богословием.
    
    И, наконец, самое главное, переводчик должен иметь живую веру, церковность и религиозную интуицию, которая дала бы возможность почувствовать, пережить, оценить каждую фразу и слово переводимого текста.
    
    Интересно, что Петербургский митрополит Гавриил и Московский митрополит Филарет, руководившие работами по переводу святоотеческих произведений с греческого языка на славянский и русский, считали нужным давать рукописи переводов для проверки простым старцам-монахам, которые не знали греческого языка, но, обладая общностью внутреннего духовного опыта с древними писателями-подвижниками и находясь с ними, так сказать, на одной духовной волне, интуитивно чувствовали всякую фальшь и ошибку перевода.
    
    III. Третье основание~---~традиция. Традиция~---~это кристаллизация истории. Это актуальное бытие прошлого в настоящем. Любовь и уважение к прошлому сохранили до наших дней старинные соборы и храмы в их первозданном величии, иконы и фрески изумительной духовной глубины. Живая традиция сохранила нам дивное, неповторимое православное богослужение.
    
    Само слово <<Православие>> значит правое, правильное богопочитание, основным признаком которого является верность апостольскому преданию.
    
    Церковная традиция возникла из общности духовного опыта христиан, опыта emph{богообщения}. emph{Одинаковое религиозное чувство создало} одинаковые или очень схожие формы. Церковное богослужение и храмовое устройство~---~это синтез жизни Церкви в эпоху ее духовного расцвета. Содержание создает формы, но и формы хранят и оберегают содержание. В данном случае древние языки имеют вспомогательное, но важное значение для сохранения в чистоте и внутренней целостности одного из видов церковного предания~---~богослужебного канона.
    
    Чем устойчивее и тверже канон, тем глубже и чище он выражает общечеловеческую и духовную потребность: каноническое есть церковное, церковное~---~соборное, соборное же~---~всечеловеческое.
    
    Славянский язык, наряду с другими древними языками, стал сакральным, священным языком Церкви.
    
    Вопрос о ясности богослужебных славянских текстов для верующих приобретает все большее значение.
    
    Следует обратить особое внимание на научную сторону новых изданий богослужебных книг, снабдить их справочным материалом, в том числе словарем, где наряду с буквальным переводом давалась бы этимология славянской лексики. Не только Священное Писание, но и следующий за ним духовный пласт~---~богослужебный канон~---~также нуждается в толкованиях и комментариях. Перед нами стоит задача составления литургической экзегезы, отвечающей интеллектуальному уровню и духовным запросам современных верующих.
    
    Славянское православное богослужение~---~это бесценное сокровище мистического гносиса, огромный потенциал духовных сил и энергии, который мы должны сохранить не только для себя, но и для будущих поколений.
    
    Велико значение церковнославянского языка для православного богослужения и православного пастыря. Но церковнославянский язык имеет очень важное значение и для православного хистианина. На этом языке совершается богослужение в нашей Православной Русской Церкви и написаны наши священные и богослужебные книги. По своему возвышенному характеру, по своей силе и звучности церковнославянский язык является наиболее совершенным средством для выражения религиозных настроений православного русского человека. Высшие стремления души и глубокие чувствования, отрешенные от земного и направленные к небесному, чистому и вечному, получают наиболее соответствующее выражение в этом языке, далеком от всего обычного, житейского. Но для этого необходимо, чтобы все читаемое в храме Божием достигало своей цели, т.е. доходило бы до сознания души христианина, научая ее истинному благочестию и указывая ей путь спасения.
    
    \section*{Краткий учебник церковнославянского языка\\
        \emph{предисловие}}
    \label{sec:brief}
    \addcontentsline{toc}{section}{\nameref{sec:brief}}
    
    Составитель сего учебника доброй памяти emph{профессор Анатолий Васильевич Ушков}.
    
    Анатолий Васильевич Ушков родился 7 августа 1894 года в г. Самаре в семье сотрудника Самарской Духовной Консистории. По окончании четырех классов Самарской Духовной семинарии, он в 1912 году успешно сдал вступительные экзамены в Казанский университет и был зачислен на физико-математический факультет. После университета, в 1916 году, А. Ушков был мобилизован на военную службу; проходил обучение в Киевском артиллерийском училище. С 1918 по 1945 г. он преподавал математику и физику в средних и специальных учебных заведениях в городах Самаре, Красноярске и Москве. Заведовал учебной частью и был руководителем методических совещаний по вопросам преподавания физики и математики. С 1939 по 1943 г. заочно учился в Московском государственном педагогическом институте на факультете русского языка и литературы, закончив который, некоторое время, наряду с преподаванием математики, вел курс литературы в средних учебных заведениях г. Москвы.
    
    В 1945 г. он осуществил свою давнишнюю мечту~---~послужить Святой Церкви и поступил в Московскую Духовную Академию. По окончании академии, в 1949 г., со званием кандидата богословия, полученным за сочинение <<Душа и ее бессмертие по христианскому учению>>, он преподавал катехизис, а затем церковнославянский язык в духовной семинарии. В 1964 г., по прочтении двух пробных лекций <<Древний мир перед пришествием Христа Спасителя>> и <<Основные положения формальной логики>>, А.В. Ушкову было присвоено звание доцента и поручено чтение лекций по логике в академии. Им составлен ряд учебных пособий для духовных школ: <<Краткая пасхалия>>, <<Логика в курсе академического образования>>, <<Краткая пасхалия в общедоступном изложении>>, <<Астрономический справочник>>, <<Краткое описание солнечной системы>>, <<Календарный счет времени>>, <<Счет и мера с древнейших времен до наших дней>>. Анатолием Васильевичем был подготовлен учебный курс церковнославянского языка, за который Совет академии присвоил ему 15 декабря 1969 г. степень магистра богословия и утвердил в должности профессора.
    
    Анатолий Васильевич отличался большой скромностью, чутким, внимательным отношением к людям; поражала его необычайная работоспособность, величайшее трудолюбие. Жажда проповеди слова Божия была его духовной потребностью. Он произнес 65 проповедей в праздничные и воскресные дни в академическом храме, в них~---~яркое выражение его глубоких христианских чувств и безграничной, сердечной любви к Богу и людям. В начале 1971---1972 учебного года Анатолий Васильевич по болезни вынужден был оставить работу в академии и перейти на пенсию.
    
    Скончался А.В. Ушков 14 января 1972 г., на 78-м году жизни, приобщившись Святых Христовых Тайн. Похоронен на кладбище в Сергиевом Посаде.
    
    Подготовил к печати доцент-иерей В. Москвич.
    
    \chapter*{1-й класс}
    \label{ch:firstgrade}
    \addcontentsline{toc}{chapter}{\nameref{ch:firstgrade}}
        \section*{Введение}
        \label{sec:intro}
        \addcontentsline{toc}{section}{\nameref{sec:intro}}
                \subsubsection[Значение церковнославянского языка]{Значение церковнославянского языка для православного богослужения и православного пастыря}
                
    Церковнославянский язык имеет очень важное значение для православного христианина. На этом языке совершается богослужение в нашей Православной Русской Церкви и написаны наши священные и богослужебные книги. По своему возвышенному характеру, по своей силе и звучности церковнославянский язык является наиболее совершенным средством для выражения религиозных настроений православного русского человека. Высшие стремления духа и глубокие чувствования, отрешенные от земного и направленные к небесному, чистому и вечному, получают наиболее соответствующее выражение в этом языке, далеком от обычного, житейского.
    
                \subsubsection{Возникновение письменности у славян}
                
    Славянский язык назван именно \emph{славянским} потому, что на нем говорили наши предки славяне.
    
    Славяне, обитавшие первоначально около Карпат, долгое время были непросвещенным языческим народом. Очень часто им случалось соприкасаться с культурными народами Византийской и Римской империй. Мало-помалу славяне стали подчиняться влиянию этих уже просвещенных христианских народов.
    
                \subsubsection{Деятельность святых братьев Кирилла и Мефодия}
        \section{Церковнославянское буквоначертание}
                \subsubsection{Церковнославянская азбука}
                \subsubsection{Основные правила церковнославянского чтения и письма}
                \subsubsection{Особенности правописания и произношения некоторых букв в церковнославянском языке}
                \subsubsection{Правописание и произношение букв, перешедших в церковнославянский язык из греческого языка}
                \subsubsection{Правописание букв\chs{ꙗ}и\chs{ѧ}}
                \subsubsection{Надстрочные знаки}
                \subsubsection{Буквы как числовые знаки (цифры)}
        \section{Общие понятия о грамматических формах}
                \subsubsection{Разделение звуков}
                \subsubsection{Чередование звуков}
                \subsubsection{Состав слова}
                \subsubsection{Слова простые и сложные}
                \subsubsection{Славянское неполногласие}
                \subsubsection{Части речи и их грамматические формы}
                \subsubsection{Понятие о предложении}
                \subsubsection{Пунктуация}
        \section{Глагол}
                \subsubsection{Понятие о глаголе}
                \subsubsection{Глаголы архаические}
                \subsubsection{Неопределенная форма глаголов}
                \subsubsection{Спряжение в настоящем времени вспомогательного глагола\chs{бы҆́ти}}
                \subsubsection{Спряжение в настоящем времени прочих глаголов}
                \subsubsection{Глагольные основы}
                \subsubsection{Глаголы тематические и разделение их на два спряжения}
                \subsubsection{Преходящее время вспомогательного глагола\chs{бы҆́ти}}
                \subsubsection{Аорист вспомогательного глагола\chs{бы҆́ти}}
                \subsubsection{Преходящее время тематических глаголов}
                \subsubsection{Аорист тематических глаголов}
                \subsubsection{Глаголы вида совершенного и несовершенного}
                \subsubsection{Будущее время глаголов}
                \subsubsection{Понятие о наклонениях глагола}
                \subsubsection{Желательное наклонение глаголов}
                \subsubsection{Повелительное наклонение глаголов}
                \subsubsection{Спряжение архаических глаголов}
                \subsubsection{Будущее время описательное}
        \section{Склоняемые части речи}
            \subsection{Имя существительное}
                \subsubsection{Родовые окончания имен существительных}
                \subsubsection{Первое склонение имен существительных}
                \subsubsection{Общие замечания к первому склонению имен существительных}
                \subsubsection{Имена существительные первого склонения с основой на шипящие и на\chs{ц}}
                \subsubsection{Второе склонение имен существительных}
                \subsubsection{Общие замечания ко второму склонению имен существительных}
                \subsubsection{Имена существительные второго склонения с основой на шипящие и на\chs{ц}}
                \subsubsection{Третье склонение имен существительных}
            \subsection{Имя прилагательное}
                \subsubsection{Разделение имен прилагательных}
                \subsubsection{Краткие имена прилагательные}
                \subsubsection{Склонение кратких имен прилагательных с твердым окончанием}
                \subsubsection{Склонение кратких имен прилагательных с мягким окончанием}
                \subsubsection{Склонение кратких имен прилагательных с основой на шипящие}
            \subsection{Местоимение}
                \subsubsection{Понятие о местоимении}
                \subsubsection{Склонение личных местоимений\chs{а҆́зъ}и\chs{ты̀}и возвратного\chs{себѐ}}
                \subsubsection{Особенности глагольного сказуемого в предложении}
                \subsubsection{Согласование слов в предложении}
                \subsubsection{Управление слов в предложении}
                \subsubsection{Обращение}
    \chapter*{2-й класс}
    \label{ch:secondgrade}
    \addcontentsline{toc}{chapter}{\nameref{ch:secondgrade}}
    \setcounter{section}{0}
    \setcounter{subsubsection}{0}
        \section{Склоняемые части речи}
            \subsection{Имя существительное}
                \subsubsection{Имена существительные неравносложные}
                \subsubsection{Имена существительные разносклоняемые}
            \subsection{Местоимение}
                \subsubsection{Склонение личного местоимения 3-го лица \chs{ѻ҆́нъ}}
                \subsubsection{Склонение притяжательных местоимений}
                \subsubsection{Склонение определительного местоимения\chs{ве́сь}}
                \subsubsection{Склонение указательного местоимения\chs{се́й}}
                \subsubsection{Склонение вспомогательных местоимений\chs{кто̀}и\chs{что̀}}
                \subsubsection{Склонение указательного местоимения\chs{то́й}(тот)}
                \subsubsection{Склонение указательного местоимения\chs{ѻ҆́нъ, ѻ҆́ный}}
            \subsection{Имя прилагательное}
                \subsubsection{Склонение полных имен прилагательных}
                \subsubsection{Общие замечания к склонению полных имен прилагательных}
                \subsubsection{Понятие о степенях сравнения имен прилагательных}
                \subsubsection{Сравнительная степень имен прилагательных}
                \subsubsection{Превосходная степень имен прилагательных}
                \subsubsection{Неправильные степени сравнения}
            \subsection{Имя числительное}
                \subsubsection{Понятие об имени числительном}
                \subsubsection{Имена числительные количественные}
                \subsubsection{Имена числительные порядковые}
                \subsubsection{Склонение количественных числительных}
        \section{Глагол}
                \subsubsection{Понятие о причастии}
                \subsubsection{Несклоняемое (спрягаемое) причастие}
                \subsubsection{Прошедшее совершенное время глаголов}
                \subsubsection{Давнопрошедшее время глаголов}
                \subsubsection{Условное наклонение глаголов}
                \subsubsection{Безличные глаголы}
                \subsubsection{Возвратная форма глагола}
                \subsubsection{Страдательная форма глагола}
                \subsubsection{Склоняемые причастия}
                \subsubsection{Краткие действительные причастия настоящего времени}
                \subsubsection{Полные действительные причастия настоящего времени}
                \subsubsection{Краткие действительные причастия прошедшего времени}
                \subsubsection{Полные действительные причастия прошедшего времени}
                \subsubsection{Краткие страдательные причастия настоящего времени}
                \subsubsection{Полные страдательные причастия настоящего времени}
                \subsubsection{Краткие страдательные причастия прошедшего времени}
                \subsubsection{Полные страдательные причастия прошедшего времени}
                \subsubsection{Сложная страдательная форма глаголов}
        \section{Неизменяемые части речи}
                \subsubsection{Наречие}
                \subsubsection{Предлог}
                \subsubsection{Союз}
                \subsubsection{Частицы}
                \subsubsection{Междометия}
        \section{Некоторые особенности церковнославянского синтаксиса}
                \subsubsection{Двойной винительный падеж}
                \subsubsection{Падеж родительный-разделительный}
                \subsubsection{Замена притяжательных местоимений личными и возвратными в дательном падеже}
                \subsubsection{Употребление местоимения и имени прилагательного в смысле имени существительного}
                \subsubsection{Отсутствие отрицания при глаголе при наличии местоимений\chs{никто̀, ничто̀}}
                \subsubsection{Славянский член и его употребление}
                \subsubsection{Особенности славянского придаточного предложения}
                \subsubsection{Особенности славянских придаточных предложений с союзными словами \chs{є҆́же, во є҆́же, ѡ є҆́же}}
                \subsubsection{Особенности славянского придаточного предложения}
                \subsubsection{Оборот <<Дательный самостоятельный>>}
                \subsubsection{Оборот <<Дательный с неопределенным>>}
                \subsubsection{Оборот <<Винительный с неопределенным>>}

\end{document}