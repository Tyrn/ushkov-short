\documentclass[11pt,a4paper]{memoir}

\usepackage[usenames,dvipsnames,svgnames,table]{xcolor}
\usepackage{fontspec}
\usepackage{xltxtra}
% code borrowed from Polyglossia documentation — Thanks!
\definecolor{myblue}{rgb}{0.02,0.04,0.48}
\definecolor{lightblue}{rgb}{0.61,.8,.8}
\definecolor{myred}{rgb}{0.65,0.04,0.07}

\usepackage{polyglossia}
\setmainlanguage{russian}
\setotherlanguages{churchslavonic,english}
\usepackage{churchslavonic}
\usepackage{lettrine}

\setmainfont[Mapping=tex-text]{Liberation Serif}
\setsansfont[Mapping=tex-text]{Liberation Sans}
\setmonofont[Mapping=tex-text]{Liberation Mono}

\newfontfamily\churchslavonicfont[Script=Cyrillic,Ligatures=TeX,Scale=1.33333333,HyphenChar="005F]{PonomarUnicode.otf} 
\newfontfamily{\slv}[Scale=MatchLowercase]{Ponomar Unicode TT}
\newfontfamily{\ust}[Scale=MatchLowercase]{Menaion Unicode TT}
\newfontfamily{\ind}{Indiction Unicode TT}

\usepackage{indentfirst}
\frenchspacing
\clubpenalty=10000
\widowpenalty=10000

\sloppy

%% Ensure sequential numbering of "paragraph" sections.
\usepackage{chngcntr}
\counterwithout{section}{chapter}

%% Insert Church Slavonic snippet into (Russian) text
\newcommand{\chs}[2][slv]{
    \begin{#1}\ignorespaces
        #2
    \ignorespacesafterend\end{#1}
}

\begin{document}
    
    \tableofcontents
    
    \frontmatter
    
    \chapter*{\small БЛАГОЛЕПИЕ ЦЕРКОВНОСЛАВЯНСКОГО ЯЗЫКА}
    Прошло уже более тысячи лет со времени Крещения Руси. Воспринятое ею Православие, благодаря трудам святых Мефодия и Кирилла, совершает свои благоговейные богослужения на церковнославянском языке. Он по своей структуре наиболее близок к греческому, и это позволило сохранить существующий в Греческой церкви строй богослужения, что явилось для многочисленных славянских народов великим благом и живительным источником благочестия и совершенствования нравственного в духе истины правой веры, а также руководством в земных делах: образования, иконографии, храмо- и градостроительства. Наша Русская Православная Церковь восприняла в неповрежденности и чистоте от Греческой Матери Церкви все догматическое, литургическое, духовное, святоотеческое наследие и предание, связав все это, с помощью славянского языка, с духом народа христианского на Руси, с жизнью и бытом и просвещением, ибо на славянский язык были переведены все богослужебные книги и писания греческих отцов, прием сохранялась точность, ясность и истинность, хотя перевод и являлся творческим делом.
    
    Славянский язык помогал памятовать о Христе, хранить Его живой образ и Его учение в сердце и душе и приносить плоды живого творческого восприятия слова Божия. В Церкви шло становление и образование самого славянского языка, его внутренняя христианизация и воцерковление, преображение самой стихии славянской мысли и слова, славянского голоса, души народа, На долю русского народа выпало редкое счастье принять христианство тогда, когда прогремели громом своим Вселенские Соборы и утверждено в чистоте апостольское Православие, и в Церкви Греческой утвердились и литургические, и догматические, и нравственные истины и самый язык славянский, ничем не запятнанный и чистый, воспринял для своего народа сосуд благодати, и в благодати Христовой народ верный обновляется, и просвещается, и спасается, и становится причастным вечной Истине, и славянский язык богатство веры и культуры передал многим полудиким племенам и народам, и в Церкви они нашли единение и братство Христово как дети Божии. На Руси потрудились многие ревнители духовного просвещения: и князь Владимир~---~просветитель, и св. Андрей Боголюбский, и свв. Александр Невский, Даниил Московский, святитель Петр, митрополит Алексий, преподобный Сергий Радонежский и многие другие подвижники благочестия. Созидались монастыри, храмы, школы. Создавались библиотеки, делались переводы на славянский язык, составлялись азбуки для малых и полудиких народностей, писались летописи.
    
    Славянский язык хранил церковное единство и православные традиции. 2 августа 1581 г. впервые была напечатана полная Библия на славянском языке в г. Остроге, что способствовало глубокому изучению Священного Писания. Русская Церковь в XVI в., когда протестанты превращали христианство в систему философских построений, как плод похи Возрождения язычества, начала вести длительную и тяжелую борьбу за сохранение чистоты веры, церковнославянского языка, церковно-национальной культуры.
    
    Славянская Библия занимает самое первое место по своей полноте и точности воспроизведения библейских древних писаний, о чем свидетельствуют и списки Кумранских рукописей, написанных за 200---300 лет до Р.Х., когда не было нужды искажать тексты Священного Писания. Древние рукописи своим текстом совпадают с острожской славянской Библией. В славянской Библии точно сохранено летоисчисление~---~5508 лет от сотворения мира до Рождества Христова, и все мессианские места. Известно, что текст 70 толковников, взятый в основу славянской Библии, долгое время считался в александрийских синагогах у евреев священным переводом, и его читали на греческом языке 200---300 лет. Масоретский же текст Библии идет от Акибы (около II века после Р.Х.).
    
    Кирилло-Мефодианский перевод Евангелия на славянский язык отличается точностью, верностью подлиннику и ясностью, чистотою выражения христианских понятий.
    
    Славянский текст Священного Писания и православного богослужения~---~свидетель первого тысячелетия, то есть неразделенного христианства, и в то же время он является драгоценным наследием равноапостольских трудов свв. Кирилла и Мефодия, на котором воспитываются чада Русской Православной Церкви.
    
    Церковнославянский богослужебный текст является важным и святым. Христиане разных конфессий стремятся осмыслить свои традиции в сравнении с традициями других исповеданий, ведь священные книги Нового Завета переведены на 1050 языков, и среди этих переводов самое важное место занимает славянский. Поскольку перевод Писания был сделан с греческого языка, а греческая текстология устанавливала единый печатный греческий текст XVII в., потом в XX веке его вытеснили из употребления, то в вопросе изысканий и оценки конечных результатов многих реконструкций текстов важную роль играет славянский текст как свидетель текстовой традиции первого тысячелетия, то есть времени неразделенного христианства.
    
    Острожская Библия 1581 года на славянском языке ни по форме, ни по содержанию не изменилась и является эталоном, по которому можно проследить реконструкцию греческого текста по времени.
    
    Анализ и сопоставление текстов показывает, что славянский текст стоит на твердой, нерушимой основе древней богослужебной практики Православия, а западное христианство отошло от древних традиций.
    
    Святые равноапостольные братья Кирилл и Мефодий использовали для перевода церковный византийский текст, который не искажался, был под контролем и оберегался всегда и всюду в Церкви, как нечто постоянное, неизменное и непреложное, как сама Истина, как церковно-текстуальное предание, уходящее в апостольскую древность. Это предание Церкви святое, соборное, апостольское, как поток, исходит из свежего родника живой воды евангельской, течет по чистому руслу под ясным небом и при тихой погоде, и его прозрачные воды менее доступны для загрязнения человеческими мудрованиями во имя кичащегося ума и в ущерб созидающей любви.
    
    В текстах встречаются несколько десятков слов труднопонятных, как тектон~---~дровосек, усмен~---~кожан, измена~---~выкуп, выну~---~всегда и др., но здравый человек за неделю их запомнит, ведь даже иностранные языки учат, да не один, а два-три языка, заучивая 40-45 слов в день, а язык, когда-то бывший родным, разговорным языком наших предков, язык, освященный веками, надо любить и хранить.
    
    Славянский язык в Церкви на Руси освящен непрерывным богослужение, и внешняя гармония, чистота, сердечность, молитвенность, мир Христов и утешение благодатное водворяются в душе православного христианина, как плод веры и любви.
    
    Большое значение славянскому языку для православного богослужения придает и Грузинская Церковь. Архимандрит Рафаил (Карелин) из Тбилиси выразил то так: <<В настоящее время в литургической жизни Грузинской Автокефальной Церкви славянский язык является по своему значению вторым, после древнегрузинского, богослужебным языком. Церковная славистика имеет для нас практическую важность и теоретический интерес>>.
    
    В связи с этим необходимо остановиться на нескольких вопросах, касающихся значения славянского языка для современного православного богослужения, рассматривая при этом славянский язык обобщенно и условно, как единую структурную и знаковую систему.
    
    В данном докладе славянский язык рассматривается обобщенно и условно, как единая структурная и знаковая система.
    
    Славянский язык насчитывает более чем тысячелетнюю литературную историю. Он не стал мертвым языком, не сделался достоянием одних археологов и лингвистов; славянский язык звучит на всех континентах, на нем совершается богослужение в православных церквах, на нем миллионы людей воссылают молитвы к Богу.
    
    В последнее время неоднократно поднимался вопрос: не является ли славянский язык литургическим архаизмом, преградой к пониманию текстов, и не следует ли перевести все богослужебные книги на современные языки?
    
    Такие попытки предпринимались в России и других странах.
    
    В России языковые литургические реформы не встретили поддержки основной массы верующих, и на сегодняшний день богослужебным языком остается церковнославянский. Можно ли объяснить это явление только одной приверженностью к традиции, или же оно имеет еще другие основания?
    
    Полагаем, что эти основания следующие.
    
    I. Славянскому языку, как и другим древним языкам (имеется в виду древнегреческий и древнегрузинский), присуща особая динамическая структура; он лучше передает пульс религиозной жизни, глубже выражает молитвенные чувства, чем современные языки.
    
    Древние языки более подходят к цельному синтетическому восприятию, новые~---~к аналитическому, дробному; древние~---~к созерцанию, новые~---~к логизированию; древние языки полны энергии и эмоций, новые, в сравнении с ними, носят рационалистический, описательный характер. Древние языки дают большую возможность соприкоснуться с глубиной явлений, с духовными субстанциями, сделать человека участником событий, современные~---~изложить явления в определенной системе и дать их анализ.
    
    Сравнивая лексическую и грамматическую ткань древних и новых языков, мы видим, что древние языки располагают меньшим по объему словарным фондом, но их грамматический строй отличается большим многообразием, пластичностью и совершенством.
    
    Мобильность и экспрессивность глагольных форм, богатство аффиксов и флексий, не имеющих аналогии в современных языках, лаконизм и динамика синтаксических структур, этимологическая глубина лексики создают неповторимую выразительность и, по словам патриарха Пимена, <<особую красоту>> славянского языка. Эти возможности славянского языка позволяют выразить в богослужебных текстах многоплановость событий, синергизм и параллелизм действий, объединить различные хронологические периоды в единые смысловые циклы (как бы преолодев линейную протяженность времени), проявить и усилить подтекст каждого предложения.
    
    Сложность и гибкость глагольных форм славянского языка делают одну и ту же по словарному составу фразу то ажурно-легкой и полетной, то~---~до физической ощутимости тяжелой и твердой, как бы изваянной из мрамора. Лаконизм, внутренняя чеканность и в то же время как бы внешняя незавершенность и обрывистость предложений, часто без последовательных переходов от одного предмета к другому, наделяют смысловые паузы эмоциональным содержанием, подчеркивая, что богослужебный текст~---~это не монолог, а диалог, таинственная беседа души с Божеством, что в священных событиях присутствует, как их постоянный субъект, личность каждого молящегося.
    
    Говоря образно, динамика древних языков созвучна динамике света. Богослужение~---~это симфония из лучей Божественных энергий. Священная история изображается в этой Божественной симфонии на светящемся фоне вечности, земные реалии~---~в их логоистическом преображении.
    
    Следует отметить, что в богослужебных текстах мастерски использован точно и строго очертанный круг изобразительных средств. Это делалось древними гимнографами для того, чтобы не подменить духовных переживаний эстетическими, т.е. душевными, не обременить ума молящегося художественной информацией и, тем самым, не оземлить молитву.
    
    Древние гимнографы мыслили Церковь как идею единства в Ее реальном и мистическом воплощении. Унификация изобразительных средств~---~одно из условий общности молитвы, духовного познания, эмоций и сопереживаний верующих~---~тела Церкви.
    
    Динамичные структуры, внутренние смысловые емкости и скульптурная пластичность славянского языка помогают воспринимать традиционные символы и метафористические образы в разных контекстах как вечно новые и неповторимые.
    
    Одним словом, древние языки более приспособлены для выражения явлений и динамики духовной жизни. Это первая и главная причина их сохранения в пра
    
    \mainmatter
    
    \part{Класс}
    
    \chapter*{Введение}
            \subsection{Параграф}
            \subsection{Параграф}
    
    \chapter{Тема}
            \subsection{Параграф}
            \subsection{Параграф}

    \chapter{Тема}
            \subsection{Параграф}
            \subsection{Параграф}

    \chapter{Тема}
            \subsection{Параграф}
            \subsection{Параграф}

    \chapter{Тема}
        \section{секция}
            \subsection{Параграф}
            \subsection{Параграф}

        \section{секция}
            \subsection{Параграф}
            \subsection{Параграф}

        \section{секция}
            \subsection{Параграф}
            \subsection{Параграф}

\end{document}